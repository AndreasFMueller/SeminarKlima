%
% burgertraining.tex -- Trainingsdaten für Burgers Gleichung
%
% (c) 2018 Prof Dr Andreas Müller, Hochschule Rapperswil
%
\subsection{Trainingsdaten für die Gleichung von Burgers\label{burgers:training}}
Damit die Gleichung von Burgers mit einem Machine-Learning-Ansatz gelöst
werden kann, müssen geeignete Trainingsdaten bereitgestellt werden.


\subsubsection{Glatte Lösungen}
Um das Verhalten für glatte Lösungen zu trainineren, generieren
wir Lösungen mit Hilfe der Charakteristiken-Methode.
Dazu müssen wie bei der Wärmeleitungsgleichungen `faire' Anfangsfunktionen 
bereitgestellt werden, welche möglichst viele Steigungen und Funktionswerte,
aber keine exotischen Fälle berücksichtigen.
Die Hermite-Polynome eignen sich wieder ausgezeichnet für diesen Zweck.

Glatte Lösungen der Burgers-Gleichung existieren nur für eine beschränkte Zeit.
Wir müssen aber sicherstellen, dass die Lösungen nur so lange verwendet
werden, bis sich Sprungstellen entwickeln.
Ist
\begin{equation}
u_{\text{max}} = \max \{ -u_0(x)\,|\, x\in\mathbb R\}
\label{burgers:zeit}
\end{equation}
die grösste negative Steigung der Anfangsfunktion, dann tritt die Sprungstelle
für Zeiten $t> 1/u_{\text{max}}$ auf.
Wir wählen daher das folgende Vorgehen.
Zunächst wählen wir eine zufällige Anfangsfunktion.
Dann bestimmen wir die Zeit
gemäss \eqref{burgers:zeit}, bis zu der die zu dieser Anfangsbedingungen
gehörende Funktion keine Sprungstelle entwickelt.
Für diese Zeit berechnen wir die Lösung mit der Charakteristiken-Methode.

\subsubsection{Sprungstellen}
Um das Verhalten bei Sprungstellen zu trainieren generieren wir
stückweise konstante Lösungen mit genau einer Sprungstelle.
Die Burgers-Gleichung besagt, dass sich die konstanten Abschnitte
der Funktion nicht verändern werden, sondern dass sich nur die Sprungstelle
verschieben wird.
Hugoniot-Rankine-Bedingung \eqref{burgers:hugoniot-rankine} besagt,
mit welcher Geschwindigkeit dies geschieht.
Damit können Trainingsfunktionen für Sprungstellen erzeugt werden.

