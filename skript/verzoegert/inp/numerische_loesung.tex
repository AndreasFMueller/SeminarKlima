\section{Numerische Lösung}
\rhead{Numerische Lösung}

Die El-Niño-DDE soll mithilfe von Matlab numerisch gelöst werden.
Matlab stellt zum Lösen von DDEs eine fertige Funktion zu Verfügung (dde23).
Da diese Funktion auf anderen Systemen (z.B. Octave) nicht verwendbar ist, soll eine eigene Lösungsfunktion geschrieben werden.
Beim Schreiben dieser Funktion wird darauf geachtet, dass die Syntax mit dde23 vergleichbar ist. 

Es werden zwei verschiedene Ansätze implementiert: 
\begin{itemize}
	\item Berechnung von endlich kurzen Zeitschritten
	\item Berechnung über die Laplacetransformation
\end{itemize}

\subsection{Analyse der Funktion dde23}
Die offizielle Syntax von dde23 \footnote{https://www.mathworks.com/help/matlab/ref/dde23.html} lautet: 
\begin{lstlisting}{dde23}
	sol = dde23(ddefun,lags,history,tspan);
\end{lstlisting}
Wir analysieren zunächst alle Parameter.

\subsubsection{Parameter: ddefun}
Die ddefun stellt die eigentliche DDE dar, welche als Funktion übergeben werden muss.
Unsere El-Niño-DDE \ref{eldde} hat als eigene Parameter die Zeit (t), den aktuellen Wert (y), die verzögerten Werte (Z) und alle Konstanten.
\begin{lstlisting}{dde_full}
	function dydt = dde_full(t,y,Z,c,a,b,e)
	ylag1 = Z(:,1);
	ylag2 = Z(:,2);
	dydt = -c*y+a*ylag1-b*ylag2-e*y.^3;
\end{lstlisting}
Damit diese Funktion akzeptiert wird, müssen natürlich die Konstanten gesetzt sein.
\begin{lstlisting}{my_dde}
	c = 1; a = 2.6; b = 3; e = 0.1;
	my_dde = @(t,y,Z) dde_full(t,y,Z,c,a,b,e);
\end{lstlisting}

\subsubsection{Parameter: lags}
Die Verzögerungen (in Jahren) entsprechen einem simplen Vektor.
\begin{lstlisting}{lags}
	tauk = 0.15; taur = 1;
	tau = [0.5*tauk 0.5*taur+tauk];
\end{lstlisting}

\subsubsection{Parameter: history}
Die history entspricht einer Funktion, welche die Werte aus der Vergangenheit ausgibt. 
Das kann mit Vektoren (mit realen Daten\footnote{http://www.cpc.ncep.noaa.gov/data/indices/sstoi.indices}) und einer Interpolation gelöst werden.
\begin{lstlisting}{hist}
	function s = dde_hist(t)
	t_v = [-0.67,-0.58,-0.5,-0.42,-0.33,-0.25,-0.167,-0.083,0];
	s_v = [0.71,0.5,-0.06,-0.4,-0.68,-0.23,-0.2,-0.66,-0.83];  
	s = @(t) interp1(t_v,s_v,t);
\end{lstlisting}

\subsubsection{Gesamte Anwendung}
Der Parameter tspan gibt die zu berechnende Zeitspanne (hier 0-3 Jahre) an.
\begin{lstlisting}{Anwendung}
	sol = dde23(my_dde,tau,dde_hist,[0, 3]);
\end{lstlisting}
Der Aufruf dde23 soll nun durch eine eigene Funktion ersetzt werden.
 

\subsection{Berchnung von endlich kurzen Zeitschritten}
Bei diesem Ansatz wird immer die Ableitung zu einer bestimmten Zeit berechnet.
Diese Ableitung wird dann für einen (kurzen) Zeitschritt als Konstant genommen und damit nächste Wert berechnet.
\begin{algorithm}
	\caption{Numerischer DDE-Solver}
	\label{algo1}
	\begin{algorithmic}[1]
		\State Initialisieren, d.h. Zeitachse erstellen, Zeitschritt dt berechnen, etc
		\For{dt in t}
		\State Alle verzögerten Werte finden (entweder in dde\_hist oder in alter Lösung)
		\For{i in tau}
		\State Korrekten verzögerten Wert für jedes $\tau$ speichern
		\EndFor
		\State dde-Funktion aufrufen und dydt speichern
		\State Nächster Wert = aktueller Wert + dydt*dt
		\EndFor
	\end{algorithmic}
\end{algorithm}

\subsection{Laplacetransformation}
Die Laplacetransformation beschreibt eine Transformation vom Zeit zum (komplexen) Frequenzbereich.
Diese Transformation ist folgendermaßen definiert:
\begin{equation}
	todo Laplace
\end{equation} 


