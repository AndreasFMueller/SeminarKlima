\section{Numerische Lösung}
\rhead{Numerische Lösung}

Die El-Niño-DDE soll mithilfe von Matlab numerisch gelöst werden.
Matlab stellt zum Lösen von DDEs eine fertige Funktion zu Verfügung (dde23).
Da diese Funktion auf anderen Systemen (z.B. Octave) nicht verwendbar ist, soll eine eigene Lösungsfunktion geschrieben werden.
Beim Schreiben dieser Funktion wird darauf geachtet, dass die Syntax mit dde23 vergleichbar ist. 

Es werden zwei verschiedene Ansätze implementiert: 
\begin{itemize}
	\item Berechnung von endlich kurzen Zeitschritten
	\item Berechnung über die Laplacetransformation
\end{itemize}

\subsection{Analyse der Funktion dde23}
Die offizielle Syntax von dde23 \footnote{https://www.mathworks.com/help/matlab/ref/dde23.html} lautet: 
\begin{lstlisting}{dde23}
	sol = dde23(ddefun,lags,history,tspan);
\end{lstlisting}
Wir analysieren zunächst alle Parameter.

\subsubsection{Parameter: ddefun}
Die ddefun stellt die eigentliche DDE dar, welche als Funktion übergeben werden muss.
Unsere El-Niño-DDE \ref{eldde} hat als eigene Parameter die Zeit (t), den aktuellen Wert (y), die verzögerten Werte (Z) und alle Konstanten.
\begin{lstlisting}{dde_full}
	function dydt = dde_full(t,y,Z,c,a,b,e)
	ylag1 = Z(:,1);
	ylag2 = Z(:,2);
	dydt = -c*y+a*ylag1-b*ylag2-e*y.^3;
\end{lstlisting}
Damit diese Funktion akzeptiert wird, müssen die Konstanten gesetzt werden.
\begin{lstlisting}{my_dde}
	c = 1; a = 2.6; b = 3; e = 0.1;
	my_dde = @(t,y,Z) dde_full(t,y,Z,c,a,b,e);
\end{lstlisting}

\subsubsection{Parameter: lags}
Die Verzögerungen (in Jahren) entsprechen einem simplen Vektor.
\begin{lstlisting}{lags}
	tauk = 0.15; taur = 1;
	tau = [0.5*tauk 0.5*taur+tauk];
\end{lstlisting}

\subsubsection{Parameter: history}
Die history entspricht einer Funktion, welche die Werte aus der Vergangenheit ausgibt. 
Das kann mit Vektoren (mit realen Daten\footnote{http://www.cpc.ncep.noaa.gov/data/indices/sstoi.indices}) und einer Interpolation gelöst werden.
\begin{lstlisting}{hist}
	function s = dde_hist(t)
	t_v = [-0.67,-0.58,-0.5,-0.42, ...];
	s_v = [0.71,0.5,-0.06,-0.4,...];  
	s = @(t) interp1(t_v,s_v,t);
\end{lstlisting}

\subsubsection{Gesamte Anwendung}
Der Parameter tspan gibt die zu berechnende Zeitspanne (hier 0-3 Jahre) an.
\begin{lstlisting}{Anwendung}
	sol = dde23(my_dde,tau,dde_hist,[0, 3]);
\end{lstlisting}
Der Aufruf dde23 soll nun durch eine eigene Funktion ersetzt werden.
 

\subsection{Berchnung von endlich kurzen Zeitschritten}
Bei diesem Ansatz wird immer die Ableitung zu einer bestimmten Zeit berechnet.
Diese Ableitung wird dann für einen (kurzen) Zeitschritt als Konstant genommen und damit nächste Wert berechnet.
\begin{algorithm}
	\caption{Numerischer DDE-Solver}
	\label{algo1}
	\begin{algorithmic}[1]
		\State Initialisieren, d.h. Zeitachse erstellen, Zeitschritt dt berechnen, etc
		\For{dt in t}
		\State Bestimmen ob die verzögerten Werte dde\_hist oder in alter Lösung (wenn t > $\tau$) vorkommen
		\For{i in tau}
		\State Korrekten verzögerten Wert für jedes $\tau$ finden
		\EndFor
		\State dde-Funktion aufrufen und dydt speichern
		\State Nächster Wert = aktueller Wert + dydt*dt
		\EndFor
	\end{algorithmic}
\end{algorithm}

Mit diesem Algorithmus wird ein identisches Ergebnis wie mit dde23 erreicht. 
Es werden jeweils 10000 Datenpunkte berechnet. 
Bei mehr Datenpunkten dauert die Berechnung sehr lange, wohl auch weil der Algorithmus nicht optimiert ist.
Bei dde23 werden keine fixen Zeitschritte dt berechnet, sondern eine Abschätzung des Fehlers wird vorgenommen.
Falls der geschätzte Fehler hoch ist, wird ein kleiner Zeitschritt berechnet. 
Bei kleinem Fehler (z.B. Sinus-Maximum) werden entsprechend große Zeitschritte gerechnet. %todo add Grafik


\subsection{Laplacetransformation}
Die Laplacetransformation beschreibt eine Transformation vom Zeit zum (komplexen) Frequenzbereich.
Diese Transformation ist folgendermaßen definiert:
\begin{equation}
	F(s)=\int_{0}^{\infty}f(t)e^{-st}dt \text{ wobei } s=j\omega
\end{equation} 
Für die Laplacetransformation gibt es einige bekannte Eigenschaften. 
Eine davon ist die Verschiebung im Zeitbereich, eine Andere die Ableitung.
\begin{align}
	f(t-t_0)\: \multimapdotbothA \: F(s)e^{-t_0 s}\\
	\frac{\partial f(t)}{\partial t}\: \multimapdotbothA \: sF(s)-f(0^+)
\end{align}
Mit dieser Eigenschaft können wir nun unsere DDE \ref{eldde} Transformieren.
Wir verwenden für die Laplacetransformierte der Temperatur $T$ das Formelzeichen $F$.
Da es keine Regeln für den kubischen Term gibt, wir $\epsilon = 0$ gesetzt.
\begin{align}
	T(t)\: \multimapdotbothA \: F(s)\\
	sF(s)-T(0^+)=-cF(s)+aF(s)e^{-\frac{1}{2}\tau_K s}-bF(s)e^{-(\frac{1}{2}\tau_R + \tau_K) s}
\end{align}
Diese Gleichung können wir nach $F(s)$ auflösen und erhalten:
\begin{equation}
	F(s) = \frac{T(0)}{s+c-ae^{-\frac{1}{2}\tau_K s}+be^{-(\frac{1}{2}\tau_R + \tau_K)s}}
\end{equation}
%todo Erklären wieso es nicht funktioniert
