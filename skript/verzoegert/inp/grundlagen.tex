%
% main.tex -- Paper zum Thema verzoegerte Differentialgleichung
%
% (c) 2018 Raphael Unterer, Hochschule Rapperswil
%
\section{Grundlagen verzögerte Differentialgleichungen}
\rhead{Grundlagen DDE}
\subsection{Definitionen}
Verzögerte Differentialgleichungen werden als DDE (engl. "\textbf{D}elayed \textbf{D}ifferential \textbf{E}quation") abgekürzt.

Die allgemeine DDE 1. Ordnung sieht folgendermaßen aus:
\begin{equation}
	\dot{x}(t) = f(x(t),x(t-\tau_1),...,x(t-\tau_n))
\end{equation}
Dabei ist $f$ eine beliebige Funktion. Die Verzögerungen $\tau_1,...,\tau_n$ sind gegeben und nach der Grösse geordnet, also $0<\tau_1<...<\tau_n$.

Im Unterschied zu einer gewöhnliche Differentialgleichung ist das Anfangswertproblem nicht mehr eindimensional, d.h. es genügt nicht mehr den Anfangszustand zu kennen.
Alle Werte von $-\tau_n$ bis $0$ müssen gegeben sein. 
Es braucht somit eine unendliche Anzahl Anfangswertvektoren, welche bekannt sein müssen.

In den folgenden Betrachtungen analysieren wir die einfachste DDE:
\begin{equation}\label{bsp}
\dot{y}(t)=ky(t-\tau)
\end{equation}

\subsection{Schrittweises Lösen}
Beim schrittweisen wird die DDE immer in Schritten von einem $\tau$ gelöst.
Wir nehmen an, dass $y$ im Bereich von $-\tau$ bis $0$ immer Konstant bleibt
\begin{equation}
	y(t)=1 \text{ wenn } -1\le t<0
\end{equation}
Daraus folgt, dass im Bereich von $0\le t<\tau$ die Ableitung
\begin{equation}\label{abl}
	\dot{y}(t)=k
\end{equation}
wird. Durch integrieren von \ref{abl} erhalten wir
\begin{equation}
	y(t)=1+kt
\end{equation}
für den Bereich $0\le t<\tau$. 
Dieses $y(t)$ kann als Anfangswert für den nächsten Schritt genommen werden.
Wir erhalten für den zweiten Schritt  $\tau\le t<2\tau$ 
\begin{equation}\label{abl2}
	\dot{y}(t)=k(1+k(t-\tau))=k+k^2(t-\tau)
\end{equation}
Es ist offensichtlich das diese Methode nur für kurze Zeiten, einfache Anfangswerte und einfache Formeln funktioniert. 
Bereits \ref{abl2} ist nicht mehr ganz einfach zu Integrieren. 
Für längere Zeiten werden die Integrale immer komplexer. %todo ev simulation matlab

\subsection{Charakteristisches Polynom}
Bei gewöhnlichen Differentialgleichungen können Lösungen mit Hilfe des charakteristischen Polynoms gefunden werden. 
Bei DDEs wird das Polynom zu einer Gleichung. 
Wir betrachten wiederum die Gleichung \ref{bsp} und verwenden als Lösungsansatz
\begin{equation}\label{ansatz}
	y(t) = ce^{\lambda t}
\end{equation}
Dieser klassische Ansatz eignet sich (fast) immer, da die Exponentialfunktion beim differenzieren erhalten bleibt. 
\ref{ansatz} eingesetzt in \ref{bsp} ergibt
\begin{equation}
	\lambda ce^{\lambda t} = kce^{\lambda (t-\tau )}
\end{equation} 
Diese Gleichung kann gekürzt werden zu
\begin{equation}
\lambda  - ke^{-\lambda \tau}= 0
\end{equation} 
Damit können nun verschiedene Werte für die Konstante $k$ berechnet werden, je nachdem wie Lösung aussehen soll. 
Die charakteristische Gleichung ist somit  eine Hilfe um Konstanten zu bestimmen, falls eine bestimmte Lösung erreicht werden soll.