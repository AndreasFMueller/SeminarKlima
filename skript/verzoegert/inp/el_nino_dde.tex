\section{El Niño DDE}
\rhead{El Niño DDE}


\subsection{Gleichung}
Zur Modellierung des El Niño Effektes wurde die folgende DDE gefunden %todo Verweis Gleichung El Niño
\begin{equation} \label{eldde}
\dot{T}(t)=-cT(t)+aT(t-\frac{1}{2}\tau_K)-bT(t-(\frac{1}{2}\tau_R+\tau_K))-\epsilon(T(t))^3
\end{equation}
Mit dieser DDE wird die Änderung der Meerestemperatur $T$ vor der Küste Südamerikas beschrieben.
Die Konstanten $c,a,b,\epsilon$ müssen so bestimmt werden, dass die DDE ein möglichst gutes Resultat ergibt.
Erst wenn diese Konstanten grob bestimmt sind, kann eine sinnvolle numerische Simulation gestartet werden.
Hilfreich sind vor allem ungefähre Verhältnisse zwischen den Konstanten, so dass man zumindest einen Anhaltspunkt für die Simulation hat.
Die Verzögerungen $\tau_K$ und $\tau_R$ sind ungefähr bekannt aus physikalischen Untersuchungen der Rossby- und Kelvinwellen.
\begin{equation}
	\tau_K \approxeq \frac{1}{6}yr \text{ und } \tau_R \approxeq 1 yr
\end{equation}
Die Konstante $\epsilon$ ist nur zur Stabilisation, d.h. wir wählen diese so klein wie möglich.


\subsection{Charakteristische Gleichung}
Die charakteristische Gleichung für die DDE \ref{eldde} scheint sehr schwierig zu sein. 
Aus diesem Grund vereinfachen wir die DDE so weit, bis wir einen Ansatz versuchen können.
Als erstes Linearisieren wir die DDE und erhalten
\begin{equation}
	\dot{T}(t)=-cT(t)+aT(t-\frac{1}{2}\tau_K)-bT(t-(\frac{1}{2}\tau_R+\tau_K))
\end{equation}
Als nächsten Schritt setzen wir $\tau_K=0$ \footnote{$\tau_K=0$ weil $\tau_K << \tau_R$}und erhalten
\begin{equation}
	\dot{T}(t)=-cT(t)+aT(t)-bT(t-(\frac{1}{2}\tau_R))
\end{equation}
Nun stellen wir eine einfache DDE auf mit $\alpha = a-c$, $\beta = b$ und $\tau = \frac{1}{2}\tau_R$.
\begin{equation}
	\dot{T}(t)=\alpha T(t)-\beta T(t-\tau)
\end{equation}
In diese DDE setzen wir nun den bekannten Ansatz $e^{-\lambda t}$ mit $\lambda \in \mathbb{C}$ ein und erhalten
\begin{equation} \label{char}
	\lambda e^{\lambda t} = \alpha e^{\lambda t} - \beta e^{\lambda(t-\tau)} \Longrightarrow \lambda = \alpha-\beta e^{-\lambda \tau}
\end{equation}
Es sind beliebig viele Lösungen für $\Lambda$ und die Konstanten möglich.
Wir suchen nun die Lösung, welche dem El Niño Phänomen entspricht.
Da der El Niño oszilliert\footnote{El Niño oszilliert annähernd Sinusförmig, ohne Dämpfung}, betrachten nur die Lösungen wo $\lambda$ rein imaginär wird. %todo Bild Oszillation El Nino
Wir setzen also $\lambda = i\omega$ und nach der Formel von Gauss wird die Gleichung \ref{char} umgeschrieben zu 
\begin{equation}
	 i\omega = \alpha-\beta(\cos(-\omega \tau)+i\sin(-\omega \tau))
\end{equation}
Es ergeben sich daraus zwei Gleichungen für Imaginär- und Realteil
\begin{equation} \label{bed1}
  	\alpha-\beta\cos(\omega \tau) = 0 \text{ und } \beta\sin(\omega\tau)=\omega
\end{equation}
Wenn diese beiden Gleichungen durcheinander dividiert werden, erhalten wir die Bedingung
\begin{equation} \label{bed}
	\tan(\omega\tau)=\frac{\omega}{\alpha}
\end{equation}
  	
\subsection{Berechnen der Konstanten}
Aus der Bedingung \ref{bed} können die Konstanten näherungsweise berechnet werden.
Zuerst geben wir die Kreisfrequenz der Oszillation $\omega$ an. 
Auf der Abbildung (todo: ref Oszillationsbild) können wir eine Periodendauer zwischen drei und sieben Jahren erkennen.
Wir bestimmen für die folgenden Berechnungen eine durchschnittliche Periodendauer von 4 Jahren. 
Alle Zeitangaben werden in Jahren angegeben.

Daraus ergibt sich $\omega = \frac{2\pi}{T_{Periode}} = \frac{\pi}{2}$ und aus der Bedingung \ref{bed} erhalten wir nun 
\begin{equation}
	a-c=\alpha=\frac{\omega}{\tan(\frac{1}{2}\tau_R \omega)}=\frac{\frac{\pi}{2}}{\tan(\frac{\pi}{4})}=\frac{\pi}{2}\approx 1.6
\end{equation}
Weiter berechnen wir $\beta$ aus \ref{bed1} und erhalten
\begin{equation}
	b=\beta=\frac{\omega}{\sin(\frac{1}{2}\tau_R \omega)}=\frac{\frac{\pi}{2}}{\sin(\frac{\pi}{4})}\approx 2.3
\end{equation}

