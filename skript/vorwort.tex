%
% vorwort.tex -- Vorwort zum Buch zum Seminar
%
% (c) 2017 Prof Dr Andreas Mueller, Hochschule Rapperswil
%
\chapter*{Vorwort}
\lhead{Vorwort}
\rhead{}
Dieses Buch entstand im Rahmen des Mathematischen Seminars
im Frühjahrssemester 2018 an der Hochschule für Technik Rapperswil.
Die Teilnehmer, Studierende der Abteilungen für Elektrotechnik,
Informatik und Bauingenieurwesen der
HSR, erarbeiteten nach einer Einführung in das Themengebiet jeweils
einzelne Aspekte des Gebietes in Form einer Seminararbeit, über
deren Resultate sie auch in einem Vortrag informierten. 

Im Frühjahr 2018 war das Thema des Seminars der Klimawandel.
Im Unterschied zu früheren Seminaren wurde diesem Seminar das Buch
\cite{skript:kaperengler}
zu Grunde gelegt.
Dies führt dazu, dass der erste Teil, der Skript-Teil etwas weniger
detailliert erarbeit wurden.
Das Hauptziel war, den Teilnehmern eine Referenz bereitzustellen,
die die Verbindung zwischen der in den Grundvorlesungen gelernten
Mathematik herzustellen.

Im zweiten Teil dieses Skripts kommen dann die Teilnehmer selbst zu Wort.
Ihre Arbeiten wurden jeweils als einzelne
Kapitel mit meist nur typographischen Änderungen übernommen.
Diese weiterführenden Kapitel sind sehr verschiedenartig.
Eine Übersicht und Einführung findet sich in der Einleitung
zum zweiten Teil auf Seite~\pageref{skript:uebersicht}.

In einigen Arbeiten wurde auch Code zur Demonstration der 
besprochenen Methoden und Resultate geschrieben, soweit
möglich und sinnvoll wurde dieser Code im Github-Repository
dieses Kurses%
\footnote{\url{https://github.com/AndreasFMueller/SeminarKlima.git}}
abgelegt.

Im genannten Repository findet sich auch der Source-Code dieses
Skriptes, es wird hier unter einer Creative Commons Lizenz
zur Verfügung gestellt.
Auf der beiliegenden DVD befinden sich die Testdaten und Programme
zu zwei der simulationsintensiveren Artikel im zweiten Teil.


