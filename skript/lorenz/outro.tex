% !TeX root = documentation.tex
% !TeX spellcheck = de_DE

Wie wir beschrieben haben, spielt das Chaos eine grosse Rolle in der Wetterprognose und ist der Grund, wieso keine langfristige, verl"assliche Prognose gemacht werden kann. Denn selbst kleinste Verf"alschungen der Messdaten f"uhren nach einer gewissen Zeit zu einem komplett anderen Resultat. Es kann sein, dass mit einem minimal anderen Input die Resultatmenge gerade auf die andere Seite des Butterflys ausschlagen k"onnte. Da wir nicht beliebig genau messen k"onnen, stellt uns das vor diese Einschr"ankung der zeitlich limitierten Prognose.

Um R"uckschluss auf das Paper von Lorenz mit dem Fl"ugelschlag des Schmetterlings zu nehmen, m"ussen wir uns "uberlegen, inwiefern ein Schmetterlingsschlag eine Auswirkung haben kann. Gem"ass dem Lorenz-Modell w"are es m"oglich, dass ein solch kleines Event zu so grossen Auswirkungen wie ein Tornado f"uhren kann, da es sich ja genau um eine winzige Parameter"anderung handelt. Diese Gedankenanregung wurde auch nach der Publikation seiner Lorenz-Gleichungen ver"offentlicht. Hingegen ein R"uckschluss zu ziehen, was genau ein Tornado ausgel"ost hat, ist praktisch unm"oglich, selbst wenn alle ben"otigten Daten vorhanden w"aren. Genauso k"onnte ein Schmetterlingsschlag einen allf"alligen Tornado verhindert haben.

Doch kann das Modell auch auf die Realit"at angewendet werden und stimmt nun diese Schlussfolgerung? F"ur das Berechnen des Lorenz-Attraktors wurden viele Vereinfachungen gemacht. So werden viele relevante Eigenschaften zum Wetter wie zum Beispiel die Luftfeuchtigkeit, die Einfl"usse der Wolken oder auch die Albedo nicht miteinbezogen. 

Es ist also nicht ein realistisches Modell. Wir kommen also zum Schluss, dass das Wetter zwar ein chaotisches Modell ist, aber durch das, dass es ein solch grosses und komplexes System ist, ein Schmetterlingsschlag schlichtweg zu irrelevant ist, um ein Tornado auszul"osen. 