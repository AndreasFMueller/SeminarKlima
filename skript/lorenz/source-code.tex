% !TeX root = ../skript.tex
% !TeX spellcheck = de_DE

\section{Anhang: Source Code}

\begin{lstlisting}[style=C, language=Java, caption=App.js]
// 3D-Visualisierung mit whsJS.

/* eslint-disable */

import * as WHS from 'whs';
import * as THREE from 'three';

export class WHSApp {

	constructor() {
		this.app = null;
		this.lorenzPoints = [];
		this.viewPoints = [];
		this.running = false;
	}

	render(element, rho = 28, sigma = 10, beta = 8 / 3) {
		// Prepare stage
		let scene = new WHS.SceneModule();
		
		this.app = new WHS.App([
			new WHS.ElementModule(element),
			scene,
			new WHS.CameraModule({
				position: {
					y: 10,
					z: 50
				}
			}),
			//0xCDCDCD
			new WHS.RenderingModule({
				bgColor:  0x4A555C,
				
				renderer: {
					antialias: true,
					shadowmap: {
						type: THREE.PCFSoftShadowMap
					}
				}
			}, {shadow: true}),
			new WHS.OrbitControlsModule(),
			new WHS.ResizeModule()
		]);
		
		// Calculate Positions
		this.lorenzPoints = WHSApp.calculate(rho, sigma, beta);
		
		// Create Visual Element for each position
		for (let i = 0; i < this.lorenzPoints.length; i++) {
			let sphere = new WHS.Sphere({
				geometry: {
				radius: 0.2,
				widthSegments: 5,
				heightSegments: 5
				},
				
				material: new THREE.MeshPhongMaterial({
					color: new THREE.Color(
						(this.lorenzPoints[i].x + 10) / 20, 
						(this.lorenzPoints[i].y + 10) / 20, 
						this.lorenzPoints[i].z) / 30
				}),
				
				
				position: this.lorenzPoints[i],
			});
			
			this.viewPoints[i] = sphere;
			this.viewPoints[i].addTo(this.app);
		}
		
		// Create Light
		new WHS.AmbientLight({
			light: {
				intensity: 0.4
			}
		}).addTo(this.app);
		
		// Display Coordinates
		
		const axis = new THREE.AxisHelper(30);
		this.app.get('scene').add(axis);
		
		
		// Start the app
		this.app.start();
		this.running = true;
	};
	
	rerender(rho, sigma, beta) {
		this.lorenzPoints = WHSApp.calculate(rho, sigma, beta);
		for (let i = 0; i < this.viewPoints.length; i++) {
			this.viewPoints[i].position = this.lorenzPoints[i];
			this.viewPoints[i].material.color.r = (this.lorenzPoints[i].x+10)/20;
			this.viewPoints[i].material.color.g = (this.lorenzPoints[i].y+10)/20;
			this.viewPoints[i].material.color.b = this.lorenzPoints[i].z/30;
		}
	
	}
	
	destruct() {
		if (this.running) {
			this.app.render = false;
			this.app.stop();
			this.app.disposeModules();
			this.app = null;
		}
	}
	
	static calculate(rho = 28, sigma = 10, beta = 8 / 3) {
		const it = 2500;
		
		let x = 0.1;
		let y = 0.1;
		let z = 0.1;
		const delta = 0.01;
		
		const arr = [];
		arr.push(new THREE.Vector3(x, y, z));
		
		for (let i = 0; i < it; i += 1) {
			/* eslint-disable no-mixed-operators */
			x = pos[i].x + ((sigma * pos[i].y) - (sigma * pos[i].x)) * delta;
			y = pos[i].y + ((-pos[i].x * pos[i].z) + (rho * pos[i].x) - pos[i].y) * delta;
			z = pos[i].z + ((pos[i].x * pos[i].y) - (beta * pos[i].z)) * delta;
			pos.push(new THREE.Vector3(x, y, z));
		}
		
		return arr;
	};
}
\end{lstlisting}

\begin{lstlisting}[style=C, language=Java, caption=Visualize.vue]
// VueJS-Adapter um die Visualsierung in die Gesammtseite einzufuegen
// Der VueJS-Code laesst Visualisierung neu zeichnen bei Parameteraenderungen
<template>
<div class="WHS-Container">
	<form id="input-form">
		<div class="form-group row">
			<label for="rho-input">rho</label>
			<input
			id="rho-input"
			v-model="rho"
			:readonly="disabled"
			:class="{'form-control-plaintext': disabled, 'form-control': !disabled}"
			type="number">
		</div>
		<div class="form-group row">
			<label for="sigma-input">sigma</label>
			<input
			id="sigma-input"
			v-model="sigma"
			:readonly="disabled"
			:class="{'form-control-plaintext': disabled, 'form-control': !disabled}"
			type="number">
		</div>
		<div class="form-group row">
			<label for="beta-input">beta</label>
			<input
			id="beta-input"
			v-model="beta"
			:readonly="disabled"
			:class="{'form-control-plaintext': disabled, 'form-control': !disabled}"
			type="number">
		</div>
	</form>
	<div id="WHS-Playground"/>
	<div id="caption">
		<p>
			<label id="x-axis" style="color:red;">X-Achse:</label>
			<label> Durchschn. Windgeschwindigkeit</label>
		</p>
		<p>
			<label id="y-axis" style="color:green;">Y-Achse:</label>
			<label>Temperatur</label>
		</p>
		<p>
			<label id="z-axis" style="color:blue;">Z-Achse:</label>
			<label>Temperaturgradient</label>
		</p>
	</div>
</div>
</template>

<script>
import * as WHS from '../../whs/App';

const app = new WHS.WHSApp();

export default {
	name: 'Visualize',
	props: {
		rho: {
			type: Number,
			required: true,
		},
		sigma: {
			type: Number,
			required: true,
		},
		beta: {
			type: Number,
			required: true,
		},
		disabled: {
			type: Boolean,
			default: true,
		},
	},

	/* eslint-disable */
	mounted: function() {
		app.render(document.getElementById('WHS-Playground'), this.rho, this.sigma, this.beta);
	},
	
	beforeDestroy() {
		app.destruct();
	},
	
	watch: {
		rho() { app.rerender(this.rho, this.sigma, this.beta); },
		sigma() { app.rerender(this.rho, this.sigma, this.beta); },
		beta() { app.rerender(this.rho, this.sigma, this.beta); },
	}
};
</script>
\end{lstlisting}