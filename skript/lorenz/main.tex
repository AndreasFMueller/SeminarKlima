%
% main.tex -- Paper zum Thema Lorenz-Attraktor
%
% (c) 2018 Matthias Baumann und Oliver Dias, Hochschule Rapperswil
%

\definecolor{lightlightgray}{gray}{0.9}

\lstset{
	language=Java,
	frame=ltrb,
	rulecolor=\color{black},
	basicstyle=\ttfamily,
	keywordstyle=\color{blue},
	commentstyle=\color{gray},
	numbers=left,
	numberstyle=\scriptsize,
	backgroundcolor=\color{lightlightgray},
	tabsize=2,
	literate={\ \ }{{\ }}1
} 

% Document

\chapter{Lorenz-Attraktor\label{chapter:lorenz}}
\lhead{Kapitel \thechapter: Lorenz-Attraktor}
\begin{refsection}
\chapterauthor{Matthias Baumann und Oliver Dias}


\subfile{lorenz/intro}
\subfile{lorenz/lorenz-modell}
\subfile{lorenz/visualisierung}
\subfile{lorenz/hintergruende}
%\subfile{lorenz/chaostheorie}
\subfile{lorenz/outro}

\subfile{lorenz/source-code}

\newpage
\rhead{Literatur}
\printbibliography[heading=subbibliography]

\end{refsection}
