% !TeX root = ../documentation.tex
% !TeX spellcheck = de_DE

\section{Chaostheorie}
Ein typisches Beispiel f"ur ein chaotisches System ist das Wetter, wie Edward Lorenz herausgefunden hat. 

Die Modelle von Lorenz werden chaotisch genannt, weil sie auf den ersten Blick keinen Gesetzmässigkeiten folgen. Dennoch haben diese Systeme bei genauer Untersuchung wiederkehrende Verhaltensweisen. Dies zeigt sich, indem sich wiederholende, zum Teil fraktale Muster und "ahnliche Figuren zustande kommen. Als Beispiel sind bei der fraktalen Menge Mandelbrot die Muster sehr gut wiedererkennbar. 

Im Generellen ist die Chaostheorie ein Bereich der Mathematik der sich mit nicht-linearen dynamischen Systemen auseinander setzt.


% Iterative Berechnungsweise von Dynamischen Systemen

\subsection{Nicht-lineare Systeme} % Viele sind Oszillatoren mit mehreren Variablen und Rückkopplungen
Bei linearen Systemen kann eine Proportionalität zwischen der Veränderung der Eingangsgrössen und den Ausgangsgrössen festgestellt werden. Währendem bei den nicht-linearen Systemen genau diese Eigenschaft nicht vorhanden ist.

Nicht-lineare Systeme besitzen Rückkopplungen von verschiedenen Ausgangsgrössen auf die Eingangsgrössen. Wichtig zum verstehen ist hier die Mehrzahl, um nicht-linear zu sein müssen mehrere Rückkopplungen vorhanden sein. Bei einer Rückkopplung ist es sehr wahrscheinlich, dass das System sich linear verhaltet.

Mathematiker beschreiben solche Systeme gerne mit Gleichungssytemen.

\subsection{Chaostheorie} ist ein Teilbereich der oben erwähnten dynamischen Systeme. Ich werde im Folgenden Systeme und Modelle als Synonym betrachten. Chaotische Systeme besitzen ganz wenige Eingangsparameter.

Die entscheidende Eigenschaft nach welcher diese Systeme kategoriesiert werden ist ihre Empfindlichkeit auf die Eingangsparameter. Eine kleine "Anderung in den Eingangsparamter kann eine grosse "Anderung der Werte auslösen.

Es gibt aber auch andere äussere Einwirkungen, die eine Veränderung der Werte auslösen kann. Zum Beispiel kann der Rundungsfehler, welcher bei modernen CPU‘s nicht-vorhersagbar ist, die Ergebnisse stark verändern. Weil jeder Wert auf den vorherigen Werten aufbaut wird der Rundungsfehler die Werte immer stärker verfälschen.

\subsection{Butterfly Effect}
