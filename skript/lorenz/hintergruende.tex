% !TeX root = documentation.tex
% !TeX spellcheck = de_DE

\section{Hintergründe}
%TODO Einleitung Hintergründe

\subsection{Dynamik} %TODO Mehr an sein Skript anlehnen
Dynamik beschreibt Systeme deren Variablen sich über die Zeit verändern. Zu jedem Zeitpunkt besitzt ein dynamisches System einen Vektor der auf einen Punkt im Ergebnisraum zeigt. Dieser Vektor wird dann als Basis für die Berechnung des nächsten Punktes im Koordinatensystem verwendet.

Vielfach sind dynamische Systeme auch deterministisch \cite{wikidynamicalsystems}. Das bedeutet, dass aus dem jetztigen Ausgangspunkt den nächsten Punkt im System berechnet werden kann.


\subsection{Empfindlichkeit der Anfangsbedingungen}
%TODO Klarer von Chaostheorie separieren
Die Chaostheorie ist ein Teilbereich der oben erwähnten dynamischen Systeme. Im Folgenden werden Systeme und Modelle als Synonym betrachtet. Chaotische Systeme besitzen ganz wenige Eingangsparameter.

Die entscheidende Eigenschaft, nach welcher diese Systeme kategoriesiert werden, ist ihre Empfindlichkeit auf die Eingangsparameter. Eine kleine änderung in den Eingangsparamter kann eine grosse änderung der Werte auslösen.

Es gibt aber auch andere äussere Einwirkungen, die eine Veränderung der Werte auslösen kann. Zum Beispiel kann der Rundungsfehler, welcher bei modernen CPU‘s nicht-vorhersagbar ist, die Ergebnisse stark verändern. Weil jeder Wert auf den vorherigen Werten aufbaut wird der Rundungsfehler die Werte immer stärker verfälschen.

\subsection{Chaostheorie}
Ein typisches Beispiel für ein chaotisches System ist das Wetter, wie Edward Lorenz herausgefunden hat. 

Die Modelle von Lorenz werden chaotisch genannt, weil sie auf den ersten Blick keinen Gesetzmässigkeiten folgen. Dennoch haben diese Systeme bei genauer Untersuchung wiederkehrende Verhaltensweisen. Dies zeigt sich, indem sich wiederholende, zum Teil fraktale Muster und ähnliche Figuren zustande kommen. Als Beispiel sind bei der fraktalen Menge Mandelbrot die Muster sehr gut wiedererkennbar. 

Im Generellen ist die Chaostheorie ein Bereich der Mathematik der sich mit nicht-linearen dynamischen Systemen auseinander setzt.

%TODO Subsection Gleichgewichtspunkte an sein Skript angelehnt inbezugnehmend auf Lorenz-Attraktor

\subsection{Attraktor}

Vorweg ist wichtig zu sagen, dass es in dynamischen Systemen keine standartisierte Definition eines Attraktors gibt. Wir verwenden in diesem Kontext folgende Definition: 
\begin{center}
	Ein Attraktor ist eine Menge an Werten, zu welchen sich ein System zu entwickeln tendiert.
\end{center} Egal welche Startwerte man in einem Attraktor verwendet, das System entwickelt sich immer auf dieselbe Art und Weise. Im Bezug auf den Lorenz Attraktor ist dies die berühmte Form des Butterflys. Mathematisch formuliert ist ein Attraktor 

\begin{align}%TODO Proper formatting?
\label{Attraktor}Attraktor = \left\{ x_a | \forall \varepsilon > 0
\forall T \exists t > T
|x(t) - x_a| < \varepsilon \right\} 
\end{align}

Für jeden Zeitpunkt $T$ existiert ein späterer Zeitpunkt $t$, bei dem sich die Funktion $x$ (Lorenz-Attraktor) so entwickelt, dass der Betrag der Differenz kleiner als die Toleranz $\varepsilon$ wird.

\subsection{Strange Attraktor}
Beim Strange Attraktor wird ein normaler Attraktor um eine chaotische Komponenten erweitert. Das heisst, dass sich Werte innerhalb des Systems chaotisch verhalten. So könnten sich Lösungen in der Definition \eqref{Attraktor} innerhalb von $\varepsilon$ beliebig bewegen. Kleinste Parameter- oder Startpunktänderungen führen zu scheinbar zusammenhangslose Resultate. Dieses Verhalten wird wie in den vorherigen Abschnitten beschrieben Chaos genannt.
