% !TeX root = documentation.tex
% !TeX spellcheck = de_DE

\section{Hintergründe}
%TODO Einleitung Hintergründe

\subsection{Empfindlichkeit der Anfangsbedingungen}
%TODO Klarer von Chaostheorie separieren
Die Chaostheorie ist ein Teilbereich der oben erwähnten dynamischen Systeme. Im Folgenden werden Systeme und Modelle als Synonym betrachtet. Chaotische Systeme besitzen ganz wenige Anfangsbedingungen. % TODO: Warum?

Die entscheidende Eigenschaft, nach welcher diese Systeme kategoriesiert werden, ist ihre Empfindlichkeit auf die Eingangsparameter. Eine kleine änderung in den Eingangsparamter kann eine grosse änderung der Werte auslösen.

Es gibt aber auch andere äussere Einwirkungen, die eine Veränderung der Werte auslösen kann. Zum Beispiel kann der Rundungsfehler eines CPU's die Ergebnisse klein wenig von Punkt-zu-Punkt verändern und so werden sich die Fehler kummulieren. Dies führt dazu, dass die Ergebnisse nicht reproduzierbar sind, da solche Einflüsse zwischen Durchläufen variieren.

\subsection{Chaostheorie}
Ein typisches Beispiel für ein chaotisches System ist das Wetter, wie Edward Lorenz herausgefunden hat. 

Die Modelle von Lorenz werden chaotisch genannt, weil sie auf den ersten Blick keinen Gesetzmässigkeiten folgen. Dennoch haben diese Systeme bei genauer Untersuchung wiederkehrende Verhaltensweisen. Dies zeigt sich, indem sich wiederholende, zum Teil fraktale Muster und ähnliche Figuren zustande kommen. Als Beispiel sind bei der fraktalen Menge Mandelbrot die Muster sehr gut wiedererkennbar. 

Im Generellen ist die Chaostheorie ein Bereich der Mathematik der sich mit nicht-linearen dynamischen Systemen auseinander setzt.

%TODO Subsection Gleichgewichtspunkte an sein Skript angelehnt inbezugnehmend auf Lorenz-Attraktor

\subsection{Attraktor}
In dynamischen Systemen ist ein Attraktor eine Untermenge eines Phasenraums(gewisse Anzahl an Zuständen), zu welchen sich ein dynamisches System im Laufe der Zeit zubewegt und diese Menge das System nicht mehr verlässt. \cite{wikiattraktor}
 Egal welche Startwerte man in einem Attraktor verwendet, das System entwickelt sich immer auf dieselbe Art und Weise. Im Bezug auf den Lorenz Attraktor ist dies die berühmte Form des Butterflys. Mathematisch formuliert ist ein Attraktor 


\begin{align}
\label{Attraktor} \text{Attraktor} &= \{ x_a | \forall \varepsilon > 0 \nonumber\\
&\qquad {} \forall T \exists t > T \nonumber\\
&\qquad {} |x(t) - x_a| < \varepsilon \} 
\end{align}

Für jeden Zeitpunkt $T$ existiert ein späterer Zeitpunkt $t$, bei dem sich die Funktion $x$ (Lorenz-Attraktor) so entwickelt, dass der Betrag der Differenz kleiner als die Toleranz $\varepsilon$ wird. 

Als Beispiel könnte man, neben dem Lorenz-Attraktor, ein Pendel nehmen. Ein Pendel entwickelt sich mit der Zeit immer näher an einen Punkt, bei welchem sich alle darauf wirkenden Kräfte zu 0 addieren. Dieser Punkt ist also der Attraktor für dieses Pendelsystem. Im Unterschied zum Lorenz-Attraktor handelt es sich beim Pendelsystem aber um ein nicht-chaotisches System. Eine kleine Änderung in der Startposition der Masse am Ende des Pendels führt auf eine kleine Änderung in der Position, wobei es bei einem chaotischen System anders ausehen würde. 

\subsection{Strange Attraktor}
Beim Strange Attraktor wird ein normaler Attraktor um eine chaotische Komponenten erweitert. Das heisst, dass sich Werte innerhalb des Systems chaotisch verhalten. So könnten sich Lösungen in der Definition \eqref{Attraktor} innerhalb von $\varepsilon$ beliebig bewegen. Kleinste Parameter- oder Startpunktänderungen führen zu scheinbar zusammenhangslose Resultate. Dieses Verhalten wird wie in den vorherigen Abschnitten beschrieben Chaos genannt.
