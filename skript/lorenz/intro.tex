% !TeX root = documentation.tex
% !TeX spellcheck = de_DE

\section{Einf"uhrung}
Der amerikanische Mathematiker und Meteorologe Edward Lorenz brachte im Jahre 1972, 3 Jahre nachdem er sein ber"uhmtes Paper zum Lorenz-Modell publizierte, ein weiteres Paper mit dem Titel: \textit{"Predictability: Does the Flap of a Butterfly’s Wings in Brazil Set Off a Tornado in Texas?"} heraus. Aussergew"ohnlicherweise stoss dieses Paper auf reges Interesse in der Bev"olkerung. Entgegen vielen anderen wissenschaftlichen Publikationen liess sich aus diesem eine scheinbar einfache Schlussfolgerung ziehen: kleine Ereignisse k"onnen grosse und weitverbreitete Konsequenzen haben. Dieser Effekt bekam den Namen "Butterfly Effect", und es wurden sogar Filme und B"ucher mit dieser Schlussfolgerung als Inspiration geschrieben. Dass das Modell auch "astethisch ansprechend ist und einer abstrakten Schmetterlingsform "ahnelt, verhalf ebenso zur Berühmtheit dieses Modells. 


%TODO Mehr auf das Paper eingehen
Wie wir feststellen werden, ist diese Schlussfolgerung eine Fehlinterpretation und basiert auf einer falschen Verallgemeinerung. Um dies erklären zu können, müssen wir uns mit dem vorhergehenden Paper von Lorenz beschäftigen, mit welchem er die drei berühmten Gleichungen entdeckte und somit den Lorenz Attraktor.
