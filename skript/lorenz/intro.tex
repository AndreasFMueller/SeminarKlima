% !TeX root = ../skript.tex
% !TeX spellcheck = de_DE

\section{Einführung}
Der amerikanische Mathematiker und Meteorologe Edward Lorenz brachte im Jahre 1972, drei Jahre nachdem er sein berühmtes Paper zum Lorenz-Modell publizierte, ein weiteres Paper mit dem Titel: \textit{"Predictability: Does the Flap of a Butterfly’s Wings in Brazil Set Off a Tornado in Texas?"} heraus. Aussergewöhnlicherweise stiess dieses Paper auf reges Interesse im allgemeinem Publikum. Entgegen vielen anderen wissenschaftlichen Publikationen liess sich aus diesem eine scheinbar einfache Schlussfolgerung ziehen: Kleine Ereignisse können grosse und weitreichende Konsequenzen haben. Dieser Effekt bekam den Namen "Butterfly Effect", und es wurden sogar Filme und Bücher mit dieser Schlussfolgerung als Inspiration geschrieben. Dass der Plot des Modells ästhetisch ansprechend ist und einer abstrakten Schmetterlingsform ähnelt, verhalf ebenso zur dessen Berühmtheit. 


Wie wir feststellen werden, ist diese Schlussfolgerung eine Fehlinterpretation und basiert auf einer falschen Verallgemeinerung. Um dies erklären zu können, müssen wir uns mit dem vorhergehenden Paper von Lorenz beschäftigen, mit welchem er die drei berühmten Gleichungen entdeckte und somit den Lorenz-Attraktor. 

Im Folgenden wird dieses Modell also genauer beschrieben sowie eine numerische Lösung vorgestellt (Abschnitt \ref{lorenz-modell}). Anschliessend werden Möglichkeiten dargelegt, wie ein solches Modell in Code umgesetzt werden kann. Als Kern dieses Papers wird dann das vorgestellte Verfahren angewendet und eine eigene Visualisierung präsentiert (Abschnitt \ref{visualisierung}), gefolgt von  Hintergrundinformationen (Abschnitt \ref{backgroundinfo}) und dem Abschluss (Abschnitt \ref{outro}), wo mithilfe von diesen Visualiserungen Schlüsse zum Lorenz-Attraktor gezogen werden. 