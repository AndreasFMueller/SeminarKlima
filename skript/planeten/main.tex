%
% main.tex -- Paper zum Thema <thema>
%
% (c) 2018 Sebastain Lenhard und Nicolas Tobler, Hochschule Rapperswil
%
\chapter{Klima auf anderen Planeten\label{chapter:thema}}
\lhead{Klima auf anderen Planeten}
\begin{refsection}
\chapterauthor{Nicolas Tobler}

\section{Einführung}
\rhead{Abschnitt}

Verschiedene Organisationen, wie unter anderem Elon Musk's SpaceX, haben sich zum Ziel gemacht, in absehbarer Zukunft den Mars für den Menschen bewohnbar zu machen. Insbesondere sollte der Mars eine erdählnliche Atmosphere erhalte, also terraformed werden. Was auf Computer-generierten bildern ziemlich simpel aussieht, wird sich in realität wahrscheinlich ziemlich schwierig herausstellen. In diesem Kapitel wird die aktuelle lage des Klimas auf dem Mars analysiert und mögliche Wege den Mars zu teraformen auf die Machbarkeit untersucht.

\section{Das Klima auf dem Mars}

Verschiedene Mars proben

\subsection{Temperatur}

\subsection{Albedo}


\subsection{Energieerhaltungs Gleichungen}




\subsection{Atmosphereische Eigenschaften}

dünne Atmosphere

Eann und wiso velohr der Mars seine Athmosphere


Rückgang der Atmosphere durch Sonnenwind
	https://www.nasa.gov/press-release/nasa-mission-reveals-speed-of-solar-wind-stripping-martian-atmosphere
	Vor 4.2 Milliarden Jahren gefrohr der Kern


\section{Den Mars terraformen}



\subsection{Aussetzen von Treibhausgasen}

Aussetzen von hocheffektiven treibhausgasen wie FCKW's

verwendung von Methan 


\section{Schlussfolgerung}
\rhead{Schlussfolgerung}

\printbibliography[heading=subbibliography]
\end{refsection}
