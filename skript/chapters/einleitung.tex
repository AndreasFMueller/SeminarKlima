%
% einleitung.tex
%
% (c) 2018 Prof Dr Andreas Müller
%
\chapter*{Einleitung\label{chapter:einleitung}}
\lhead{Einleitung}
\addcontentsline{toc}{chapter}{Einleitung}
Dass sich das Klima verändert ist unbestritten,
Diskussionen entstehen nur über die Ursache.
Das Klima-System umfasst die gesamte Atmosphäre, die Weltmeere und auch in
den Landmassen kann wärme gespeichert werden.
Je nach Bodenbedeckung, sei es durch Vegetation oder Eis, wird mehr
Strahlung absorbiert oder reflektiert.
Diese grosse Zahl von auf das Klima einwirkenden Faktoren macht es
schwierig, einfache Kausalketten zu konstruieren zum Beispiel zwischen
$\text{CO}_2$ und Anstieg der globalen Mitteltemperatur.
Aber ist auch genauso unangemessen, aus einzelnen Datenpunkten 
abzuleiten, dass der Klimawandel gar nicht existiert, wie es einzelne
interessierte Akteure tun.

Die Lösung dieses Dilemmas ist natürlich, das Klimasystem mathematisch
zu modellieren.
Genau dies wird seit der Mitte des letzten Jahrhunderts versucht.
Doch auch dies ist keine einfache Aufgabe.
Zunächst müssen Klimafaktoren, also Aussagen über längerfristige
Mittelwerte, von den kurzfristigen Schwankungen des Wetters getrennt werden.
Die globale Jahres-Mitteltemperatur ist ein einfacher solcher Faktor,
doch mit dieser Mittelung geht der Blick auf die jahreszeitlichen
Schwankungen verloren. 
Es könnte ja sein, die Mitteltemperatur nur langsam ansteigt, aber die
mit den Jahreszeiten verbundene Amplitude der Temperatur über das Jahr
derart gross wird, dass sich Ökosysteme derart verändern, dass zum
Beispiel die Nahrungsversorgung gefährdet wird.
Erfolgreiche Modellierung bedeutet also, dass geeignete Variablen 
gefunden werden, die die wesentlichen Aspekte des Klimas darstellen
können, sie nicht übermässig vereinfachen und immer noch die Fähigkeit
aufrechterhalten, Vorhersagen über die zukünftige Entwicklung des Klimas 
zu machen.

Es ist naheliegend, dass eine detaillierte Beschreibung der Atmosphäre,
der Weltmeere, des Wärmeaustausches mit den Landmassen mit allen bekannten
Gesetzen der Physik im Prinzip korrekte Vorhersagen ermöglichen dürfte.
Es stellt sich aber heraus, dass dies nicht realistisch ist.
Schon in einfachen Strömungsmodellen tritt das Phänomen der sensitiven
Abhängigkeit von Anfangsbedingungen auf.
Selbst winzigste Unterschiede in den Anfangsbedingungen führen zu völlig
verschiedener langfristiger Entwicklung des Modells.
Solche Modelle mögen zum Beispiel für kurzfristige Wetterprognosen
noch geeignet sein, sie sind aber offenbar ungeeignet für Klimaprognosen.
Dieses Phänomen wird an Hand des Lorenz-Modells in
\index{Lorenz-Modell}
Abschnitt~\ref{section:lorenz-modell} illustriert.

Ein weiteres Problem ist, dass eine detaillierte Modellierung nach dem
Vorbild der Fluiddynamik zu derart komplexen Gleichungen führt, dass 
sie für lange Zeiträume nicht mehr effizient gelöst werden können.
Zwar steigt mit der Weiterenticklung der Computer-Technologie die 
zur Verfügung stehende Rechenleistung und damit der technisch erreichbare
Prognosehorizont, er bleibt aber immer noch weit entfernt von den
angestrebten Zeiträumen von mindestens 100 Jahren.
Die Modelle müssen daher soweit vereinfacht werden und durch relativ
niedrigdimensionale Modelle ersetzt werden, die robuste Vorhersagen
über wenige Jahrhunderte ermöglichen.
Die detaillierte Kenntnis der physikalischen Prozesse kann solche 
Variablen nahelegen, daher werden in Kapitel~\ref{chapter:wetter und klima}
die wichtigsten physikalischen Grundlagen zusammengestellt.
Die in Kapitel~\ref{chapter:thc} beschriebene thermohaline Zirkulation
oder die Zonenmodelle von Kapitel~\ref{chapter:zonenmodelle}
kommt ebenfalls mit einer sehr kleinen Zahl von physikalisch motivierten
Parametern aus.

Manchmal können jedoch auch mathematische Methoden dazu verwendet werden,
geeignete Variablen zu finden.
Zum Beispiel kann dies mit Hilfe der Fourier-Analyse geschehen, die
im Kapitel~\ref{chapter:fourier} entwickelt wird.
Die Fourier-Koeffizienten $a_0$ und $a_1,b_1$ trennen zum Beispiel die
langfristigen Mittelwerte und die jährlichen Schwankungen, sie geben
also mathematisch wieder, was man auch aus einer physikalischen
Argumentation hätte ableiten können.
In Abschnitt~\ref{section:pdeloesungen} wird das Separationsverfahren 
für partielle Differentialgleichungen dargestellt.
Es liefert eine Basis von Funktionen, mit denen die unendlichdimensionalen
partiellen Differentialgleichungen in endlichdimensionale gewöhnliche
Differentialgleichungen umgewandelt werden können.
Das Lorenz-Modell von Abschnitt~\ref{section:lorenz-modell} ist ein
Beispiel, wie die Dimensionszahl reduziert werden kann.
Die spektrale Methoden, die in Abschnitt \ref{section:spektrale methoden}
beschrieben werden, sind nicht nur in der Wetterprognose üblich, sondern
sind auch für Klimamodelle nützlich, zum Beispiel auch, weil sie auf
ganz natürliche Weise die für Klimauntersuchungen häufig irrelevante
Erdrotation von der Breitenabhängigkeit zu trennen gestatten.

Das Phänomen des El Niño illustriert, dass Koppelungen wesentlicher
\index{El Niño}
Klimaparameter sowohl über grosse Distanzen wie auch lange Zeiträume
existieren.
Die Modellierung dieser Koppelungen kann zum Beispiel auf der Basis
der Strömungsdynamik von Kapitel~\ref{chapter:fluiddynamik} erfolgen.
Damit lässt sich dann ein realistischeres Modell des El Niño-Phänomens
in Form einer verzögerten Differentialgleichung konstruieren.
Kapitel~\ref{chapter:elnino} zeigt den Weg dazu.
Die numerischen Probleme in diesem Modell werden in
Kapitel~\ref{chapter:verzoegert} untersucht.

Woher weiss man, ob ein Modell die Zukunft korrekt vorhersagen wird?
Da je nach Prognose möglicherweise dringender Handlungsbedarf für die
Menschheit besteht, kann man nicht warten, bis klar wird, dass die
Prognose direkt überprüft werden kann.
Vertrauen in das Modell muss daher dadurch hergestellt werden, dass
man die Vorhersagen des Modells mit der bisherigen Klimageschichte
der Erde vergleicht.
Dies wird zum Beispiel ansatzweise in Kapitel~\ref{chapter:planeten}
durchgeführt.

Häufig stellt sich das Problem, dass gar nicht alle Parameter eines
Modelles überhaupt bekannt sind.
Noch schwieriger ist es, den Anfangszustand zu ermitteln.
Woher kennen wir zum Beispiel den Atmosphärenzustand zu Beginn des
Kambriums, der Phase der Erdgeschichte, in der zum ersten mal komplexe
\index{Kambrium}
Fossilien auftauchen.
Diese Informationen sind nicht alle der direkten Bestimmung
zugänglich.
Man kann aber versuchen, diejenigen Parameter zu finden, die zusammen
mit den messbaren Grössen eine Vorhersage des Modells ergeben, die
gut mit der tatsächlich überlieferten Klimageschichte der Erde 
übereinstimmen.
Dieser Prozess der Datenassimilation wird in
Kapitel~\ref{chapter:assimilation} abstrakt beschrieben und in
Kapitel~\ref{chapter:kalman} an einem Modellbeispiel durchgeführt.

Die Darstellungen im ersten Teil des Buches erheben nicht den Anspruch
einer vollständigen Darstellung.
Gegebenüber dem Buch \cite{skript:kaperengler}, welches dieses Seminar
begleitet hat, wird nur versucht, die Schwergewichte etwas klarer auf
die in dieser Einleitung genannten Punkte zu leiten.
Diese Prinzipien, mit denen man zweckmässige Modelle vernünftiger
Dimensionalität konstruieren kann, die stabile langfristige
Aussagen ermöglichen, sind nicht nur in der Klimamodellierung
nützlich, sondern können in jeder Modellierungsaufgabe in den Ingenieur-
oder Naturwissenschaften angewendet werden.








