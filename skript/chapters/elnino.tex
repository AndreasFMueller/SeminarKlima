%
% elnino.tex
%
% (c) 2018 Prof Dr Andreas Müller, Hochschule Rapperswil
%
\chapter{El Niño Southern Oscillation}
\lhead{El Niño Southern Oscillation}
Die Klimaentwicklung hängt wesentlich davon ab, wie Energie an der
Erdoberfläche verteilt wird.
Aus diesem Grund haben wir in Kapitel~\ref{chapter:fluiddynamik}
die Strömungsdynamik als den wesentlichen Mechanismus des 
Energietransportes in der Atmoshpäre studiert.
Und in Kapitel~\ref{chapter:thc} haben wir mit der Modellierung der
thermohalinen Zirkulation eine alternative Möglichkeit kennengelernt,
den Energie-Transport in den Weltmeeren zu beschreiben.

Das El Niño-Phänomen im Pazifik ist ein interessantes Teilsystem des
Klimasystems, welches einigermassen gut als isoliertes Teilsystem 
behandelt werden kann.
Die Modellierung, die wir in diesem Kapitel anstreben, braucht
einerseits die Ideen der Fluiddynamik, um die Energietransportmechanismen
zu beschreiben, und andererseits die Idee der Box-Modelle, um aus diesen
Mechanismen eine einfache gewöhnliche Differentiagleichung abzuleiten,
mit deren Hilfe die Dynamik des El~Niño studiert werden kann.

%
% kelvinrossby.tex
%
% (c) 2018 Prof Dr Andreas Müller, Hochschule Rapperswil
%
\section{Rossby- und Kelvin-Wellen\label{section:elnino:kelvinrossby}}
\rhead{Rossby- und Kelvin-Wellen}
In diesem Abschnitt soll die Ausbreitung von Anomalien in der Höhe
der Meeresoberfläche in der Höhe des Äquators studiert werden.

\subsection{Kelvin-Wellen\label{subsection:kelvin}}
Ein der einfachsten Form führt ein Tiefdruckgebiet über dem zentralen
Pazifik dazu, dass die Meeresoberfläche über die Normalhöhe ansteigt.
Wenn sich das Tiefdruckgebiet auffüllt und die Druckkraft zur Aufrechterhaltung
dieser Anomalie wegfällt, wird dieser ``Wasserberg'' zerfallen.
Mit Hilfe der Gleichungen der Strömungsdynamik sollte sich die Ausbreitung
dieser Wellen beschreiben lassen.

Da die Anfangsbedingungen symmetrisch bezüglich einer Spiegelung an
der Äquatorebenen sind, dürfen wir annehmen, dass auch die resultierende
Strömung diese Symmetrie hat.
In unmittelbarer Nähe des Äquators brauchen wir die Krümmung der
Erdoberfläche nicht zu berücksichtigen, und können daher mit einem
$x$-$y$-Koordinatensystem arbeiten, in dem $x$ die Richtung entlang des
Äquators und $y$ die Richtung entlang der Längenkreise ist.
Die Strömungsgeschwindigkeitskomponenten nennen wir $u$ entlang des
Äquators und $v$ entlang der Längenkreise.
Die Anomalie der Höhe der Meeresoberfläche bezeichnen wir mit $h(x,y)$.

\subsection{Bewegungsgleichung für Kelvin-Wellen}
Die zeitliche Änderung der Geschwindigkeit, also die Beschleunigung,
ist nach dem zweiten Newtonschen Gesetz proportional zu den Kräften.
Die Schwerkraft versucht, den Wasserberg abzubauen.
Wasser wird in die Richtung beschleunigt, in die die Höhe $h(x,y)$
abnimmt.
Ausserdem wirkt die Coriolis-Kraft, die die Strömung auf der Nordhalbkugel
nach rechts ablenkt, und auf der Südhalbkugel nach links.
\begin{equation}
\begin{aligned}
\frac{\partial u}{\partial t}
&=
\phantom{-}
fv - g\frac{\partial h}{\partial x}
\\
\frac{\partial v}{\partial t}
&=
-fu - g\frac{\partial h}{\partial y}
\end{aligned}
\label{elnino:kelvin:newton}
\end{equation}
Dies ist ein System von zwei partiellen Differentialgleichung für 
drei unbekannte Funktionen $h$, $u$ und $v$, es ist also mindestens
noch eine weitere Gleichung nötig, damit das Problem überhaupt gelöst
werden kann.

Abfliessendes Wasser reduziert die Höhenanomalie.
Die Kontinuitätsgleichung
besagt, dass die Abnahme der der Höhenanomalie proportional ist zur
Divergenz des Geschwindigkeitsfeldes ist.
Die fehlende Differentialgleichung ist daher
\begin{equation}
\frac{\partial h}{\partial t}
=
-H\biggl(
\frac{\partial u}{\partial x} + \frac{\partial v}{\partial y}
\biggr).
\label{elnino:kelvin:kont}
\end{equation}
Gesucht ist jetzt also eine Lösung der Differentialgleichungen
\eqref{elnino:kelvin:newton} und \eqref{elnino:kelvin:kont}.

\subsection{Wellengleichung}
Wir interessieren uns nur für eine Lösung in unmittelbarer Nähe des
Äquators und dürfen daher annehmen, dass sich das Wasser nicht
entlang der Längenkreise bewegt, dass also $v=0$ gilt.
Die Differentialgleichungen
\eqref{elnino:kelvin:newton} und \eqref{elnino:kelvin:kont}.
vereinfachen sich damit zu
\begin{align}
\frac{\partial u}{\partial t}
&=
\phantom{-fu}
 - g\frac{\partial h}{\partial x}
\label{kelvin:naeherung:1}
\\
0
&=
-fu - g\frac{\partial h}{\partial y}
\label{kelvin:naeherung:2}
\\
\frac{\partial h}{\partial t}
&=
-H
\frac{\partial u}{\partial x}
\label{kelvin:naeherung:3}
\end{align}
Indem wir \eqref{kelvin:naeherung:1} nach $x$ und 
\eqref{kelvin:naeherung:3} nach $t$ ableiten, erhalten wir
\begin{equation}
\left.
\begin{aligned}
\frac{\partial^2 u}{\partial x\,\partial t}
&=
-g\frac{\partial^2h}{\partial x^2}
\\
\frac{\partial^2 h}{\partial t^2}
&=
-H\frac{\partial^2 u}{\partial t\,\partial x}
\end{aligned}
\;
\right\}
\quad
\Rightarrow
\quad
\frac{\partial^2 h}{\partial t^2}
=
gH\frac{\partial^2 h}{\partial x^2}
\label{kelvin:wellengleichung}
\end{equation}
Dies ist eine Wellengleichung für eine Welle mit Ausbreitungsgeschwindigkeit
$c=\sqrt{gH}$.

\subsection{Approxmative Lösung der Wellengleichung}
Bis jetzt wurde die zweite Gleichung~\eqref{kelvin:naeherung:2}
nicht verwendet, es wurde eigentlich nur das Verhalten der Welle auf
dem Äquator modelliert.
Da wir jetzt aber wissen, dass mindestens entlang des Äquators die Lösung
eine Welle mit Ausbreitungsgeschwindigkeit $c=\sqrt{gH}$ ist, können
wir versuchen, auch die $y$-Abhängigkeit zu modellieren.

\subsubsection{Dispersionsrelation}
Eine in $x$-Richtung laufende Welle mit Wellenzahl $k$ kann beschrieben werden
als
$
\sin(kx-\omega t).
$
Die Wellenzahl $k$ ist positiv für eine nach Osten laufende Welle und negativ
für eine nach Westen laufende Welle.
Wir suchen also eine Lösung des Gleichungssystems
\eqref{kelvin:naeherung:1}--\eqref{kelvin:naeherung:3}
in der Form
\[
h_k(t,x,y) = \gamma(y)\cdot \sin(kx-\omega t)
\]
zu finden.
Die Funktion $\gamma(y)$ beschreibt das Profil des ``Wasserberges'' 
in der Nähe des Äquators, wir nehmen daher an, dass $\gamma(y)$ für grosse
Werte von $y$ rasch abnimmt.

Einsetzen des Lösungsansatzes $h_k(t,x,y)$ in die
Gleichung~\eqref{kelvin:wellengleichung} liefert
\begin{equation}
\left.
\begin{aligned}
\frac{\partial^2 h_k}{\partial t^2}
&=
- \omega^2 \gamma(y) \sin(kx-\omega t)
=
-\omega^2 h_k(t,x,y)
\\
\frac{\partial^2 h_k}{\partial x^2}
&=
-
k^2
\gamma(y)
\sin (kx-\omega t)
=
-k^2 h_k(t,x,y)
\end{aligned}
\;\right\}
\qquad
\Rightarrow
\qquad
\omega^2=gHk^2
\quad\text{oder}\quad
\biggl|
\frac{\omega}{k}\biggr|
=\sqrt{gH}=c
\end{equation}
Aus dieser Dispersionsrelation
liest man ab, dass die Phasengeschwindigkeit einer solchen
Welle unabhängig ist von der Frequenz.

\subsubsection{$y$-Abhängigkeit}
Bis jetzt haben wir die Gleichung~\eqref{kelvin:naeherung:2} nicht
verwendet.
Sie erlaubt, $u$ zu berechnen, es gilt
\[
u=-\frac{g}{f}\,\frac{\partial h}{\partial y}
\qquad
\text{oder für $h_k$}
\qquad
u_k=-\frac{g}{f} \gamma'(y) \sin(kx-\omega t).
\]
Setzt man dies in \eqref{kelvin:naeherung:3} ein, erhält man
\begin{equation}
\left.
\begin{aligned}
\frac{\partial h_k}{\partial t}
&=
-\omega
\gamma(y) \cos(kx-\omega t)
\\
\frac{\partial u_k}{\partial x}
&=
k \frac{g}{f}\gamma'(y) \cos(kx-\omega t)
\end{aligned}
\;\right\}
\qquad\Rightarrow\qquad
-\omega
\gamma(y) \cos(kx-\omega t)
=
-H
k \frac{g}{f}\gamma'(y) \cos(kx-\omega t)
\end{equation}
nach Kürzen gemeinsamer Faktoren und Umstellen folgt
\begin{equation}
\gamma'(y)
=
\gamma(y) \frac{f}{gH}\frac{\omega}{k}
=
\pm
\frac{f}{c} \gamma(y).
\label{kelvin:gamma:dgl1}
\end{equation}
Das Vorzeichen in \eqref{kelvin:gamma:dgl1} hängt vom Vorzeichen der
Wellenzahl $k$ ab, das obere Vorzeichen steht für eine nach Osten
laufende Welle.

Die Coriolis-Kraft $f$ verschwindet am Äquator, in erster Näherung
ist sie proportional zur $y$, wir schreiben daher $f=\beta y$.
Die Differentialgleichung~\eqref{kelvin:gamma:dgl1} wird damit zu
\begin{equation}
\gamma'(y)
=
\pm
\frac{\beta}{c} 
y\gamma(y).
\label{kelvin:gamma:dgl2}
\end{equation}

Für zunehmende $y$ muss $\gamma$ abnehmen, es muss also $\gamma'(y)<0$ sein
für genügend grosse $y$.
Dies ist aber nur möglich für das negative Vorzeichen, und damit nur
für eine nach Osten laufende Welle.
Im folgenden konzentrieren wir uns daher auf das negative Zeichen
in \eqref{kelvin:gamma:dgl2}.

Um eine Lösung von \eqref{kelvin:gamma:dgl2} zu finden, teilen wir
durch $\gamma(y)$
und verwenden, dass $\gamma'(y)/\gamma(y)$ die Ableitung des
Logarithmus ist:
\begin{equation}
\frac{\gamma'(y)}{\gamma(y)}
=
\frac{d}{dy}\log \gamma(y) = -\frac{\beta}{c} y
\qquad\Rightarrow\qquad
\log\gamma(y) = -\frac{\beta}{2c}y^2
\qquad\Rightarrow\qquad
\gamma(y) = \exp\biggl(
- \frac{\beta}{2c}y^2
\biggr)
\end{equation}
Das $y$-Profil der Welle ist also eine Gauss-Funktion.
Die Zone, in der sich eine Kelvin-Welle ausbreiten kann, 
wird breiter, wenn $\beta$ grösser wird, wenn also die 
Rotationsgeschwindigkeit des Planeten grösser wird.
Sie wird kleiner, wenn $c=\sqrt{gH}$ grösser wird, also
bei grösserer Gravitation.




\section{El Niño}
\rhead{El Niño}

\section{Oszillator-Modell}
\rhead{Oszillator-Modell}

