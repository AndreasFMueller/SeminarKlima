%
% verzoegert.tex
%
% (c) 2018 Prof Dr Andreas Müller, Hochschule Rapperswil
%
\section{Verzögertes Oszillator-Modell\label{section:dde-nino}}
\rhead{Verzögertes Oszillator-Modell}
In den Kelvin- und Rossby-Wellen haben wir Mechanismen kennengelernt,
mit dem Energietransport entlang des Äquators innerhalb der vorherrschenden
Strömung sowohl in östlicher wie auch in westlicher Richtung möglich ist.
Entscheidend für die Dynamik dieses Energietransports ist die Zeit,
die eine Welle benötigt, um den Pazifik zu durchqueren.
Sei $\tau_K$ die Zeit, die eine Kelvin-Welle braucht, um den Pazifik
zu durchqueren, und $\tau_R$ die entsprechende Zeit für eine Rossby-Welle.

Wir versuchen jetzt ein für die Temperaturanomalie $T(t)$ im östlichen
Pazifik aufzustellen, die diese Mechanismen berücksichtigt.
Ohne einen Transportmechanismus würde die Temperaturanomalie einfach
zerfallen, was beschrieben werden kann mit einer Differentialgleichung
der Form
\[
\frac{d}{dt}T(t)
=
-cT(t),
\]
mit einer Konstanten $c$.

Die Diskussionen über die dem El~Niño-Phänomen zugrunde liegenden
Mechanismen haben wir gesehen, dass die Thermoklinenanomalie im
zentralen Pazifik eine Auswirkung auf Temperaturanomalie $T(t)$ haben.
Eine Thermoklinenanomalie $h_0(t)$ auf dem Äquator wandert als
Kelvin-Welle in der Zeit $\frac12\tau_K$ in den Ostpazifik.
Eine Thermoklinenanomalie $h_1(t)$ nördlich oder südlich des Äquators
kann als Rossby-Welle nach Westen wandern, am Westrand des Pazifik
reflektiert werden und als Kelvin-Welle in den Ostpazifik zurückkehren.
Dazu ist die Zeit $\frac12\tau_R+\tau_K$ nötig.
Dies lässt sich mit der Differentialgleichung
\begin{equation}
\frac{d}{dt}T(t)
=
-cT(t) + a_0h_0(t-\frac12\tau_K) + b_0h_1(t-(\frac12\tau_R+\tau_K))
\label{elninodde:1}
\end{equation}
erreichen.

Nun scheint es aber auch einen Zusammenhang zwischen der Temperaturanomalie
im Ostpazifik und den Thermoklinenanomalien im zentralen Pazifik zu geben.
Im einfachsten Fall kännen wir diesen Zusammenhang linear modellieren,
also 
\[
h_0(t-\frac12\tau_K)
\sim
T(t-\frac12\tau_K),
\qquad
\text{und}
\qquad
h_1(t-(\frac12\tau_R+\tau_K))
\sim
T(t-(\frac12\tau_R+\tau_K)).
\]
Eingesetzt in die Differentialgleichung~\eqref{elninodde:1} wird dabei zu
\begin{equation}
\frac{d}{dt}T(t)
=
-cT(t) + aT(t-\frac12\tau_K) + bT(t-(\frac12\tau_R+\tau_K))
\label{elninodde:2}
\end{equation}
Die Ableitung $\dot T(t)$ hängt also nicht nur von der Temperaturanomalie
zur Zeit $t$ ab, sondern auch von den Temperaturanomalien zu den
früheren Zeitpunkten $t-\frac12\tau_K$ und $t-(\frac12\tau_R-\tau_K)$.
Man nennt dies eine verzögerte Differentialgleichung.

Die Theorie der linearen, verzögerten Differentialgleichungen wie 
\eqref{elninodde:2} zeigt, dass sie ähnlich wie gewöhnliche linear
Differentialgleichungen erster Ordnung Lösungen haben, die exponentiell
schnell zerfallen oder exponentiell anwachsen.
Daher können wir nicht erwarten, dass \eqref{elninodde:2} das El~Niño-Phänomen
adäquat beschreiben kann.
Um das exponentielle Anwachsen zu verhindern, braucht es einen zusätzlichen
Term, der dem Anwachsen von $T(t)$ überproportional entgegenwirkt.
Zum Beispiel könnte dies man mit einem Term der Form $-\varepsilon T(t)^3$
erreichen.
Die verzögerte Differentialgleichung ist dann
\begin{equation}
\frac{d}{dt}T(t)
=
-cT(t) + aT(t-\frac12\tau_K) + bT(t-(\frac12\tau_R+\tau_K))
-\varepsilon T(t)^3.
\label{elninodde:3}
\end{equation}
Ein andere Möglichkeit, ein ähnliches Ziel zu erreichen, wird in Kapitel
\ref{chapter:planeten} dargestellt.

In Kapitel~\ref{chapter:verzoegert} wird das Verhalten der Lösungen
solcher verzögerter Differentialgleichungen etwas mehr im Detail
diskutiert und es wird gezeigt, dass sie tatsächlich wesentliche
Aspekte des El~Niño-Phänomens adäquat wiedergeben können.







