%
% elnino.tex
%
% (c) 2018 Prof Dr Andreas Müller, Hochschule Rapperswil
%
\section{El Niño\label{section:elnino}}
\rhead{El Niño}
El Niño oder genauer die El Niño-Southern Oscillation (ENSO) ist ein
quasiperiodisches Klima\-phänomen mit einer Periode von drei bis sieben
Jahren, welches im Pazifik in Äquatornähe beobachtet werden kann.

\subsection{Ablauf eines El Niño-Ereignisses\label{subsection:elnino:ablauf}}
Unter normalen Bedingungen treiben die vorherrschenden Winde Wasser vom
Ostpazifik vor der Küste von Südamerika in Richtung des zentralen Pazifik.
Dadurch senkt sich die Thermokline
(siehe Abschnitt~\ref{section:thc:schichtung})
im Westpazifik ab, während im 
Ostpazifik kaltes Tiefenwasser aufsteigt.
Da dieses Wasser nährstoffreich ist, äussert sich dieser Normalzustand
in grossem Fischreichtum, daher war dieses Phänomen seit Jahrhunderten
bei Fischern in Peru bekannt.

Während eines El Niño-Ereignisses bricht der Antrieb dieses Ungleichgewichts
zusammen.
In der Folge sinkt die Thermokline im Ostpazifik ab,
das kalte nährstoffreiche
Wasser bleibt aus.
Im Westpazifik steigt die Thermokline dagegen an.
Der Wegfall der atmosphärischen Strömung bedeutet auch, dass in der Nähe
von Südamerika vermehrt Niederschläge auftreten, was zu Überschwemmungen
und Missernten führen kann.
Die stark veränderten Strömungsverhältnisse können aber auch weiter entfernte
Auswirkungen haben.
Man kann zeigen, dass sich El Niño-Ereignisse auf der ganzen Welt auswirken
können und es wird vermutet, dass ein besonders starkes Ereignis zwischen
1789 und 1793 an Missernten in Europa mitschuldig war und dazu beigetragen
hat, die französische Revolution auszulösen.

\subsection{Modellierung mit einer Differentialgleichung%
\label{subsection:elnino:modellierung}}
Wie in Abschnitt~\ref{subsection:elnino:ablauf} dargestellt, bedingen sich
bei der ENSO die Thermoklinen-Anomalien im Ost- und Westpazifik und die
entsprechenden Temperatur-Anomalien gegenseitig.
Wir verwenden daher die folgenden Variablen:
\[
\begin{aligned}
T_E&&&\text{Temperaturanomalie im Ostpazifik}\\
T&&&\text{Temperaturanomalie im Westpazifik}\\
h_E&&&\text{Termoklinentiefenanomalie im Ostpazifik}\\
h_W&&&\text{Termoklinentiefenanomalie im Westpazifik}\\
S&&&\text{Windkräfte im westlichen und zentralen Pazifik}\\
S_E&&&\text{Windkräfte im Ostpazifik}\\
\end{aligned}
\]
Wir müssen jetzt Gleichungen herleiten, die diese Grössen und ihre
Änderungsraten, also ihre zeitlichen Ableitungen, miteinander verbinden.

Die Windkräfte $S$ führen im Normalfall dazu, dass die Thermoklinentiefe
im Ostpazifik kleiner wird, während sie im Westpazifik ansteigt.
Fällt $S$ weg, gleichen sich die Thermoklinentiefen an.
Man darf daher eine Gleichgewichtsbedingung der Form
\begin{equation}
h_E = h_W + S
\label{skript:elnino:gleichgewicht}
\end{equation}
annehmen.
Durch die Gleichung~\ref{skript:elnino:gleichgewicht} legt auch
das Vorzeichen sowie die Masseinheit von $S$ fest.
Im Normalzustand ist $S$ negativ.

Die Thermoklinentiefenanomalie $h_W$ im Westpazifik würde ohne die Wirkung
von $S$ langsam verschwinden, was mit einer Differentialgleichung
der Form
\[
\frac{dh_W}{dt} = -rh_W
\]
beschrieben werden kann.
Die Windkraft $S$ wirkt dem entgegen.
Je grösser der Betrag von $S$ ist, desto schneller sinkt $h_W$ ab.
Dies können wir mit der Differentialgleichung
\begin{equation}
\frac{dh_W}{dt}
=
-rh_W -\alpha S
\end{equation}
modelliert werden kann.

Auf der anderen Seite des Pazifik vor der Küste Südamerikas
betrachten wir die Temperaturanomalie
$T_E$.
Ohne irgendwelche Einflüsse verschwindet die Temperaturanomlie mit der
Zeit, also
\[
\frac{dT_E}{dt} = -cT_E.
\]
Je grösser die Thermoklinentiefenanomalie $h_E$ im Ostpazifik ist, desto weiter
weg ist das kalte Wasser von der Oberfläche, die Temperaturanomalie an
der Oberfläche nimmt also langsamer ab, was die Differentialgleichung
\begin{equation}
\frac{dT_E}{dt} = -cT_E + \gamma h_E
\end{equation}
ergibt.

Die bisher gefundenen Gleichungen reichen nicht aus, die vielen Variablen
zu bestimmen.
Wir brauchen daher zusätzliche Bedingungen, die die Variablen koppeln.
Es kann davon ausgegangen werden, dass eine höhere Temperaturanomalie
im Ostpazifik zu erhöhter Windströmung führt, dass es also eine
Proportionalität 
\begin{equation}
S=bT_E
\label{skript:elnino:sbte}
\end{equation}
gibt.

Mit \eqref{skript:elnino:gleichgewicht} können wir $h_E$ eliminieren,
mit \eqref{skript:elnino:sbte} die Grösse $S$.
Dies führt auf $h_E=h_W+S=h_W+bT_E$.
Das verbleibende Gleichungssystem ist
\begin{align*}
\frac{dT_E}{dt}
&=
-cT_E + \gamma(h_W+bT_E)
=
(-c+\gamma b)T_E
+
\gamma h_W
\\
\frac{dh_W}{dt}
&=
-rh_W - \alpha b T_E.
\end{align*}
Mit einigermassen plausiblen Werten der Parameter ergibt sich eine
Oszillation mit einer Periode von etwa $3.4$ Jahren, was ungefähr der
für die ENSO bekannten Periodizität entspricht.
Die Analyse ist in \cite{skript:kaperengler} im Detail durchgeführt.


\subsection{Mängel des Modells\label{subsection:maengel des modells}}
Die Modellierung des El Niño illustriert ein weiteres Problem, nämlich
die Tatsache, dass Grössen miteinander über grosse Distanzen gekoppelt
sind.
Eine Änderung der Meereshöhe im Westpazifik kann nicht ohne Zeitverzögerung
mit einer Thermoklinen-Anomalie im Ostpazifik gekoppelt sein.
Ein einfaches Modell wie bei der relativ langsam ablaufenden thermohalinen
Zirkulation kann daher kaum erfolgreich sein.
Vielmehr muss in diesem Modell die Zeit berücksichtigt werden, die ein
Signal benötigt, um sich durch den Pazifik zu bewegen.

Solche Mechanismen sind bekannt und können mit relativ einfachen
Mitteln modelliert werden.
Es sind die Kelvin-Wellen, die weiter unten in
Abschnitt~\ref{section:elnino:kelvin}
behandelt werden und die Rossby-Wellen, beschrieben in
Abschnitt~\ref{section:elnino:rossby}.
Daraus lässt sich mit Hilfe einer verzögerten Differentialgleichung
ein Modell bauen (Abschnitt~\ref{section:dde-nino}), welches in
Kapitel~\ref{chapter:verzoegert} vertieft studiert wird.


