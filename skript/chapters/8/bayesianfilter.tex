%
% bayesianfilter.tex
%
% (c) 2018 Prof Dr Andreas Müller, Hochschule Rapperswil
%

\newtheorem*{lemma*}{Lemma}

\section{Kalman-Filter als Bayesscher Filter}
In Satz~\ref{satz:filter und vorhersage} wurde gezeigt, wie das allgemeine
Filterproblem gelöst werden kann.
Ohne zusätzliche Voraussetzungen über den Prozess $X(k)$ sind die Formeln
des Satzes jedoch kaum brauchbar.
Einerseits sind die Übergangswahrscheinlichkeiten im Allgemeinen kaum
bekannt, andererseits sind die vielen Integrale nur in einfachen
Fällen ausführbar.

Im Kalman-Filter ist jedoch sehr viel mehr über den stochastischen 
Prozess bekannt.
Bis auf den Systemfehler sind die Übergänge durch lineare Abbildungen,
die Matrizen $\varphi_k$, gegeben und der Messprozess, beschrieben durch
die Matrize $H_k$, ist ebenfalls linear.
Die Verteilungen sind jeweils Normalverteilungen, daher sollten
alle Integrale leicht zu berechnen sein und es sollten sich als Lösung
des Filterproblems der Formelsatz des Kalman-Filters und insbesondere
die Formel~\eqref{kalman:kmatrix} für die Matrix $K_k$ ergeben.

Für die Zwecke dieses Seminars könnten wir es damit bewenden lassen.
Das im Seminar verwendete Buch \cite{skript:kaperengler} beweist die
Kalman-Filter-Formeln aus den Formeln von Satz~\ref{satz:filter und vorhersage}.
Das zentrale Lemma in dieser Rechnung ist Lemma~20.1, leider ist
der Beweis sehr fehlerhaft.
Ausserdem ist auch die Übungsaufgabe 4 in den Exercises, die im Beweis in
Abschnitt 20.3.1 benötigt wird, ebenfalls falsch.
Das Ziel dieses Abschnitts ist, diese Fehler zu korrigieren und damit
den Seminarteilnehmern das Verständnis des Kapitels~20 von
\cite{skript:kaperengler} zu erleichtern.

In Abschnitt~\ref{subsection:bayesian:normalverteilt} wird gezeigt,
wie die bedingten Wahrscheinlichkeitsdichten für normalverteilte
Zufallsvariablen dargestellt werden können.
Der Beweis des Lemmas~20.1 und damit der Nachweis, dass der in
Abschnitt~\ref{section:kalman-filter} dargestellte Kalmanfilter nichts
anderes als ein Spezialfall des in
Abschnitt~\ref{section:assimilationsproblem} beschriebenen Bayesschen
Filterprozesses ist, wird in Abschnitt~\ref{subsection:lemma 20.1}
geführt.
Die fehlerhafte Aufgabe 4 in Kapitel 20 von \cite{skript:kaperengler}
wird in Abschnitt~\ref{subsection:aufgabe4} korrigiert.

\subsection{Normalverteilte Variablen\label{subsection:bayesian:normalverteilt}}
Wir gehen aus von einem Zustandsvektor $X$ und einer Messmatrix $H$,
die die Messwerte $Y=HX + w$ bis auf einen Messfehler $w$ mit
Kovarianzmatrix $R$ aus $X$ berechnen kann.
In \cite{skript:kaperengler} wird darauf hingewiesen, dass die Berechnung
von $f_{X|y}$ mit Hilfe des Satzes von Bayes 
unpraktikabel ist ausser in dem Spezialfall, wo $X$ und $Y|X$ 
normalverteilt sind.
Dies bedeutet, dass die Wahrscheinlichkeitsdichte von $X$ geschrieben
werden kann als
\begin{equation}
f_X(x) = \exp\biggl(
-\frac12 (x-\mu)^t P^{-1} (x-\mu)
\biggr).
\label{xverteilung}
\end{equation}
Darin ist $\mu=E(X)$ der Vektor der Erwartungswerte und $P$ die
Kovarianzmatrix $P=E(XX^t)$ von $X$.

Die Zufallsvariable $Y|X$ beschreibt die Messwerte von $Y$, die im
Systemzustand $X$ zu erwarten sind.
Die Messmatrix $H$ ermittelt bis auf ebenfalls normalverteilten
Messfehler die Messwerte, $Y|X$ ist also normalverteilt mit
Erwartungswert $E(HX)=H\mu$.
Die Wahrscheinlichkeitsdichte von $Y|X$ ist daher
\begin{equation}
f_{Y|x}(y)
=
\exp\biggl(-\frac12
(y-Hx)^tR^{-1}(y-Hx)
\biggr).
\label{yverteilung}
\end{equation}

\subsection{Lemma 20.1\label{subsection:lemma 20.1}}
In \cite{skript:kaperengler} wird dann das folgende Lemma formuliert,
welches im wesentlichen die Formel \eqref{kalman:kmatrix} für die
Kalman-Filter-Matrix beweist.

\begin{lemma*}
Seien $X$ und $Y$ normalverteilte Zufallsvariablen wie oben ausgeführt.
Dann ist $X|Y$ ebenfalls normalverteilt mit
Erwartungswert $\mu^*$ und Kovarianzmatrix $P^*$ mit
\begin{align}
\mu^*
&=
\mu + K(Y - H\mu)
\label{mustern}
\\
P^*
&=
(I-KH)P,
\label{Pstern}
\\
\intertext{wobei $K$ die Kalman-Gain-Matrix}
K
&=
PH^t(R+HPH^t)^{-1}
\label{kalman-formel}
\end{align}
ist.
\end{lemma*}

\begin{proof}[Beweis]
Das Lemma behauptet, dass es möglich ist, die Wahrscheinlichkeitsdichte
$f_{X|y}$ bis auf einen Normierungsfaktor, der nur von $y$ abhängen darf,
in der Form
\[
f_{X|y}(x)
\sim
\exp\biggl(
-\frac12 (x-\mu^*)^t{P^*}^{-1}(x-\mu^*)
\biggr)
\]
zu schreiben.
Dies bedeutet, dass nach
\eqref{xverteilung} und \eqref{yverteilung}
\begin{align*}
\exp\biggl(
-\frac12 (x-\mu^*)^t{P^*}^{-1}(x-\mu^*)
\biggr)
&\sim
f_X(x)\, f_{Y|x}(y)
\\
&=
\exp\biggl(
-\frac12\bigl(
(x-\mu)^tP^{-1}(x-\mu)
+
(y-Hx)^tR^{-1}(y-Hx)
\bigr)
\biggr)
\end{align*}
gelten muss.
Die Terme in den Exponentialfunktionen müssen also bis auf eine Konstante,
die nur von $y$ abhängen darf, übereinstimmen:
\[
(x-\mu^*)^t{P^*}^{-1}(x-\mu^*)
=
(x-\mu)^tP^{-1}(x-\mu)
+
(y-Hx)^tR^{-1}(y-Hx)
+c(y).
\]
Wir multiplizieren beide Seiten aus und verwenden dabei, dass die
Kovarianzmatrizen und ihre Inversen symmetrisch sind.
Die linke Seite ergibt:
\[
\text{LHS}
=
x^t{P^*}^{-1}x
-2{\mu^*}^t {P^*}^{-1}x
+{\mu^*}^t{P^*}^{-1}\mu^*.
\]
Die rechte Seite ist:
\begin{align*}
\text{RHS}
&=
x^tP^{-1}x-2\mu^tP^{-1}x+\mu^tP^{-1}\mu+
x^tH^tR^{-1}Hx-2y^tR^{-1}Hx+y^tR^{-1}y+c(y)
\\
&=
x^t(P^{-1}+H^tR^{-1}H)x
-2(\mu^tP^{-1}+y^tR^{-1}H)x
+\mu^tP^{-1}\mu +y^tR^{-1}y+c(y).
\end{align*}
Nur der erste Term auf beiden Seiten enthält die quadratischen Terme, 
es folgt
\[
{P^*}^{-1} = P^{-1}+H^tR^{-1}H
\qquad\Rightarrow\qquad
P^*
=
(P^{-1}+H^tR^{-1}H)^{-1}.
\]
Ebenso lesen wir an den Termen linear in $x$ ab, dass
\[
{\mu^*}^t {P^*}^{-1}
=
\mu^tP^{-1}+y^tR^{-1}H
\]
sein muss.
Durch Transponieren finden wir die Gleichung
\[
{P^*}^{-1}\mu^*
=
P^{-1}\mu + H^tR^{-1}y,
\]
was wir durch Multiplikation mit $P^*$ von links nach $\mu^*$ auflösen
können:
\begin{align*}
\mu^*
&=
P^*( P^{-1}\mu + H^tR^{-1}y)
\end{align*}
Schreiben wir $P^{-1}={P^*}-H^tR^{-1}H$, erhalten wir
\begin{align*}
\mu^*
&=
P^*( ({P^*}-H^tR^{-1}H)
\mu + H^tR^{-1}y)
\\
&=
\mu + P^*H^tR^{-1}y-P^*H^tR^{-1}H\mu
\\
&=
\mu + P^*H^tR^{-1}(y-H\mu).
\end{align*}
Mit der Abkürzung
\begin{equation}
K=P^*H^tR^{-1}
=
(P^{-1}+H^tR^{-1}H)^{-1}H^tR^{-1}
\label{kalman-gain}
\end{equation}
wird daraus
\[
\mu^* = \mu + K(y-H\mu).
\]
Damit ist gezeigt, dass sich die Wahrscheinlichkeitsdichte von $X|Y$
tatsächlich als Normalverteilung schreiben lässt, und dass der Erwartungswert
$\mu^*$ von $X|Y$ in der Form \eqref{mustern} geschrieben werden kann.

Jetzt muss nur noch gezeig werden, dass für $P^*$ und $K$ die im Lemma
genannten Formeln gelten.
Wir rechnen zuerst nach, dass die Formel für die $K$ im Lemma gleichbedeutend
ist mit der Formel~\eqref{kalman-gain}.
Wir multiplizieren zu diesem Zweck die Differenz der beiden zu vergleichenden
Terme von links mit
$P^{-1}+H^tR^{-1}H$
und von rechts mit $R+HPH^t$.
Wir erhalten
\begin{align*}
(P^{-1}+H^tR^{-1}H)PH^t
-
H^tR^{-1}(R+HPH^t)
&=
H^t + H^tR^{-1}HPH^t
- H^t - H^tR^{-1}HPH^t
=0,
\end{align*}
womit die Formel~\eqref{kalman-formel} nachgewiesen ist.

Es bleibt, die Formel \eqref{Pstern} für $P^*$ nachzuweisen.
Zunächst lesen wir aus der Abkürzung \eqref{kalman-gain} ab, dass
\begin{align*}
KH
&=
P^*
H^tR^{-1}H
\\
&=
(P^{-1}+H^tR^{-1}H)^{-1}(P^{-1}
+H^tR^{-1}H
-P^{-1}
)
\\
&=
I- P^* P^{-1}
\\
\Rightarrow\qquad
P^*P^{-1}&=I-KH.
\end{align*}
Multiplizieren wir jetzt beide Seiten von rechts mit $P$, erhalten wir
\[
P^*
=
(I-KH)P,
\]
wie im Lemma behauptet.
\end{proof}

\subsection{Übungsaufgabe 4\label{subsection:aufgabe4}}
Die Formeln für
$\operatorname{var}(X_1|y)$
und
$\operatorname{var}(X_3|y)$
auf Seite~266 von \cite{skript:kaperengler} sind ebenfalls falsch.
In diesem Abschnitt leiten wir korrigierte Versionen her.

Zunächst ist die Matrix $P$ (oder $\Sigma$, wie sie in der Aufgabe~4 und 
im Abschnitt~20.3.1 heisst) zu finden.
Aus der Rekursionsformel
\begin{align*}
\operatorname{var}(X_{i+1})
&=
\operatorname{var}(\alpha X_i + \xi_{i+1})
=
\alpha^2 \operatorname{var}(X_i) + \operatorname{var}(\xi_{i+1})
=
\alpha^2 \operatorname{var}(X_i) + \sigma^2
\end{align*}
für die Varianzen lesen wir zum Beispiel ab
\begin{align*}
\operatorname{var}(X_1)
&=
\sigma^2
\\
\operatorname{var}(X_2)
&=
\alpha^2\sigma^2 +\sigma^2 = (\alpha^2 + 1)\sigma^2
\\
\operatorname{var}(X_3)
&=
(\alpha^4+\alpha^2+1)\sigma^2
\\
\operatorname{var}(X_4)
&=
(\alpha^6+\alpha^4+\alpha^2+1)\sigma^2
\end{align*}
Analog kann man aus der Rekursionsformel
\begin{align*}
\operatorname{cov}(X_i,X_{i+k})
&=
\operatorname{cov}(X_i,\alpha X_{i+k-1}+ \xi_{i+k})
=
\alpha \operatorname{cov}(X_i,X_{i+k-1}) + \operatorname{cov}(X_i,\xi_{i+k})
=
\alpha \operatorname{cov}(X_i,X_{i+k-1})
\end{align*}
ablesen:
\begin{align*}
\operatorname{cov}(X_1,X_2)
&=
\alpha \operatorname{var}(X_i)=\alpha\sigma^2
\\
\operatorname{cov}(X_1,X_3)
&=
\alpha\operatorname{cov}(X_1,X_2)
=
\alpha^2\sigma^2
\\
\operatorname{cov}(X_1,X_4)
&=
\alpha^3\sigma^2
\\
\operatorname{cov}(X_2,X_3)
&=
\alpha\operatorname{var}(X_2)
=
\alpha(\alpha^2+1)\sigma^2
\\
\operatorname{cov}(X_2,X_4)
&=
\alpha^2(\alpha^2+1)\sigma^2
\\
\operatorname{cov}(X_3,X_4)
&=
\alpha\operatorname{var}(X_3)
=
\alpha(\alpha^4+\alpha^2+1)\sigma^2
\end{align*}
Die Kovarianzmatrix ist daher
\[
P
=
\sigma^2
\begin{pmatrix}
1       &\alpha              &\alpha^2           &\alpha^3                    \\
\alpha  &\alpha^2 + 1        &\alpha(\alpha^2+1) &\alpha^2(\alpha^2+1)        \\
\alpha^2&\alpha(\alpha^2+1)  &\alpha^4+\alpha^2+1&\alpha(\alpha^4+\alpha^2+1) \\
\alpha^3&\alpha^2(\alpha^2+1)&\alpha(\alpha^4+\alpha^2+1)&\alpha^6+\alpha^4+\alpha^2+1\\
\end{pmatrix}
\]
Aus den Formel für $K$ und $P^*$ im Lemma kann man jetzt beide berechnen
und die verlangten Varianzen als Diagonalelemente von $P^*$ ablesen:
\begin{align*}
\operatorname{var}(X_1|y)
&=
\frac{\sigma^2\,\tau^4+(\alpha^2+2)\sigma^4\,\tau^2+\sigma^6
 }{\tau^4+(\alpha^4+2\alpha^2+2)\sigma^2\tau^2+(\alpha^2+1
 )\sigma^4}
\\
&=
\sigma^2
\frac{\tau^4+(\alpha^2+2)\sigma^2\,\tau^2+\sigma^4}%
{\tau^4+(\alpha^4+2\alpha^2+2)\sigma^2\,\tau^2+(\alpha^2+1
 )\sigma^4} < \sigma^2
\\
\operatorname{var}(X_3|y)
&=
\frac{(\alpha^4+a^2+1)\,\sigma^2\,\tau^4+(\alpha^2+1)\,
 \sigma^4\,\tau^2}{\tau^4+(\alpha^4+2\alpha^2+2)\sigma^2
 \tau^2+(\alpha^2+1)\sigma^4}
\\
&=
\tau^2\sigma^2
\frac{(\alpha^4+a^2+1)\tau^2+(\alpha^2+1)
 \sigma^2}{\tau^4+(\alpha^4+2\alpha^2+2)\sigma^2
 \tau^2+(\alpha^2+1)\sigma^4}
=
O(\tau^2)\qquad \text{für $\tau\to 0$}.
\end{align*}
Die Schlussfolgerungen der Aufgabe bleiben somit gültig.

%$${{\sigma^2\,\tau^4+\left(a^2+2\right)\,\sigma^4\,\tau^2+\sigma^6
% }\over{\tau^4+\left(a^4+2\,a^2+2\right)\,\sigma^2\,\tau^2+\left(a^2+1
% \right)\,\sigma^4}}$$
%$${{\left(a^4+a^2+1\right)\,\sigma^2\,\tau^4+\left(a^2+1\right)\,
% \sigma^4\,\tau^2}\over{\tau^4+\left(a^4+2\,a^2+2\right)\,\sigma^2\,
% \tau^2+\left(a^2+1\right)\,\sigma^4}}$$
