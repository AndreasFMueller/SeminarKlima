%
% assimilation.tex
%
% (c) 2018 Prof Dr Andreas Müller, Hochschule Rapperswil
%
\section{Das Assimilationsproblem\label{section:assimilationsproblem}}
Der Kalman-Filter ist ein Beispiel dafür, wie laufend neue Messdaten und
ein Modell des untersuchten System ermöglichen, selbst dann einigermassen
vollständige Informationen über den Zustand des Systems zu erhalten, wenn
nicht alle Zustandsgrössen direkt gemessen werden.
Dass dies sogar für ein chaotisches System bis zu einem gewissen Grad
möglich ist, zeigen die Experimente in Kapitel~\ref{chapter:kalman}.

Allerdings wurden bei der Konstruktion des Kalman-Filters einige
einschränkende Annahmen gemacht.
Das System und der Messprozess musste linear beschreibbar sein, 
und Fehler mussten normalverteilt, unabhängig und mit Mittelwert $0$ sein.
Ausserdem wurden immer nur Messungen des aktuellen Zeitpunktes verwendet,
um die aktuellste Zustandsschätzung zu verbessern.
Dieses Vorgehen ist natürlich in vielen technischen Anwendungen angemessen
und auch passend zum Beispiel für Wetterprognosen, wo fast nur der aktuelle
Zustand der Atmosphäre interessiert.

In einem Klimamodell ist jedoch nicht nur der aktuelle Zustand der Atmosphäre
interessant.
Vielmehr geht es darum, die Geschichte des Erdklimas möglichst genau
zu rekonstruieren und dabei immer weitere, neu zugänglich gemachte 
Daten in das Modell zu integrieren und die Qualität der Vorhersagen zu
verbessern.

In diesem Kapitel soll daher angedeutet werden, wie ein allgemeineres
Datenassimilationsproblem formuliert werden kann, welches als Spezialfall
das im vorangegangenen Abschnitt~\ref{section:kalman-filter} gelöste
Filterproblem hat.

\subsection{Markov-Kette und Markov-Eigenschaft\label{subsection:markov}}
Der Zustand des Klimasystems zu einem beliebigen Zeitpunkt kann mit einem
Vektor $x\in\mathbb R^n$ von Variablen beschrieben werden, im einfachsten
Fall bestehend allein aus der globalen Mitteltemperatur.
Natürlich ist $x$ nicht bekannt, wir können bestenfalls $X$ ist eine
Zufallsvariable betrachten, die den Wert $x$ angenommen hat.

Da die ganze Klimageschichte rekonstruiert werden soll, sind die Werte von
$x$ zu diskreten Zeitpunkten $0,\dots,N$ zu bestimmen.
Die Klimageschichte ist also eine Folge $X(k)$ mit $k=0,\dots,N$ von
Zufallsvariablen.
Im Folgenden wird davon ausgegangen, dass die Wahrscheinlichkeitsdichte
$f_{X(k)}(x)$ von $X(k)$ bekannt ist.
Man nennt $X(k)$ einen diskreten {\em stochastischen Prozess}.
\index{stochastischer Prozess}%
\index{Prozess, stochastischer}%
In der Realität steht uns jeweils nur ein Teil der Werte des Prozesses zur
Verfügung, so bezeichnen wir mit $X(i:j)$ die Folge
\[
X(i),X(i+1),\dots,X(j-1),X(j).
\]
Damit ist $X(0:n)$ die Klimageschichte bis zur Zeit $n$, und $X(n:N)$ die
Zukunft\footnote{Die Notation $X(i:j)$ ist motiviert durch die 
von Matlab oder Octave verwendete Notation \texttt{i:j} für die
Folge der Zahlen von $i$ bis $j$.}.

Das Assimilationsproblem besteht darin, dass Messwerte $Y(k)$ zur Verfügung
gestellt werden, mit denen sich die Wahrscheinlichkeitsdichte von $X(k)$
``genauer'' bestimmen lässt.
Um den Formalismus herauszuarbeiten seien $X$ und $Y$ Zufallsvariablen,
es soll die Frage beantwortet werden, wie sich die Verteilung von $X$
ändert, wenn $Y$ den Wert von $y$  annimmt.
Die Wahrscheinlichkeitsdichte von $(X,Y)$ sei $f_{X,Y}(x,y)$, sie kann
auf zwei verschiedene Arten als Produkt
\begin{equation}
f_{X,Y}(x,y)
= 
f_{X|y}(x)\,f_Y(y)
=
f_{Y|x}(y)\,f_X(x)
\label{skript:bedpdf}
\end{equation}
geschrieben werden.
\index{Satz von Bayes}%
Der {\em Satz von Bayes} für die {\em bedingten Wahrscheinlichkeitsdichten}
\index{bedingte Wahrscheinlichkeit}
\begin{equation}
f_{X|y}(x)
=
\frac{f_X(x)}{f_Y(y)}\,f_{Y|x}(y).
\label{assim:bayes}
\end{equation}
stellt en Zusammenhang zwischen den bedingten Wahrscheinlichkeiten her.

Wenn die Zufallsvariablen $X$ und $Y$ unabhängig sind, dann hat der
Wert von $y$ keinen Einfluss auf die Wahrscheinlichkeitsdichte von $X$,
es ist also
\[
f_{X|y}(x)
=
f_X(x)\quad\forall y.
\]
Gegeben $f_{X|y}(x)$ lässt sich auch die Wahrscheinlichkeitsdichte
$f_X$ berechnen, es gilt
\begin{align}
f_X(x)
&=
\int f_{X|y}(x)\,f_Y(y) dy.
\label{assim:total}
\end{align}
Dies ist der Satz von der {\em totalen Wahrscheinlichkeit}.
\index{totale Wahrscheinlichkeit}%

Kausalität verlangt, dass der Zustand $X(k)$ zur Zeit $k$ abhängig ist
von den Zustandswerten $x(j)$ mit $j<k$.
Da $X(k)$ Zufallsvariablen sind, muss sich diese Eigenschaft mit den
Wahrscheinlichkeitsdichten für $X(k)$ ausdrücken lassen.

Mit Ausnahme der verzögerten Differentialgleichung, die zur Modellierung
des El~Niño-Phänomens in Kapitel~\ref{chapter:elnino} verwendet wurde,
haben alle bisher studierten Modelle die Eigenschaft, dass der Zustand
zur Zeit $k$ vollständig bestimmt ist durch den Zustand zur Zeit $k-1$.
Frühere Zustände zu Zeiten $j < k-1$ müssen nicht berücksichtigt werden.
Mit Hilfe der bedingten Wahrscheinlichkeitsdichte~\eqref{skript:bedpdf}
kann dies wie folgt ausgedrückt werden:

\begin{definition}
\index{Markov-Eigenschaft}%
\index{Markov-Kette}%
Der stochastische Prozess $X(k)$ hat die {\em Markov-Eigenschaft}, wenn gilt
\begin{equation}
f_{X(n:N)|x(0:n-1)}(x(n:N)) 
=
f_{X(n:N)|x(n-1)}(x(n:N)).
\end{equation}
Ein solcher stochastischer Prozess heiss {\em Markov-Kette}.
\end{definition}
Die Definition besagt, dass die Verteilung von $X(n:N)$ sich nicht
ändert, wenn man ausser den Werten $x(n-1)$ auch noch die früheren
Werte $x(0:n-2)$ kennt.

Die Markov-Eigenschaft eines stochastischen Prozesses bedeutet,
dass für Vorhersagen von $X(n:N)$ nur die Verteilung von $X(n-1)$
bekannt sein muss.
Weiteres Wissen über weiter zurück liegende Werte $X(0:n-2)$
liefert keine Verbesserung der Vorhersage von $X(n)$.
Durch wiederholte Anwendung von \eqref{skript:bedpdf} erhält man
daraus
\begin{align*}
f_{X(0:n)}(x(0:n))
&=
f_{X(n)|x(0:n-1)}(x(n))\,
f_{X(0:n-1)}(x(0:n-1))
=
f_{X(n)|x(n-1)}(x(n))\,
f_{X(0:n-1)}(x(0:n-1))
\\
&=
f_{X(n)|x(n-1)}(x(n))\,
f_{X(n-1)|x(n-2)}(x(n-1))\,
\dots\,
f_{X(0}(x(0))
\end{align*}
Die Wahrscheinlichkeitsdichten $f_{X(j)|x(j-1)}(x(j))$ heissen
die {\em Übergangswahrscheinlichkeiten } der Markov-Kette.

\subsection{Filterung, Vorhersage und Reanalyse}
Die Aufgabe, die sich jetzt stellt, ist aus Ausschnitten der
Messwertegeschichte $Y(0:N)$ Ausschnitte der Klimageschichte zu
rekonstruieren.
Die folgenden Spezialfälle werden unterschieden:
\begin{enumerate}
\item
\index{Filterung}%
Filterung: Schätze $X(n)$ aus den Werten von $Y(1:n)$.
\item
\index{Vorhersage}%
Vorhersage: Schätze $X(n)$ aus den Werten von $Y(1:n-1)$.
\item
\index{Reanalyse}%
Reanalyse: Schätze $X(n)$ aus den Werten von $Y(1:N)$.
\end{enumerate}
Man beachte, dass bei der Reanalyse auch die späteren Werte zur
Verfügung stehen.
Reanalyse liegt zum Beispiel vor, wenn aktuelle Klimaentwicklungen dazu
verwendet werden, die Klimageschichte der letzten Jahrzehnte genauer 
zu rekonstruieren.

Um eines dieser Probleme zu lösen, muss der Zusammenhang zwischen den
Zustandsvariablen $X(k)$ und den Messgrössen $Y(j)$ bekannt sein.
Die Messgrösse $Y(k)$ soll nur vom Zustand zur Zeit $k$ abhängen,
nicht von früheren oder späteren Zeitpunkten.
Die Zufallsvariablen $Y(j)$ und $X(k)$ mit $j\ne k$ müssen daher
unabhängig sein.
Dies bedeutet, dass sich die Wahrscheinlichkeitsdichte als Produkt
\begin{align*}
f_{Y(1:n)|x(0:n)} (y(1:n))
&=
f_{Y(n)|x(n)}(y(n))\,
f_{Y(n-1)|x(n-1)}(y(n-1))\,
\dots\,
f_{Y(1)|x(1)}(y(1))
\end{align*}
zerlegen lässt.

Damit lassen sich jetzt die Problemstellungen genauer formulieren 
und auch lösen.

\begin{satz}[Filter und Vorhersage]
\label{satz:filter und vorhersage}
Gegeben sei
\begin{enumerate}
\item die Wahrscheinlichkeitsdichte $f_{X(0)}$ für den Anfangszustand,
\item die Übergangswahrscheinlichkeiten $f_{X(j)|x(j-1)}$ der
Markov-Kette $X(0:N)$ und
\item die bedingten Wahrscheinlichkeitsdichten $f_{Y(j)|x(j)}$ für die
Messungen.
\end{enumerate}
Dann können wie Wahrscheinlichkeitsdichten $f_{X(i)|y(1:i)}$ für das
Filterproblem sowie die Wahrscheinlichkeitsdichten $f_{X(i)|y(1:i-1)}$
für das Vorhersage-Problem wie folgt bestimmt werden.
\begin{enumerate}
\item Schritt:
\begin{align}
f_{X(1)}(x(1))
&=
\int f_{X(1)|x(0)}(x(1)) \, f_{X(0)}(x(0))\, dx(0)
\label{assim:totalstep1}
\\
f_{X(1)|y(1)}(x(1))
&\sim
f_{Y(1)|x(1)}(y(1))\, f_{X(1)}(x(1)),
\label{assim:bayesstep1}
\end{align}
wobei der Proportionalitätsfaktor so gewählt werden muss, dass tatsächlich
eine Wahrscheinlichkeitsdichte entsteht.
\item Schritt: für $i=2,\dots,N$
\begin{align}
f_{X(i)|y(1:i-1)}(x(i))
&=
\int f_{X(i)|x(i-1)}(x(i))\, f_{X(i-1)|y(1:i-1)}(x(i-1))\,dx(i-1)
\label{assim:totalstep2}
\\
f_{X(i)|y(1:i)}(x(i))
&\sim
f_{Y(i)|x(i)}(y(i))\, f_{X(i)|y(1:i-1)}(x(i)),
\label{assim:bayesstep2}
\end{align}
liefert,
wobei wiederum der Proportionalitätsfaktor so gewählt werden muss, dass
eine Wahrscheinlichkeitsdichte entsteht.
\end{enumerate}
\end{satz}

\begin{proof}[Beweis]
Die Formeln von Schritt~1 sind nichts anderes als der Satz von
der totalen Wahrscheinlichkeit für \eqref{assim:bayesstep1} und
der Satz von Bayes~\eqref{assim:bayes} für \eqref{bayesstep2}.

Für den Beweis der Formeln für den zweiten Schritt verwenden wir 
vollständig Induktion.
Die Verankerung für $i=2$ ist gegeben durch den Satz von der totalen
Wahrscheinlichkeit~\eqref{assim:total}
\begin{align*}
f_{X(2)|y(1)}(x(2))
&=
\int f_{X(2)|x(1)}(x(2))  f_{X(1)}(x(1))\, dx(1).
\end{align*}
und erhalten damit die Formel~\eqref{assim:totalstep2} für $i=2$.
Die zweite Formel~\eqref{assim:bayesstep2} ergibt sich aus dem Satz
von Bayes~\eqref{assim:bayes}.

Für den Induktionsschritt verwenden wir wieder zuerst den Satz von der
totalen Wahrscheinlichkeit~\eqref{assim:total} und erhalten
\begin{align*}
f_{X(i)|y(1:i-1)}(x(i))
&=
\int f_{X(i)|x(i-1)}(x(i)) \, f_{X(i-1)|y(1:i-1)}(x(i-1))\,dx(i-1)
\end{align*}
also die Formel~\eqref{assim:totalstep2}.
Die zweite Formel~\eqref{assim:bayesstep2} folgt daraus mit dem Satz
von Bayes~\eqref{assim:bayes}.
\end{proof}


\begin{satz}[Reanalyse]
\label{satz:reanalyse}
Gegeben seien
\begin{enumerate}
\item
die Übergangswahrscheinlichkeiten $f_{X(i)|x(i-1)}$ für die
Markov-Kette $X(0:n)$
\item
die Wahrscheinlichkeitsdichte für das Filterproblem
$f_{X(i)|y(1:i)}$.
\end{enumerate}
Dann kann die Wahrscheinlichkeitsdichte für das Reanalyse-Problem wie folgt
bestimmt werden
\begin{enumerate}
\item Falls $i=N$, dann ist die Wahrscheinlichkeitsdichte für das
Reanalyse-Problem dieselbe dieselbe wie jene für das Filterproblem.
\item Für $i=2,\dots,N-1$ und für die Wahrscheinlichkeitsdichte 
$f_{X(i+1)|y(1:N)}$  für das Reanalyse-Problem für $X(i+1)$ 
wird die Wahrscheinlichkeitsdichte für das Reanalyse-Problem für $X(i)$
durch
\begin{align}
f_{X(i)|x(i+1),y(1:i)}(x(i))
&\sim
f_{X(i+1)|x(i)}(x(i+1))\, f_{X(i)|y(1:i)}(x(i))
\label{assim:reanabayes}
\\
f_{X(i)|y(1:N)}(x(i))
&=
\int f_{X(i)|x(i+1),y(1:i)}(x(i))\, f_{X(i+1)|y(1:N)}(x(i+1))\,dx(i+1)
\label{assim:reanatotal}
\end{align}
berechnet,
wobei der Proportionalitätsfaktor wieder so gewählt werden muss, dass
eine Wahrscheinlichkeitsdichte entsteht.
\end{enumerate}
\end{satz}

\begin{proof}[Beweis]
Fall~1 ist klar, das Reanalyse-Problem für $X(N)$ ist dasselbe wie des
Filterproblem für $X(N)$.

Es bleibt daher nur noch, die Formeln
\eqref{assim:reanabayes}
und
\eqref{assim:reanatotal}
nachzuweisen.
Die Formel
\eqref{assim:reanabayes}
ist nichts anderes als der 
als der Satz von Bayes
\eqref{assim:bayes}.
Die Formel 
\eqref{assim:reanatotal}
ist der Satz von der totalen Wahrscheinlichkeit
\eqref{assim:total}.
\end{proof}






