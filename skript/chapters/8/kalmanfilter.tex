%
% kalmanfilter.tex
%
% (c) 2018 Prof Dr Andreas Müller, Hochschule Rapperswil
%
\section{Kalman-Filter\label{section:kalman-filter}}
\index{Kalman-Filter}%
Die bisherige Sicht auf das Problem war, dass aus dem Modell und den
Anfangsbedingungen der zukünftige Zustand des Modells abgeleitet werden
kann.
Doch bei fehlerhaften Anfangsdaten wird auch der zukünftige Zustand
fehlerhaft sein.
Ein erster Versuch, die Situation wieder zurecht zu rücken ist, auch für
den späteren Zustand ein Messung durchzuführen und zu versuchen,
den Systemzustand derart zu korrigieren, dass auch die Vorhersagen des
Modells besser mit der Realität übereinstimmen.
Dies ist genau, was der Kalman-Filter ermöglicht.

In diesem Abschnitt soll eine kurze Übersicht über den Kalman-Filter
gegeben werden.
Sie stützt sich auf \cite{skript:catlin}, eine detailliertere Darstellung
findet sich in \cite[Kapitel 8]{skript:wrstatskript}.

\subsection{Messung und System\label{subsection:messung und system}}
Wir gehen davon aus, dass das System, zum Beispiel das Klimasystem der Erde,
mit einem Vektor $x\in\mathbb R^n$ von Zustandsvariablen beschrieben werden
kann.
Den detaillierten Zustand des Systems können wir nicht erfahren, die
Grössen $x$ sind also bestenfalls teilweise bekannt.
Die beschränkte Kenntnis des Systemzustand äussert sich darin, dass
nur eine Schätzung $\hat x$ für $x$ gefunden werden kann.
$\hat x$ stellt also das aktuelle Wissen über den Systemzustand dar.

Durch direkte Messung können möglicherweise einzelne Systemvariablen
bestimmt werden.
Häufiger können viele der Variablen aber nicht direkt gemessen werden.
Das Modell braucht vielleicht die Temperatur, aber gemessen werden kann nur
die Strahlungsintensität.
Die gemessenen Variablen $z\in\mathbb R^m$ sind also bestenfalls eine
Funktion $z=f(x)$ der Zustandsvariablen $x$.

Der Kalman-Filter soll ermöglichen, das Wissen über den System-Zustand
$\hat x$ durch Messung der Grössen $z$ zu verbessern, und so eine
verbesserte Schätzung zu erhalten.
Dies bedeutet, dass durch wiederholte Messung die Zustandsschätzung
laufen verbessert werden kann, so dass die Fehler der ursprünglich gewählten 
Anfangsbedingungen immer unbedeutender werden.
Dies bedeutet aber, dass die zeitliche Entwicklung des Systems mit einer Folge
$x_k$, $k\in\mathbb N$, von Zustandsvariablen modelliert werden muss.
Diese Zustände sind nur näherungsweise bekannt als eine Folge
von Schätzungen $\hat{x}_k$, $k\in\mathbb N$.
Mit jeder Messung $z_k$ erfahren wir etwas über den Systemzustand
zur Zeit $k$.
Die Messung $z_k$ und das bisherige Wissen $\hat{x}_{k-1}$ ergeben
die neueste beste Schätzung $\hat{x}_k$.
Das Ziel des Kalman-Filters ist, den Zusammenhang zwischen den
genannten Grössen in Form eines funktionalen Zusammenhangs der Form
\begin{equation}
\hat{x}_k = F(\hat{x}_{k-1},z_n)
\label{skript:filterbasis}
\end{equation}
zu finden.

Die implizite Annahme hinter \eqref{skript:filterbasis} ist,
dass es einen Zusammenhang zwischen $x_{k-1}$ und $x_k$ und damit auch
zwischen den Schätzungen $\hat{x}_{k-1}$ und $\hat{x}_k$ gibt.
Es ist daher denkbar, \eqref{skript:filterbasis} in zwei Teilschritte
aufzuteilen.
Das vorhandene Wissen $\hat{x}_{k-1}$ macht eine Vorhersage über den
Zustand $\hat{x}_{k|k-1}$, welches anschliessend mit den neuen 
Messwerten $z_k$ zur bestmöglichen Information $\hat{x}_k$ über den
aktuellen Zustand verbunden werden kann.
Dazu braucht es einerseits eine Beschreibung der Systementwicklung
$x_{k+1} = \varphi(x_k)$, aus der sich auch eine Entwicklung der
Schätzungen $\hat{x}_{k|k-1} = \Phi(\hat{x}_{k-1})$ ableiten lässt.
Andererseits brauchen wir eine Funktion
\begin{equation}
\hat{x}_k = K(\hat{x}_{k|k-1}, z_k)
\end{equation}
mit der der Zustand korrigiert werden kann.
Setzen wir dies wieder zusammen, folgt
\[
\hat{x}_k
=
F(\hat{x}_{k|k-1},z_k)
=
K(\Phi(\hat{x}_{k-1}),z_k).
\]

\subsection{Systemmodellierung\label{subsection:systemmodellierung}}
Die früher entwickelten Modelle zur Klimaentwicklung resultierten in
zum Teil sehr komplizierten Abhängigkeiten zwischen Systemvariablen
zu weit auseinander liegenden Zeitpunkten.
Sie hatten jedoch gemeinsam, dass sie durch eine Differentialgleichungen
erster Ordnung der Form
\[
\frac{dx}{dt} = f(x,t)
\]
beschrieben werden konnten.
Für genügend kurze Zeitschritte $\Delta t$ können die Zustandsvariablen
$x_k = x(k\Delta t)$ zu den Zeitpunkten $t_k=k\Delta t$
durch lineare Approximation
\[
x_k = x_{k-1} + f(x_{k-1}, t_{k-1}) \cdot \Delta t
\]
bestimmt werden.
Eine Schätzung $\hat{x}_{k-1}$ des Zustandes zur Zeit $k-1$ 
kann dann mit Hilfe von
\begin{equation}
\hat{x}_{k|k-1} = \hat{x}_{k-1} + f(\hat{x}_{k-1},t_{k-1})\cdot \Delta t
\label{kalman:filter-schaetzung}
\end{equation}
zu einem Zustand $\hat{x}_{k|k-1}$ entwickelt werden.

Um zu beurteilen, wie gross die Korrektur durch die Messung sein soll,
muss bekannt sein, in welchem Ausmass die Schätzung vom tatsächlichen
Zustand abweicht.
Insbesondere muss \eqref{kalman:filter-schaetzung} dazu verwendet werden
können, auch die Entwicklung der Abweichung $x_{k|k-1}-\hat{x}_{k|k-1}$
aus $x_{k-1} - \hat{x}_{k-1}$ zu berechnen.
Für eine beliebige Funktion $f$ ist dies im Allgemeinen zu schwierig.
Durch erneute Linearisierung kann aus
\eqref{skript:filterbasis} eine lineare Entwicklungsgleichung
\begin{equation}
x_k = \varphi_k x_{k-1}
\end{equation}
gewonnen werden, wobei $\varphi_k$ eine $n\times n$-Matrix ist.

Nur die allereinfachsten Systeme lassen sich einigermassen korrekt modellieren.
Fast immer enthält ein Modell Vereinfachungen, die nötig sind, um über
überhaupt ein Modell zu erhalten, mit dem Vorhersagen möglich sind.
Entsprechend können die Vorhersagen auch nicht exakt sein, es verbleibt
immer ein Fehler.
Die Systementwicklung kann dies mit
\begin{equation}
x_k = \varphi_k x_{k-1} + u_k
\label{skript:kalman-system-mitfehler}
\end{equation}
berücksichtigen, worin $u_k$ alle nicht modellierten Einflüsse
enthält, auch genannt der {\em Systemfehler}. 
\index{Systemfehler}
Wir nehmen an, dass das Modell alle wesentlichen Einflüsse umfasst.
Die Fehler $u_k$ müssen daher zufällig und unvorhersagbar erscheinen,
die sich im Mittel nicht auswirken.
Es wird daher angenommen, dass die Komponenten $u_k$ normalverteilte
und unabhängige Zufallsvariablen mit Erwartungswert $0$ sind, dass also
\[
E(u_k)=0
\qquad\text{und}\qquad
E(u_ku_k^t)
=
\begin{pmatrix}
\sigma_1^2&0         &\dots &0\\
0         &\sigma_2^2&\dots &0\\
\vdots    &\vdots    &\ddots&\vdots\\
0         &0         &\dots &\sigma_n^2
\end{pmatrix}
=
Q_k
\]
ist\footnote{%
Aus der Argumentation oben lässt sich nicht ableiten, dass
die Fehler $u_k$ unabhängig sind, genau genommen gehen wir daher zu
weit, wenn wir annehmen, dass $Q_k$ Diagonalform hat, auch wenn dies
in der Literatur der Einfachheit halber üblicherweise angenommen wird.
Etwas allgemeiner sollte man nur voraussetzen, dass $Q_k$ eine symmetrische,
positiv definite Matrix ist.
Die nachfolgend beschriebene Theorie lässt sich damit unverändert
entwickeln.
}.
Die Matrix $Q_k$, die {\em Systemfehler-Kovarianz}
enthält auf der Diagonalen die Varianzen der Systemfehler $u_k$ 
und ausserhalb der Diagonalen die Kovarianzen.
\index{Systemfehler-Kovarianz}

\subsection{Schätzfehler\label{subsection:schaetzfehler}}
Wenn Vorhersage $\hat{x}_{k|k-1}$ und Messung $\hat{z}_k$ 
miteinander kombiniert werden sollen um die bestmögliche Schätzung
$\hat{x}_k$ zu erhalten, wird Kenntnis über den zu erwartenden
Fehler der Schätzung benötigt.

Die Schätzung $\hat{x}_k$ sollte mindestens im Mittel korrekt sein.
Dies lässt sich dadurch ausdrücken, dass die Fehler $x_k - \hat{x}_k$
als Zufallsvariablen mit Erwartungswert $0$ modelliert werden können.
Als Mass für den Schätzfehler kann man daher die Kovarianzmatrix von
\index{Schätzfehler-Kovarianz}%
$x_k-\hat{x}_k$ verwenden, die mit
\[
P_k =  E\bigl(\, (x_k-\hat{x}_k) (x_k-\hat{x}_k)^t\,\bigr)
\]
bezeichnet wird.
Die Zeitentwicklung~\eqref{skript:kalman-system-mitfehler} führt einen
zusätzlichen Fehler ein, der mit
\begin{equation}
P_{k|k-1} = \varphi_k P_{k-1} \varphi_k^t + Q_k
\label{skript:kalman:fehlerentwicklung}
\end{equation}
bezeichnet werden soll.

\subsection{Messprozess\label{subsection:messprozess}}
\index{Messprozess}
Früher wurde gezeigt, wie der Messprozess die Werte der Messgrössen $z_k$
aus den Zustandsvariablen ableitet.
Auch hier ist die Modellierung mit einer allgemeinen Funktion $z=f(x)$ 
zu kompliziert.
Auch für den Messprozes wird daher angenommen, dass er durch eine
lineare Abbildung $z=Hx$ beschrieben werden kann, wobei $H$ eine
$m\times n$-Matrix ist.
Um einen Rest von Nichtlinearität von $f$ zu retten, können wir davon
ausgehen, dass $H$ zusätzlich vom Zeitpunkt abhängt, also $z_k=H_kx_k$.

Jede Messung ist aber mit Fehlern behaftet, die mit einem Fehler-Vektor
$w_k$ in
\begin{equation}
z_k = H_kx_k + w_k,
\label{skript:kalman:messfehler}
\end{equation}
modelliert werden soll,
dessen Komponenten normalverteile Zufallsvariable mit Erwartungswert $0$
sind.
Meist wird zudem angenommen, dass die Fehler der einzelnen Messgrössen
unabhängig sind, dass die Kovarianzen von $w_k$ also eine
Diagonalmatrix
\[
R_k = E(w_kw_k^t) =
\begin{pmatrix}
\varrho_1^2&\dots & 0\\
\vdots     &\ddots&\vdots\\
0          &\dots &\varrho_m^2
\end{pmatrix},
\]
die {\em Messfehler-Kovarianz-Matrix},
ist\footnote{Wie bei der Systemfehler-Kovarianz würde es auch hier
genügen anzunehmen, dass $R_k$ eine symmetrisch positiv definite Matrix
ist.}.
\index{Messfehler}%
\index{Messfehler-Kovarianz}%

\subsection{Filterung\label{subsection:filterung}}
Wenn Vorhersage $\hat{x}_{k|k-1}$ und Messung $z_k$ nicht im Konflikt
stehen, dann ist auch keine Korrektur nötig, es kann
$\hat{x}_k = \hat{x}_{k|k-1}$ gewählt werden.
Offenbar kann man $\hat{x}_{k|k-1}$ nicht direkt mit $z_k$ vergleichen,
da diese Vektoren verschiedene Dimension haben.
Man kann aber $z_k$ vergleichen mit den Messwerten, die sich ergeben
müssten, wenn $\hat{x}_{k|k-1}$ der tatsächliche Zustand des Systems
wäre.
Diese Messwerte wären $H_k\hat{x}_{k|k-1}$.
Es ist daher naheliegend,
die Korrektur linear in der Differenz $z_k-H_k\hat{x}_{k|k-1}$
anzusetzen.
Es ist daher eine $n\times m$-Matrix $K_k$ gesucht, mit der sich
die Korrektur als
\begin{equation}
\hat{x}_{k}
=
\hat{x}_{k|k-1} + K_k(z_k-H_k\hat{x}_{k|k-1})
=
(I-K_kH_k)\hat{x}_{k|k-1} + K_kz_k
\label{skript:kalman:filter}
\end{equation}
berechnen lässt.
Die Matrix $K_k$ soll so bestimmt werden, dass der Fehler der
Schätzung $\hat{x}_k$ möglichst klein wird.

Der Fehler von $\hat{x}_k$ ist $P_k$, mit \eqref{skript:kalman:filter}
kann er aus
\begin{equation}
P_k
=
(I-K_kh_k)P_{k|k-1}(I-K_kH_k)^t + K_kR_kK_k^t
\label{skript:fehler:korrektur}
\end{equation}
berechnet werden.
Als Mass für die Grösse des Fehlers können die Fehler jeder einzelnen 
Systemvariablen, also die Varianzen der Komponenten von $x_k$ 
verwendet werden.
Diese stehen in der Matrix $P_k$ auf der Diagonale, 
das gesuchte Fehlermass ist also die Spur der Matrix $P_k$.
$K_k$ muss jetzt so bestimmt werden, dass $\operatorname{Spur} P_k$ 
minimal wird.
Wie in \cite{skript:wrstatskript} gezeigt wird, das erreicht, wenn
\[
\begin{aligned}
&&
\frac{\partial}{\partial K_k} \operatorname{Spur}P_k
&=
-2(I-K_kH_k)P_{k|k-1}H_k^t + 2K_kR_k
=
0
\\
&\Rightarrow&
P_{k|k-1}H_k^t
&=
K_k(H_kP_{k|k-1}H_k^t+R_k)
\\
&\Rightarrow&
K_k
&=
P_{k|k-1}H_k^t
(H_kP_{k|k-1}H_k^t+R_k)^{-1},
\end{aligned}
\]
sofern die Matrix in Klammern tatsächlich invertierbar ist.
Die Matrix $K_k$ heisst die {\em Kalman-Filter-Matrix}.
\index{Kalman-Filter-Matrix}%

Damit haben wir alle Formeln für den Kalman-Filter zusammengetragen.
In jedem Zeitschritt führen wir folgende Schritte durch:
\begin{enumerate}
\item
Vorhersage des Zustandes und des Schätzfehlers:
\begin{align*}
\hat{x}_{k|k-1}&= \varphi_k\hat{x}_{k-1} \\
P_{k|k-1}      &= \varphi_kP_{k-1}\varphi_k^t + Q_k
\end{align*}
\item
Berechnung der Kalman-Filter-Matrix
\begin{align}
K_k
&=
P_{k|k-1}H_k^t
(H_kP_{k|k-1}H_k^t+R_k)^{-1}
\label{kalman:kmatrix}
\end{align}
\item
Korrektur:
\begin{align*}
\hat{x}_k&= (I-K_kH_k)\hat{x}_{k|k-1} + K_kz_k \\
P_k      &= 
(I-K_kh_k)P_{k|k-1}(I-K_kH_k)^t + K_kR_kK_k^t
\end{align*}
\end{enumerate}

\subsection{Beispiel}
Als Beispiel betrachten wir ein stark vereinfachtes Modell für die
Temperatur der Erde und die Albedo in der Nähe eines Gleichgewichtes.
Wir sind nur daran interessiert, die Temperaturanomalie $T$ und die
Albedo $a$ in der Umgebung eines Gleichgewichtspunktes zu
modellieren.
In einem Zeitinterval $\Delta t$ ändert sich die Temperatur
einerseits durch Ausstrahlung, andererseits in Abhängigkeit von
der Coalbedo $c$. 
Die Coalbedo wiederum steigt umso schneller, je grösser die
Temperaturanomalie ist.
Dies führt auf die Systemgleichungen
\[
\begin{aligned}
T_k &= (1-\alpha) T_k  + \beta c_k \\
c_k &= c_{k-1} + \gamma T_k
\end{aligned}
\]
mit geeigneten Koeffizienten $\alpha$, $\beta$ und $\gamma$.
$\alpha$ beschreibt den Anteil der gespeicherten Wärme-Energie, der durch
Ausstrahlung verlorgen geht und daher die Temperatur absenkt.
$\beta$ bestimmt, wie stark die Coalbedo die Temperatur beeinflussen kann,
modelliert also die Einstrahlung.
Schliesslich berechnet $\gamma$, wie eine Temperturerhöhung sich auf die
Coalbedo auswirkt.

Schreiben wir $x_k=(T_k,c_k)^t$, können wir die Systementwicklung als
\[
x_{k+1}
=
\begin{pmatrix}
T_{k+1}\\c_{k+1}
\end{pmatrix}
=
\underbrace{
\begin{pmatrix}
(1-\alpha)&\beta\\
\gamma& 1
\end{pmatrix}}_{\displaystyle\varphi_k}
\begin{pmatrix}
T_k\\c_k
\end{pmatrix}
\]
ausdrücken.
Wir nehmen an, dass wir nur die Temperaturanomalie $T_k$ messen können,
die zugehörige Messmatrix ist
\[
H_k=\begin{pmatrix}1&0\end{pmatrix}
\qquad\Rightarrow\qquad
z_k = H_k\begin{pmatrix} T_k\\c_k\end{pmatrix}
=
T_k.
\]
Zur Vervollständigung des Modells müssen jetzt nur noch die Mess- und
Systemfehler ermittelt werden.
Wenn die Temperaturanomalie mit einer Genauigkeit von $0.1\,\text{K}$
ermittelt werden kann, muss $\varrho_1^2=0.01\,\text{K}^2$ verwendet
werden.
Die Systemfehlerkovarianzmatrix $Q$ ist dagegen weniger leicht zu ermitteln.
In der Praxis wird in dieser Situation häufig experimentell vorgegangen,
wie dies auch in Kapitel~\ref{chapter:kalman} vorgeführt wird.






