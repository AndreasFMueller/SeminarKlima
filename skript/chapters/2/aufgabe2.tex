Betrachten Sie die Differentialgleichung
\[
y'' - xy = 0.
\]
Finden Sie eine Lösung mit Anfangswerten $y(x)=1$ und $y'(x)=0$.

\begin{loesung}
Wir setzen den Potenzreihenansatz
\[
y(x) = a_0 + a_1x + a_2x^2 + a_3x^3+\dots
\]
in die Differentialgleichung ein, und erhalten
\begin{align*}
y''(x)
&=
2\cdot 1\, a_2 + 3\cdot 2\, a_3x + 4\cdot 3\, a_4x^2 + 5\cdot 4\,a_5x^3+\dots
\\
xy(x)
&=
a_0x + a_1x^2 + a_2x^3+a_3x^4+\dots
\end{align*}
Koeffizientenvergleich ergibt
\begin{align*}
2\cdot 1\,a_2 &= 0    &a_2&=0\\
3\cdot 2\,a_3 &= a_0  &a_3&=\frac{a_0}{3\cdot 2}\\
4\cdot 3\,a_4 &= a_1  &a_4&=\frac{a_1}{4\cdot 3}\\
5\cdot 4\,a_5 &= a_2  &a_5&=0\\
6\cdot 5\,a_6 &= a_3  &a_6&=\frac{a_3}{6\cdot 5}=\frac{a_0}{6\cdot 5\cdot 3\cdot 2}\\
7\cdot 6\,a_7 &= a_4  &a_7&=\frac{a_4}{7\cdot 6}=\frac{a_1}{7\cdot 6\cdot 4\cdot 3}\\
8\cdot 7\,a_8 &= a_5  &a_8&=0\\
              &\qquad\vdots
\end{align*}
Die Anfangsbedingungen ergeben $y(0)=a_0=1$ und $y'(0)=a_1=0$.
Damit ist die Lösung
\begin{align*}
y(x)
&=
1
+ \frac{x^3}{3\cdot 2}
+ \frac{x^6}{6\cdot 5\cdot 3\cdot 2}
+ \frac{x^9}{9\cdot 8\cdot 6\cdot 5\cdot 3\cdot 2}
+ \dots
\end{align*}
Die Nenner sehen aus wie Fakultäten, nur dass jeder dritte Faktor fehlt.
Erweitert man die Brüche so, dass im Nenner eine Fakultät steht, bekommt
$y(x)$ die Form
\begin{align*}
y(x)
&=
1
+ \frac{1}{3!}x^3
+ \frac{1\cdot 4}{6!}x^6
+ \frac{1\cdot 4\cdot 7}{9!}x^9
+ \frac{1\cdot 4\cdot 7\cdot 10}{12!}x^{12}
+ \dots
\\
&=
1 + \sum_{k=1}^\infty \frac{1\cdot 4\cdot 7\cdots (3k-2)}{(3k)!} x^{3k}.
\qedhere
\end{align*}
\end{loesung}

Die gefundene Lösung kann durch Airy-Funktionen ausgedrückt werden,
wie in \cite{skript:airy} gezeigt wird.

