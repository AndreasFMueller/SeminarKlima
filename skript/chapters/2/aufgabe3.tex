Finden Sie eine gerade Lösung der Differentialgleichung 
\begin{equation}
y''(x)\cdot\cos x=y(x)
\label{aufgabe3-dgl}
\end{equation}
in Form einer Reihe
\begin{equation}
y(x)
=
a_0 + a_1\cos x + a_2 \cos 2x + a_3 \cos3x+\dots
\label{aufgabe3-ansatz}
\end{equation}
von Kosinus-Funktionen.
Berechnen Sie die Koeffizienten bis $a_5$.

\begin{hinweis}
Die Lösung ist proportional zu $a_0$, Sie dürfen daher annehmen, dass
$a_0=1$.
Verwenden Sie ausserdem die Formeln
\[
\cos\alpha \cdot \cos\beta
=
\frac12\bigl(\cos(\alpha-\beta) + \cos(\alpha+\beta)\bigr)
\]
und Koeffizientenvergleich.
\end{hinweis}

\begin{loesung}
Wir berechnen die zweite Ableitung des Ansatzes
\begin{align*}
y''(x)
&=
-a_1\cos x - 2^2 a_2\cos 2x -3^2a_3\cos 3x-4^2a_4\cos4x-\dots
\\
y''(x)\cdot\cos x
&=
-a_1\cos x\cdot\cos x
-2^2a_2\cos2x\cdot\cos x
-3^2a_3\cos3x\cdot\cos x
-4^2a_4\cos4x\cdot\cos x
-\dots
\end{align*}
Die Produkte von Kosinus-Funktionen können mit der Formel
\[
\cos x \cdot \cos kx
=
\frac12\bigl(\cos (k-1)x + \cos(k+1)x\bigr)
\]
vereinfacht werden.
\begin{align*}
y''(x)\cdot\cos x
&=
-a_1\frac12(1+\cos2x)
-2^2a_2\frac12(\cos x + \cos3x)
-3^2a_3\frac12(\cos2x + \cos4x)
\\
&\qquad
-4^2a_4\frac12(\cos3x + \cos5x)
-\dots
\\
&=
-\frac{a_1}2
-2^2a_2\frac12
\cos x
-\frac12(a_1+3^2a_3)
\cos2x
-\frac12(2^2a_2+4^2a_4)
\cos3x
\\
&\qquad
-\frac12(a_4+5^2a_5)
\cos4x
+\dots
\end{align*}
Durch Koeffizientenvergleich erhalten wir
\begin{align*}
a_0&=-\frac12 a_1         &&\Rightarrow& a_1 &=-2a_0
\\
a_1&=-2^2\frac12a_2       &&\Rightarrow& a_2 &=-\frac{2}{2^2}a_1
\\
a_2&=-\frac12(a_1+3^2a_3) &&\Rightarrow& a_3 &= -\frac{1}{3^2}(2a_2+a_1)
\\
a_3&=-\frac12(2^2a_2+4^2a_4) &&\Rightarrow& a_4 &= -\frac{1}{4^2}(2a_3+2^2a_2)
\\
a_4&=-\frac12(3^2a_3+5^2a_5) &&\Rightarrow& a_5 &= -\frac{1}{5^2}(2a_4+3^2a_3)
\intertext{oder allgemein}
a_k&=-\frac12\bigl((k-1)^2a_{k-1} + (k+1)^2a_{k+1}\bigr)
&&\Rightarrow&
a_{k+1}&=-\frac{1}{(k+1)^2}\bigl(2a_k+(k-1)^2a_{k-1}\bigr).
\end{align*}
Offenbar sind die Koeffizienten $a_k$ mit $k>0$ proportional zu $a_0$,
die Lösungsfunktion ist daher ein Vielfaches der Funktion, welche man
erhält, wenn man $a_0=1$ setzt.

Etwas formaler können wir diese Rekursionsformel auch aus der
Summendarstellung des Ansatzes herleiten:
\begin{align*}
y(x)
&=
\sum_{j=0}^\infty a_j \cos jx
\\
y''(x)\cdot \cos x
&=
-
\sum_{k=0}^\infty a_k k^2\cos kx \cos x
\\
&=
-\frac12\sum_{k=0}^\infty a_k k^2\bigl(\cos(k+1)x + \cos(k-1)x\bigr)
\\
&=
-\frac12 \sum_{j=1}^\infty (j-1)^2a_{j-1} \cos jx
-\frac12 \sum_{j=-1}^\infty (j+1)^2a_{j+1} \cos jx
\end{align*}
Durch Koeffizientenvergleich finden wir
\begin{align*}
j&=0:
&
a_0
&=
-\frac12 a_1
&
a_1&=-2a_0=-2
\\
j&=1:
&
a_1
&=
-\frac122^2a_2
&
a_2
&=
-\frac12a_1
=
1
\\
j&>1:
&
a_j
&=
-\frac12
\bigl(
(j+1)^2a_{j+1} + (j-1)^2 a_{j-1}
\bigr).
\end{align*}
Für $j>1$ lesen wir ab
\begin{align*}
a_{j+1}
&=
-\frac{1}{(j+1)^2} \bigl(
2a_j+(j-1)^2a_{j-1}
\bigr).
\end{align*}
Damit ist der Koeffizient $a_3$ zum Beispiel:
\begin{align*}
a_3
&= 
-\frac1{3^2}(2a_2 + 2^2 a_1)
=
-\frac1{9}(2\cdot(2\cdot 1-1^2\cdot 2)=0.
\end{align*}
Für die Koeffizienten $a_k$ mit grösseren $k$ verwenden wir die 
Rekursionsformel
\[
a_k=-\frac{1}{k^2}(2a_{k-1}+(k-2)^2a_{k-2})
\]
und erhalten die Koeffizienten in Tabelle~\ref{aufgabe3-tabelle}.
Im Speziellen ist
\begin{align*}
a_4&=-\frac1{16}(2\cdot a_3 + 2^2 a_2)=-\frac14
\\
a_5
&=
-\frac1{25}\bigl(2\cdot a_4 + 3^2 a_3\bigr)
=
\frac1{25}\biggl(-2\cdot\frac14\biggr)
=
-\frac1{50}
\end{align*}
\begin{table}
\centering
\begin{tabular}{>{$}l<{$}|>{$}r<{$}}
k&a_k\\
\hline
 0&   1\phantom{.000000000000000}\\
 1&  -2\phantom{.000000000000000}\\
 2&   1\phantom{.000000000000000}\\
 3&   0\phantom{.000000000000000}\\
 4&  -0.25\phantom{0000000000000}\\
 5&   0.02\phantom{0000000000000}\\
 6&   0.11\phantom{0000000000000}\\
 7&  -0.014693877551020\\
 8&  -0.061415816326531\\
 9&   0.010405328798186\\
10&   0.039098015873016\\
11&  -0.007611798879331\\
12&  -0.027045680482937\\
13&   0.005769935061331\\
14&   0.019811418976634\\
15&  -0.004509963836970\\
16&  -0.015132883561509\\
17&   0.003615943357928\\
18&   0.011934525632810\\
19&  -0.002960877234645\\
\end{tabular}
\caption{Tabelle der Koeffizienten für die
Lösungsfunktion~\eqref{aufgabe3-ansatz} der Differentialgleichung
\eqref{aufgabe3-dgl}.
\label{aufgabe3-tabelle}}
\end{table}%
\begin{figure}
\centering
\begin{tikzpicture}[>=latex,thick]
\draw[->] (-6.6,0)--(6.6,0) coordinate[label=$x$];
\draw[->] (0,-1.3)--(0,4.5) coordinate[label={right:$y$}];
\draw (-0.1,1)--(0.1,1); \node at (0,1) [left] {$1$};
\draw (-0.1,2)--(0.1,2); \node at (0,2) [left] {$2$};
\draw (-0.1,3)--(0.1,3); \node at (0,3) [left] {$3$};
\draw (-0.1,4)--(0.1,4); \node at (0,4) [left] {$4$};
%\draw (-0.1,5)--(0.1,5); \node at (0,5) [left] {$5$};
\draw (-0.1,-1)--(0.1,-1); \node at (0,-1) [left] {$-1$};
\input{chapters/2/a3.path}

\draw (-3.1415,-0.1)--(-3.1415,0.1);
\draw (3.1415,-0.1)--(3.1415,0.1);
\draw (-6.2830,-0.1)--(-6.2830,0.1);
\draw (6.2830,-0.1)--(6.2830,0.1);
\draw (1.5708,-0.1)--(1.5708,0.1);
\draw (-1.5708,-0.1)--(-1.5708,0.1);
\draw (4.7124,-0.1)--(4.7124,0.1);
\draw (-4.7124,-0.1)--(-4.7124,0.1);
\node at (-3.1415,0) [below] {$-\pi$};
\node at (3.1415,0) [below] {$\pi$};
\node at (6.2830,0) [below] {$2\pi$};
\node at (-6.2830,0) [below] {$-2\pi$};
\end{tikzpicture}
\caption{Graphische Darstellung der Lösungsfunktion der Differentialgleichung
\eqref{aufgabe3-dgl}.}
\end{figure}%
In Tabelle~\ref{aufgabe3-tabelle} kann man erkennen, dass die Koeffizienten
nur sehr langsam abnehmen.
Da die Lösungsfunktion an den Stellen $(2k+1)\pi/2$ mit $k\in\mathbb Z$
nicht differenzierbar ist, ist dieses Verhalten sogar zu erwarten.
Genauer, man erwartet, dass die Koeffizienten $a_k$ wie $1/k$ abnehmen.
Und wie kann man sich überhaupt versichern, dass die Fourierreihe
an anderen Stellen konvergiert?

Man kann aber auch ablesen, dass die Elemente $a_{k+1}$ und $a_{k-1}$ 
entgegengesetztes Vorzeichen haben und betragsmässig sehr nahe beeinander
sind.
Um das besser zu verstehen berechnen wir die Summe $a_{k+1}-a_{k-1}$
und erhalten
\begin{align*}
a_{k+1}+a_{k-1}
&=
-\frac{1}{(k+1)^2}(2a_k + (k-1)^2 a_{k-1}) + a_{k-1}
\\
&=
-\frac{1}{(k+1)^2}(2a_k + (k-1)^2 a_{k-1}+) - (k+1)^2a_{k-1})
\\
&=
-\frac{1}{(k+1)^2}(2a_k + ((k-1)^2-(k+1)^2) a_{k-1}+))
\\
&=
-\frac{1}{(k+1)^2}(2a_k + 4ka_{k-1})).
\end{align*}
Daraus liest man ab, dass tatsächlich die Summe $a_{k+1}+a_{k-1}$
wie $1/k^2$ abnimmt.
Man kann daraus auch eine Abschätzung für $y(0)$ bekommen, nämlich
\begin{align*}
y(0)
&=
\sum_{j=0}^\infty a_k 
\\
&=
a_0 + a_1 + a_2 + a_3 + a_4+ a_5 + a_6 + a_7 + a_8 + \dots
\\
&=
a_0 + a_1 + (a_2 + a_4) + (a_3 + a_5) + (a_6 + a_8) + (a_7 + a_9)+\dots
\\
|y(0)|
&=
\biggl|
a_0 + a_1
+
\sum_{k=1}^\infty (a_{2k} + a_{2k+2})
+
\sum_{k=1}^\infty (a_{2k+1} + a_{2k+3})
\biggr|
\\
&\le
|a_0| + |a_1|
+
\sum_{k=1}^\infty |a_{2k}+a_{2k+2}|
+
\sum_{k=1}^\infty |a_{2k+1}+a_{2k+3}|
\end{align*}
Da in beiden Reihen die Terme wie $1/k^2$ gegen $0$ gehen, konvergieren
sie.
\end{loesung}
