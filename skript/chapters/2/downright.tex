%
% downright.tex
%
% (c) 2018 Prof Dr Andreas Müller, Hochschule Rapperswil
%
\documentclass[tikz]{standalone}
\usepackage{times}
\usepackage{txfonts}
\usepackage[utf8]{inputenc}
\usepackage{graphics}
\usepackage{ifthen}
\usepackage{color}
\usetikzlibrary{arrows,intersections}
\usetikzlibrary{math}
\begin{document}

\def\feld#1{
\draw[->,color=red] (0,1.2000)--({#1*0.0000},{1.2000+#1*-1.0000});
\fill[color=red] (0,1.2000) circle[radius=0.01];
\draw[->,color=red] (0,1.1000)--({#1*-0.1094},{1.1000+#1*-1.0020});
\fill[color=red] (0,1.1000) circle[radius=0.01];
\draw[->,color=red] (0,1.0000)--({#1*-0.2141},{1.0000+#1*-0.9986});
\fill[color=red] (0,1.0000) circle[radius=0.01];
\draw[->,color=red] (0,0.9000)--({#1*-0.3138},{0.9000+#1*-0.9907});
\fill[color=red] (0,0.9000) circle[radius=0.01];
\draw[->,color=red] (0,0.8000)--({#1*-0.4081},{0.8000+#1*-0.9793});
\fill[color=red] (0,0.8000) circle[radius=0.01];
\draw[->,color=red] (0,0.7000)--({#1*-0.4963},{0.7000+#1*-0.9654});
\fill[color=red] (0,0.7000) circle[radius=0.01];
\draw[->,color=red] (0,0.6000)--({#1*-0.5774},{0.6000+#1*-0.9503});
\fill[color=red] (0,0.6000) circle[radius=0.01];
\draw[->,color=red] (0,0.5000)--({#1*-0.6500},{0.5000+#1*-0.9352});
\fill[color=red] (0,0.5000) circle[radius=0.01];
\draw[->,color=red] (0,0.4000)--({#1*-0.7122},{0.4000+#1*-0.9214});
\fill[color=red] (0,0.4000) circle[radius=0.01];
\draw[->,color=red] (0,0.3000)--({#1*-0.7623},{0.3000+#1*-0.9103});
\fill[color=red] (0,0.3000) circle[radius=0.01];
}


\begin{tikzpicture}[thick, >= latex, scale=4]

\foreach \xs in {0,3,...,48}{
\begin{scope}[xshift=\xs]
\feld{0.1}
\end{scope}
}

\draw[line width=0.2pt] (-0.07,0.2)--(1.75,0.2);
\draw[line width=0.2pt] (-0.07,1.2)--(1.75,1.2);
\draw (-0.07,0.2)--(-0.03,0.2);
\draw (-0.07,1.2)--(-0.03,1.2);
\node at (-0.07,0.2) [left] {$0.2$};
\node at (-0.07,1.2) [left] {$1.2$};
\draw[->] (-0.07,0)--(1.8,0) coordinate[label={$x$}];
\draw[->] (-0.05,-0.02)--(-0.05,1.4) coordinate[label={right:$y$}];

\end{tikzpicture}
\end{document}

