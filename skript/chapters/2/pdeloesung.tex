%
% pdeloesung.tex -- Loesungsprinzip für partielle Differentialgleichungen
%
% (c) 2018 Prof Dr Andreas Müller, Hochschule Rapperswil
%
\section{Lösungen von partiellen Differentialgleichungen}
In diesem Kapitel haben wir Strömungen mit Hilfe partieller
Differentialgleichungen beschrieben.
Funktionenräume sind unendlich dimensional, was sehr erschwert,
dafür überhaupt eine Lösung zu finden.

\subsection{Diskretisation}
Der einfachste Ansatz, partielle Differntialgleichungen zu lösen,
folgt den Verfahren, die auch für gewöhnliche Differentialgleichungen
\cite{skript:mathsem-dgl} 

\subsection{Basisfunktionen}
Lineare Differentialgleichungen haben die Eigenschaft, dass 
mit zwei Lösungen auch deren Linearkombinationen wieder Lösungen sind.
In der Theorie der gewöhnlichen linearen Differentialgleichungen endlicher
Ordnung geben die Koeffizienten der Linearkombination die nötige Flexibilität,
die Anfangsbedingungen zu erfüllen.
Dasselbe gilt auch für partielle Differentialgleichungen, mit dem Unterschied
allerdings, dass es meinstens unendlich viele linear unabhängige Lösungen
gibt.
Ist $L$ ein linearer Differentialoperator und $u_k$, $k\in\mathbb N$ eine
Familie von Lösungen der Gleichung $Lu=0$.
Im besten Fall lässt sich dann jede andere Lösung $u$ in einem noch
zu definierenden Sinn beliebig genau durch Linearkombinationen
\[
u=\sum_{k\in\mathbb N} a_ku_k
\]
approximiert werden.
Wir verzichten darauf, auf die analytischen Details einzugehen.

Bei einer nichtlinearen Differentialgleichung ist es sicher nicht
mehr möglich, Lösungen durch Linearkombination anderer Lösungen
zu bilden.
Aber es spricht nichts dagegen, dass es eine Familie $u_k$, $k=1,\dots,N$,
von Funktionen gibt, mit der man Lösungen genügend genau approximieren
kann.

\subsubsection{Die Potenzreihenmethode}
Diese Idee liegt zum Beispiel der Potenzreihenmethode zu Grunde.
Dabei nimmt man an, dass die Lösung einer Differentialgleichung als
Linearkombination der Potenzfunktionen $u_k(x)=x^k, k=0\dots,N$
approximiert werden kann.
Als Beispiel betrachten wir die Differentialgleichung
\[
y''=-\lambda y.
\]
Eine Potenzfunktion $u_k(x)=x^k$ ist offensichtlich keine Lösung.
Man kann aber die Lösung als eine Linearkombination
der Potenzfunktionen
\[
y(x)
=
a_0+a_1x+a_2x^2+a_3x^3+\dots
\]
ansetzen und in die Differentialgleichung einsetzen.
Man erhält
\[
2\cdot 1 \cdot a_2
+
3\cdot 2 \cdot a_3x
+
4\cdot 3 \cdot a_4x^2
+
\dots
=
-\lambda(
a_0+a_1x+a_2x^2+a_3x^3+\dots)
\]
Man liest daraus die Gleichungen für die Koeffizienten $a_k$ ab:
\begin{equation*}
\left.
\begin{aligned}
2\cdot 1\cdot a_2&=-\lambda a_0\\
3\cdot 2\cdot a_3&=-\lambda a_1\\
4\cdot 3\cdot a_4&=-\lambda a_2\\
5\cdot 4\cdot a_5&=-\lambda a_3\\
&\;\;\vdots
\end{aligned}
\right\}
\qquad\Rightarrow\qquad
\left\{
\begin{aligned}
a_{2k}  &=(-1)^k\frac{\lambda^k}{ 2k!   }a_0\\
a_{2k+1}&=(-1)^k\frac{\lambda^k}{(2k+1)!}a_1
\end{aligned}
\right.
\end{equation*}
Daraus kann man erkennen, dass jede Lösung die Form
\begin{align*}
u(x)
&=
a_0
\biggl(
1-\frac{(\sqrt{\lambda}x)^2}{2!} + \frac{(\sqrt{\lambda}x)^4}{4!}-\dots
\biggr)
+
\frac{a_1}{\sqrt{\lambda}}
\biggl(
\sqrt{\lambda}x
-
\frac{(\sqrt{\lambda}x)^3}{3!} + \frac{(\sqrt{\lambda}x)^5}{5!}-\dots
\biggr)
\\
&= a_0 \cos \sqrt{\lambda}x
+ \frac{a_1}{\sqrt{\lambda}} \sin\sqrt{\lambda}x
\end{align*}
hat.

\subsubsection{Erfolgsfaktoren}
Aus dem eben entwickelten Beispiel kann man einige heuristische Regeln ableiten,
wie die Funktionenfamilie $u_k$ beschaffen sein muss, damit es
durchführbar ist.
\begin{enumerate}
\item Die Ableitungen $u_k'(x)$ der Funktionen können durch
Linearkombinationen derselben ausgedrückt werden.
Im Beispiel ist $u'_k(x)=kx^{k-1}=ku_{k-1}(x)$.
\item Produkte von Funktionen $u_k(x)$ lassen sich durch Linearkombinationen
approximieren.
Im Beispiel ist $u_k(x)u_l(x)=x^kx^k = x^{k+l}=u_{k+l}(x)$.
\item Beim Einsetzen des Ansatzes in die Differentialgleichungen
entstehen algebraische Gleichungen, aus denen die Koeffizienten
bestimmt werden können.
Im Beispiel 
\end{enumerate}

\subsubsection{Alternative Lösung für das Beispiel}

\subsubsection{Ein etwas komplexeres Beispiel}
Wir versuchen, diese Idee auf die Lösung der Gleichung von
Burgers
\[
\frac{\partial u}{\partial t} + u\frac{\partial u}{\partial x}=0
\]
anzuwenden.
Wir suchen $2\pi$-periodische Lösungen für ebensolche Anfangsbedingung
$u(0,x)=g(x)$.
Wie im Kapitel~\ref{chapter:fourier} dargelegt wird, sind die Funktionen
$\cos kx$ und $\sin kx$ geeignete Basisfunktionen für das skizzierte Verfahren.
Zur Vereinfachung der Rechnung verwenden wir statt der reellen Funktion
$\cos kx$ und $\sin kx$ die komplexen Exponentialfunktionen $e^{ikx}$.
Als Ansatz verwenden wir daher
\[
u(x) = \sum_{k\in\mathbb Z} c_k(t) e^{ikx}
\]
und setzen dies in die Differentialgleichung ein, die dadurch zu
\[
\sum_{k\in\mathbb Z} \dot c_k(t) e^{ikx}
=
\sum_{j,l\in\mathbb Z} c_j(t)e^{ijx} c_l(t) le^{ilx}
=
\sum_{j,l\in\mathbb Z} c_j(t)c_l(t) le^{i(j+l)x}
\]
Daraus lesen wir die Gleichungen
\begin{equation}
\dot c_k(t) = \sum_{l} lc_l(t)c_{k-l}(t)
\label{burgers:gewoedgl}
\end{equation}
für die Koeffizienten ab.
Wir haben also ein System von gewöhnlichen linearen Differentialgleichungen
erster Ordnung gefunden, und damit das partielle Differentialgleichungssystem
auf ein einfachers System reduziert.
Allerdings ist von der rechten Seite der Differentialgleichung
\eqref{burgers:gewoedgl}
nicht einmal garantiert, dass diese Summen konvergieren.

Für eine approximative Lösung vernachlässigen wir die Koeffizienten $c_k$
mit $|k|>N$.


\subsection{Separation}
\index{Separation}%
\subsubsection{Motivation}
Besonders gut geeignet als Basisfunktionen sind Funktionen, die bereits
Lösungen einer vereinfachten Variante der partiellen Differentialgleichung
sind oder die die Randbedingungen leicht wiederzugeben erlauben.
Zum Beispiel kommt in den Strömungsdifferentialgleichungen häufig der
Laplace-Operator vor, also könnte es nützlich sein, als Basisfunktionen
eine Familie von Funktionen $u_n$ heranzuziehen, welche bereits Lösungen
der Gleichung $\Delta u_n$ sind. In Kapitel~\ref{chapter:lorenz2}) wird
gezeigt, wie man diese Idee auf das in
Abschnitt~\label{section:lorenz-modell} vorgestellte Lorenz-Modell
andwenden kann.
Wenn man dann die Lösung als Linearkombination $\sum a_nu_n$ dieser Funktionen
ansetzt, verschwinden alle Terme der Form $\Delta u_n$, was die gewöhnlichen
Differentialgleichungen für die Koeffizienten $a_n$ dramatisch vereinfachen
kann.
In diesem Abschnitt zeigen wir daher an einem Beispiel, wie solche Funktionen
gefunden werden können. 

\subsubsection{Die Separationsmethode an einem Beispiel}
Gegeben sei die Differentialgleichung
\[
\begin{aligned}
\Delta u &= \lambda u&&
\text{auf dem Gebiet $\Omega=\{ (x,y)\,|\, 0<x<a\text{ und }0<y<b\}$}
\\
\text{mit}
\qquad
u=&0&&\text{auf dem Rand $\partial\Omega$.}
\end{aligned}
\]
Das Gebiet ist ein Rechteck mit Seitenlängen $a$ und $b$.
Die Relation $\Delta u=\lambda u$ ermöglicht, jedes Vorkommen eines auf
die Funktion $u$ wirkenden Laplace-Operators durch die Multiplikation
mit $\lambda$ zu ersetzen, und damit die Differentialgleichung zu vereinfachen.

Die Separationsmethode nimmt jetzt an, dass sich die Lösung der
Differentialgleichung in der besonders einfachen Form eines Produktes
$u(x,y)=X(x)\cdot Y(y)$ finden lässt.
Setzt man diesen Separationsansatz
\index{Separationsansatz}
in die Differentialgleichung ein, erhält man
\[
\Delta u
=
\biggl(
\frac{\partial^2}{\partial x^2}
+
\frac{\partial^2}{\partial y^2}
\biggr)
X(x)\cdot Y(y)
=
X''(x)\cdot Y(y) + X(x)\cdot Y''(y)
=
\lambda X(x)\cdot Y(y).
\]
Wir interessieren uns für nicht verschwindende Lösungen, wir nehmen daher
an, dass die Funktionen $X(x)$ und $Y(y)$ nur isolierte Nullstellen haben.
Ausserhalb dieser Nullstellen ist es erlaubt, durch $X(x)\cdot Y(y)$ zu
divideren. 
Damit wird es möglich, die Variablen $x$ und $y$ zu trennen, man
erhält
\[
0
=
\frac{X''(x)\cdot Y(y) + X(x)\cdot Y''(y)}{X(x)\cdot Y(y)} - \lambda
=
\frac{X''(x)}{X(x)}
+
\frac{Y''(y)}{Y(y)}
-
\lambda
\qquad\Rightarrow\qquad
\frac{X''(x)}{X(x)}=-\frac{Y''(y)}{Y(y)}+\lambda.
\]
Die linke Seite der Gleichung hängt nur von $x$ ab, die rechte nur von $y$.
Ändert man $x$ während man $y$ konstant hält, dann kann sich die rechte
Seite nicht ändern, die linke muss daher auch konstant sein.
Wenn aber die linke Seite eine konstante ist, dann muss auch die rechte
eine Konstante sein. 
Es gibt also eine vorerst noch unbekannte Konstante $\mu$, so dass
die Funktionen $X(x)$ und $Y(y)$ die Differentialgleichungen
\begin{equation}
X''(x) = -\mu X(x)
\qquad\text{und}\qquad
Y''(y) = (\lambda+\mu)Y(y)
\label{skript:separation:gleichungen}
\end{equation}
erfüllen.
Mögliche Werte für $\mu$ werden sich daraus ergeben, welche Werte die
Differentialgleichungen überhaupt lösbar sind, oder für welche Werte
sich die gegebenen Randbedinungen erfüllen lassen.

Wir lösen die Differntialgleichungen~\eqref{skript:separation:gleichungen}
so, dass auch die Randbedingungen $u=0$ auf $\partial\Omega$ erfüllt sind.
Für die Funktionen $X(x)$ und $Y(y)$ bedeutet dies:
\[
\begin{aligned}
X(0)&=0 & Y(0) &= 0 \\
X(a)&=0 & Y(b) &= 0.
\end{aligned}
\]
Die linearen Differntialgleichungen~\eqref{skript:separation:gleichungen}
haben Exponentialfunktionen oder trigonometrische Funktionen als
Lösungen.
Da sich mit Exponentialfunktionen die Randbedingungen ohnehin nicht
einhalten lassen, suchen wir Lösungen als trigonometrische Funktionen.
Da $X(0)=0$ und $Y(0)=0$ gilt, müssen die Lösungen Sinus-Funktionen sein,
wir nehmen daher an, dass $X(x)=\sin \kappa x$ und $Y(y)=\sin \nu y$.
Für diese Funktionen gilt
\[
\begin{aligned}
X''(x) &= -\kappa^2\sin \kappa x = -\kappa^2 X(x)
&&\text{und}&
Y''(y) &= -\nu^2\sin \nu x = -\nu^2 Y(y).
\end{aligned}
\]
Die Randbedingungen am rechten Rand sind
\[
\begin{aligned}
X(a) = \sin \kappa a = 0
&
Y(b) = \sin \nu b = 0.
\end{aligned}
\]
Da die Sinus-Funktion ihre Nullstellen genau bei den ganzzahligen Vielfachen
von $\pi$ hat, schliessen wir, dass $\kappa a = k\pi$ und $\nu b = l\pi$
mit $k,l\in\mathbb Z$ gilt.
Die Lösungen sind daher von der Form
\[
\begin{aligned}
X(x) &= \sin \frac{xk\pi}{a}
\qquad \text{und} \qquad
Y(y) &= \sin\frac{yl\pi}{b}
\end{aligned}
\]
Und damit für die Familie der Lösungesfunktionen
\[
u_{kl}(x,y) = \sin \frac{xk\pi}{a} \sin\frac{yl\pi}{b}.
\]
Wir berechnen auch noch die Werte von $\mu$ und $\lambda$:
\begin{align*}
X''(x)
&=
-\mu X(x) = -\frac{k^2\pi^2}{a^2}
\\
\Delta u_{kl}
&=
-\frac{k^2\pi^2}{a^2} u_{kl}
-\frac{l^2\pi^2}{b^2} u_{kl}
=
-\biggl(\frac{k^2}{a^2}+\frac{l^2}{b^2}\biggr)\pi^2 u_{kl}
=
\lambda  u_{kl}
&&\Rightarrow&
\lambda_{kl} = - \biggl(\frac{k^2}{a^2}+\frac{l^2}{b^2}\biggr)\pi^2.
\end{align*}
Nur ein diskrete Menge von Zahlenwerten $\mu$ sind also möglich, und nur 
für eine diskrete Menge von Eigenwerten $\lambda$ ist die ursprüngliche 
Differentialgleichung überhaupt lösbar.

Damit haben wir eine Familie $u_{kl}$, $k,l\in\mathbb Z$ von Funktionen
gefunden, die Lösungen der Differentialgleichung
$\Delta u_{kl}=\lambda_{kl} u_{kl}$ sind.
Das Anfangs gestellt Ziel ist erreicht.

\subsubsection{Verallgemeinerungen}
Das Beispiel deutet auch an, wie die Methode verallgemeinert werden kann.
\begin{enumerate}
\item
Wenn möglich, wähle ein Koordinatensystem, mit welchem das Definitionsgebiet
einfach beschrieben werden kann.
Formuliere die Differentialgleichung in diesen Koordinaten.
Darstellungen des Laplace-Operator für fast alle wichtigen Koordinatensysteme
können zum Beispiel in guten Formelsammlungen gefunden werden.
\item
Separationsansatz: setze die Lösung als Produkt von Funktionen an,
jeweils von nur einer Koordinate abhängen, also
$u(x,y)=X(x)\cdot Y(y)$ in einem $x$-$y$-Koordinatensystem.
\item
Setze den Ansatz in die Differentialgleichung ein und versuche die
Variablen zu trennen. 
Gesucht ist eine Gleichung, deren linke Seite nur von einer Variablen
abhängt, die rechte nur von allen anderen.
Wie im Beispiel folgt, dass beide Seiten konstant sein müssen.
Damit hat man das Problem darauf reduziert, eine gewöhnliche
Differentialgleichung für eine Funktion der einen Variable zu lösen.
Integrationskonstation, die in der Lösung der gewöhnlichen
Differentialgleichung immer auftreten wird, können oft mit Hilfe
eines Teils der Randbedingungen eliminiert werden.
\item
Durch Iteration von Schritt~3 können Teillösungen $u_n$ gefunden werden,
die von den Konstanten abhängen, die bei der Separation eingefügt werden
müssen.
\end{enumerate}

