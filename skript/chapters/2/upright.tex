%
% upright.tex
%
% (c) 2018 Prof Dr Andreas Müller, Hochschule Rapperswil
%
\documentclass[tikz]{standalone}
\usepackage{times}
\usepackage{txfonts}
\usepackage[utf8]{inputenc}
\usepackage{graphics}
\usepackage{ifthen}
\usepackage{color}
\usetikzlibrary{arrows,intersections}
\usetikzlibrary{math}
\begin{document}

\def\feld#1{
\draw[->,color=red] (0,0.2000)--({#1*0.0000},{0.2000+#1*1.0000});
\fill[color=red] (0,0.2000) circle[radius=0.01];
\draw[->,color=red] (0,0.3000)--({#1*0.0298},{0.3000+#1*0.9948});
\fill[color=red] (0,0.3000) circle[radius=0.01];
\draw[->,color=red] (0,0.4000)--({#1*0.0712},{0.4000+#1*0.9887});
\fill[color=red] (0,0.4000) circle[radius=0.01];
\draw[->,color=red] (0,0.5000)--({#1*0.1241},{0.5000+#1*0.9810});
\fill[color=red] (0,0.5000) circle[radius=0.01];
\draw[->,color=red] (0,0.6000)--({#1*0.1885},{0.6000+#1*0.9709});
\fill[color=red] (0,0.6000) circle[radius=0.01];
\draw[->,color=red] (0,0.7000)--({#1*0.2643},{0.7000+#1*0.9572});
\fill[color=red] (0,0.7000) circle[radius=0.01];
\draw[->,color=red] (0,0.8000)--({#1*0.3521},{0.8000+#1*0.9386});
\fill[color=red] (0,0.8000) circle[radius=0.01];
\draw[->,color=red] (0,0.9000)--({#1*0.4531},{0.9000+#1*0.9136});
\fill[color=red] (0,0.9000) circle[radius=0.01];
\draw[->,color=red] (0,1.0000)--({#1*0.5698},{1.0000+#1*0.8803});
\fill[color=red] (0,1.0000) circle[radius=0.01];
\draw[->,color=red] (0,1.1000)--({#1*0.7070},{1.1000+#1*0.8359});
\fill[color=red] (0,1.1000) circle[radius=0.01];
}


\begin{tikzpicture}[thick, >= latex, scale=4]

\foreach \xs in {0,3,...,48}{
\begin{scope}[xshift=\xs]
\feld{0.1}
\end{scope}
}

\draw[line width=0.2pt] (-0.07,0.2)--(1.75,0.2);
\draw[line width=0.2pt] (-0.07,1.2)--(1.75,1.2);
\draw (-0.07,0.2)--(-0.03,0.2);
\draw (-0.07,1.2)--(-0.03,1.2);
\node at (-0.07,0.2) [left] {$0.2$};
\node at (-0.07,1.2) [left] {$1.2$};
\draw[->] (-0.07,0)--(1.8,0) coordinate[label={$x$}];
\draw[->] (-0.05,-0.02)--(-0.05,1.4) coordinate[label={right:$y$}];

\end{tikzpicture}
\end{document}

