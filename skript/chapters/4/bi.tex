%
% drei.tex
%
% (c) 2018 Prof Dr Andreas Müller, Hochschule Rapperswil
%
\documentclass[tikz]{standalone}
\usepackage{times}
\usepackage{amsmath}
\usepackage{txfonts}
\usepackage[utf8]{inputenc}
\usepackage{graphics}
\usetikzlibrary{arrows,intersections}
\usetikzlibrary{math}
\begin{document}
\begin{tikzpicture}[thick, >= latex, xscale = 20, yscale = 6]

\draw[domain=0:0.5,samples=100,color=red]
	plot ({0.25 - (\x-0.5)*(\x-0.5)},{\x});

\draw[domain=0.5:1,samples=100,color=blue]
	plot ({0.25 - (\x-0.5)*(\x-0.5)},{\x});

\draw[domain=1:1.3,samples=100,color=red]
	plot ({(\x-0.5)*(\x-0.5)-0.25},{\x});

\draw[->] (-0.03,0)--(0.4,0) coordinate[label={above:$\lambda$}];
\draw[->] (0,-0.03)--(0,1.35) coordinate[label={right:$x$}];

\draw (-0.003,1)--(0.003,1);
\node at (-0.003,1) [left] {$1$};

\draw (0.25,-0.03)--(0.25,0.03);
\node at (0.25,-0.03) [below] {$\frac14$};

\node at (0.25,0.5) [right] {$S$};
\fill[color=red] (0.25,0.5) circle[x radius={0.07/20},y radius={0.07/6}];
\draw[color=red,line width=0.5] (0.25,0.5)--(0.25,{0.5+sqrt(0.5)});
\node at (0.25,{0.5+sqrt(0.5)}) [below right] {$P$};
\fill[color=red] (0.25,{0.5+sqrt(0.5)}) circle[x radius={0.07/20}, y radius={0.07/6}];

\end{tikzpicture}
\end{document}

