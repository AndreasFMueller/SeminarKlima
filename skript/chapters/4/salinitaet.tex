%
% salinitaet.tex -- Salinität
%
% (c) 2018 Prof Dr Andreas Müller
%
\section{Salinität und Dichte}
\rhead{Salinität}
Der Salzgehalt des Meerwassers ist nicht konstant.
Er steigt an, wenn Wasser verdampf oder sich Eis bildet.
Er sinkt, wenn das Salz durch Niederschläge verdünnt wird.
Mit der Veränderung des Salzgehaltes geht auch eine Änderung
der Dichte einher.

Den genauen Zusammenhang zwischen Salinität, Temperatur und Dichte
kann nicht aus Naturgesetzen abgeleitet werden.
Verschiedene Untersuchungen haben empirische Formeln für die
Dichte in Abhängigkeit von Temperatur und Salinität zu Tage
gefördert.
Zum Beispiel  in der Form
\[
\varrho
=
\varrho_0(T)
+
A(T)\cdot S + B(T)\cdot S^{\frac32}+C\cdot S^2,
\]
wobei die Koeffizienten $A(T)$ und $B(T)$ Polynome der Temperatur $T$ in
$\mathstrut^\circ\text{C}$ sind:
\begin{align*}
A(T)
&=
 0.824493 - 0.0040899\,T + 0.000076438\,T^2 - 0.00000082467\,T^3 + 0.0000000053875\,T^4
\\
B(T)
&=
 -0.00572466 + 0.00010227\,T - 0.0000016546\,T^2
\\
C
&=
0.00048314
\\
\varrho_0(T)
&=
 0.824493 - 0.0040899\,T + 0.000076438\,T^2
 - 0.00000082467\,T^3 + 0.0000000053875\,T^4,
\end{align*}
die man etwa in
\cite{skript:millero}
findet.
Diese Formeln geben die Dichte über einen weiten Parameterbereich
mit einem relativen Fehler $<0.001$ wieder.
Für unsere qualitativen Überlegungen ist diese Genauigkeit
nicht nötig.

Im Folgenden betrachten wir die Dichte-Anomalie $\varrho-\varrho_0$
in Abhängigkeit von der Temperatur-Anomalie $T-T_0$ und der
Salinitäts-Anomalie $S-S_0$.
Man kann die Dichte-Anomalie immer als Taylor-Reihe
\[
\varrho 
=
\varrho_0
+
\frac{\partial \varrho}{\partial T}(T-T_0)
+
\frac{\partial \varrho}{\partial S}(S-S_0)
+
\text{Terme höherer Ordnung}.
\]
Solange $T-T_0$ und $S-S_0$ nicht allzu gross sind, kann man sich auf
die linearen Terme beschränken und
\begin{equation}
\varrho
=
\varrho_0(1-\alpha(T-T_0)+\beta(S-S_0))
\label{skript:salinity-linear}
\end{equation}
schreiben.
Darin sind $\alpha$ und $\beta$ positive Zahlen, die Vorzeichen in
\eqref{skript:salinity-linear} sind so gewählt, dass die Dichte
mit höherer Temperatur abnimmt und mit höherer Salinität zunimmt.
Typische Werte sind
\[
\alpha = 1.5\cdot 10^{-4}\frac{1}{\text{K}}
\qquad
\text{und}
\qquad
\beta = 8\cdot 10^{-4}\frac{1}{\text{psu}}.
\]

Mit diesem Modell für die Dichte könnten wir jetzt versuchen, die
Bewegungsgleichungen der Fluiddynamik und die Wärmeleitungsgleichung
zu lösen.
Wegen der komplizierten Form der Weltmeere ist das eine Aufgabe,
die ausschliesslich numerisch gelöst werden kann.
Ausserem benötigen wir detaillierte Informationen über den Wärmeaustsusch
mit der Atmosphäere oder dem Meeresboden.
Eine derart detaillierte Modellierung scheint daher aussichtslos.
Für eine qualitative Aussage über die Zirkulation benötigen wir daher
ein dramatisch vereinfachtes Modell.





