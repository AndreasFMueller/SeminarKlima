%
% vergleich.tex
%
% (c) 2018 Prof Dr Andreas Müller, Hochschule Rapperswil
%
\documentclass[tikz]{standalone}
\usepackage{times}
\usepackage{amsmath}
\usepackage{txfonts}
\usepackage[utf8]{inputenc}
\usepackage{graphics}
\usetikzlibrary{arrows,intersections}
\begin{document}
\begin{tikzpicture}[thick, >= latex, xscale=4, yscale=5]

\draw[->] (-0.03,0)--(3.1,0) coordinate[label={above:$\lambda$}];
\draw[->] (0,-0.02)--(0,1.5)
	coordinate[label={right:$E$}];
%	coordinate[label={right:$E\;[\mu\text{W}/\text{nm}\cdot\text{m}^2]$}];

\input{vplot.tex}
\erdflaeche
\sonnenflaeche
\erdkurven
\sonnenkurve

\foreach \x in {0,1,...,3}{
	\draw ({\x},-0.02)--({\x},0.02);
}
\node at (0,-0.01) [below] {$100$nm};
\node at (1,-0.01) [below] {$1\mu$m};
\node at (2,-0.01) [below] {$10\mu$m};
\node at (3,-0.01) [below] {$100\mu$m};

%\def\maximum{7.67}
%\draw[line width=0.1] (0,\maximum)--(3,\maximum);

\node at (0.8,1.45) {Sonne};
\node at (2.1,1.45) {Erde};
\node[color=red] at (0.82,0.6) {\Huge $+$};
\node[color=red] at (0.85,0.3) {Einstrahlung};
\node[color=blue] at (2.13,0.6) {\Huge $-$};
\node[color=blue] at (2.17,0.3) {Ausstrahlung};

\end{tikzpicture}
\end{document}

