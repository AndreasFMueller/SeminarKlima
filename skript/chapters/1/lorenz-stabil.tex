%
% lorenz-stabil.tex -- Stabilität der Atmosphäre
%
% (c) 2018 Prof Dr Andreas Müller, Hochschule Rapperswil
%
\documentclass[tikz]{standalone}
\usepackage{times}
\usepackage{txfonts}
\usepackage[utf8]{inputenc}
\usepackage{graphics}
\usepackage{ifthen}
\usepackage{color}
\usetikzlibrary{arrows,intersections}
\begin{document}

\def\paket{
\fill[color=blue!20] (-0.25,-0.25)--(0.25,-0.25)--(0.25,0.25)--(-0.25,0.25)--cycle;
\draw[color=blue] (-0.25,-0.25)--(0.25,-0.25)--(0.25,0.25)--(-0.25,0.25)--cycle;
}

\definecolor{darkgreen}{RGB}{0,128,0}

\newboolean{gradient}

\def\graphik{

\ifthenelse{\boolean{gradient}}{
\draw[color=red] (2.5,0)--(-2.5,5);
\draw[->,color=blue] (-0.55,3.5)--(-0.55,4.2);
\node at (-0.55,4.2) [left] {\color{blue}$F$};
\draw[->,color=blue] (1.55,0.5)--(1.55,-0.2);
\node at (1.55,-0.2) [right] {\color{blue} $F$};
}{
\draw[color=red] (1.5,0)--(-1.,5);
\draw[->,color=blue] (-0.55,3.5)--(-0.55,2.8);
\node at (-0.55,2.8) [left] {\color{blue}$F$};
\draw[->,color=blue] (1.55,0.5)--(1.55,1.2);
\node at (1.55,1.2) [right] {\color{blue} $F$};
}

\draw[->] (-3,0)--(3,0) coordinate[label={$T$}];
\draw[->] (-1,-0.1)--(-1,5.1) coordinate[label={left:$y$}];


% rectangle around (0.5,2)
\begin{scope}[xshift = 0.5cm, yshift = 2cm]
\paket
\end{scope}

\begin{scope}[xshift = -0.55cm, yshift = 3.5cm]
\paket
\end{scope}

\begin{scope}[xshift = 1.55cm, yshift = 0.5cm]
\paket
\end{scope}

\draw[->,line width=1.5pt,color=darkgreen] (0.5,2)--(-0.55,3.5);
\draw[->,line width=1.5pt,color=darkgreen] (0.5,2)--(1.55,0.5);
\fill[color=darkgreen] (0.5,2) circle[radius=0.1];
}

\begin{tikzpicture}[thick, >= latex]

\setboolean{gradient}{false}
\graphik
\setboolean{gradient}{true}
\begin{scope}[xshift = 7cm]
\graphik
\end{scope}

\end{tikzpicture}
\end{document}

