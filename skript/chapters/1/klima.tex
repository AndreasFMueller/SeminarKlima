%
% klima.tex -- Klima
%
% (c) 2018 Prof Dr Andreas Müller, Hochschule Rapperswil
%

\section{Klima\label{section:klima}}
\rhead{Klima}
In der Wikipedia kann man die folgenden Definitionen für die Begriffe Wetter
und Klima finden:

\begin{definition}
Als {\em Wetter} bezeichnet man den
spürbaren, kurzfristigen Zustand der Atmosphäre (auch: messbarer
Zustand der Troposphäre) an einem bestimmten Ort der Erdoberfläche,
der unter anderem als Sonnenschein, Bewölkung, Regen, Wind, Hitze
oder Kälte in Erscheinung tritt
\cite{skript:wetter}.
\end{definition}

\begin{definition}
Das {\em Klima} steht als Begriff für die Gesamtheit aller meteorologischen
Vorgänge, die für die über Zeiträume von mindestens 30 Jahren
regelmässig wiederkehrenden durchschnittlichen Zustände der Erdatmosphäre
an einem Ort verantwortlich sind
\cite{skript:klima}.
\end{definition}

Was also Donald Trump in seinem Tweet beschrieben hat ist das Wetter.
Selbst wenn die Temperatur in New York unter den Gefrierpunkt fällt, 
heisst das nicht, dass die mittlere Temperatur in New York über mehrere
Jahre nicht doch ansteigen kann.
Tatsächlich bedeutet ``globale Erwärmung'' nicht, dass die mittlere
Temperatur an jedem Punkt der Erde zunehmen wird.
Im Gegenteil ist es durchaus möglich, dass zwar die mittlere Temperatur
der Erde ständig zunimmt, wie wir in den letzten Jahren auch messtechnisch
nachweisen konnten, dass aber auch die Temperaturunterschiede stark zunehmen,
so dass es am Ende an einzelnen Stelle der Erdoberfläche zu einer 
Abkühlung kommen kann.
Um dieser Komplexität Rechnung zu tragen, spricht man nicht mehr von
der ``globalen Erwärmung'', sondern vom Klimawandel.

Auch wenn sich das Wetter nur sehr eingeschränkt vorhersagen lässt,
bedeutet das noch lange nicht, dass das Klima nicht doch sehr
genau vorhergesagt werden kann.
Eine Analogie kann den Unterschied zwischen der Vorhersagbarkeit
von Wetter und Klima verdeutlichen.
Wenn man in einem Kochtopf Wasser zum Kochen bringt, stellt sich
eine unvorhersagbare chaotische Bewegung kleiner und grosser
Gasblasen ein.
Es ist unmöglich vorherzusagen, wann und wo sich die nächste Blase
bilden wird und welchen Weg sie an die Oberfläche des Wasser nehmen
wird.
Wenn wir aber nur die mittlere Temperatur betrachten, können wir
aus der Heizleistung der Kochplatte, der Masse und der spezifischen
Wärmekapazität des Wassers genau berechnen, welche Temperatur zu welcher
Zeit im Wasser herschen wird und wir können den Zeitpunkt exakt
vorhersagen, wann das Wasser zu sieden beginnt.
Die mittlere Temperatur des Wassers beschreibt das ``Klima''
in der Pfanne, die kleinräumigen und kurzfristigen Blasen und anderen
Turbulenzen beschreiben das ``Wetter''.

