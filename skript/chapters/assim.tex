%
% assim.tex
%
% (c) 2018 Prof Dr Andreas Müller, Hochschule Rapperswil
%
\chapter{Datenassimilation}
\lhead{Datenassimilation}
In den vorangegangenen Kapiteln wurden Methoden vorgestellt
und Modelle entwickelt, mit denen sich die Klimaentwicklung auf der
Erde untersuchen lässt.
Diese Modelle enthalten eine Anzahl von Parametern, zum Beispiel
Naturkonstanten oder Eigenschaften des Planeten, die erst bestimmt
werden müssen, bevor das Modell verwendet werden kann.
Ausserdem werden Anfangs- oder Randbedingungen benötigt, um die
Differentialgleichungen überhaupt lösen zu können.

Gerade die Bestimmung der Anfangsbedingungen kann sehr schwierig sein.
Wie soll man den Zustand der gesamten Atmosphäre im Detail ermitteln.
Das Lorenz-Modell (Abschnitt~\ref{section:lorenz-modell}) zeigt, dass
ungenaue Anfangsbedingungen sehr schnell zu unbrauchbaren Vorhersagen
des Modells führen können.
Dies bedeutet natürlich auch, dass nur Modelle verwendet werden sollen,
die nicht empfindlich sind auf die Anfangsbedingungen und vernünftige
Aussagen ermöglichen, auch wenn die Anfangsbedingungen nicht sehr genau 
sind.
Aber auch ein solches Modell kann bei ungenauen Anfangsbedingungen 
nicht lang zuverlässige Vorhersagen machen.
Wir müssen also das Problem lösen, die unzulänglichen Anfangsdaten 
später zu korrigieren.
Dieses Problem wird vom Kalman-Filter dadurch gelöst, dass spätere
Messungen den aktuellen Zustand korrigieren.
Dieser Spezialfall wird in Abschnitt~\ref{section:kalman-filter}
vorgestellt.
Dies ist jedoch nicht ausrechend, denn in er Klimaforschung sind wir
manchmal mit der Situation konfrontiert, dass wir zusätzliche
Daten über frühere Zustände erhalten oder Informationen über Zustandsvariablen,
die früher nicht gemessen werden konnten.
Dieses erweiterte sogenannte Datenassimilations-Problem wird in
Abschnitt~\ref{section:assimilationsproblem} untersucht.

%
% kalmanfilter.tex
%
% (c) 2018 Prof Dr Andreas Müller, Hochschule Rapperswil
%
\section{Kalman-Filter\label{section:kalman-filter}}
\index{Kalman-Filter}%
\rhead{Kalman-Filter}
Die bisherige Sicht auf das Problem war, dass aus dem Modell und den
Anfangsbedingungen der zukünftige Zustand des Modells abgeleitet werden
kann.
Doch bei fehlerhaften Anfangsdaten wird auch der zukünftige Zustand
fehlerhaft sein.
Ein erster Versuch, die Situation wieder zurecht zu rücken ist, auch für
den späteren Zustand ein Messung durchzuführen und zu versuchen,
den Systemzustand derart zu korrigieren, dass auch die Vorhersagen des
Modells besser mit der Realität übereinstimmen.
Dies ist genau, was der Kalman-Filter ermöglicht.

In diesem Abschnitt soll eine kurze Übersicht über den Kalman-Filter
gegeben werden.
Sie stützt sich auf \cite{skript:catlin}, eine detailliertere Darstellung
findet sich in \cite[Kapitel 8]{skript:wrstatskript}.

\subsection{Messung und System\label{subsection:messung und system}}
Wir gehen davon aus, dass das System, zum Beispiel das Klimasystem der Erde,
mit einem Vektor $x\in\mathbb R^n$ von Zustandsvariablen beschrieben werden
kann.
Den detaillierten Zustand des Systems können wir nicht erfahren, die
Grössen $x$ sind also bestenfalls teilweise bekannt.
Die beschränkte Kenntnis des Systemzustand äussert sich darin, dass
nur eine Schätzung $\hat x$ für $x$ gefunden werden kann.
$\hat x$ stellt also das aktuelle Wissen über den Systemzustand dar.

Durch direkte Messung können möglicherweise einzelne Systemvariablen
bestimmt werden.
Häufiger können viele der Variablen aber nicht direkt gemessen werden.
Das Modell braucht vielleicht die Temperatur, aber gemessen werden kann nur
die Strahlungsintensität.
Die gemessenen Variablen $z\in\mathbb R^m$ sind also bestenfalls eine
Funktion $z=f(x)$ der Zustandsvariablen $x$.

Der Kalman-Filter soll ermöglichen, das Wissen über den System-Zustand
$\hat x$ durch Messung der Grössen $z$ zu verbessern, und so eine
verbesserte Schätzung zu erhalten.
Dies bedeutet, dass durch wiederholte Messung die Zustandsschätzung
laufend verbessert werden kann, so dass die Fehler der ursprünglich gewählten 
Anfangsbedingungen immer unbedeutender werden.
Dies bedeutet aber, dass die zeitliche Entwicklung des Systems mit einer Folge
$x_k$, $k\in\mathbb N$, von Zustandsvariablen modelliert werden muss.
Diese Zustände sind nur näherungsweise bekannt als eine Folge
von Schätzungen $\hat{x}_k$, $k\in\mathbb N$.
Mit jeder Messung $z_k$ erfahren wir etwas über den Systemzustand
zur Zeit $k$.
Die Messung $z_k$ und das bisherige Wissen $\hat{x}_{k-1}$ ergeben
die neueste beste Schätzung $\hat{x}_k$.
Das Ziel des Kalman-Filters ist, den Zusammenhang zwischen den
genannten Grössen in Form einer Funktion
\begin{equation}
\hat{x}_k = F(\hat{x}_{k-1},z_n)
\label{skript:filterbasis}
\end{equation}
zu finden.

Die implizite Annahme hinter \eqref{skript:filterbasis} ist,
dass es einen Zusammenhang zwischen $x_{k-1}$ und $x_k$ und damit auch
zwischen den Schätzungen $\hat{x}_{k-1}$ und $\hat{x}_k$ gibt.
Es ist daher denkbar, \eqref{skript:filterbasis} in zwei Teilschritte
aufzuteilen.
Das vorhandene Wissen $\hat{x}_{k-1}$ macht eine Vorhersage über den
Zustand $\hat{x}_{k|k-1}$, welches anschliessend mit den neuen 
Messwerten $z_k$ zur bestmöglichen Information $\hat{x}_k$ über den
aktuellen Zustand verbunden werden kann.
Dazu braucht es einerseits eine Beschreibung der Systementwicklung
$x_{k+1} = \varphi(x_k)$, aus der sich auch eine Entwicklung der
Schätzungen $\hat{x}_{k|k-1} = \Phi(\hat{x}_{k-1})$ ableiten lässt.
Andererseits brauchen wir eine Funktion
\begin{equation}
\hat{x}_k = K(\hat{x}_{k|k-1}, z_k)
\end{equation}
mit der der Zustand korrigiert werden kann.
Setzen wir dies wieder zusammen, folgt
\[
\hat{x}_k
=
F(\hat{x}_{k|k-1},z_k)
=
K(\Phi(\hat{x}_{k-1}),z_k).
\]

\subsection{Systemmodellierung\label{subsection:systemmodellierung}}
Die früher entwickelten Modelle zur Klimaentwicklung resultierten in
zum Teil sehr komplizierten Abhängigkeiten zwischen Systemvariablen
zu weit auseinander liegenden Zeitpunkten.
Sie hatten jedoch gemeinsam, dass sie durch eine Differentialgleichungen
erster Ordnung der Form
\[
\frac{dx}{dt} = f(x,t)
\]
beschrieben werden konnten.
Für genügend kurze Zeitschritte $\Delta t$ können die Zustandsvariablen
$x_k = x(k\Delta t)$ zu den Zeitpunkten $t_k=k\Delta t$
durch lineare Approximation
\[
x_k = x_{k-1} + f(x_{k-1}, t_{k-1}) \cdot \Delta t
\]
bestimmt werden.
Eine Schätzung $\hat{x}_{k-1}$ des Zustandes zur Zeit $k-1$ 
kann dann mit Hilfe von
\begin{equation}
\hat{x}_{k|k-1} = \hat{x}_{k-1} + f(\hat{x}_{k-1},t_{k-1})\cdot \Delta t
\label{kalman:filter-schaetzung}
\end{equation}
zu einem Zustand $\hat{x}_{k|k-1}$ entwickelt werden.

Um zu beurteilen, wie gross die Korrektur durch die Messung sein soll,
muss bekannt sein, in welchem Ausmass die Schätzung vom tatsächlichen
Zustand abweicht.
Insbesondere muss \eqref{kalman:filter-schaetzung} dazu verwendet werden
können, auch die Entwicklung der Abweichung $x_{k|k-1}-\hat{x}_{k|k-1}$
aus $x_{k-1} - \hat{x}_{k-1}$ zu berechnen.
Für eine beliebige Funktion $f$ ist dies im Allgemeinen zu schwierig.
Durch erneute Linearisierung kann aus
\eqref{skript:filterbasis} eine lineare Entwicklungsgleichung
\begin{equation}
x_k = \varphi_k x_{k-1}
\end{equation}
gewonnen werden, wobei $\varphi_k$ eine $n\times n$-Matrix ist.

Nur die allereinfachsten Systeme lassen sich einigermassen korrekt modellieren.
Fast immer enthält ein Modell Vereinfachungen, die nötig sind, um über
überhaupt ein Modell zu erhalten, mit dem Vorhersagen möglich sind.
Entsprechend können die Vorhersagen auch nicht exakt sein, es verbleibt
immer ein Fehler.
Die Systementwicklung kann dies mit
\begin{equation}
x_k = \varphi_k x_{k-1} + u_k
\label{skript:kalman-system-mitfehler}
\end{equation}
berücksichtigen, worin $u_k$ alle nicht modellierten Einflüsse
enthält, auch genannt der {\em Systemfehler}. 
\index{Systemfehler}
Wir nehmen an, dass das Modell alle wesentlichen Einflüsse umfasst.
Die Fehler $u_k$ müssen zufällig und unvorhersagbar erscheinen,
die sich im Mittel nicht auswirken.
Es wird daher angenommen, dass die Komponenten $u_k$ normalverteilte
und unabhängige Zufallsvariablen mit Erwartungswert $0$ sind, dass also
\[
E(u_k)=0
\qquad\text{und}\qquad
E(u_ku_k^t)
=
\begin{pmatrix}
\sigma_1^2&0         &\dots &0\\
0         &\sigma_2^2&\dots &0\\
\vdots    &\vdots    &\ddots&\vdots\\
0         &0         &\dots &\sigma_n^2
\end{pmatrix}
=
Q_k
\]
ist\footnote{%
Aus der Argumentation oben lässt sich nicht ableiten, dass
die Fehler $u_k$ unabhängig sind, genau genommen gehen wir daher zu
weit, wenn wir annehmen, dass $Q_k$ Diagonalform hat, auch wenn dies
in der Literatur der Einfachheit halber üblicherweise angenommen wird.
Etwas allgemeiner sollte man nur voraussetzen, dass $Q_k$ eine symmetrische,
positiv definite Matrix ist.
Die nachfolgend beschriebene Theorie lässt sich damit unverändert
entwickeln.
}.
Die Matrix $Q_k$, die {\em Systemfehler-Kovarianz}
enthält auf der Diagonalen die Varianzen der Systemfehler $u_k$ 
und ausserhalb der Diagonalen die Kovarianzen.
\index{Systemfehler-Kovarianz}

\subsection{Schätzfehler\label{subsection:schaetzfehler}}
Wenn Vorhersage $\hat{x}_{k|k-1}$ und Messung $\hat{z}_k$ 
miteinander kombiniert werden sollen um die bestmögliche Schätzung
$\hat{x}_k$ zu erhalten, wird Kenntnis über den zu erwartenden
Fehler der Schätzung benötigt.

Die Schätzung $\hat{x}_k$ sollte mindestens im Mittel korrekt sein.
Dies lässt sich dadurch ausdrücken, dass die Fehler $x_k - \hat{x}_k$
als Zufallsvariablen mit Erwartungswert $0$ modelliert werden können.
Als Mass für den Schätzfehler kann man daher die Kovarianzmatrix von
\index{Schätzfehler-Kovarianz}%
$x_k-\hat{x}_k$ verwenden, die mit
\[
P_k =  E\bigl(\, (x_k-\hat{x}_k) (x_k-\hat{x}_k)^t\,\bigr)
\]
bezeichnet wird.
Die Zeitentwicklung~\eqref{skript:kalman-system-mitfehler} führt einen
zusätzlichen Fehler ein, der mit
\begin{equation}
P_{k|k-1} = \varphi_k P_{k-1} \varphi_k^t + Q_k
\label{skript:kalman:fehlerentwicklung}
\end{equation}
bezeichnet werden soll.

\subsection{Messprozess\label{subsection:messprozess}}
\index{Messprozess}
Früher wurde gezeigt, wie der Messprozess die Werte der Messgrössen $z_k$
aus den Zustandsvariablen ableitet.
Auch hier ist die Modellierung mit einer allgemeinen Funktion $z=f(x)$ 
zu kompliziert.
Auch für den Messprozes wird daher angenommen, dass er durch eine
lineare Abbildung $z=Hx$ beschrieben werden kann, wobei $H$ eine
$m\times n$-Matrix ist.
Um einen Rest von Nichtlinearität von $f$ zu retten, können wir davon
ausgehen, dass $H$ zusätzlich vom Zeitpunkt abhängt, also $z_k=H_kx_k$.

Jede Messung ist mit Fehlern behaftet, die mit einem Fehler-Vektor
$w_k$ in
\begin{equation}
z_k = H_kx_k + w_k,
\label{skript:kalman:messfehler}
\end{equation}
modelliert werden soll,
dessen Komponenten normalverteile Zufallsvariable mit Erwartungswert $0$
sind.
Meist wird zudem angenommen, dass die Fehler der einzelnen Messgrössen
unabhängig sind, dass die Kovarianzen von $w_k$ also eine
Diagonalmatrix
\[
R_k = E(w_kw_k^t) =
\begin{pmatrix}
\varrho_1^2&\dots & 0\\
\vdots     &\ddots&\vdots\\
0          &\dots &\varrho_m^2
\end{pmatrix},
\]
die {\em Messfehler-Kovarianz-Matrix},
ist\footnote{Wie bei der Systemfehler-Kovarianz würde es auch hier
genügen anzunehmen, dass $R_k$ eine symmetrisch positiv definite Matrix
ist.}.
\index{Messfehler}%
\index{Messfehler-Kovarianz}%

\subsection{Filterung\label{subsection:filterung}}
Wenn Vorhersage $\hat{x}_{k|k-1}$ und Messung $z_k$ nicht im Konflikt
stehen, dann ist auch keine Korrektur nötig, es kann
$\hat{x}_k = \hat{x}_{k|k-1}$ gewählt werden.
Offenbar kann man $\hat{x}_{k|k-1}$ nicht direkt mit $z_k$ vergleichen,
da diese Vektoren verschiedene Dimension haben.
Man kann aber $z_k$ vergleichen mit den Messwerten, die sich ergeben
müssten, wenn $\hat{x}_{k|k-1}$ der tatsächliche Zustand des Systems
wäre.
Diese Messwerte wären $H_k\hat{x}_{k|k-1}$.
Es ist daher nahe liegend,
die Korrektur linear in der Differenz $z_k-H_k\hat{x}_{k|k-1}$
anzusetzen.
Es ist daher eine $n\times m$-Matrix $K_k$ gesucht, mit der sich
die Korrektur als
\begin{equation}
\hat{x}_{k}
=
\hat{x}_{k|k-1} + K_k(z_k-H_k\hat{x}_{k|k-1})
=
(I-K_kH_k)\hat{x}_{k|k-1} + K_kz_k
\label{skript:kalman:filter}
\end{equation}
berechnen lässt.
Die Matrix $K_k$ soll so bestimmt werden, dass der Fehler der
Schätzung $\hat{x}_k$ möglichst klein wird.

Der Fehler von $\hat{x}_k$ ist $P_k$, mit \eqref{skript:kalman:filter}
kann er aus
\begin{equation}
P_k
=
(I-K_kh_k)P_{k|k-1}(I-K_kH_k)^t + K_kR_kK_k^t
\label{skript:fehler:korrektur}
\end{equation}
berechnet werden.
Als Mass für die Grösse des Fehlers können die Fehler jeder einzelnen 
Systemvariablen, also die Varianzen der Komponenten von $x_k$ 
verwendet werden.
Diese stehen in der Matrix $P_k$ auf der Diagonale, 
das gesuchte Fehlermass ist also deren Summe, die Spur der Matrix $P_k$.
$K_k$ muss jetzt so bestimmt werden, dass $\operatorname{Spur} P_k$ 
minimal wird.
Wie in \cite{skript:wrstatskript} gezeigt wird, das erreicht, wenn
\[
\begin{aligned}
&&
\frac{\partial}{\partial K_k} \operatorname{Spur}P_k
&=
-2(I-K_kH_k)P_{k|k-1}H_k^t + 2K_kR_k
=
0
\\
&\Rightarrow&
P_{k|k-1}H_k^t
&=
K_k(H_kP_{k|k-1}H_k^t+R_k)
\\
&\Rightarrow&
K_k
&=
P_{k|k-1}H_k^t
(H_kP_{k|k-1}H_k^t+R_k)^{-1},
\end{aligned}
\]
sofern die Matrix in Klammern tatsächlich invertierbar ist.
Die Matrix $K_k$ heisst die {\em Kalman-Filter-Matrix}.
\index{Kalman-Filter-Matrix}%

Damit haben wir alle Formeln für den Kalman-Filter zusammengetragen.
In jedem Zeitschritt führen wir folgende Schritte durch:
\begin{enumerate}
\item
Vorhersage des Zustandes und des Schätzfehlers:
\begin{align*}
\hat{x}_{k|k-1}&= \varphi_k\hat{x}_{k-1},\\
P_{k|k-1}      &= \varphi_kP_{k-1}\varphi_k^t + Q_k.
\end{align*}
\item
Berechnung der Kalman-Filter-Matrix
\begin{align}
K_k
&=
P_{k|k-1}H_k^t
(H_kP_{k|k-1}H_k^t+R_k)^{-1}.
\label{kalman:kmatrix}
\end{align}
\item
Korrektur:
\begin{align*}
\hat{x}_k&= (I-K_kH_k)\hat{x}_{k|k-1} + K_kz_k,\\
P_k      &= 
(I-K_kh_k)P_{k|k-1}(I-K_kH_k)^t + K_kR_kK_k^t.
\end{align*}
\end{enumerate}

\subsection{Beispiel}
Als Beispiel betrachten wir ein stark vereinfachtes Modell für die
Temperatur der Erde und die Albedo in der Nähe eines Gleichgewichtes.
Wir sind nur daran interessiert, die Temperaturanomalie $T$ und die
Albedo $a$ in der Umgebung eines Gleichgewichtspunktes zu
modellieren.
In einem Zeitintervall $\Delta t$ ändert sich die Temperatur
einerseits durch Ausstrahlung, andererseits in Abhängigkeit von
der Coalbedo $c$. 
Die Coalbedo wiederum steigt umso schneller, je grösser die
Temperaturanomalie ist.
Dies führt auf die Systemgleichungen
\[
\begin{aligned}
T_k &= (1-\alpha) T_k  + \beta c_k \\
c_k &= c_{k-1} + \gamma T_k
\end{aligned}
\]
mit geeigneten Koeffizienten $\alpha$, $\beta$ und $\gamma$.
$\alpha$ beschreibt den Anteil der gespeicherten Wärme-Energie, der durch
Ausstrahlung verloren geht und daher die Temperatur absenkt.
$\beta$ bestimmt, wie stark die Coalbedo die Temperatur beeinflussen kann,
modelliert also die Einstrahlung.
Schliesslich berechnet $\gamma$, wie eine Temperturerhöhung sich auf die
Coalbedo auswirkt.

Schreiben wir $x_k=(T_k,c_k)^t$, können wir die Systementwicklung als
\[
x_{k+1}
=
\begin{pmatrix}
T_{k+1}\\c_{k+1}
\end{pmatrix}
=
\underbrace{
\begin{pmatrix}
(1-\alpha)&\beta\\
\gamma& 1
\end{pmatrix}}_{\displaystyle\varphi_k}
\begin{pmatrix}
T_k\\c_k
\end{pmatrix}
\]
ausdrücken.
Wir nehmen an, dass wir nur die Temperaturanomalie $T_k$ messen können,
die zugehörige Messmatrix ist
\[
H_k=\begin{pmatrix}1&0\end{pmatrix}
\qquad\Rightarrow\qquad
z_k = H_k\begin{pmatrix} T_k\\c_k\end{pmatrix}
=
T_k.
\]
Zur Vervollständigung des Modells müssen jetzt nur noch die Mess- und
Systemfehler ermittelt werden.
Wenn die Temperaturanomalie mit einer Genauigkeit von $0.1\,\text{K}$
ermittelt werden kann, muss $\varrho_1^2=0.01\,\text{K}^2$ verwendet
werden.
Die Systemfehlerkovarianzmatrix $Q$ ist dagegen weniger leicht zu ermitteln.
In der Praxis wird in dieser Situation häufig experimentell vorgegangen,
wie dies auch in Kapitel~\ref{chapter:kalman} vorgeführt wird.








%
% assimilation.tex
%
% (c) 2018 Prof Dr Andreas Müller, Hochschule Rapperswil
%
\section{Das Assimilationsproblem\label{section:assimilationsproblem}}
Der Kalman-Filter ist ein Beispiel dafür, wie laufend anfallende neue
Messdaten und
ein Modell des untersuchten System ermöglichen, selbst dann einigermassen
vollständige Informationen über den Zustand des Systems zu erhalten, wenn
nicht alle Zustandsgrössen direkt gemessen werden.
Dass dies sogar für ein chaotisches System bis zu einem gewissen Grad
möglich ist, zeigen die Experimente in Kapitel~\ref{chapter:kalman}.

\rhead{Das Assimilationsproblem}
Allerdings wurden bei der Konstruktion des Kalman-Filters einige
einschränkende Annahmen gemacht.
Das System und der Messprozess musste linear beschreibbar sein, 
und Fehler mussten normalverteilt, unabhängig und mit Mittelwert $0$ sein.
Ausserdem wurden immer nur Messungen des aktuellen Zeitpunktes verwendet,
um die aktuellste Zustandsschätzung zu verbessern.
Dieses Vorgehen ist natürlich in vielen technischen Anwendungen angemessen
und auch passend zum Beispiel für Wetterprognosen, wo fast nur der aktuelle
Zustand der Atmosphäre interessiert.

In einem Klimamodell ist jedoch nicht nur der aktuelle Zustand der Atmosphäre
interessant.
Vielmehr geht es darum, die Geschichte des Erdklimas möglichst genau
zu rekonstruieren und dabei immer weitere, neu zugänglich gemachte 
Daten in das Modell zu integrieren und die Qualität der Vorhersagen zu
verbessern.

In diesem Kapitel soll daher angedeutet werden, wie ein allgemeineres
Datenassimilationsproblem formuliert werden kann, welches als Spezialfall
das im vorangegangenen Abschnitt~\ref{section:kalman-filter} gelöste
Filterproblem hat.

\subsection{Markov-Kette und Markov-Eigenschaft\label{subsection:markov}}
Der Zustand des Klimasystems zu einem beliebigen Zeitpunkt kann mit einem
Vektor $x\in\mathbb R^n$ von Variablen beschrieben werden, im einfachsten
Fall bestehend allein aus der globalen Mitteltemperatur.
Natürlich ist $x$ nicht bekannt, wir können bestenfalls $X$ als eine
Zufallsvariable betrachten, die den Wert $x$ angenommen hat.

Da die ganze Klimageschichte rekonstruiert werden soll, sind die Werte von
$x$ zu diskreten Zeitpunkten $0,\dots,N$ zu bestimmen.
Die Klimageschichte ist also eine Folge $X(k)$ mit $k=0,\dots,N$ von
Zufallsvariablen.
Im Folgenden wird davon ausgegangen, dass die Wahrscheinlichkeitsdichte
$f_{X(k)}(x)$ von $X(k)$ bekannt ist.
Man nennt $X(k)$ einen diskreten {\em stochastischen Prozess}.
\index{stochastischer Prozess}%
\index{Prozess, stochastischer}%
In der Realität steht uns jeweils nur ein Teil der Werte des Prozesses zur
Verfügung, so bezeichnen wir mit $X(i:j)$ die Folge
\[
X(i),X(i+1),\dots,X(j-1),X(j).
\]
Damit ist $X(0:n)$ die Klimageschichte bis zur Zeit $n$, und $X(n:N)$ die
Zukunft\footnote{Die Notation $X(i:j)$ ist motiviert durch die 
von Matlab oder Octave verwendete Notation \texttt{i:j} für die
Folge der Zahlen von $i$ bis $j$.}.

Das Assimilationsproblem besteht darin, dass Messwerte $Y(k)$ zur Verfügung
gestellt werden, mit denen sich die Wahrscheinlichkeitsdichte von $X(k)$
``genauer'' bestimmen lässt.
Um den Formalismus herauszuarbeiten seien $X$ und $Y$ Zufallsvariablen,
es soll die Frage beantwortet werden, wie sich die Verteilung von $X$
ändert, wenn $Y$ den Wert von $y$  annimmt.
Die Wahrscheinlichkeitsdichte von $(X,Y)$ sei $f_{X,Y}(x,y)$, sie kann
auf zwei verschiedene Arten als Produkt
\begin{equation}
f_{X,Y}(x,y)
= 
f_{X|y}(x)\,f_Y(y)
=
f_{Y|x}(y)\,f_X(x)
\label{skript:bedpdf}
\end{equation}
geschrieben werden.
\index{Satz von Bayes}%
Der {\em Satz von Bayes} für die {\em bedingten Wahrscheinlichkeitsdichten}
\index{bedingte Wahrscheinlichkeit}
\begin{equation}
f_{X|y}(x)
=
\frac{f_X(x)}{f_Y(y)}\,f_{Y|x}(y)
\label{assim:bayes}
\end{equation}
stellt den Zusammenhang zwischen den bedingten Wahrscheinlichkeiten her.

Wenn die Zufallsvariablen $X$ und $Y$ unabhängig sind, dann hat der
Wert von $y$ keinen Einfluss auf die Wahrscheinlichkeitsdichte von $X$,
es ist also
\[
f_{X|y}(x)
=
f_X(x)\quad\forall y.
\]
Gegeben $f_{X|y}(x)$ lässt sich auch die Wahrscheinlichkeitsdichte
$f_X$ berechnen, es gilt
\begin{align}
f_X(x)
&=
\int f_{X|y}(x)\,f_Y(y) dy.
\label{assim:total}
\end{align}
Dies ist der Satz von der {\em totalen Wahrscheinlichkeit}.
\index{totale Wahrscheinlichkeit}%

Kausalität verlangt, dass der Zustand $X(k)$ zur Zeit $k$ abhängig ist
von den Zustandswerten $x(j)$ mit $j<k$.
Da $X(k)$ Zufallsvariablen sind, muss sich diese Eigenschaft mit den
Wahrscheinlichkeitsdichten für $X(k)$ ausdrücken lassen.

Mit Ausnahme der verzögerten Differentialgleichung, die zur Modellierung
des El~Niño-Phä\-no\-mens in Kapitel~\ref{chapter:elnino} verwendet wurde,
haben alle bisher studierten Modelle die Eigenschaft, dass der Zustand
zur Zeit $k$ vollständig bestimmt ist durch den Zustand zur Zeit $k-1$.
Frühere Zustände zu Zeiten $j < k-1$ müssen nicht berücksichtigt werden.
Mit Hilfe der bedingten Wahrscheinlichkeitsdichte~\eqref{skript:bedpdf}
kann dies wie folgt ausgedrückt werden:

\begin{definition}
\index{Markov-Eigenschaft}%
\index{Markov-Kette}%
Der stochastische Prozess $X(k)$ hat die {\em Markov-Eigenschaft}, wenn gilt
\begin{equation}
f_{X(n:N)|x(0:n-1)}(x(n:N)) 
=
f_{X(n:N)|x(n-1)}(x(n:N)).
\end{equation}
Ein solcher stochastischer Prozess heiss {\em Markov-Kette}.
\end{definition}
Die Definition besagt, dass die Verteilung von $X(n:N)$ sich nicht
ändert, wenn man ausser den Werten $x(n-1)$ auch noch die früheren
Werte $x(0:n-2)$ kennt.

Die Markov-Eigenschaft eines stochastischen Prozesses bedeutet,
dass für Vorhersagen von $X(n:N)$ nur die Verteilung von $X(n-1)$
bekannt sein muss.
Weiteres Wissen über weiter zurück liegende Werte $X(0:n-2)$
liefert keine Verbesserung der Vorhersage von $X(n)$.
Durch wiederholte Anwendung von \eqref{skript:bedpdf} erhält man
daraus
\begin{align*}
f_{X(0:n)}(x(0:n))
&=
f_{X(n)|x(0:n-1)}(x(n))\,
f_{X(0:n-1)}(x(0:n-1))
=
f_{X(n)|x(n-1)}(x(n))\,
f_{X(0:n-1)}(x(0:n-1))
\\
&=
f_{X(n)|x(n-1)}(x(n))\,
f_{X(n-1)|x(n-2)}(x(n-1))\,
\dots\,
f_{X(0}(x(0))
\end{align*}
Die Wahrscheinlichkeitsdichten $f_{X(j)|x(j-1)}(x(j))$ heissen
die {\em Übergangswahrscheinlichkeiten } der Markov-Kette.

\subsection{Filterung, Vorhersage und Reanalyse}
Die Aufgabe, die sich jetzt stellt, ist aus Ausschnitten der
Messwertegeschichte $Y(0:N)$ Ausschnitte der Klimageschichte zu
rekonstruieren.
Die folgenden Spezialfälle werden unterschieden:
\begin{enumerate}
\item
\index{Filterung}%
Filterung: Schätze $X(n)$ aus den Werten von $Y(1:n)$.
\item
\index{Vorhersage}%
Vorhersage: Schätze $X(n)$ aus den Werten von $Y(1:n-1)$.
\item
\index{Reanalyse}%
Reanalyse: Schätze $X(n)$ aus den Werten von $Y(1:N)$.
\end{enumerate}
Man beachte, dass bei der Reanalyse auch die späteren Werte zur
Verfügung stehen.
Reanalyse liegt zum Beispiel vor, wenn aktuelle Klimaentwicklungen dazu
verwendet werden, die Klimageschichte der letzten Jahrzehnte genauer 
zu rekonstruieren.

Um eines dieser Probleme zu lösen, muss der Zusammenhang zwischen den
Zustandsvariablen $X(k)$ und den Messgrössen $Y(j)$ bekannt sein.
Die Messgrösse $Y(k)$ soll nur vom Zustand zur Zeit $k$ abhängen,
nicht von früheren oder späteren Zeitpunkten.
Die Zufallsvariablen $Y(j)$ und $X(k)$ mit $j\ne k$ müssen daher
unabhängig sein.
Dies bedeutet, dass sich die Wahrscheinlichkeitsdichte als Produkt
\begin{align*}
f_{Y(1:n)|x(0:n)} (y(1:n))
&=
f_{Y(n)|x(n)}(y(n))\,
f_{Y(n-1)|x(n-1)}(y(n-1))\,
\dots\,
f_{Y(1)|x(1)}(y(1))
\end{align*}
zerlegen lässt.

Damit lassen sich jetzt die Problemstellungen genauer formulieren 
und auch lösen.

\begin{satz}[Filter und Vorhersage]
\label{satz:filter und vorhersage}
Gegeben seien
\begin{enumerate}
\item die Wahrscheinlichkeitsdichte $f_{X(0)}$ für den Anfangszustand,
\item die Übergangswahrscheinlichkeiten $f_{X(j)|x(j-1)}$ der
Markov-Kette $X(0:N)$ und
\item die bedingten Wahrscheinlichkeitsdichten $f_{Y(j)|x(j)}$ für die
Messungen.
\end{enumerate}
Dann können wie Wahrscheinlichkeitsdichten $f_{X(i)|y(1:i)}$ für das
Filterproblem sowie die Wahrscheinlichkeitsdichten $f_{X(i)|y(1:i-1)}$
für das Vorhersage-Problem wie folgt bestimmt werden.
\begin{enumerate}
\item Schritt:
\begin{align}
f_{X(1)}(x(1))
&=
\int f_{X(1)|x(0)}(x(1)) \, f_{X(0)}(x(0))\, dx(0)
\label{assim:totalstep1}
\\
f_{X(1)|y(1)}(x(1))
&\sim
f_{Y(1)|x(1)}(y(1))\, f_{X(1)}(x(1)),
\label{assim:bayesstep1}
\end{align}
wobei der Proportionalitätsfaktor so gewählt werden muss, dass tatsächlich
eine Wahrscheinlichkeitsdichte entsteht.
\item Schritt: für $i=2,\dots,N$ gilt
\begin{align}
f_{X(i)|y(1:i-1)}(x(i))
&=
\int f_{X(i)|x(i-1)}(x(i))\, f_{X(i-1)|y(1:i-1)}(x(i-1))\,dx(i-1)
\label{assim:totalstep2}
\\
f_{X(i)|y(1:i)}(x(i))
&\sim
f_{Y(i)|x(i)}(y(i))\, f_{X(i)|y(1:i-1)}(x(i)),
\label{assim:bayesstep2}
\end{align}
wobei wiederum der Proportionalitätsfaktor so gewählt werden muss, dass
eine Wahrscheinlichkeitsdichte entsteht.
\end{enumerate}
\end{satz}

\begin{proof}[Beweis]
Die Formeln von Schritt~1 sind nichts anderes als der Satz von
der totalen Wahrscheinlichkeit für \eqref{assim:bayesstep1} und
der Satz von Bayes~\eqref{assim:bayes} für \eqref{assim:bayesstep2}.

Für den Beweis der Formeln für den zweiten Schritt verwenden wir 
vollständig Induktion.
Die Verankerung für $i=2$ ist gegeben durch den Satz von der totalen
Wahrscheinlichkeit~\eqref{assim:total}
\begin{align*}
f_{X(2)|y(1)}(x(2))
&=
\int f_{X(2)|x(1)}(x(2))  f_{X(1)}(x(1))\, dx(1).
\end{align*}
und erhalten damit die Formel~\eqref{assim:totalstep2} für $i=2$.
Die zweite Formel~\eqref{assim:bayesstep2} ergibt sich aus dem Satz
von Bayes~\eqref{assim:bayes}.

Für den Induktionsschritt verwenden wir wieder zuerst den Satz von der
totalen Wahrscheinlichkeit~\eqref{assim:total} und erhalten
\begin{align*}
f_{X(i)|y(1:i-1)}(x(i))
&=
\int f_{X(i)|x(i-1)}(x(i)) \, f_{X(i-1)|y(1:i-1)}(x(i-1))\,dx(i-1)
\end{align*}
also die Formel~\eqref{assim:totalstep2}.
Die zweite Formel~\eqref{assim:bayesstep2} folgt daraus mit dem Satz
von Bayes~\eqref{assim:bayes}.
\end{proof}


\begin{satz}[Reanalyse]
\label{satz:reanalyse}
Gegeben seien
\begin{enumerate}
\item
die Übergangswahrscheinlichkeiten $f_{X(i)|x(i-1)}$ für die
Markov-Kette $X(0:n)$
\item
die Wahrscheinlichkeitsdichte für das Filterproblem
$f_{X(i)|y(1:i)}$.
\end{enumerate}
Dann kann die Wahrscheinlichkeitsdichte für das Reanalyse-Problem wie folgt
bestimmt werden
\begin{enumerate}
\item Falls $i=N$, dann ist die Wahrscheinlichkeitsdichte für das
Reanalyse-Problem dieselbe dieselbe wie jene für das Filterproblem.
\item Für $i=2,\dots,N-1$ und für die Wahrscheinlichkeitsdichte 
$f_{X(i+1)|y(1:N)}$  für das Reanalyse-Problem für $X(i+1)$ 
wird die Wahrscheinlichkeitsdichte für das Reanalyse-Problem für $X(i)$
durch
\begin{align}
f_{X(i)|x(i+1),y(1:i)}(x(i))
&\sim
f_{X(i+1)|x(i)}(x(i+1))\, f_{X(i)|y(1:i)}(x(i))
\label{assim:reanabayes}
\\
f_{X(i)|y(1:N)}(x(i))
&=
\int f_{X(i)|x(i+1),y(1:i)}(x(i))\, f_{X(i+1)|y(1:N)}(x(i+1))\,dx(i+1)
\label{assim:reanatotal}
\end{align}
berechnet,
wobei der Proportionalitätsfaktor wieder so gewählt werden muss, dass
eine Wahrscheinlichkeitsdichte entsteht.
\end{enumerate}
\end{satz}

\begin{proof}[Beweis]
Fall~1 ist klar, das Reanalyse-Problem für $X(N)$ ist dasselbe wie des
Filterproblem für $X(N)$.

Es bleibt daher nur noch, die Formeln
\eqref{assim:reanabayes}
und
\eqref{assim:reanatotal}
nachzuweisen.
Die Formel
\eqref{assim:reanabayes}
ist nichts anderes als der 
als der Satz von Bayes
\eqref{assim:bayes}.
Die Formel 
\eqref{assim:reanatotal}
ist der Satz von der totalen Wahrscheinlichkeit
\eqref{assim:total}.
\end{proof}










