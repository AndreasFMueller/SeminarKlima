%
% fourier.tex -- oszillationsphänomene insbesondere Milankowitsch-Zyklern
%
% (c) 2018 Prof Dr Andreas Müller, Hochschule Rapperswil
%
\chapter{Fourier-Analysis\label{chapter:fourier}}
\lhead{Kapitel \thechapter: Fourier-Analysis}
\rhead{ }
Im Kapitel~\ref{chapter:wetter und klima} wurde gezeigt, dass einige der
Einflüsse
auf das Klimasystem periodisch sind mit einer Periode, die vergleichbar
oder grösser ist als die bei der Definition des Begriffes Klima üblicherweise
verwendeten Mittelungszeitspanne.
Diese Anregungen führen daher zu periodischen Klimaschwankungen.
Die Fourier-Analysis ermöglicht, solche periodischen Einflüsse in
einem Signal zu erkennen und sie von anderen Phänomenen zu trennen.

Ziel dieses Kapitels ist, die diskrete Fourier-Theorie zu entwickeln und
an Beispielen zu zeigen, wie die Fourier-Koeffizienten periodische Phänomene
zu analysieren gestattet.
Im ersten Abschnitt wird die Problemstellung konkretisiert und die
Notation festgelegt.
Im zweiten Abschnitt werden Formeln zur Berechnung der Fourierkoeffizienten
hergeleitet, die im dritten Abschnitt dann geometrisch gedeutet werden.

\input{chapters/6/periodisch.tex}
%
% fourierkoef.tex -- Bestimmung der Fourier-Koeffizienten
%
% (c) 2018 Prof Dr Andreas Müller, Hochschule Rapperswil
%
\section{Fourier-Koeffizienten}
\rhead{Fourier-Koeffizienten}
In diesem Abschnitt bestimmen wir die Fourierkoeffizienten für ein
trigonometrisches Polynom der Form
\begin{equation}
p_n(t)
=
a_0 + \sum_{k=1}^{n-1} a_k\cos kt + \sum_{k=1}^{n-1} b_k\sin kt + a_n\cos nt,
\label{skript:fourier:ansatz}
\end{equation}
derart, dass die Werte
\begin{equation}
y_j \quad\text{zu den Zeitpunkten}\quad t_j=2\pi\frac{j}{N},\qquad 1\le j< N
\label{skript:fourier:gleichungen}
\end{equation}
möglichst genau die Funktionswerte $p_n(t_j)$ reproduzieren.

Im Ansatz~\eqref{skript:fourier:ansatz} für $p_n(t)$ finden wir
genau $2n$ zu bestimmende Koeffizienten, nämlich $a_0$,
$n$ Koeffizienten $a_k$ mit $k=1,\dots,n$
und $n-1$ Koeffizienten $b_k$ mit $k=1,\dots,n-1$.
Mit $N$ Datenpunkten $y_j$ haben wir genau die $N$ Bedingungen
$p_n(t_j)=y_j$, um diese Koeffizienten zu bestimmen.
Die Bedingungen $p_n(t_j)=y_j$ sind $N$ lineare Gleichungen für die
$2n$ Koeffizienten, sie dürften sich also genau dann exakt lösen
lassen, wenn $N=2n$, wenn also die Zahl der Datenpunkte gerade ist.

Für $N<2n$ haben wir weniger Gleichungen als Koeffizienten, es ist
also unmöglich, die Koeffizienten ohne zusätzliche Annahmen zu bestimmen.
Das Problem ist also nur dann überhaupt lösbar, wenn $N\ge 2n$ ist.
Für $N>2n$ haben wir zu viele Daten, wir können nicht erwarten, dass
das Gleichungssystem $p_n(t_j)=y_j$ überhaupt eine Lösung hat.
Im optimalen Fall, für $N=2n$ sollte es möglich sein, die Koeffizienten so
zu bestimmen, dass die Funktionswerte $y_j$ exakt reproduziert werden.

\subsection{Least Squares}
Für $N>2n$ können wir nicht erwarten, dass der Ansatz
\eqref{skript:fourier:ansatz} die Daten exakt reproduzieren kann,
wir müssen uns also mit einer Näherungslösung begnügen.
Wir verlangen stattdessen, dass der Fehler der Lösung möglichst gering
wird, dass also
\[
L=L(a_0,a_1,\dots,a_n,b_1,\dots,b_{n-1})= \sum_{j=1}^N (y_j - p_n(t_j))^2
\]
möglichst klein wird.

Die Grösse $L$ wird minimal, wenn alle Ableitungen nach den Koeffizienten
verschwinden:
\begin{equation}
\frac{\partial L}{\partial a_0}=0,
\qquad
\frac{\partial L}{\partial a_l}=0, \;{1\le l\le n}
\qquad
\frac{\partial L}{\partial b_l}=0, \;{1\le l\le n-1}
\label{skript:fourier:leastsquaresableitungen}
\end{equation}
Um die Koeffizienten zu bestimmen, müssen wir die Ableitungen in
\eqref{skript:fourier:leastsquaresableitungen}
berechnen und erhalten die Gleichungen
\begin{align}
\frac{\partial L}{\partial a_0}
&=
-2 \sum_{j=1}^N (y_j-p_n(t_j))\cdot \frac{\partial p_n}{\partial a_0}(t_j)=0,
&&
\label{skript:fourier:a0ableitung}
\\
\frac{\partial L}{\partial a_l}
&=
-2 \sum_{j=1}^N (y_j-p_n(t_j))\cdot \frac{\partial p_n}{\partial a_l}(t_j)=0
&k&=1,\dots,n,
\label{skript:fourier:akableitung}
\\
\frac{\partial L}{\partial b_l}
&=
-2 \sum_{j=1}^N (y_j-p_n(t_j))\cdot \frac{\partial p_n}{\partial b_l}(t_j)=0
&k&=1,\dots,n-1.
\label{skript:fourier:bkableitung}
\end{align}
Man beachte, dass $p_n(t_j)$ die Koeffizienten $a_0$, $a_k$ und $b_k$
linear enthält.
Der erste Klammerausdruck $y_j-p_n(t_j)$ in der Summe enthält daher die
Koeffizienten ebenfalls nur linear, die Ableitungen nach den Koeffizienten
enhalten daher die Koeffizienten überhaupt nicht mehr.
Tatsächlich ergibt die Berechnung der Ableitungen
\begin{align}
\frac{\partial p_n}{\partial a_0}(t_j)
&=
1
&&
\label{skript:fourier:a0abl}
\\
\frac{\partial p_n}{\partial a_l}(t_j)
&=
\cos lt_j
&l&=1,\dots,n
\label{skript:fourier:akabl}
\\
\frac{\partial p_n}{\partial b_l}(t_j)
&=
\sin lt_j
&l&=1,\dots,n-1.
\label{skript:fourier:bkabl}
\end{align}
\definecolor{darkblue}{rgb}{0,0,0.6}%
Setzen wir diese Ableitungen als {\color{darkblue}dunkelblaue} Terme in
\eqref{skript:fourier:a0ableitung}--\eqref{skript:fourier:bkableitung} ein
und vertauschen die Reihenfolge der Summationen über $k$ und $j$,
erhalten wir die Gleichungen
\definecolor{darkred}{rgb}{0.6,0,0}%
\begin{equation}
\left.
\begin{aligned}
0&=
\sum_{j=1}^N
\biggl(
y_j - a_0 - \sum_{k=1}^n a_k\cos kt_j - \sum_{k=1}^{n-1} b_k\sin kt_j
\biggr)\cdot\mathstrut{\color{darkblue}1}
\\
&=
\sum_{j=1}^N y_j
-Na_0
-\sum_{k=1}^n a_k {\color{darkred}\sum_{j=1}^N\cos kt_j}
-\sum_{k=1}^{n-1} b_k {\color{darkred}\sum_{j=1}^N\sin kt_j},
\\
0&=
\sum_{j=1}^N
\biggl(
y_j - a_0 - \sum_{k=1}^n a_k\cos kt_j - \sum_{k=1}^{n-1} b_k\sin kt_j
\biggr)\cdot\mathstrut {\color{darkblue}\cos lt_j}
\\
&=
\sum_{j=1}^N y_j \cos lt_j
-a_0\sum_{j=1}^N \cos lt_j
-\sum_{k=1}^n a_k {\color{darkred}\sum_{j=1}^N\cos kt_j\cos lt_j}
-\sum_{k=1}^{n-1} b_k {\color{darkred}\sum_{j=1}^N\sin kt_j\cos lt_j},
%&l&=1,\dots,n-1
\\
0
&=
\sum_{j=1}^N
\biggl(
y_j - a_0 - \sum_{k=1}^n a_k\cos kt_j \sum_{k=1}^{n-1} b_k\sin kt_j
\biggr)\cdot\mathstrut {\color{darkblue}\sin lt_j}
\\
&=
\sum_{j=1}^N y_j \sin lt_j
-a_0\sum_{j=1}^N \sin lt_j
-\sum_{k=1}^n a_k {\color{darkred}\sum_{j=1}^N\cos kt_j\sin lt_j}
-\sum_{k=1}^{n-1} b_k {\color{darkred}\sum_{j=1}^N\sin kt_j\sin lt_j}.
%&l&=1,\dots,n
\end{aligned}
\right\}
\label{skript:fourier:gl2}
\end{equation}
Hier haben wir die Faktoren $-2$ ebenfalls weggelassen.
Auf den ersten Blick scheinen diese Gleichungen nicht einfach lösbar zu
sein, die {\color{darkred}dunkelrot} hervorgehobenen Koeffizienten der
Unbekannten $a_0$, $a_k$ und $b_k$ scheinen ziemlich kompliziert zu sein.
Es stellt sich aber heraus, dass diese direkt ausgewertet werden können,
was im nächsten Abschnitt geschehen soll.

\subsection{Trigonometrische Summen}
In den Gleichungen~\eqref{skript:fourier:gl2} treten als Koeffizienten
für die Unbekannteno $a_0$, $a_k$ und $b_k$ trigonometrische Summen
der Form
\begin{equation}
\sum_{j=1}^N \cos kt_j
\qquad
\text{oder}
\qquad
\sum_{j=1}^N \sin kt_j,
\label{skript:fourier:trigosum}
\end{equation}
sowie Summen von Produkten trigonmetrischer Funktionen wie
\begin{equation}
\sum_{j=1}^N \cos kt_j\cos lt_j,\quad
\sum_{j=1}^N \cos kt_j\sin lt_j
\quad
\text{oder}
\quad
\sum_{j=1}^N \sin kt_j\sin lt_j
\label{fourier:produkte}
\end{equation}
auf.
In diesem Abschnitt sollen diese Summen mit Hilfe einer geometrischen
Überlegung berechnet werden.

Wir befassen uns zunächst mit Summen der Form
\eqref{skript:fourier:trigosum} und beweisen den folgenden Satz.

\begin{figure}
\centering
\includegraphics{chapters/6/trigosum.pdf}
\caption{Verteilung der Punkte $(\cos t_j, \sin t_j)$  auf dem Einheitskreis 
in rot.
Die Punkte $(\cos kt_j,\sin kt_j)$ für $k=9$ bilden eine Teilmenge, die
blau dargestellt ist.
Jeder blaue Punkt wird genau dreimal besucht, sie bilden ein gleichseitiges
Fünfeck mit den Punkten $(\cos 3t_j,\sin 3t_j)$ als Ecken.
Deren Schwerpunkt ist wieder der Nullpunkt.
\label{fourier:einheitskreis}
}
\end{figure}

\begin{satz}
\label{skript:fourier:orthogonalitaet1}
Ist $t_j=2\pi j/N$, dann gelten für beliebige ganze Zahlen $l$,
mit $0\le l\le n$, die Identitäten
\begin{equation*}
\begin{aligned}
\sum_{j=1}^N \cos lt_j
&=
\begin{cases}
N&\qquad l=0\\
0&\qquad\text{sonst}
\end{cases}
\\
\sum_{j=1}^N \sin lt_j
&=0.
\end{aligned}
\end{equation*}
\end{satz}

\begin{proof}[Beweis]
Wir betrachten zunächst den Fall $l=0$.
In diesem Fall ist $\cos lt_j=1$ und $\sin lt_j=0$ und damit
\[
\sum_{j=1}^N \cos lt_j = N
\qquad\text{und}\qquad
\sum_{j=1}^N \sin lt_j = 0.
\]
Im Folgenden können wir daher annehmen, dass $l\ne 0$.

In Abbildung~\eqref{fourier:einheitskreis} kann man sehen, dass die Punkte
$(\cos t_j,\sin t_j)$ auf dem Einheitskreis ein regelmässiges Polygon
mit $N$ Ecken bilden.
Der Schwerpunkt des Polygons ist ganz offensichtlich der Mittelpunkt.
Daraus folgt
\[
\sum_{j=1}^N \cos t_j = 0
\qquad\text{und}\qquad
\sum_{j=1}^N \sin t_j = 0.
\]
Damit ist der Satz für den Fall $l=1$ bewiesen.

Für beliebiges $l\ne 0$ beobachten wir, dass die Punkte 
$(\cos lt_j,\sin lt_j)$ eine Teilmenge der Punkte $(\cos t_j, \sin t_j)$
sind.
Wenn $l$ und $N$ teilerfremd sind, sind die Mengen gleich.
Wenn $l$ und $N$ dagegen den grössten gemeinsamen Teiler
$r=\operatorname{ggT}(l,N)$ haben, dann
ist die Menge 
\[
\{ 
(\cos lt_j,\sin lt_j)\,|\,j=1,\dots,N
\}
=
\{
(\cos rt_j,\sin rt_j)\,|\,j=1,\dots,N/r
\}
\]
ein regelmässiges Polygon mit $N/r$ Ecken.
Diese Situation ist in Abbildung~\ref{fourier:einheitskreis} mit den
blauen Punkten für den Fall $r=3=\operatorname{ggT}(9,15)$
illustriert.
Wie im Falle von $l=1$ folgt\footnote{Etwas formeller könnten wir sagen,
dass wir hier vollständige Induktion nach $N$ machen.
Was wir im letzten Schritt nämlich brauchen ist der Wert einer
trigonometrischen Summe mit $N/r<N$ Summanden, deren Werte wir gemäss
der naheligenden Induktionsannahme bereits kennen.},
dass der Schwerpunkt des Polygons der
Nullpunkt ist, und damit, dass
\begin{align*}
\sum_{j=1}^N \cos lt_j 
=
\sum_{j=1}^N \cos rt_j 
=
0,
\\
\sum_{j=1}^N \sin lt_j 
=
\sum_{j=1}^N \sin rt_j 
=
0.
\end{align*}
Damit ist alles gezeigt.
\end{proof}

In \eqref{fourier:produkte} werden die Summen von Produkten benötigt.
Mit üblichen trigonometrischen Umformungen kann man diese in Summen
von einfachen trigonometrischen Funktionen umwandeln.
Wir verwenden dazu die Formeln
\begin{align}
\cos\alpha\cos\beta
&=
\frac12\bigl(\cos(\alpha-\beta)+\cos(\alpha+\beta)\bigr),
\label{fourier:coscos}
\\
\sin\alpha\sin\beta
&=
\frac12\bigl(\cos(\alpha-\beta) + \cos(\alpha+\beta)\bigr).
\label{fourier:sinsin}
\\
\sin\alpha\cos\beta
&=
\frac12\bigl(\sin(\alpha-\beta) + \sin(\alpha+\beta)\bigr),
\label{fourier:sincos}
\end{align}
Damit können wir die Summen in \eqref{fourier:produkte} umwandeln:
\begin{align*}
\sum_{j=1}^N \cos kt_j\cos lt_j
&=
\sum_{j=1}^N \frac12\bigl(\cos (k-l)t_j +\cos(k+l)t_j\bigr)
\\
&=
\frac12\sum_{j=1}^N \cos (k-l)t_j
+ \frac12\underbrace{\sum_{j=1}^N\cos(k+l)t_j}_{\displaystyle=0}
\\
&=
\begin{cases}
\displaystyle\frac{N}2&\qquad k=l\\
0&\qquad\text{sonst}
\end{cases}
\\
\sum_{j=1}^N \sin kt_j \sin lt_j
&=
\sum_{j=1}^N \frac12\bigl(\cos(k-l)t_j +\cos(k+l)t_j\bigr)
\\
&=\frac12\sum_{j=1}^N \cos(k-l)t_j
+\frac12\underbrace{\sum_{j=1}^N \cos(k+l)t_j}_{\displaystyle=0}
\\
&=
\begin{cases}
\displaystyle\frac{N}2&\qquad k=l\\
0&\qquad\text{sonst}
\end{cases}
\\
\sum_{j=1}^N \sin kt_j \cos lt_j
&=
\sum_{j=1}^N \frac12\bigl(\sin(k-l)t_j +\sin(k+l)t_j\bigr)
\\
&=
\frac12\underbrace{\sum_{j=1}^N \sin(k-l)t_j}_{\displaystyle=0}
+
\frac12\underbrace{\sum_{j=1}^N \sin(k+l)t_j}_{\displaystyle=0}
=0.
\end{align*}
Damit haben wir den folgenden Satz bewiesen:

\begin{satz}
\label{skript:fourier:satzprodukte}
Für beliebige $k,l\in \mathbb N$ gilt
\begin{align*}
\sum_{j=1}^N
\cos kt_j \cos lt_j
&=
\begin{cases}
N                     &\qquad k=l=0\\
\displaystyle\frac{N}2&\qquad k=l > 0\\
0                     &\qquad\text{sonst}
\end{cases}
\\
\sum_{j=1}^N
\sin kt_j \sin l_j
&=
\begin{cases}
\displaystyle \frac{N}2&\qquad k=l\\
0                      &\qquad\text{sonst}
\end{cases}
\\
\sum_{j=1}^N
\sin kt_j \cos lt_j
&=
0
\end{align*}
\end{satz}
Mit dem Kronecker-$\delta$ 
\index{Kronecker-$\delta$}%
\[
\delta_{kl}
=
\begin{cases}
1&\qquad k=l\\
0&\qquad\text{sonst}
\end{cases}
\]
können wir die ersten zwei Formeln für $k,l>0$ noch etwas kompakter 
als
\begin{equation}
\sum_{j=1}^N
\cos kt_j \cos lt_j
=
\sum_{j=1}^N
\sin kt_j \sin l_j
=
\delta_{kl}\frac{N}2
\label{skript:fourier:trigsumsummary}
\end{equation}
schreiben.

In den folgenden Abschnitten verwenden wir diese Formeln, um die
Koeffizienten $a_k$ und $b_k$ zu bestimmen.
Der Koeffizient $a_0$ muss gesondert behandelt werden.

\subsection{Bestimmung von $a_0$}
In der Gleichung
\[
0
=
\sum_{j=1}^Ny_j
-Na_0
-\sum_{k=1}^na_k\underbrace{\sum_{j=1}^N\cos kt_j}_{\displaystyle=0}
-\sum_{k=1}^nb_k\underbrace{\sum_{j=1}^N\sin kt_j}_{\displaystyle=0}
\]
verschwinden die trigonometrischen Summen über $j$ nach
Satz~\ref{skript:fourier:orthogonalitaet1} und es bleibt die Gleichung
\begin{align*}
0
&=
\sum_{j=1}^Ny_j
-Na_0
\\
\Rightarrow\qquad
a_0&=\frac1{N}\sum_{j=1}^N y_j.
\end{align*}
Der Koeffizient $a_0$ ist der Mittelwert der Werte $y_j$.

\subsection{Bestimmung von $a_k, k>0$}
Zur Bestimmung von $a_k$ mit $k>0$ müssen wir die Gleichung
\[
0
=
\sum_{j=1}^N y_j\cos lt_j
-
a_0\underbrace{\sum_{j=1}^N\cos lt_j}_{\displaystyle=0}
-\sum_{k=1}^na_k\sum_{j=1}^N\cos kt_j\cos lt_j
-\underbrace{\sum_{k=1}^nb_k\sum_{j=1}^N\sin kt_j\cos lt_j}_{\displaystyle=0}
\]
heranziehen.
Die zweite und vierte Summe verschwindet wegen
Satz~\ref{skript:fourier:satzprodukte}, so dass wir die die Gleichung
\begin{align*}
0&=
\sum_{j=1}^N y_j\cos lt_j
-\sum_{k=1}^na_k\sum_{j=1}^N\cos kt_j\cos lt_j
\end{align*}
erhalten.
Die innere Summe über $j$ im zweiten Term verschwindet, wieder gemäss 
Satz~\ref{skript:fourier:satzprodukte}, für alle Werte
von $k$ ausser für $k=l$, in diesem Fall ist sie $N/2$. 
Damit können wir nach $a_k$ auflösen:
\begin{align*}
0&
=
\sum_{j=1}^N y_j\cos lt_j
-\sum_{k=1}^na_k\delta_{kl}\frac{N}2
=
\sum_{j=1}^N y_j\cos lt_j
-a_l\frac{N}2
\\
\Rightarrow\qquad 
a_l &= \frac{2}{N}\sum_{j=1}^Ny_j\cos lt_j.
\end{align*}

\subsection{Bestimmung von $b_k$}
Zur Bestimmung von $b_k$ müssen wir die Gleichung
\[
0=\sum_{j=1}^N y_j\sin lt_j 
-a_0\sum_{j=1}^N \sin lt_j
-\sum_{k=1}^na_k\sum_{j=1}^N\cos kt_j\sin lt_j
-\sum_{k=1}^nb_k\sum_{j=1}^N\sin kt_j\sin lt_j
\]
heranziehen.
Nach Satz~\ref{skript:fourier:satzprodukte}
verschwindet die zweite und die dritte Summe 
und in der letzten Summe
verschwinden alle Terme ausser der Term mit $k=l$, für den die
innere Summe über $j$ den Wert $N/2$ hat.
Damit wird die Gleichung vereinfacht zu
\begin{align*}
0
&=
\sum_{j=1}^Ny_j\sin lt_j - \sum_{k=1}^{n-1}b_l\delta_{kl}\frac{N}2
=
\sum_{j=1}^Ny_j\sin lt_j - b_l\frac{N}2
\\
\Rightarrow\qquad
b_l
&=
\frac{2}{N}\sum_{j=1}^Ny_j\sin lt_j.
\end{align*}

\subsection{Zusammenstellung der Resultate}
Sei $N=2n$ eine gerade natürliche Zahl.
Eine $2\pi$-periodische Funktion $f(t)$ kann als trigonometrisches Polynom
der Form
\[
p_n(t)
=
a_0 + \sum_{k=1}^{n-1} (a_k\cos kt + b_k\sin kt) + a_n\cos nt
\]
derart approximiert werden, dass zu den Zeiten $t_j=2\pi j/N, j=1,\dots,N$ 
die Funktion und das trigonometrische Polynom übereinstimmen:
\[
f(t_j) = y_j = p_n(t_j).
\]
Dazu müssen die Koefizienten
\begin{align*}
a_0
&=
\frac{1}N
\sum_{j=1}^N y_j,
\\
a_k
&=
\frac{2}N
\sum_{j=1}^N y_j\cos t_j
=
\frac{2}N
\sum_{j=1}^N y_j\cos \frac{2\pi j}{N}&\text{für }k&=1,\dots,n
\\
\text{und}\qquad
b_k
&=
\frac{2}N
\sum_{j=1}^N y_j\sin t_j
=
\frac{2}N
\sum_{j=1}^N y_j\sin \frac{2\pi j}{N}&\text{für }k&=1,\dots,n-1,
\end{align*}
verwendet werden.

\subsection{Beispiel: Dreiecksfunktion\label{subsection:fourier:dreiecksfunktion}}
\begin{figure}
\centering
\includegraphics{chapters/6/dreieck.pdf}
\caption{Dreiecksfunktion ({\color{blue}blau}) approximiert mit
trigonometrischen Polynomen  $p_n(t)$ ({\color{red}rot})
mit verschiedenen Werten von $N=2n$.
\label{skript:fourier:beispiel}}
\end{figure}
\begin{table}
\centering
\setlength{\tabcolsep}{5pt}
\begin{tabular}{>{$}l<{$}>{$}r<{$}>{$}r<{$}>{$}r<{$}>{$}r<{$}>{$}r<{$}>{$}r<{$}>{$}r<{$}>{$}r<{$}}
&N=4&N=8&N=12&N=16&N=20&N=24&\dots&N=\infty\\
\hline
b_1&1& 0.85355& 0.82934& 0.82107& 0.81727& 0.81522&& 0.8105695\\
b_3& &-0.14645&-0.11111&-0.10124&-0.09704&-0.09484&&-0.0900633\\
b_5& &        & 0.05954& 0.04520& 0.04000& 0.03748&& 0.0324228\\
b_7& &        &        &-0.03249&-0.02519&-0.02207&&-0.0165422\\
b_9& &        &        &        & 0.02050& 0.01627&& 0.0100070\\
b_{11}&&      &        &        &        &-0.01413&&-0.0066989\\
%\hline
\end{tabular}
\caption{Nicht verschwindende Fourier-Koeffizienten der
Dreiecksfunktion~\eqref{skript:fourier:dreieck}
für verschiedene Werte von $N$.
In der Spalte ganz rechts unter $N=\infty$ die Werte für die
Fourierkoeffizienten der stetigen Fourier-Reihe
nach \eqref{fourier:normalekoeffizienten}.
\label{skript:fourier:dreieckkoef}}
\end{table}
Als Beispiel untersuchen wir die Approximation der
$2\pi$-periodischen Dreiecksfunktion, die auf dem Interval $[0,2\pi)$
durch
\begin{equation}
f(t)
=
\begin{cases}
\displaystyle t\cdot\frac{2}{\pi}    &\displaystyle \qquad 0\le t < \frac{\pi}2\\[8pt]
\displaystyle 2-t\cdot\frac{2}{\pi}  &\displaystyle \qquad \frac{\pi}2\le t < \frac{3\pi}2\\[8pt]
\displaystyle t\cdot\frac{2}{\pi} - 4&\displaystyle \qquad \frac{3\pi}2\le t <2\pi
\end{cases}
\label{skript:fourier:dreieck}
\end{equation}
gegeben ist,
mit Hilfe eines trigonometrischen Polynoms.
In Abbildung~\ref{skript:fourier:beispiel} ist die Dreiecksfunktion
hellblau dargestellt.

Weil die Funktion antisymmetrisch ist, verschwinden alle $a_k$-Koeffizienten.
Da die Funktion ausserdem symmetrisch ist bezüglich $\frac{\pi}2$ verschwinden
alle geraden $b_k$-Koeffizienten.
Für $N=4$ besteht $y$ nur aus vier Werten: $1$, $0$, $-1$ und $0$,
in diesem Fall muss die Funktion mit nur einem einzigen $\sin$-Term
mit dem Fourier-Koeffizienten $b_1$ darstellbar sein.
Tatsächlich ist $f(t) = \sin t$ an den Stellen $t_j=2\pi j/4$, $j=1,\dots,4$
(blau in Abbildung~\ref{skript:fourier:beispiel}),
dies wird in Abbildung~\ref{skript:fourier:beispiel} ganz oben gezeigt.
Erhöht man $N$, wird die Approximation immer besser, dies zeigen die
weiteren Graphiken in Abbildung~\ref{skript:fourier:beispiel}.

Die Berechnung der Fourier-Koeffizienten mit den Integralformeln
\eqref{fourier:normalekoeffizienten}
liefert für die Dreiecksfunktion~\eqref{skript:fourier:dreieck}
die Formel
\[
b_k = (-1)^{(k-1)/2}\frac{8}{\pi^2k^2}
\]
(siehe auch \eqref{skript:fourier:bkstetig})
für ungereade Werte von $k$.
Diese Werte sind in der letzten Spalte unter $N=\infty$ dargestellt.
Für zunehmendes $N$ konvergieren die diskreten Koeffizienten $b_k$ gegen 
diese stetigen Werte.


%
% vektorgeometrie.tex
%
% (c) 2018 Prof Dr Andreas Müller, Hochschule Rapperswil
%
\section{Vektorgeometrische Interpretation}
\rhead{Vektorschreibweise}
Die bisherigen rein analytischen Betrachtungen verdecken den geometrischen
Gehalt der bisher entwickelten Theorie.
In diesem Abschnitt soll daher zunächst eine vektorielle Darstellung
aufgebaut werden, die dann erlauben soll, einerseits die Formeln für die 
Fourierkoeffizienten geometrisch zu verstehen und andererseits auf
komplexere Situationen zu verallgemeinern.

\subsection{Vektoren}
Die Operationen zur Bestimmung der Fourier-Koeffizienten können in 
vektorieller Schreibweise etwas übersichtlicher dargestellt werden.
Zunächst fassen wir die Funktionswerte $y_j$ in einem Vektor
\begin{equation}
y = \begin{pmatrix}y_1\\\vdots\\y_N\end{pmatrix}
\end{equation}
zusamen.
Zur Berechnung der Fourier-Koeffizienten brauchen wir auch noch die
Werte der trigonometrischen Funktionen zu den Zeiten $t_j$, die wir
ebenfalls als Vektoren
\begin{align*}
c_0&=\begin{pmatrix}1\\\vdots\\1\end{pmatrix},
&
c_k&=\begin{pmatrix}\cos kt_1\\\vdots\\\cos kt_N\end{pmatrix},\;(k=1,\dots,n)
&&\text{und}
&
s_k&=\begin{pmatrix}\sin kt_1\\\vdots\\\sin kt_N\end{pmatrix},\;(k=1,\dots,n-1)
\end{align*}
schreiben.

\subsubsection{Skalarprodukt und Norm}
Die Fourier-Koeffizienten sind Summen von Produkten von Komponenten von $y$
und Werten von trigonometrischen Funktionen, die wir ebenfalls als Komponenten
der Vektoren $c_k$ und $s_k$ ansehen können.
Dies sieht aus wie das Skalarprodukt zweier Vektoren.

Wir verenden die folgende Notation.
Das Skalarprodukt zweier Vektoren $x$ mit Komponenten $x_j$ und $y$ mit
Komponenten $y_j$ ist
\begin{equation}
x\cdot y
=
\sum_{j=1}^N x_jy_j
=
x^ty.
\end{equation}
Die Norm eines Vektors ist
\begin{equation}
\| x\| = \sqrt{x\cdot x} = \sqrt{\sum_{j=1}^N x_j^2}.
\end{equation}
Die Entfernung zwischen zwei Vektoren ist dann $\| x-y\|$.
Die Fourier-Analyse löst das Problem, Koeffzienten $a_k$ und $b_k$
zu finden, so dass der Vektor $y$ der Funktionswerte $y_j$ und
der Vektor $p_n$ der Werte $p_n(t_j)$ des trigonometrischen Polynoms
möglichst nahe beeinander sind.
Genauer wurde 
\begin{equation}
L=L(a_0,a_1,\dots,a_n,b_1,\dots,b_{n-1})
=
\sum_{j=1}^N (y_j - p_n(t_j))^2
=
\| y - p_n\|^2
\qquad\text{mit}\quad
p_n=\begin{pmatrix}
p_n(t_1)\\\vdots\\p_n(t_N)
\end{pmatrix}
\end{equation}
als zu minimierendes Mass für den Abstand der beiden Vektoren
definiert.

Die Fourier-Koeffizienten können jetzt als Skalarprodukte geschrieben werden:
\begin{align*}
a_0
&=
\frac1N\sum_{j=1}^N y_j
=
\frac1N\sum_{j=1}^N 1\cdot y_j 
=
\frac1N c_0\cdot y,
\\
a_k
&=
\frac{2}{N}\sum_{j=1}^N
\cos kt_j \cdot y_j
=
\frac2N c_k\cdot y,
&k&=1,\dots n
\\
b_k
&=
\frac2N \sum_{j=1}^N \sin kt_j \cdot y_j
=
\frac2N s_k\cdot y,&k&=1,\dots,n-1
\end{align*}

\subsubsection{Rekonstruktion der Funktion}
Auch die Darstellung der Funktion kann man wieder als Skalarprodukt schreiben.
Dazu schreiben wir die Fourier-Koeffizienten und die Werte der
trigonometrischen Funtionen also Vektoren
\[
\begin{aligned}
a
&=
\begin{pmatrix}
a_0\mathstrut\\
a_1\mathstrut\\
b_1\mathstrut\\
a_2\mathstrut\\
b_2\mathstrut\\
\vdots\\
b_{n-1}\mathstrut\\
a_n\mathstrut
\end{pmatrix}
&&\text{und}
&
e(t)
&=
\begin{pmatrix}
1\\
\cos t\\
\sin t\\
\cos2t\\
\sin2t\\
\vdots\\
\sin(n-1)t\\
\cos nt
\end{pmatrix}.
\end{aligned}
\]
Damit wird 
\begin{align*}
p_n(t)
&=
a_0 + a_1\cos t + b_1\sin t + a_2\cos2t+b_2\sin2t+\dots+a_n\cos nt
\\
&=
\begin{pmatrix}
a_0\mathstrut\\
a_1\mathstrut\\
b_1\mathstrut\\
a_2\mathstrut\\
b_2\mathstrut\\
\vdots\\
b_{n-1}\mathstrut\\
a_n\mathstrut
\end{pmatrix}
\cdot
\begin{pmatrix}
1\\
\cos t\\
\sin t\\
\cos2t\\
\sin2t\\
\vdots\\
\sin(n-1)t\\
\cos nt
\end{pmatrix}
=
a\cdot e(t).
\end{align*}
Der Vektor $p_n$ besteht aus Funktionswerten $p_n(t_j)$, man kann dies
auch als Linearkombination der Vektoren $c_0$, $c_k$ und $s_k$ schreiben:
\begin{equation}
p_n = a_0c_0 + \sum_{k=1}^n a_kc_k + \sum_{k=1}^{n-1} b_ks_k.
\label{skript:fourier:pn}
\end{equation}

\subsubsection{Gleichungen für die Koeffizienten}
Sogar die Herleitung der Gleichungen zur Bestimmung der Fourierkoeffizienten
lässt sich in dieser vektoriellen Schreibweise kompakter durchführen.
Die Ableitung von $L$ nach einem Koeffizienten ist
\begin{align*}
\frac{\partial L}{\partial a_0}
&=
\frac{\partial}{\partial a_0} (y-p_n)\cdot (y-p_n)
=
-2(y-p_n)\cdot \frac{\partial p_n}{\partial a_0}
\\
\frac{\partial L}{\partial a_k}
&=
\frac{\partial}{\partial a_k} (y-p_n)\cdot (y-p_n)
=
-2(y-p_n)\cdot \frac{\partial p_n}{\partial a_k}
\\
\frac{\partial L}{\partial b_k}
&=
\frac{\partial}{\partial b_k} (y-p_n)\cdot (y-p_n)
=
-2(y-p_n)\cdot \frac{\partial p_n}{\partial b_k}.
\end{align*}
Die Ableitungen des Vektors $p_n$ nach den Koeffizienten sind
\begin{align*}
\frac{\partial p_n}{\partial a_0}
&=
\begin{pmatrix}
1\\
\vdots\\
1
\end{pmatrix}
=
c_0,&
\frac{\partial p_n}{\partial a_k}
&=
\begin{pmatrix}
\cos kt_1\\\vdots\\\cos kt_N
\end{pmatrix} = c_k,
&
\frac{\partial p_n}{\partial b_k}
&=
\begin{pmatrix}
\sin kt_j\\\vdots\\\sin kt_N,
\end{pmatrix}
\end{align*}
womit die Gleichungen zur Bestimmung der Fourierkoeffizienten zu
\begin{align*}
(y-p_n)\cdot c_0 &= 0&
(y-p_n)\cdot c_k &= 0&
(y-p_n)\cdot s_k &= 0
\end{align*}
werden.
Die Darstellung~\eqref{skript:fourier:pn} des Vektors $p_n$ erlaubt,
diese Gleichungen weiter zu vereinfachen, sie werden zu
\begin{equation}
\begin{aligned}
(y-p_n)\cdot c_0&=0 
&&\Rightarrow&
y\cdot c_0
&=
\biggl(a_0c_0 + \sum_{k=1}^n a_kc_k +\sum_{k=1}^{n-1}b_ks_k\biggr)\cdot c_0
\\
(y-p_n)\cdot c_k&=0 
&&\Rightarrow&
y\cdot c_k
&=
\biggl(a_0c_0 + \sum_{k=1}^n a_kc_k +\sum_{k=1}^{n-1}b_ks_k\biggr)\cdot c_k
\\
(y-p_n)\cdot s_k&=0 
&&\Rightarrow&
y\cdot s_k
&=
\biggl(a_0c_0 + \sum_{k=1}^n a_kc_k +\sum_{k=1}^{n-1}b_ks_k\biggr)\cdot s_k
\end{aligned}
\label{skript:fourier:skalarprodgl}
\end{equation}
Die Summen auf der rechten Seite können ausgewertet werden, wenn man die
die Skalarprodukte der Vektoren $c_k$ und $s_k$ kennt.

\subsubsection{Orthogonalität}
Die Aussagen von Satz~\ref{skript:fourier:orthogonalitaet1}
lassen sich jetzt in geometrische Form fassen.
\begin{satz}
Es gilt
\begin{align*}
c_k\cdot c_l
&=
\begin{cases}
N&\qquad k=l=0\\
\displaystyle\frac{N}2&\qquad k=l>0\\
0&\qquad\text{sonst}
\end{cases}
\\
s_k\cdot s_l
&=
\begin{cases}
\displaystyle\frac{N}2&\qquad k=l\\
0&\qquad\text{sonst}
\end{cases}
\\
c_k\cdot s_l
&=
0
\end{align*}
\label{skript:fourier:orthogonalitaet}
\end{satz}

\begin{proof}[Beweis]
Die genannten Skalarprodukte sind nichts anderes als die Summen in
Satz~\ref{skript:fourier:orthogonalitaet1}:
\begin{align*}
c_k\cdot c_l
&=
\sum_{j=1}^N \cos kt_j \cos lt_j
=
\begin{cases}
N&\qquad k=l=0\\
\displaystyle\frac{N}2&\qquad k=l>0\\
0&\qquad\text{sonst}
\end{cases}
\end{align*}
und analog für die anderen Skalarprodukte.
Die Aussage des Satzes ist daher nichts anders als eine geometrische
Umformulierung der Aussagen des
Satzes~\ref{skript:fourier:orthogonalitaet1}.
\end{proof}

In den Gleichungen~\eqref{skript:fourier:skalarprodgl} können jetzt
die Skalarprodukte berechnet werden.
Es bleibt jeweils nur ein einziger Term übrig, nämlich
\begin{equation*}
\begin{aligned}
y\cdot c_0 &= a_0 c_0\cdot c_0
&&\Rightarrow&
a_0 = \frac{1}{c_0\cdot c_0}c_0\cdot y = \frac1N c_0\cdot y
\\
y\cdot c_k &= a_k c_k\cdot c_k
&&\Rightarrow&
a_k = \frac{1}{c_k\cdot c_k}c_k\cdot y = \frac2N c_k\cdot y
\\
y\cdot s_k &= b_k s_k\cdot s_k
&&\Rightarrow&
b_k = \frac{1}{s_k\cdot s_k}s_k\cdot y = \frac2N s_k\cdot y,
\end{aligned}
\end{equation*}
wie früher auch schon gefunden.

\subsubsection{Die Identität von Parseval}
Die Relationen von
Satz~\ref{skript:fourier:orthogonalitaet1}
besagen, dass die Vektoren $c_k$ und $s_k$ orthogonal sind.
Wir wenden Sie auf das Skalarprodukt der Funktion $f$ mit sich selbst an.
\begin{align*}
f\cdot f
&=
a_0^2 c_0\,\cdot c_0
+
\sum_{k=1}^na_k^2 \,c_k\cdot c_k
+
\sum_{k=1}^{n-1} b_k^2\,s_k\cdot s_k
\\
&=
Na_0^2
+
\frac{N}2\sum_{k=1}^n a_k^2
+
\frac{N}2\sum_{k=1}^{n-1} b_k^2
=
\frac{N}2
\biggl(
2a_0^2
+
\sum_{k=1}^n a_k^2
+
\sum_{k=1}^{n-1} b_k^2
\biggr).
\end{align*}
Damit haben wir den folgenden Satz bewiesen:
\begin{satz}[Parseval]
\[
\|f\|^2
=
\sum_{j=1}^N y_j^2
=
\frac{N}2
\biggl(
2a_0^2
+
\sum_{k=1}^n a_k^2
+
\sum_{k=1}^{n-1} b_k^2
\biggr).
\]
\end{satz}
Die Parseval-Gleichung hat zur Konsequenz, dass eine Funktion mit nur
kleinen Werten auch nur kleine Fourier-Koeffizienten haben kann.
Sind zwei Funktionen $f$ und $g$ nahe beeinandern, dann unterscheiden
sich die Fourier-Koefizienten nur wenig.

Umgekehrt können wir auch schliessen, dass Terme mit kleine
Fourier-Koeffizienten nur zu geringen Unterschieden der Funktionen
führen.
Indem wir Terme mit kleinen Fourier-Koeffizienten weglassen, 
können wir die Approximation einer Funktion durch ein trigonometrisches
Polynom vereinfachen, ohne dass ein grosser Fehler entsteht.

\subsection{$2\pi$-Periodische Funktionen auf $\mathbb R$\label{subsection:fourier:stetig}}
Die eben vektoriell dargestellte Analyse diskreter periodischer Funktionen 
kann verallgemeinert werden auf die Analyse von Funktionen auf
anderen Definitionsgebieten.
Benötigt wird eine Familie von Basisfunktionen und ein Skalarprodukt
$\langle\;,\;\rangle$ derart, dass die Basisfunktionen $g_i$ bezüglich
dieses Skalarproduktes orthonormiert sind, dass also
\[
\langle g_i,g_j\rangle
=
\delta_{ij}
=
\begin{cases}
1&\qquad i=j\\
0&\qquad\text{sonst}.
\end{cases}
\]
Jede Linearkombination
\[
f = \sum_{i} \alpha_i g_i
\]
von Basisfunktionen kann ebenfalls mit dem Skalarprodukt rekonstruiert
werden.
Dazu berechnet man
\[
\langle g_i,f\rangle
=
\biggl\langle
g_i,\sum_j\alpha_jg_j
\biggr\rangle
=
\sum_j \langle g_i,\alpha_jg_j\rangle
=
\sum_j \alpha_j\delta_{ij}
=
\alpha_i.
\]
Das Skalarprodukt kann auch verwendet werden, um einen Abstand zwischen
Vektoren als
\[
\| f-g\|^2
=
\langle f-g,f-g\rangle
\]
zu definieren.

Dieselbe Situation lässt sich auch für $2\pi$-periodische Funktionen 
auf $\mathbb R$ herbeiführen.
Als Basisfunktionen kann man die Funktionen 
\begin{equation}
\frac{1}{\sqrt{2}},\; \cos kx,\; \sin lx\quad k>0
\label{fourier:basis}
\end{equation}
verwenden.
Das Skalarprodukt $\langle f,g\rangle$ muss linear in $f$ und $g$ sein.
Eine naheliegende Wahl ist
\[
\langle f, g\rangle
=
\frac{1}{\pi}\int_{-\pi}^{\pi} f(x)\,g(x)\,dx.
\]
Wir überprüfen, ob die Funktionen orthogonal sind:
\begin{align*}
\left\langle \frac1{\sqrt{2}},\frac1{\sqrt{2}}\right\rangle
&=
\frac1{\pi}
\int_{-\pi}^{\pi} \frac12\,dx
=
1
\\
\left\langle \frac1{\sqrt{2}},\cos kx\right\rangle
&=
\frac1{\pi}\int_{-\pi}^{\pi}
\frac1{\sqrt{2}}\cos kx
\,dx
=0
\\
\left\langle \frac1{\sqrt{2}},\sin kx\right\rangle
&=
\frac1{\pi}\int_{-\pi}^{\pi}
\frac1{\sqrt{2}}\sin kx
\,dx
=0
\\
\langle \cos kx,\cos lx\rangle
&=
\frac1{\pi}
\int_{-\pi}^\pi \cos kx\cos lx\,dx
\\
&=
\frac1{\pi}
\int_{-\pi}^\pi
\frac12\bigl(
\cos (k-l)x+\cos (k+l)x
\bigr)
\,dx
=
\begin{cases}
1&\qquad k=l\\
0&\qquad\text{sonst}
\end{cases}
\\
\langle \sin kx,\sin lx\rangle
&=
\frac1{\pi}
\int_{-\pi}^\pi \sin kx\,\sin lx\,dx
\\
&=
\frac1{\pi}
\int_{-\pi}^\pi
\frac12
\bigl(
\cos (k-l)x - \cos (k+l)x
\bigr)
\,dx
=
\begin{cases}
1&\qquad k=l\\
0&\qquad\text{sonst}
\end{cases}
\\
\langle \sin kx,\cos lx\rangle
&=
\frac1{\pi}
\int_{-\pi}^{\pi} 
\frac12\bigl(
\sin (k-l)x + \sin (k+l)x
\bigr)
\,dx
=0
\end{align*}
Zu einer $2\pi$-periodischen Funktion $f(x)$ kann man daher immer
die Koeffizienten
\begin{equation}
\begin{aligned}
\bar{a}_0&=\frac1{\pi\sqrt{2}}\int_{-\pi}f(x)\,dx
\\
a_k&=\frac1{\pi}\int_{-\pi}^\pi f(x)\cos kx\,dx
\\
b_k&=\frac1{\pi}\int_{-\pi}^\pi f(x)\sin kx\,dx
\end{aligned}
\label{fourier:normalekoeffizienten}
\end{equation}
berechnen.
Die Linearkombination
\begin{equation}
\tilde f(x)
=
\bar{a}_0\cdot\frac1{\sqrt{2}}
+ 
\sum_{k=1}^\infty (a_k\cos kx+b_k\sin kx)
\label{fourier:reihe}
\end{equation}
ist natürlich wieder eine $2\pi$-periodische Funktion.

Ist $f(x)$ eine Linearkombination von Funktionen~\eqref{fourier:basis},
dann sind nur endlich viele der Koeffizienten $\bar{a}_0$, $a_k$ und $b_k$
sind von $0$ verschieden und es gilt $f(x)=\tilde f(x)$, die Summe
\eqref{fourier:reihe} rekonstriert die Funktion $f(x)$ also exakt..

Für eine beliebige $2\pi$-periodische Funktion $f(x)$ ist die Funktion
$\tilde f(x)$ nach \eqref{fourier:reihe} im Allgemeinen eine unendliche
Reihe.
Die Reihe \eqref{fourier:reihe} heisst die Fourier-Reihe der Funktion 
$f(x)$.
\index{Fourier-Reihe}

In der Literatur wird $a_0$ meistens anders definiert, nämlich als
\[
a_0 = \frac1{\pi}\int_{-\pi}^{\pi} f(x)\,dx = \sqrt{2}\bar{a}_0
\qquad\Rightarrow\qquad
\bar{a}_0 = \frac{a_0}{\sqrt{2}}
\]
Der erste Term der Reihe~\eqref{fourier:basis} wird dann
\[
\bar{a}_0\cdot\frac1{\sqrt{2}}
=
\frac{a_0}{\sqrt{2}}\cdot\frac{1}{\sqrt{2}}
=
\frac{a_0}2
\]
und die Fourier-Reihe ist
\begin{equation}
\tilde f(x)
=
\frac{a_0}2
+
\sum_{k=1}^\infty (a_k\cos kx+b_k\sin kx).
\end{equation}

\subsubsection{Beispiel: Dreiecksfunktion}
Wir berechnen die Fourier-Koeffizienten für die Dreiecksfunktion
\eqref{skript:fourier:dreieck}, für die wir die diskreten Fourier-Koeffizienten
bereits in Abschnitt~\ref{subsection:fourier:dreiecksfunktion}
berechnet haben.
Wie im diskreten Fall folgt aus den Symmetrieeigenschaften, dass $a_k=0$
ist für alle $k$ und $b_k=0$ für alle geraden $k$.
Es bleibt also nur noch, die $b_k$ für ungerades $k$ zu bestimmen.
\begin{align}
b_k
&=
\frac{1}{\pi} \int_0^{2\pi} f(x) \sin kx\,dx
\notag
\\
&=
\frac{4}{\pi} \int_{0}^{\frac{\pi}2} f(x)\sin kx\,dx
=
\frac{4}{\pi} \int_{0}^{\frac{\pi}2} \frac2{\pi}x\sin kx\,dx
=
\frac{8}{\pi^2} \int_{0}^{\frac{\pi}2} x\sin kx\,dx
\notag
\\
&=
\frac{8}{\pi^2} \biggl(
\biggl[-\frac{x}{k}\cos kx\biggr]_0^{\frac{\pi}2} + \int_{0}^{\frac{\pi}2}\frac{1}{k}\cos kx\,dx
\biggr)
=
\frac{8}{\pi^2}
\biggl[
-\frac{x}{k}\cos kx +\frac{1}{k^2}\sin kx
\biggr]_0^{\frac{\pi}2}
\notag
\\
&=
\frac{8}{\pi^2} \biggl[ \frac{\sin kx - kx\cos kx}{k^2}\biggr]_0^{\frac{\pi}2}
=
\frac{8}{\pi^2 k^2} \biggl[\sin kx - kx\cos kx\biggr]_0^{\frac{\pi}2}
\notag
\\
&=
\frac{8}{\pi^2k^2}\biggl(
(-1)^{(k-1)/2} - k\frac{\pi}2\underbrace{\cos k\frac{\pi}2}_{\displaystyle=0}
\biggr)
=
\frac{8}{\pi^2k^2}(-1)^{(k-1)/2}.
\label{skript:fourier:bkstetig}
\end{align}
In Tabelle~\ref{skript:fourier:beispiel} werden die diskreten
Fourierkoeffizienten den hier gefundenen stetigen Koeffizienten
gegenübergestellt.

\subsection{Diskrete Fourier-Transformation}
Die Fourier-Koeffizienten $a_k$ und $b_k$ hängen linear von den
Funktionswerten $y_j$ ab.
Der Vektor der Fourier-Koeffizienten muss daher der Bildvektor des
Vektors $\vec y$ der Funktionswerte unter einer linearen Transformation
sein.
In diesem Abschnitt soll die diskrete Fourier-Transformation in etwas
mehr Detail hergeleitet werden.
Ausserdem soll gezeigt werden, wie die Fourier-Transformation
mit der schnellen Fouriertransformation
effizient berechnet werden kann, die in vielen Softwarepaketen zur
numerischen Berechnung implementiert ist.

\subsubsection{Transformationsmatrix}
Die Berechnung der Fourier-Koeffizienten ist eine lineare Operation
mit der $N\times N$-Matrix:
\[
A
=
\begin{pmatrix}
c_0^t\\
c_1^t\\
s_1^t\\
\vdots\\
c_{n-1}^t\\
s_{n-1}^t\\
c_n^t
\end{pmatrix}
=
\begin{pmatrix}
1           &1           &\dots &1            \\
\cos t_1    &\cos t_2    &\dots &\cos t_N     \\
\sin t_1    &\sin t_2    &\dots &\sin t_N     \\
\cos 2t_1   &\cos 2t_2   &\dots &\cos 2t_N    \\
\sin 2t_1   &\sin 2t_2   &\dots &\sin 2t_N    \\
\vdots      &\vdots      &\ddots&\vdots       \\
\sin(n-1)t_1&\sin(n-1)t_2&\dots &\sin(n-1)t_N \\
\cos nt_1   &\cos nt_2   &\dots &\cos nt_N    
\end{pmatrix}
\]
Die Orthogonalitätsrelationen von
Satz~\ref{skript:fourier:orthogonalitaet}
können jetzt neu geschrieben werden:
\begin{align*}
AA^t
&=
\begin{pmatrix}
c_0\cdot c_0&
	c_0\cdot c_1&
		c_0\cdot s_1&
			\dots&
				c_0\cdot c_{n-1}&
					c_0\cdot s_{n-1}&
						c_0\cdot c_n\\
c_1\cdot c_0&
	c_1\cdot c_1&
		c_1\cdot s_1&
			\dots&
				c_1\cdot c_{n-1}&
					c_1\cdot s_{n-1}&
						c_1\cdot c_n\\
s_1\cdot c_0&
	s_1\cdot c_1&
		s_1\cdot s_1&
			\dots&
				s_1\cdot c_{n-1}&
					s_1\cdot s_{n-1}&
						s_1\cdot c_n\\
\vdots	&\vdots	&\vdots	&\ddots	&\vdots	&\vdots	&\vdots	\\
c_{n-1}\cdot c_0&
	c_{n-1}\cdot c_1&
		c_{n-1}\cdot s_1&
			\dots&
				c_{n-1}\cdot c_{n-1}&
					c_{n-1}\cdot s_{n-1}&
						c_{n-1}\cdot c_n\\
s_{n-1}\cdot c_0&
	s_{n-1}\cdot c_1&
		s_{n-1}\cdot s_1&
			\dots&
				s_{n-1}\cdot c_{n-1}&
					s_{n-1}\cdot s_{n-1}&
						s_{n-1}\cdot c_n\\
c_n\cdot c_0&
	c_n\cdot c_1&
		c_n\cdot s_1&
			\dots&
				c_n\cdot c_{n-1}&
					c_n\cdot s_{n-1}&
						c_n\cdot c_n\\
\end{pmatrix}
\\
&=
\begin{pmatrix}
N     &0        &0        &\dots    &0        &0        &0        \\
0     &\frac{N}2&0        &\dots    &0        &0        &0        \\
0     &0        &\frac{N}2&\dots    &0        &0        &0        \\
\vdots&\vdots   &\vdots   &\ddots   &\vdots   &\vdots   &\vdots   \\
0     &0        &0        &\dots    &\frac{N}2&0        &0        \\
0     &0        &0        &\dots    &0        &\frac{N}2&0        \\
0     &0        &0        &\dots    &0        &0        &\frac{N}2
\end{pmatrix}.
\end{align*}
Bis auf die Faktoren $N$ und $\frac{N}2$ auf der Diagonalen ist
${\cal F}{\cal F}^t$ 
eine Diagonalmatrix.
Wir können die Matrix zu einer Einheitsmatrix machen, indem wir 
sie mit der Diagonalmatrix
\begin{equation}
D
=
\begin{pmatrix}
\sqrt{\frac1N}&0&\dots&0\\
0&\sqrt{\frac{2}{N}}&\dots&0\\
\vdots&\vdots&\ddots&\vdots\\
0&0&\dots&\sqrt{\frac{2}{N}}
\end{pmatrix}
\end{equation}
multiplizieren.
Wir schreiben
\begin{align*}
{\cal F}
&=
D\, A
\end{align*}
Wir nennen $\cal F$ die {\em Fourier-Matrix}.
\index{Fourier-Matrix}%
Die Fourier-Matrix $\cal F$ ist orthogonal, es gilt
\[
{\cal F}{\cal F}^t
=
DAA^tD^t
=
DD^tAA^t
=
E,
\]
wobei wir im letzten Schritt $D^t$ mit $AA^t$ vertauschen durften,
weil beide Diagonalmatrizen sind und damit vertauschen.
Insbesondere erhält $\cal F$ das Skalarprodukt, womit wir natürlich
nur die Parseval-Identität anders formuliert haben.

%
% komplex.tex
%
% (c) 2018 Prof Dr Andreas Müller, Hochschule Rapperswil
%
\subsubsection{Komplexe Fouriertransformation}
Bisher wurden alle Rechnungen nur mit reellen Zahlen durchgeführt.
Es stellt sich aber heraus, dass komplexe Zahlen für die Beschreibung
der Fourier-Transformation sehr viel praktischer sind.
Der Grund dafür ist die Eulersche Beziehung
\index{Euler-Gleichung}%
\[
e^{it} = \cos t + i \sin t
\]
und die Rechenregel
\[
e^{a+b}=e^a\cdot e^b
\qquad\Rightarrow\qquad
e^{ikt}=\cos kt+i\sin kt = (\cos t + i \sin t)^k
\]
für die Exponentialfunktion.
Im Folgenden gehen wir wieder von $N=2n$ aus.

Die reellen Fourier-Koeffizienten werden durch die Summen
\[
a_0
=
\frac{1}{N}\sum_{j=1}^N y_j,\quad
a_l
=
\frac{2}{N}\sum_{j=1}^N y_j \cos lt_j
\quad\text{und}\quad
b_l
=
\frac{2}{N}\sum_{j=1}^N y_j \sin lt_j
\]
berechnet.
Für $l=0$ liefert die Formel für $b_l$ trivialerweise $b_0=0$.
Für $l=n$ ist $\sin lt_j = \sin nt_j = \sin \pi j=0$, also verschwindet auch
die Summe für $b_n$.
Fassen wir $a_l$ und $b_l$ als Real- und Imaginärteil einer komplexen
Zahl auf, dann können wir 
\begin{align}
c_0
&=
a_0-ib_0
=
\frac{1}{N}
\sum_{j=1}^N y_j
=
\frac{1}{N}
\sum_{j=1} y_j e^{-0t_j}
\label{skript:complex:c0}
\\
c_l
&=
\frac12(a_l-ib_l)
=
\frac{1}{N} \sum_{j=1}^N y_j (\cos lt_j - i \sin lt_j)
=
\frac1{N} \sum_{j=1}^N y_j e^{-lt_j}
%=
%\frac{1}{N} \sum_{j=1}^N y_j e^{-2\pi ilj/N}
\label{skript:complex:cl}
\\
c_n
&=
\frac12(a_n-ib_n)
=
\frac{1}{N} \sum_{j=1}^N y_j (\cos nt_j  -i\sin nt_j)
=
\frac{1}{N} \sum_{j=1}^N y_j e^{-int_j}
=
\frac{1}{N} \sum_{j=1}^N y_j (-1)^j
\label{skript:complex:cn}
\end{align}
berechnen.

Für reelle Werte $y_j$ haben die Koeffizienten $c_l$ zusätzliche
Symmetrieeigenschaften.
Die komplex konjugierten Koeffizienten $\bar c_l$ ist
\[
\bar c_l
=
\overline{\sum_{j=0}^{N-1} y_j e^{-ilt_j}}
=
\sum_{j=0}^{N-1} y_j e^{ilt_j}
=
c_{-l}.
\]
Ausserdem ist $c_{N-l}=c_{-l}$ wegen $e^{\pm iNt_j}=1$.
Zusammen mit der Beziehung $\bar c_l=c_{-l}$, können wir die komplexen
Fourier-Koeffizienten auch verwenden, um die reellen Koeffizienten
\begin{equation}
\begin{aligned}
a_l
&=
c_l + c_{-l}
&&\text{und}&
ib_l
&=
-c_l + c_{-l}
\end{aligned}
\end{equation}
durch die komplexen Koeffizienten auszudrücken.

Auch die Rekonstruktion~\eqref{skript:fourier:rekonstruktion} ist
mit komplexen Zahlen darstellbar.
Dazu verwendet man 
\[
\cos x = \operatorname{Re} e^{ix} = \frac{e^{ix}+e^{-ix}}{2}
\qquad\text{und}\qquad
\sin x = \operatorname{Im} e^{ix} = \frac{e^{ix}-e^{-ix}}{2i}.
\]
Damit wird das trigonometrische Polynom
\begin{align*}
f(t_j)
&=
a_0
+\sum_{k=1}^{n-1} (a_k \cos kt_j + b_k \sin kt_j)
+a_n \cos nt_j
\\
&=
a_0
+\sum_{k=1}^{n-1} (a_k \operatorname{Re} e^{ikt_j} + b_k\operatorname{Im}e^{ikt_j})
+ a_n e^{int_j}
\\
&=
c_0
+
\sum_{k=0}^{n}
\frac12
\bigl(
(a_k-ib_k) e^{ikt_j}
+
(a_k+ib_k) e^{-ikt_j}
\bigr)
+
c_n e^{int_j}
\\
&=
c_0
+
\sum_{k=0}^{n}
\biggl(
\frac12
(a_k-ib_k) e^{ikt_j}
+
\frac12
(a_k+ib_k) e^{-ikt_j}
\biggr)
+
c_n e^{int_j}.
\intertext{Zusammen mit den Koeffizienten $c_{-l}$ folgt}
&=
c_0
+
\sum_{k=0}^{n-1} \bigl(c_k e^{ikt_j} + \bar{c}_k e^{-ikt_j})
+
c_n e^{int_j}
\\
&=
c_0
+
\sum_{k=1}^{n-1}(c_ke^{ikt_j} + c_{-k}e^{-ikt_j})
+
c_n e^{int_j}
=
\sum_{k=-n+1}^n c_k e^{ikt_j}
=
\sum_{k=0}^{N-1} c_k e^{ikt_j}.
\end{align*}
Damit sieht die Rekonstruktionsformel bis auf das Vorzeichen
im Exponenten gleich aus wie die
Transformationsformel~\eqref{skript:complex:cl}.
Diese Eigenschaft kann dazu verwendet werden, die Rücktransformation
im Wesentlichen mit demselben Code zu berechnen wie die Transformation.
Wir fassen die Resultate im folgenden Satz zusammen.

\begin{satz}
Sei $N=2n$ und $t_j=2\pi j/N$, und seien Funktionswerte $y_j\in\mathbb C$
geben.
Dann ist
\[
c_l = \sum_{j=1}^N y_j e^{-ilt_j}
\qquad
\Rightarrow
\qquad
y_j = \sum_{k=0}^{N-1} c_ke^{ikt_j}.
\]
Falls $y_j\in\mathbb R$, dann ist $c_{-l}=c_l$, $c_0,c_n\in\mathbb R$.
\end{satz}

\subsubsection{Spektrum}
\begin{figure}
\centering
\includegraphics{chapters/6/spektrum.pdf}
\caption{Graphische Darstellung von Funktion und Spektrum
\label{skript:komplex:spektrum}}
\end{figure}
Die Koeffizienten $c_k$ zeigen an, mit welcher Amplitude die Schwingung
$e^{ikt}$ in der Funktion $f(t)$ vertreten ist.
In Abbildung~\ref{skript:komplex:spektrum} ist oben die Funktion 
\[
f(t)
=
0.5 \cos 40t
+0.1\sin 80t
+0.12\sin 81t
+0.12\sin 82t
+0.1\sin 83t
+g(t)
\]
dargestellt, wobei $g(t)$ ein zufälliges Rauschsignal ist.
In der unteren Graphik sind die Beträge $|c_k|$ der Koeffizienten $c_k$ 
dargestellt.
Die Skala ist so gewählt, dass
\[
c_{\text{max}} = \max_{1\le k\le N} |c_k|.
\]
Man nennt $c_k$ auch das {\em Spektrum} von $f$.
\index{Spektrum}%
Die Extrema von $|c_k|$ machen die in der Funktion $f(t)$ vertretenen
Frequenzen bei $k=40$ und in der Umgebung von $k=80$ sichtbar.


%
% fft.tex -- Fast Fourier Transform
%
% (c) 2018 Prof Dr Andreas Müller, Hochschule Rapperswil
%
\subsubsection{Fast Fourier Transform}
Die komplexe Darstellung der diskreten Fourier-Transformation ermöglicht
eine Formulierung, in der die Berechnung der Koeffizienten wie auch die
Auswertung des trigonometrischen Polynoms sehr viel schneller erfolgen
kann, wenigstens wenn $N$ gerade ist.
Für die Bestimmung der $c_l$ verlangt die Berechnung der Produkte
$y_j e^{lt_j}$ für alle $j$ und $l$.
Ausserdem ist $t_j=2\pi j/n=lt_1$, die Exponentialfaktoren sind daher
$e^{lt_j}=(e^{t_1})^{lj}$.
Die Details dieses Algorithmus sollen hier nicht entwickelt werden,
es soll nur der reduzierte Rechenaufwand abgeschätzt werden.
Die Anzahl der Multiplikationen dominiert die Laufzeit der Berechnung,
daher soll $g(N)$ die Anzahl der Multiplikationen für eine
Fourier-Transformation mit $N$ Termen bezeichnen.

Da $N$ gerade ist, kann man die Summe zur Berechnung der Koeffizienten
\begin{align*}
c_l
&=
\sum_{j=0}^{N-1} y_j e^{ilt}
\intertext{aufteilen in gerade und ungerade Terme}
&=
\sum_{j=0}^{N/2-1} y_{2j} e^{ilt_{2j}}
+
\sum_{j=0}^{N/2-1} y_{2j+1} e^{ilt_{2j+1}}
\\
&=
\sum_{j=0}^{N/2-1} y_{2j} e^{ilj t_{2}}
+
e^{it_1}
\sum_{j=0}^{N/2-1} y_{2j+1} e^{ilj t_{2}},
\end{align*}
ist zu erkennen, dass jeder Summand eine Fourier-Transformation
mit der halben Anzahl von Datenpunkten und der doppelten Schrittweite
$t_2$ statt $t_1$ ist.
Der Aufwand für die Berechnung der Koeffizienten $c_l$ ist
also mit $g(N)=N/2 + g(N/2)$ Operationen möglich.
Wenn $N$ sogar eine Zweierpotenz ist, dann $2^m$, dann lässt sich dies
Idee iterieren, und die Summe aufteilen
\[
g(N)
=
g(2^m)
=
2^{m-1}
+
g(2^{m-1})
=
2^{m-1}
+
2^{m-2}
+
g(2^{m-2})
=
\dots
=
2^m+2^{m-1}+2^{m-2}+\dots + 1 + g(1)
=
O(Nm)
=
O(N\log(N)).
\]






\section*{Übungen}
\begin{uebungsaufgaben}
\item
\input{chapters/6/aufgabe1.tex}
\end{uebungsaufgaben}

