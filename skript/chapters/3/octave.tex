%
% octave.tex -- Differentialgleichungen mit Octave lösen
%
% (c) 2018 Prof Dr Andreas Müller, Hochschule Rapperswil
%
\subsection{Differentialgleichungen mit dem Computer lösen\label{section:octave}}
Im Abschnitt~\ref{subsection:euler} wurde das Euler-Verfahren
vorgestellt und seine Fehlerordnung untersucht.
Es wurde darauf hingewiesen, dass die lineare Fehlerordnung das Verfahren
für praktische Zwecke ungeeignet macht, da der Rechenaufwand für jede
zusätzliche Stelle Genauigkeit um einen Faktor 10 steigt.
In Computer-Programmen zur Lösung numerischer Probleme findet man daher
meistens nur Verfahren mit höherer Fehlerordnung wie zum Beispiel
das Verfahren von Runge-Kutta.

In {\em Octave} stellt die Funktion \texttt{lsode} die notwendige
Funktionalität auf eine Art zur Verfügung, bei der sich der
Benutzer kaum Gedanken über die im Hintergrund verwendeten
numerischen Techniken machen muss.
Tatsächlich verwendet \texttt{lsode} ein Mehrschrittverfahren, welches
Funktionswerte an verschiedenen Stellen verwendet, um den nächsten
Funktionswert zu berechnen.
Solche Verfahren werden ebenfalls in \cite{skript:mathsem-dgl} beschrieben.

\begin{table}
\centering
\begin{tabular}{ >{$}c<{$} >{$}c<{$} >{$}c<{$} >{$}c<{$} >{$}c<{$} }
x&e^{-x}&\text{Euler}&\text{beschleunigt}&\texttt{lsode}\\
\hline
0.0&1.0000000000&1.000000&1.00000000&1.0000000000\\
0.1&0.9048374180&0.\underline{904}792&0.\underline{9048374}3& 0.\underline{90483741}56\\
0.2&0.8187307530&0.\underline{818}648&0.\underline{8187307}8& 0.\underline{8187307}365\\
0.3&0.7408182206&0.\underline{740}707&0.\underline{7408182}5& 0.\underline{740818}1957\\
0.4&0.6703200460&0.\underline{670}185&0.\underline{6703200}9& 0.\underline{6703}199847\\
0.5&0.6065306597&0.\underline{606}378&0.\underline{606530}71& 0.\underline{6065306}196\\
0.6&0.5488116360&0.\underline{548}646&0.\underline{5488116}9& 0.\underline{5488116}160\\
0.7&0.4965853037&0.\underline{496}411&0.\underline{4965853}6& 0.\underline{496585}2810\\
0.8&0.4493289641&0.\underline{449}149&0.\underline{449329}02& 0.\underline{4493289}524\\
0.9&0.4065696597&0.\underline{406}386&0.\underline{406569}72& 0.\underline{40656965}50\\
1.0&0.3678794411&0.\underline{367}695&0.\underline{367879}50& 0.\underline{36787944}04\\
\hline
\end{tabular}
\caption{Vergleich der numerische Lösung der Beispieldifferentialgleichung 
gefunden mit verschiedenen Verfahren.
Die Spalten ``Euler'' und ``beschleunigt'' enthalten die mit 
Euler-Verfahren bzw.~mit dem beschleunigten Verfahren für $h=0.001$
gefundenen Werte.
In der letzten Spalte die Resultate, die mit der Octave-Funktion 
\texttt{lsode} gefunden wurden.
\label{skript:numerik:lsode}}
\end{table}
Zur Illustration soll hier nur gezeigt werden, wie man mit \texttt{lsode}
die Beispieldifferentialgleichung lösen kann.
Dazu braucht man zunächst eine Implementation $f(x,t)$ der
Differentialgleichung.
Im Beispiel ist dies die Funktion $f(y,x)=-y$, die wir als
\lstinputlisting[style=Matlab]{chapters/3/eulerbeispiel.m}
implementieren können.
Der Funktion \texttt{lsode} müssen wir dann ausserdem noch die
Anfangsbedingung übergeben und die $x$-Werte
übergeben, für die wir Funktionswerte erhalten wollen.
Der Code
\lstinputlisting[style=Matlab]{chapters/3/eulerloesung.m}
leistet dies.
Man beachte, dass wir nirgends angeben müssen, mit welcher
Schrittweite gerechnet werden soll.
\texttt{lsode} wählt geeignete Schritte, um eine voreingestellte
Genauigkeit von 6--8 Stellen zu erreichen.
Die gewünschte Genauigkeit kann über Optionen zur Funktion
\texttt{lsode} modifiziert werden, wir verweisen in diesem Zusammenhang
auf die {\em Octave}-Dokumentation.

Die Resultate der Berechnung sind in Tabelle~\ref{skript:numerik:lsode}
zusammengstellt.
Man beachte, dass am Ende des Intervalls eine Genauigkeit von
8 Stellen erreicht wird, besser als mit dem beschleunigten
Verfahren.
Nicht aus der Tabelle ersichtlich ist die Anzahl der Aufrufe der
Funktion \texttt{f}.
Für das Eulerverfahren und das beschleunigte Verfahren sind bei
der Schrittweite $h=0.001$ genau 1000 Aufrufe notwendig.
Die Funktion \texttt{lsode} ruft dagegen die Funktion \texttt{f}
nur 49 mal auf.
Da die Auswertung der Funktion \texttt{f} in der Praxis meistens
der dominierende rechnerische Aufwand ist, ist \texttt{lsode}
den anderen beiden Verfahren auf gerade zu spektakuläre Weise überlegen.








