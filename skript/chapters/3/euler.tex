%
% euler.tex -- Euler Verfahren
%
% (c) 2018 Prof Dr Andreas Müller, Hochschule Rapperswil
%
\subsection{Das Euler-Verfahren\label{subsection:euler}}
Es ist also eine Differentialgleichung der Form
\[
\frac{dx}{dt} = f(x),\qquad x\in\mathbb R^n
\]
mit der Anfangsbedingung $x(0)=x_0$ numerisch zu lösen.
Die Ableitung ist definiert als der Grenzwert des Differenzenquotienten
\[
\frac{dx(t)}{dt}
=
\lim_{\Delta t\to 0}
\frac{x(t+\Delta t)-x(t)}{\Delta t}.
\]
Für genügend kleines $\Delta t$ ist der Differenzenquotient daher eine
(hoffentlich) ausreichend genaue Approximation für die Ableitung,
also
\begin{equation}
\frac{x(t+\Delta t)-x(t)}{\Delta t} = f(x(t))
\qquad\Rightarrow\qquad
x(t+\Delta t) = x(t) + f(x(t))\cdot \Delta t.
\label{skript:euler:differenzenquotient}
\end{equation}
Diese Formel erlaubt, die Werte $x(t)$ zu berechnen für diejenigen
Zeitpunkte $t$, die Vielfache von $\Delta t$ sind.
Schreibt man $x_k = x(k\cdot \Delta t)$, dann gilt
\begin{align*}
x(0) &= x_0\\
x_k &= x_{k-1} + \Delta t \cdot f(x_{k-1}).
\end{align*}
Dieses Verfahren heisst das {\em Euler-Verfahren}.
\index{Euler-Verfahren}%

\subsubsection{Beispiel: $y'=-ay$}
Die lineare Differentialgleichung $y'=-ay$ mit der Anfangsbedingung 
$y(0)=y_0 > 0$ kann mit dem Euler-Verfahren mit Schritten der Länge $\Delta x=h$
wie folgt gelöst werden:
\begin{align*}
y(0)&=y_0\\
y_1&=y(h) = y_0 + h\cdot y'(0) = y_0  -ah y(0) = y_0(1 - ah)
\\
y_2&=y(2h) = y_1 + h\cdot y'(h) = y_1  -ah y_1 = y_1(1-ah)
\\
&\;\;\vdots
\\
y_k&= (1-ah)^k y_0.
\end{align*}
Um den Funktionswert $x=Nh$ zu errechnen, müssen also $N$ Schritte der
Länge $h=x/N$ durchgeführt werden, oder
\[
y(x) = y_0(1-ah)^N =
y_0
\biggl(
1-\frac{a}{N}
\biggr)^N.
\]
Erhöht man die Zahl der Schritte, erwartet man, dass das Resultat genauer
wird.
Tatsächlich ist der Grenzwert für $N\to\infty$
\[
\lim_{N\to\infty} a_0\biggl(1-\frac{a}{N}\biggr)^N
=
y_0
e^{-ax},
\]
was natürlich die bekannte exakte Lösung der Differentialgleichung ist.
Das Eulerverfahren reproduziert also für genügend kleine Schritte 
die exakte Lösung der Differentialgleichung beliebig genau.

Das Beispiel zeigt aber auch, dass für zu grosse Schrittweite das
Verfahren unsinnige Resultate liefert.
Wenn nämlich $ah>1$ ist, dann folgt, dass
$y_k=(1-ah)^k y_0  <0$ für ungerade $k$ und $y_k < 0$ für gerade $k$.
Die numersiche Lösung liefert also negative und positive Werte,
während die exakte Lösung $y_0e^{-ax}$ nie negativ wird.

\subsubsection{Genauigkeit}
Die Genauigkeit des Euler-Verfahrens ist ziemlich beschränkt.
Um $x(1)$ in $N$ Schritten zu berechnen, muss man Zeitschritte der
Länge $\Delta t=1/N$ verwenden.
Die Approximation \eqref{skript:euler:differenzenquotient} kann mit
Hilfe der Taylor-Reihe noch etwas verbessert werden, es gilt
\begin{align}
x(t+\Delta t)
&=
x(t) + \frac{d x(t) }{dt}\cdot \Delta t
+ \frac{1}{2!}\frac{d^2x(t)}{dt^2}\cdot \Delta t^2
+ \frac{1}{3!}\frac{d^3x(t)}{dt^3}\cdot \Delta t^3
+\dots
\label{skript:euler:taylor}
\\
&=
x(t) + \frac{d x(t) }{dt}\cdot \Delta t
+ O(1/N^2).
\notag
\end{align}
Der Fehler in jedem Einzelschritt ist also von der Grössenordnung $1/N^2$.
Nach $n$ Schritten verbleibt ein Fehler in der Grössenordnung von $1/N$,
\[
x_N = x(1) + O(1/N).
\]
Um eine zusätzliche Stelle Genauigkeit zu erreichen, um den Fehler um
den Faktor 10 zu reduzieren, muss die die Zahl der Schritte also zehnmal 
grösser werden.
Der Rechenaufwand für eine Steigerung der Genauigkeit um den Faktor
$a$ steigt also um den Faktor $1/a$, man sagt, das Euler-Verfahren ist
ein lineares Verfahren.

\subsubsection{Verfahren höherer Ordnung}
Die Genauigkeit des Euler-Verfahrens könnte dadurch gesteigert
werden, dass man nicht nur den ersten Term in der
Entwicklung~\eqref{skript:euler:taylor} verwendet, sondern auch
noch höhere Ableitungen berücksichtig.
Zum Beispiel könnte man die zweite Ableitung verwenden.
Diese lässt sich aus der Differentialgleichung mit Hilfe von
\[
\frac{d^2x}{dt^2}
=
\frac{d}{dt} \frac{dx(t)}{dt}
=
\frac{d}{dt} f(x(t))
=
Df(x(t))\cdot \frac{dx(t)}{dt}
=
Df(x(t))\cdot f(x(t))
\]
berechnen.
Allerdings ist es meistens ziemlich aufwendig die Ableitung $Df$ von $f$ zu
berechnen.

Wir untersuchen den Genauigkeitsgewinn an Hand des Beispiels $y'=-ay$.
Die Funktion $f(y) = -ay$ hat als Ableitung die Konstante $-a$ und
die zweite Ableitung von $y$ ist
\[
y''(x) = \frac{d}{dx} y'(x) = \frac{d}{dx} (-ay(x)) = -ay'(x)=a^2 y(x)
\]
Die Taylor-Entwicklung \eqref{skript:euler:taylor} für die Schrittweite
$h=\Delta x$ wird daher
\begin{align}
y(x+h)
&=
y(x) + y'(x)h + \frac12y''(x)h^2
\notag
\\
&= y(x) - ay(x)\Delta t + \frac12 a^2 y(x)
\notag
\\
&= y(x)\cdot \biggl(1-ah+\frac12 a^2h^2\biggr)
\notag
\\
\Rightarrow\qquad\qquad
y_k&=y_0\biggl(1-ah+\frac12 a^2h^2\biggr)^k
\label{skript:euler:beschleunigt}
\end{align}
Wir erwarten, dass dieses Verfahren deutlich genauere Resultate liefert,
und zwar erwarten wir, dass der Fehler von der Grössenordnung $1/N^2$ ist.
Man nennt dies ein {\em quadratisches} Verfahren.

\begin{table}
\setlength{\tabcolsep}{5pt}
\centering
\begin{tabular}{>{$}c<{$}>{$}c<{$}|>{$}c<{$}>{$}c<{$}>{$}c<{$}|>{$}c<{$}>{$}c<{$}>{$}c<{$}}
x&e^{-x}&\multicolumn{3}{c|}{Euler-Verfahren}&\multicolumn{3}{c}{beschleunigtes Verfahren \eqref{skript:euler:beschleunigt}}\\
&&h=0.1&h=0.01&h=0.001&h=0.1&h=0.01&h=0.001\\
\hline
0.0&1.00000000&1.000000&1.000000&1.000000&1.00000000&1.00000000&1.00000000\\
0.1&0.90483741&0.\underline{90}0000&0.\underline{90}4382&0.\underline{904}792&0.\underline{90}500000&0.\underline{90483}893&0.\underline{9048374}3\\
0.2&0.81873075&0.\underline{81}0000&0.\underline{81}7906&0.\underline{818}648&0.\underline{81}902500&0.\underline{81873}350&0.\underline{8187307}8\\
0.3&0.74081822&0.\underline{7}29000&0.\underline{7}39700&0.\underline{740}707&0.\underline{74}121762&0.\underline{7408}2195&0.\underline{7408182}5\\
0.4&0.67032004&0.\underline{6}56100&0.\underline{6}68971&0.\underline{670}185&0.\underline{670}80195&0.\underline{6703}2454&0.\underline{6703200}9\\
0.5&0.60653065&0.\underline{5}90490&0.\underline{60}5006&0.\underline{606}378&0.\underline{60}707576&0.\underline{60653}575&0.\underline{606530}71\\
0.6&0.54881163&0.\underline{5}31441&0.\underline{54}7156&0.\underline{548}646&0.\underline{54}940356&0.\underline{54881}716&0.\underline{5488116}9\\
0.7&0.49658530&0.\underline{4}78296&0.\underline{49}4838&0.\underline{496}411&0.\underline{49}721022&0.\underline{4965}9114&0.\underline{4965853}6\\
0.8&0.44932896&0.\underline{4}30467&0.\underline{44}7523&0.\underline{449}149&0.\underline{449}97525&0.\underline{4493}3500&0.\underline{449329}02\\
0.9&0.40656965&            0.387420&0.\underline{40}4731&0.\underline{406}386&0.\underline{40}722760&0.\underline{4065}7580&0.\underline{406569}72\\
1.0&0.36787944&0.\underline{3}48678&0.\underline{36}6032&0.\underline{367}695&0.\underline{36}854098&0.\underline{3678}8561&0.\underline{367879}50\\
\hline
\end{tabular}
\caption{Numerische Lösungen der Differentialgleichung $y'=-y$
mit dem Eulerverfahren und dem beschleunigten Verfahren 
\eqref{skript:euler:beschleunigt}.
Der Fehler im Eulerverfahren ist von der gleichen Grössenordnung wie $h$,
während er im beschleunigten Verfahren von der Grössenordnung von $h^2$ ist.
Korrekte Stellen unterstrichen.
\label{skript:euler:numerik}}
\end{table}
In Tabelle~\ref{skript:euler:numerik} sind die Resultate der numerischen
Rechnung mit beiden Verfahren mit verschiedenen Schrittweiten $h$
zusammengestellt.
Es sind nur die Werte von $y$ für ganzzahlige Vielfache von $0.1$
für $x$ gelistet.
Die korrekten Stellen sind jeweils unterstrichen und bestätigen, dass
beim Euler-Verfahren der Fehler von der Grössenordnung $h$, während
er beim beschleunigten Verfahren von der Grössenordnung $h^2$ ist.

Leider ist die praktische Durchführung dieses Verfahrens erschwert durch
die Notwendigkeit der Berechnung der Matrix $Df(x)$.
Es sind jedoch Verfahren gefunden worden, mit denen die Fehlerordnung
vergrössert werden kann, ohne dass es notwendig wird, die Ableitungen
von $f$ zu berechnen.
Das wohl bekannteste solche Verfahren ist das {\em Runge-Kutta-Verfahren},
welches von vierter Ordnung ist.
Es wird im Detail vorgeführt in \cite{skript:mathsem-dgl}.



