%
% lin.tex
%
% (c) 2018 Prof Dr Andreas Müller, Hochschule Rappreswil
%
\section{Linearisierung und Stabilität\label{section:linearisierung}}
\rhead{Linarisierung und Stabilität}
Die Technik der Phasendiagramme hat es sehr einfach gemacht, auf graphische
Art zwischen stabilen und instabilen Gleichgewichten zu unterscheiden.
Tatsächlich kann sie weiterentwickelt werden, in \cite{skript:mathsem-dgl}
wird gezeigt, wie man Gleichgewichtslösungen auch zweidimensionaler
Systeme ganz ähnlich graphisch analysieren kann.
In diesem Abschnitt soll aber nur gezeigt werden, wie auch mit relativ
bescheidenem Aufwand auch auf algebraischem oder analytischem Weg
Aussagen über die Stabilität von Gleichgewichtslösungen gefunden werden
können.

\subsection{Lineare Differentialgleichungen}
Eine lineare Differentialgleichung
\begin{equation}
\frac{dx}{dt}
=
Ax
\label{skript:lin:dgl}
\end{equation}
kann mit der Matrix-Exponentialfunktion
\begin{equation}
x(t) = e^{At} x(0)
=
\biggl(
\sum_{k=0}^\infty \frac{t^k}{k!}A^k
\biggr) x(0)
=
\biggl(
1+ At + \frac{A^2t^2}{2!} +\frac{A^3t^3}{3!}+\dots
\biggl)\cdot x(0)
\label{skript:lin:potenzreihe}
\end{equation}
gelöst werden.

Für eine diagonalisierbare Matrix $A$ ist die Berechnung der
Potenzreihe~\eqref{skript:lin:potenzreihe} sehr viel einfacher, da
gilt
\begin{equation}
A^k
=
\begin{pmatrix}
\lambda_1&\dots &0        \\
\vdots   &\ddots&\vdots   \\
0        &\dots &\lambda_n
\end{pmatrix}^k
=
\begin{pmatrix}
\lambda_1^k&\dots &0        \\
\vdots     &\ddots&\vdots   \\
0          &\dots &\lambda_n^k
\end{pmatrix}
\qquad\Rightarrow\qquad
e^{tA}
=
\begin{pmatrix}
e^{\lambda_1 t} & \dots  & 0             \\
\vdots          & \ddots &\vdots         \\
0               & \dots  &e^{\lambda_n t}
\end{pmatrix}.
\label{skript:lin:expreihe}
\end{equation}
Insbesondere lässt sich damit sehr einfach entscheiden, ob wie Lösungen
für zunehmendes $t$ anwachsen oder gegen $0$ konvergieren.
Ist $\lambda_j = a_j + ib_j$ die Aufteilung in Real- und Imaginärteil,
dann ist
\[
e^{\lambda_jt} = e^{a_jt}(\cos b_jt +i\sin b_jt)
\qquad\Rightarrow\qquad
|e^{\lambda_j t}| = e^{a_jt}.
\]
Ist $a_j < 0$, dann konvergiert $e^{\lambda_jt}$ gegen $0$.
Da die $0$-Lösung eine Gleichgewichtslösung von~\eqref{skript:lin:dgl}
ist, kann gefolgert werden, dass diese Lösung genau dann stabil ist, 
wenn alle Eigenwerte von $A$ negativen Realteil haben.

Natürlich lässt sich nicht jede Matrix diagonalisieren, aber man kann
mindestens eine Basis finden, in der $A$ Jordan-Normalform hat, also aus
\index{Jordan-Normalform}%
Blöcken der Form
\[
A_{\lambda}
=
\begin{pmatrix}
\lambda&   1   &       &       &       \\
       &\lambda&   1   &       &       \\
       &       &\lambda&\ddots &       \\
       &       &       &\ddots &    1  \\
       &       &       &       &\lambda
\end{pmatrix}
=
\begin{pmatrix}
\lambda&       &       &       \\
       &\lambda&       &       \\
       &       &\ddots &       \\
       &       &       &\lambda
\end{pmatrix}
+
\begin{pmatrix}
   0   &   1   &       &       \\
       &   0   &\ddots &       \\
       &       &\ddots &   1   \\
       &       &       &   0
\end{pmatrix}
=
\lambda E + N.
\]
Die Matrizen $\lambda E$ und $N$ vertauschen, so dass die
Produkteigenschaft der Exponentialfunktion verwendet werden kann,
es gilt
\begin{equation}
e^{A_\lambda t}
=
e^{\lambda tE}\cdot e^{Nt}
=
e^{\lambda t} e^{Nt}.
\label{skript:lin:faktorisierung}
\end{equation}
Bezeichnet man mit $N_i$ die Matrix, die in der $i$-ten Nebendiagonale
Einsen enthält und sonst nur Nullen, dann ist $N=N_1$.
Die Potenzen von $N$ können dann sofort in der Form $N^k=N_k$ ausgedrückt
werden, und damit kann man auch
\begin{equation}
e^{Nt}
=
\begin{pmatrix}
1&t&\frac{t^2}{2!}&\frac{t^3}{3!}& \dots & \frac{t^{n-1}}{(n-1)!}\\
0&1&       t      &\frac{t^2}{2!}& \dots & \frac{t^{n-2}}{(n-2)!}\\
      &\ddots&\ddots& \ddots     &       & \vdots                \\
      &      &\ddots& \ddots     &\ddots & \vdots                \\
      &      &      &         0  &    1  & t \\
      &      &      &            &    0  & 1
\end{pmatrix}
\end{equation}
sofort berechnen.
Der Faktor $e^{Nt}$ von \eqref{skript:lin:faktorisierung}
hängt nicht von $\lambda$ ab und wächste höchstens mit $t^{n-1}$ an.
Dies bedeutet, dass das Stabilitätskriterium an die Eigenwerte von $A$
weiterhin zutrifft: Die Gleichgewichtslösung $x=0$ ist genau dann stabil,
wenn alle Eigenwerte von $A$ negativen Realteil haben.

\subsection{Linearisierung\label{subsection:linearisierung}}
Ist $x_0$ ein kritischer der Punkt der autonomen Differentialgleichung
\[
\frac{dx}{dt} = f(x),
\]
dann können wir mit Hilfe einer Koordinatentransformation
\[
x = x_0 + \tilde x
\qquad
\text{und}
\qquad
\tilde f(\tilde x) = f(x_0 + \tilde x)
\]
die Differentialgleichung in die Form
\[
\frac{d\tilde x}{dt}
=
\tilde f(\tilde x)
\]
bringen, welche $0$ als Gleichgewichtslösung hat.
Im Folgenden schreiben wir wieder $f$ und $x$ für $\tilde f$ und $\tilde x$.

Da $f(0)=0$ gilt, kann $f(x)$ in eine Potenzreihe der Form
\[
f(x) = Ax + O(|x|^2)
\]
entwickelt werden.
Darin ist $A$ die Ableitungsmatrix
\[
A=(a_{ij})=Df
\qquad\text{mit}\qquad
(Df)_{ij}
=
\frac{\partial f_i(x)}{\partial x_j} = a_{ij}
\]
der Funktion $f$.
Dies ist die {\em linearisierte} Version der
Differentialgleichung~\eqref{skript:lin:dgl}.

In unmittelbarer Nähe des Gleichgewichtspunktes $0$ sind analog wie
bei der linearen Differentialgleichung~\eqref{skript:lin:dgl}
wieder die Eigenwerte der Matrix $A=Df$ darüber entscheidend, ob der
Gleichgewichtspunkt stabil ist.
$x_0$ ist genau dann ein stabilier Gleichgewichtspunkt, wenn alle
Eigenwerte der Ableitungsmatrix $Df$ negativ sind.








