%
% lin.tex
%
% (c) 2018 Prof Dr Andreas Müller, Hochschule Rappreswil
%
\section{Linearisierung und Stabilität\label{section:linearisierung}}
\rhead{Linarisierung und Stabilität}
Die Technik der Phasendiagramme hat es sehr einfach gemacht, auf graphische
Art zwischen stabilen und instabilen Gleichgewichten zu unterscheiden.
Tatsächlich kann sie weiterentwickelt werden, in \cite{skript:mathsem-dgl}
wird gezeigt, wie man Gleichgewichtslösungen auch zweidimensionaler
Systeme ganz ähnlich graphisch analysieren kann.
In diesem Abschnitt soll aber nur gezeigt werden, wie auch mit relativ
bescheidenem Aufwand auch auf algebraischem oder analytischem Weg
Aussagen über die Stabilität von Gleichgewichtslösungen gefunden werden
können.

\subsection{Lineare Differentialgleichungen}
Eine lineare Differentialgleichung
\begin{equation}
\frac{dx}{dt}
=
Ax
\label{skript:lin:dgl}
\end{equation}
kann mit der Matrix-Exponentialfunktion
\begin{equation}
x(t) = e^{At} x(0)
=
\biggl(
\sum_{k=0}^\infty \frac{t^k}{k!}A^k
\biggr) x(0)
=
\biggl(
1+ At + \frac{A^2t^2}{2!} +\frac{A^3t^3}{3!}+\dots
\biggl)\cdot x(0)
\label{skript:lin:potenzreihe}
\end{equation}
gelöst werden.

Für eine diagonalisierbare Matrix $A$ ist die Berechnung der
Potenzreihe~\eqref{skript:lin:potenzreihe} sehr viel einfacher, da
gilt
\begin{equation}
A^k
=
\begin{pmatrix}
\lambda_1&\dots &0        \\
\vdots   &\ddots&\vdots   \\
0        &\dots &\lambda_n
\end{pmatrix}^k
=
\begin{pmatrix}
\lambda_1^k&\dots &0        \\
\vdots     &\ddots&\vdots   \\
0          &\dots &\lambda_n^k
\end{pmatrix}
\qquad\Rightarrow\qquad
e^{tA}
=
\begin{pmatrix}
e^{\lambda_1 t} & \dots  & 0             \\
\vdots          & \ddots &\vdots         \\
0               & \dots  &e^{\lambda_n t}
\end{pmatrix}.
\label{skript:lin:expreihe}
\end{equation}
Insbesondere lässt sich damit sehr einfach entscheiden, ob wie Lösungen
für zunehmendes $t$ anwachsen oder gegen $0$ konvergieren.
Ist $\lambda_j = a_j + ib_j$ die Aufteilung in Real- und Imaginärteil,
dann ist
\[
e^{\lambda_jt} = e^{a_jt}(\cos b_jt +i\sin b_jt).
\qquad\Rightarrow\qquad
|e^{\lambda_j t}| = e^{a_jt}
\]
Ist $a_j < 0$, dann konvergiert $e^{\lambda_jt}$ gegen $0$.
Da die $0$-Lösung eine Gleichgewichtslösung von~\eqref{skript:lin:dgl}
ist kann gefolgert werden, dass diese Lösung genau dann stabil ist, 
wenn alle Eigenwerte von $A$ negativen Realteil ist.

\subsection{Linearisierung}













