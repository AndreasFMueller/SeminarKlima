%
% dgl.tex
%
% (c) 2018 Prof Dr Andreas Müller, Hochschule Rapperswil
%
\section{Grundlagen}
\rhead{Grundlagen}
Eine Differentialgleichung ist eine Beziehung zwischen einer Funktion und
ihren Ableitungen.
Wir betrachten Funktionen der Zeit $t$ mit Werten in $\mathbb R^n$
und schreiben sie $x(t)$.
Sei $f$ eine Funktion
\[
f\colon \mathbb R^n\times \mathbb R \to \mathbb R^n: (x,t) \mapsto f(x,t).
\]

\begin{definition}
Eine Funktion $x(t)$ heisst Lösung der Differentialgleichung
\begin{equation}
\frac{dx}{dt} = f(x,t)
\label{skript:dgl:dgldef}
\end{equation}
zur Anfangsbedingung $x_0$, wenn gilt $x(0)=x_0$ und
\[
\frac{dx}{dt} = f(x(t),t)
\]
für alle $t>0$.
\end{definition}

Unter einigermassen milden Bedingungen an die Funktion $f(x,t)$ ist
sichergestellt, dass eine Differentialgleichung immer eine Lösung hat.

\subsection{Autonome Differentialgleichungen}
Wenn die Funktion $f$ von der Zeit abhängt, wird es im allgemeinen
keine konstanten Lösungen geben.
Für die Klimadiskussion sind wir allerdings daran interessiert, ob
ein Modell Lösungen hat, die sich mit der Zeit nicht ändern.
Solche Lösungen zeigen uns, dass wir alle kurzfristigen
Schwankungen, die wir dem Wetter zuordnen würden, ausgemittelt haben.

\begin{definition}
Eine Differentialgleichung der Form~\eqref{skript:dgl:dgldef}
heisst {\em autonom},
\index{autonom}%
wenn die Funktion $f$ nicht von der Zeit abhängt.
Eine autonome Differentialgleichung kann als
\[
\frac{dx}{dt} = f(x)
\]
geschrieben werden.
\end{definition}

Die Forderung, dass die Differentialgleichung autonom sein soll, ist
allerdings auf triviale Art zu erfüllen, indem man zu einer neuen
unabhängigen Variablen übergeht und die bisherige Zeitvariable 
als letzte Komponente der Funktion $x(t)$ hinzufügt.
Wir schreiben die Lösungsfunktionen als
\begin{align*}
x(t)
&=
\begin{pmatrix}
x_1(t) \\ \vdots \\x_n(t)
\end{pmatrix}
&&\text{und erweitern dies zu}&
\bar x(s)
=
\begin{pmatrix}
x_1(s) \\ \vdots \\ x_n(s) \\ s
\end{pmatrix}
\in\mathbb R^{n+1}.
\end{align*}
Die rechte Seite der Differentialgleichung, also die Funktion $f(x,t)$
schreiben wir
\begin{align*}
f(x,t) 
&=
f(x_1,\dots,x_n,t)
&&\text{mit Anfangsbedingung}&
x_0 
&=
\begin{pmatrix} x_{01}\\\vdots \\ x_{0n}\end{pmatrix}
\\
\intertext{und erweitern dies nun zu einer Funktion $\bar f$ für eine
autonome Differentialgleichung für $\bar x$}
\bar f(\bar x)
&=
\begin{pmatrix}
f_1(\bar x_1,\dots,\bar x_n,\bar x_{n+1})\\
\vdots\\
f_n(\bar x_1,\dots,\bar x_n,\bar x_{n+1})\\
1
\end{pmatrix}
&&\text{mit Anfangsbedingung}&
\bar x_0
&=
\begin{pmatrix} x_{01} \\ \vdots \\ x_{0n} \\ 0\end{pmatrix}.
\end{align*}
Die Differentialgleichung für $\bar x$ ist
\begin{equation}
\frac{d\bar x}{ds}
=
\bar f(\bar x),
\label{skript:dgl:autodgl}
\end{equation}
dies ist offensichtlich eine autonome Differentialgleichung.
Die letzte Komponenten von \eqref{skript:dgl:autodgl} ist die
Differentialgleichung für $\bar x_{n+1}$
\[
\frac{d\bar x_{n+1}}{ds} = 1
\]
mit der Anfangsbedingung $x_{n+1}(0)=0$, sie hat die Lösung
$\bar x_{n+1}(s)=s$.
Die Koordinate $\bar x_{n+1}$ ist also nichts anderes als die
ursprüngliche Zeitkoordinate.
Aus der Lösung $\bar x(s)$ der autonomen Differentialgleichung
kann die Lösung der ursprünglichen Differentialgleichung gewonnen
werden, indem man einfach die letzte Koordinate weglässt:
\[
x(t)
=
\begin{pmatrix}
\bar x_1(t) \\ \vdots \\ \bar x_n(t)
\end{pmatrix}.
\]
Der Übergang zur autonomen Differentialgleichung erhöht die Dimension
des Vektors.
Dadurch wird die Diskussion kritischer Punkte und Gleichgewichtslösungen
leider nicht vereinfacht.
Statt eine Differentialgleichung nachträglich autonom zu machen
ist daher im allgemeinen anzustreben, dass sie von vornherein
autonom ist.
In den nachfolgenden Beispielen gehen wir daher immer von autonomen
Differentialgleichungssystemen aus.



