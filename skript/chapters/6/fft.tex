%
% fft.tex -- Fast Fourier Transform
%
% (c) 2018 Prof Dr Andreas Müller, Hochschule Rapperswil
%
\subsubsection{Fast Fourier Transform}
Die komplexe Darstellung der diskreten Fourier-Transformation ermöglicht
eine Formulierung, in der die Berechnung der Koeffizienten wie auch die
Auswertung des trigonometrischen Polynoms sehr viel schneller erfolgen
kann, wenigstens wenn $N$ gerade ist.
Für die Bestimmung der $c_l$ verlangt die Berechnung der Produkte
$y_j e^{lt_j}$ für alle $j$ und $l$.
Ausserdem ist $t_j=2\pi j/n=lt_1$, die Exponentialfaktoren sind daher
$e^{lt_j}=(e^{t_1})^{lj}$.
Die Details dieses Algorithmus sollen hier nicht entwickelt werden,
es soll nur der reduzierte Rechenaufwand abgeschätzt werden.
Die Anzahl der Multiplikationen dominiert die Laufzeit der Berechnung,
daher soll $g(N)$ die Anzahl der Multiplikationen für eine
Fourier-Transformation mit $N$ Termen bezeichnen.

Da $N$ gerade ist, kann man die Summe zur Berechnung der Koeffizienten
\begin{align*}
c_l
&=
\sum_{j=0}^{N-1} y_j e^{ilt}
\intertext{aufteilen in gerade und ungerade Terme}
&=
\sum_{j=0}^{N/2-1} y_{2j} e^{ilt_{2j}}
+
\sum_{j=0}^{N/2-1} y_{2j+1} e^{ilt_{2j+1}}
\\
&=
\sum_{j=0}^{N/2-1} y_{2j} e^{ilj t_{2}}
+
e^{it_1}
\sum_{j=0}^{N/2-1} y_{2j+1} e^{ilj t_{2}},
\end{align*}
ist zu erkennen, dass jeder Summand eine Fourier-Transformation
mit der halben Anzahl von Datenpunkten und der doppelten Schrittweite
$t_2$ statt $t_1$ ist.
Der Aufwand für die Berechnung der Koeffizienten $c_l$ ist
also mit $g(N)=N/2 + g(N/2)$ Operationen möglich.
Wenn $N$ sogar eine Zweierpotenz ist, dann $2^m$, dann lässt sich dies
Idee iterieren, und die Summe aufteilen
\[
g(N)
=
g(2^m)
=
2^{m-1}
+
g(2^{m-1})
=
2^{m-1}
+
2^{m-2}
+
g(2^{m-2})
=
\dots
=
2^m+2^{m-1}+2^{m-2}+\dots + 1 + g(1)
=
O(Nm)
=
O(N\log(N)).
\]


