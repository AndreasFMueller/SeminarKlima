%
% vektorgeometrie.tex
%
% (c) 2018 Prof Dr Andreas Müller, Hochschule Rapperswil
%
\section{Vektorgeometrische Interpretation}
\rhead{Vektorschreibweise}
\subsection{Vektoren}
Die Operationen zur Bestimmung der Fourier-Koeffizienten können in 
vektorieller Schreibweise etwas übersichtlicher dargestellt werden.
Zunächst fassen wir die Funktionswerte $y_j$ in einem Vektor zusamen.
\begin{equation}
y = \begin{pmatrix}y_1\\\vdots\\y_N\end{pmatrix}
\end{equation}
Zur Berechnung der Fourier-Koeffizienten brauchen wir auch noch die
Werte der trigonometrischen Funktionen zu den Zeiten $t_j$, die wir
ebenfalls als Vektoren
\begin{align*}
c_0&=\begin{pmatrix}0\\\vdots\\0\end{pmatrix},
&
c_k&=\begin{pmatrix}\cos kt_1\\\vdots\\\cos kt_N\end{pmatrix},\;(k=1,\dots,n)
&&\text{und}
&
s_k&=\begin{pmatrix}\sin kt_1\\\vdots\\\sin kt_N\end{pmatrix},\;(k=1,\dots,n-1)
\end{align*}
schreiben.
Die Fourier-Koeffizienten können jetzt als Skalarprodukte geschrieben werden:
\begin{align*}
a_0 &=\frac1N c_0\cdot y,&
a_k &=\frac2N c_k\cdot y,\;(k=1,\dots,n),&
b_k &=\frac2N s_k\cdot y,\;(k=1,\dots,n-1).
\end{align*}

\subsubsection{Rekonstruktion der Funktion}
Auch die Darstellung der Funktion kann man wieder als Skalarprodukt schreiben.
Dazu schreiben wir die Fourier-Koeffizienten und die Werte der
trigonometrischen Funtionen also Vektoren
\begin{align*}
a
&=
\begin{pmatrix}
a_0\mathstrut\\
a_1\mathstrut\\
b_1\mathstrut\\
a_2\mathstrut\\
b_2\mathstrut\\
\vdots\\
b_{n-1}\mathstrut\\
a_n\mathstrut
\end{pmatrix}
&&\text{und}
&
e(t)
&=
\begin{pmatrix}
1\\
\cos t\\
\sin t\\
\cos2t\\
\sin2t\\
\vdots\\
\sin(n-1)t\\
\cos nt
\end{pmatrix}.
\end{align*}
Damit wird 
\[
p(t) = a\cdot e(t)
\]

\subsubsection{Orthogonalität}
Die Aussagen von Satz~\ref{skript:fourier:orthogonalitaet1}
lassen sich jetzt in geometrische Form fassen.
\begin{satz}
Es gilt
\begin{align*}
c_k\cdot c_l
&=
\begin{cases}
N&\qquad k=l=0\\
\displaystyle\frac{N}2&\qquad k=l>0\\
0&\qquad\text{sonst}
\end{cases}
\\
s_k\cdot s_l
&=
\begin{cases}
\displaystyle\frac{N}2&\qquad k=l\\
0&\qquad\text{sonst}
\end{cases}
\\
c_k\cdot s_l
&=
0
\end{align*}
\label{skript:fourier:orthogonalitaet}
\end{satz}

\begin{proof}[Beweis]
Die genannten Skalarprodukte sind nichts anderes als die Summen in
Satz~\ref{skript:fourier:orthogonalitaet1}:
\begin{align*}
c_k\cdot c_l
&=
\sum_{j=1}^N \cos kt_j \cos lt_j
=
\begin{cases}
N&\qquad k=l=0\\
\displaystyle\frac{N}2&\qquad k=l>0\\
0&\qquad\text{sonst}
\end{cases}
\end{align*}
und analog für die anderen Skalarprodukte.
Die Aussage des Satzes ist daher nichts anders als eine geometrische
Umformulierung der Aussagen des
Satzes~\ref{skript:fourier:orthogonalitaet1}.
\end{proof}

\subsubsection{Die Identität von Parseval}
Die Relationen von
Satz~\ref{skript:fourier:orthogonalitaet1}
besagen, dass die Vektoren $c_k$ und $s_k$ orthogonal sind.
Wir wenden Sie auf das Skalarprodukt der Funktion $f$ mit sich selbst an.
\begin{align*}
f\cdot f
&=
a_0^2 c_0\,\cdot c_0
+
\sum_{k=1}^na_k^2 \,c_k\cdot c_k
+
\sum_{k=1}^{n-1} b_k^2\,s_k\cdot s_k
\\
&=
Na_0^2
+
\frac{N}2\sum_{k=1}^n a_k^2
+
\frac{N}2\sum_{k=1}^{n-1} b_k^2
=
\frac{N}2
\biggl(
2a_0^2
+
\sum_{k=1}^n a_k^2
+
\sum_{k=1}^{n-1} b_k^2
\biggr)
\end{align*}
Damit haben wir den folgenden Satz bewiesen:
\begin{satz}[Parseval]
\[
\|f\|^2
=
\sum_{j=1}^N y_j^2
=
\frac{N}2
\biggl(
2a_0^2
+
\sum_{k=1}^n a_k^2
+
\sum_{k=1}^{n-1} b_k^2
\biggr)
\]
\end{satz}

\subsection{Fourier-Transformation}
\subsubsection{Fourier-Transformation}
Die Berechnung der Fourier-Koeffizienten ist eine lineare Operation
mit der $N\times N$-Matrix:
\[
A
=
\begin{pmatrix}
c_0^t\\
c_1^t\\
s_1^t\\
\vdots\\
c_{n-1}^t\\
s_{n-1}^t\\
c_n^t
\end{pmatrix}
=
\begin{pmatrix}
1           &1           &\dots &1            \\
\cos t_1    &\cos t_2    &\dots &\cos t_N     \\
\sin t_1    &\sin t_2    &\dots &\sin t_N     \\
\cos 2t_1   &\cos 2t_2   &\dots &\cos 2t_N    \\
\sin 2t_1   &\sin 2t_2   &\dots &\sin 2t_N    \\
\vdots      &\vdots      &\ddots&\vdots       \\
\sin(n-1)t_1&\sin(n-1)t_2&\dots &\sin(n-1)t_N \\
\cos nt_1   &\cos nt_2   &\dots &\cos nt_N    
\end{pmatrix}
\]
Die Orthogonalitätsrelationen von
Satz~\ref{skript:fourier:orthogonalitaet}
können jetzt neu geschrieben werden:
\begin{align*}
AA^t
&=
\begin{pmatrix}
c_0\cdot c_0&
	c_0\cdot c_1&
		c_0\cdot s_1&
			\dots&
				c_0\cdot c_{n-1}&
					c_0\cdot s_{n-1}&
						c_0\cdot c_n\\
c_1\cdot c_0&
	c_1\cdot c_1&
		c_1\cdot s_1&
			\dots&
				c_1\cdot c_{n-1}&
					c_1\cdot s_{n-1}&
						c_1\cdot c_n\\
s_1\cdot c_0&
	s_1\cdot c_1&
		s_1\cdot s_1&
			\dots&
				s_1\cdot c_{n-1}&
					s_1\cdot s_{n-1}&
						s_1\cdot c_n\\
\vdots	&\vdots	&\vdots	&\ddots	&\vdots	&\vdots	&\vdots	\\
c_{n-1}\cdot c_0&
	c_{n-1}\cdot c_1&
		c_{n-1}\cdot s_1&
			\dots&
				c_{n-1}\cdot c_{n-1}&
					c_{n-1}\cdot s_{n-1}&
						c_{n-1}\cdot c_n\\
s_{n-1}\cdot c_0&
	s_{n-1}\cdot c_1&
		s_{n-1}\cdot s_1&
			\dots&
				s_{n-1}\cdot c_{n-1}&
					s_{n-1}\cdot s_{n-1}&
						s_{n-1}\cdot c_n\\
c_n\cdot c_0&
	c_n\cdot c_1&
		c_n\cdot s_1&
			\dots&
				c_n\cdot c_{n-1}&
					c_n\cdot s_{n-1}&
						c_n\cdot c_n\\
\end{pmatrix}
\\
&=
\begin{pmatrix}
N     &0        &0        &\dots    &0        &0        &0        \\
0     &\frac{N}2&0        &\dots    &0        &0        &0        \\
0     &0        &\frac{N}2&\dots    &0        &0        &0        \\
\vdots&\vdots   &\vdots   &\ddots   &\vdots   &\vdots   &\vdots   \\
0     &0        &0        &\dots    &\frac{N}2&0        &0        \\
0     &0        &0        &\dots    &0        &\frac{N}2&0        \\
0     &0        &0        &\dots    &0        &0        &\frac{N}2
\end{pmatrix}.
\end{align*}
Bis auf die Faktoren $N$ und $\frac{N}2$ auf der Diagonalen ist
${\cal F}{\cal F}^t$ 
eine Diagonalmatrix.
Wir können die Matrix zu einer Einheitsmatrix machen, indem wir 
sie mit der Diagonalmatrix
\begin{equation}
D
=
\begin{pmatrix}
\sqrt{\frac1N}&0&\dots&0\\
0&\sqrt{\frac{2}{N}}&\dots&0\\
\vdots&\vdots&\ddots&\vdots\\
0&0&\dots&\sqrt{\frac{2}{N}}
\end{pmatrix}
\end{equation}
multiplizieren.
Wir schreiben
\begin{align*}
{\cal F}
&=
D\, A
\end{align*}
Wir nennen $\cal F$ die {\em Fourier-Matrix}.
\index{Fourier-Matrix}%
Die Fourier-Matrix $\cal F$ ist orthogonal, es gilt
\[
{\cal F}{\cal F}^t
=
DAA^tD^t
=
DD^tAA^t
=
E,
\]
wobei wir im letzten Schritt $D^t$ mit $AA^t$ vertauschen durften,
weil beide Diagonalmatrizen sind und damit vertauschen.
Insbesondere erhält $\cal F$ das Skalarprodukt, womit wir natürlich
nur die Parseval-Identität anders formuliert haben.

\subsubsection{Fast Fourier Transform}









