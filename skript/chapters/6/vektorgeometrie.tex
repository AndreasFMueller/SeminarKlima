%
% vektorgeometrie.tex
%
% (c) 2018 Prof Dr Andreas Müller, Hochschule Rapperswil
%
\section{Vektorgeometrische Interpretation}
\rhead{Vektorschreibweise}
Die bisherigen rein analytischen Betrachtungen verdecken den geometrischen
Gehalt der bisher entwickelten Theorie.
In diesem Abschnitt soll daher zunächst eine vektorielle Darstellung
aufgebaut, die dann erlauben soll, einerseits die Formeln für die 
Fourierkoeffizienten geometrisch zu verstehen und andererseits auf
komplexere Situationen zu verallgemeinern.

\subsection{Vektoren}
Die Operationen zur Bestimmung der Fourier-Koeffizienten können in 
vektorieller Schreibweise etwas übersichtlicher dargestellt werden.
Zunächst fassen wir die Funktionswerte $y_j$ in einem Vektor zusamen.
\begin{equation}
y = \begin{pmatrix}y_1\\\vdots\\y_N\end{pmatrix}
\end{equation}
Zur Berechnung der Fourier-Koeffizienten brauchen wir auch noch die
Werte der trigonometrischen Funktionen zu den Zeiten $t_j$, die wir
ebenfalls als Vektoren
\begin{align*}
c_0&=\begin{pmatrix}0\\\vdots\\0\end{pmatrix},
&
c_k&=\begin{pmatrix}\cos kt_1\\\vdots\\\cos kt_N\end{pmatrix},\;(k=1,\dots,n)
&&\text{und}
&
s_k&=\begin{pmatrix}\sin kt_1\\\vdots\\\sin kt_N\end{pmatrix},\;(k=1,\dots,n-1)
\end{align*}
schreiben.
Die Fourier-Koeffizienten können jetzt als Skalarprodukte geschrieben werden:
\begin{align*}
a_0 &=\frac1N c_0\cdot y,&
a_k &=\frac2N c_k\cdot y,\;(k=1,\dots,n),&
b_k &=\frac2N s_k\cdot y,\;(k=1,\dots,n-1).
\end{align*}

\subsubsection{Rekonstruktion der Funktion}
Auch die Darstellung der Funktion kann man wieder als Skalarprodukt schreiben.
Dazu schreiben wir die Fourier-Koeffizienten und die Werte der
trigonometrischen Funtionen also Vektoren
\begin{align*}
a
&=
\begin{pmatrix}
a_0\mathstrut\\
a_1\mathstrut\\
b_1\mathstrut\\
a_2\mathstrut\\
b_2\mathstrut\\
\vdots\\
b_{n-1}\mathstrut\\
a_n\mathstrut
\end{pmatrix}
&&\text{und}
&
e(t)
&=
\begin{pmatrix}
1\\
\cos t\\
\sin t\\
\cos2t\\
\sin2t\\
\vdots\\
\sin(n-1)t\\
\cos nt
\end{pmatrix}.
\end{align*}
Damit wird 
\[
p(t) = a\cdot e(t)
\]

\subsubsection{Orthogonalität}
Die Aussagen von Satz~\ref{skript:fourier:orthogonalitaet1}
lassen sich jetzt in geometrische Form fassen.
\begin{satz}
Es gilt
\begin{align*}
c_k\cdot c_l
&=
\begin{cases}
N&\qquad k=l=0\\
\displaystyle\frac{N}2&\qquad k=l>0\\
0&\qquad\text{sonst}
\end{cases}
\\
s_k\cdot s_l
&=
\begin{cases}
\displaystyle\frac{N}2&\qquad k=l\\
0&\qquad\text{sonst}
\end{cases}
\\
c_k\cdot s_l
&=
0
\end{align*}
\label{skript:fourier:orthogonalitaet}
\end{satz}

\begin{proof}[Beweis]
Die genannten Skalarprodukte sind nichts anderes als die Summen in
Satz~\ref{skript:fourier:orthogonalitaet1}:
\begin{align*}
c_k\cdot c_l
&=
\sum_{j=1}^N \cos kt_j \cos lt_j
=
\begin{cases}
N&\qquad k=l=0\\
\displaystyle\frac{N}2&\qquad k=l>0\\
0&\qquad\text{sonst}
\end{cases}
\end{align*}
und analog für die anderen Skalarprodukte.
Die Aussage des Satzes ist daher nichts anders als eine geometrische
Umformulierung der Aussagen des
Satzes~\ref{skript:fourier:orthogonalitaet1}.
\end{proof}

\subsubsection{Die Identität von Parseval}
Die Relationen von
Satz~\ref{skript:fourier:orthogonalitaet1}
besagen, dass die Vektoren $c_k$ und $s_k$ orthogonal sind.
Wir wenden Sie auf das Skalarprodukt der Funktion $f$ mit sich selbst an.
\begin{align*}
f\cdot f
&=
a_0^2 c_0\,\cdot c_0
+
\sum_{k=1}^na_k^2 \,c_k\cdot c_k
+
\sum_{k=1}^{n-1} b_k^2\,s_k\cdot s_k
\\
&=
Na_0^2
+
\frac{N}2\sum_{k=1}^n a_k^2
+
\frac{N}2\sum_{k=1}^{n-1} b_k^2
=
\frac{N}2
\biggl(
2a_0^2
+
\sum_{k=1}^n a_k^2
+
\sum_{k=1}^{n-1} b_k^2
\biggr)
\end{align*}
Damit haben wir den folgenden Satz bewiesen:
\begin{satz}[Parseval]
\[
\|f\|^2
=
\sum_{j=1}^N y_j^2
=
\frac{N}2
\biggl(
2a_0^2
+
\sum_{k=1}^n a_k^2
+
\sum_{k=1}^{n-1} b_k^2
\biggr)
\]
\end{satz}

\subsubsection{$2\pi$-Periodische Funktionen auf $\mathbb R$}
Die eben vektoriell dargestellte Analyse diskreter periodischer Funktionen 
kann verallgemeinert werden auf die Analyse von Funktionen auf
anderen Definitionsgebieten.
Benötigt wird eine Familie von Basisfunktionen und ein Skalarprodukt
$\langle\;,\;\rangle$ derart, dass die Basisfunktionen $g_i$ bezüglich
dieses Skalarproduktes orthonormiert sind, dass also
\[
\langle g_i,g_j\rangle
=
\delta_{ij}
=
\begin{cases}
1&\qquad i=j\\
0&\qquad\text{sonst}.
\end{cases}
\]
Jede Linearkombination
\[
f = \sum_{i} \alpha_i g_i
\]
von Basisfunktionen kann ebenfalls mit dem Skalarprodukt rekonstruiert
werden.
Dazu berechnet man
\[
\langle g_i,f\rangle
=
\biggl\langle
g_i,\sum_j\alpha_jg_j
\biggr\rangle
=
\sum_j \langle g_i,\alpha_jg_j\rangle
=
\sum_j \alpha_j\delta_{ij}
=
\alpha_i.
\]
Das Skalarprodukt kann auch verwendet werden, um einen Abstand zwischen
Vektoren als
\[
\| f-g\|^2
=
\langle f-g,f-g\rangle
\]
zu definieren.

Dieselbe Situation lässt sich auch für $2\pi$-periodische Funktionen 
auf $\mathbb R$ herbeiführen.
Als Basisfunktionen kann man die Funktionen 
\begin{equation}
\frac{1}{\sqrt{2}},\; \cos kx,\; \sin lx\quad k>0
\label{fourier:basis}
\end{equation}
verwenden.
Das Skalarprodukt $\langle f,g\rangle$ muss linear in $f$ und $g$ sein.
Eine naheliegende Wahl ist
\[
\langle f, g\rangle
=
\frac{1}{\pi}\int_{-\pi}^{\pi} f(x)\,g(x)\,dx.
\]
Wir überprüfen, ob die Funktionen orthogonal sind:
\begin{align*}
\left\langle \frac1{\sqrt{2}},\frac1{\sqrt{2}}\right\rangle
&=
\frac1{\pi}
\int_{-\pi}^{\pi} \frac12\,dx
=
1
\\
\left\langle \frac1{\sqrt{2}},\cos kx\right\rangle
&=
\frac1{\pi}\int_{-\pi}^{\pi}
\frac1{\sqrt{2}}\cos kx
\,dx
=0
\\
\left\langle \frac1{\sqrt{2}},\sin kx\right\rangle
&=
\frac1{\pi}\int_{-\pi}^{\pi}
\frac1{\sqrt{2}}\sin kx
\,dx
=0
\\
\langle \cos kx,\cos lx\rangle
&=
\frac1{\pi}
\int_{-\pi}^\pi \cos kx\cos lx\,dx
\\
&=
\frac1{\pi}
\int_{-\pi}^\pi
\frac12\bigl(
\cos (k-l)x+\cos (k+l)x
\bigr)
\,dx
=
\begin{cases}
1&\qquad k=l\\
0&\qquad\text{sonst}
\end{cases}
\\
\langle \sin kx,\sin lx\rangle
&=
\frac1{\pi}
\int_{-\pi}^\pi \sin kx\,\sin lx\,dx
\\
&=
\frac1{\pi}
\int_{-\pi}^\pi
\frac12
\bigl(
\cos (k-l)x - \cos (k+l)x
\bigr)
\,dx
=
\begin{cases}
1&\qquad k=l\\
0&\qquad\text{sonst}
\end{cases}
\\
\langle \sin kx,\cos lx\rangle
&=
\frac1{\pi}
\int_{-\pi}^{\pi} 
\frac12\bigl(
\sin (k-l)x + \sin (k+l)x
\bigr)
\,dx
=0
\end{align*}
Zu einer $2\pi$-periodischen Funktion $f(x)$ kann man daher immer
die Koeffizienten
\begin{equation}
\begin{aligned}
\bar{a}_0&=\frac1{\pi\sqrt{2}}\int_{-\pi}f(x)\,dx
\\
a_k&=\frac1{\pi}\int_{-\pi}^\pi f(x)\cos kx\,dx
\\
b_k&=\frac1{\pi}\int_{-\pi}^\pi f(x)\sin kx\,dx
\end{aligned}
\label{fourier:normalekoeffizienten}
\end{equation}
berechnen.
Die Linearkombination
\begin{equation}
\tilde f(x)
=
\bar{a}_0\cdot\frac1{\sqrt{2}}
+ 
\sum_{k=1}^\infty (a_k\cos kx+b_k\sin kx)
\label{fourier:reihe}
\end{equation}
ist natürlich wieder eine $2\pi$-periodische Funktion.

Ist $f(x)$ eine Linearkombination von Funktionen~\eqref{fourier:basis},
dann sind nur endlich viele der Koeffizienten $\bar{a}_0$, $a_k$ und $b_k$
sind von $0$ verschieden und es gilt $f(x)=\tilde f(x)$, die Summe
\eqref{fourier:reihe} rekonstriert die Funktion $f(x)$ also exakt..

Für eine beliebige $2\pi$-periodische Funktion $f(x)$ ist die Funktion
$\tilde f(x)$ nach \eqref{fourier:reihe} im Allgemeinen eine unendliche
Reihe.
Die Reihe \eqref{fourier:reihe} heisst die Fourier-Reihe der Funktion 
$f(x)$.
\index{Fourier-Reihe}

In der Literatur wird $a_0$ meistens anders definiert, nämlich als
\[
a_0 = \frac1{\pi}\int_{-\pi}^{\pi} f(x)\,dx = \sqrt{2}\bar{a}_0
\qquad\Rightarrow\qquad
\bar{a}_0 = \frac{a_0}{\sqrt{2}}
\]
Der erste Term der Reihe~\eqref{fourier:basis} wird dann
\[
\bar{a}_0\cdot\frac1{\sqrt{2}}
=
\frac{a_0}{\sqrt{2}}\cdot\frac{1}{\sqrt{2}}
=
\frac{a_0}2
\]
und die Fourier-Reihe ist
\begin{equation}
\tilde f(x)
=
\frac{a_0}2
+
\sum_{k=1}^\infty (a_k\cos kx+b_k\sin kx).
\end{equation}

\subsection{Fourier-Transformation}
Die Fourier-Koeffizienten $a_k$ und $b_k$ hängen linear von den
Funktionswerten $y_j$ ab.
Der Vektor der Fourier-Koeffizienten muss daher der Bildvektor des
Vektors $\vec y$ der Funktionswerte unter einer linearen Transformation
sein.
In diesem Abschnitt soll zunächst diese diskrete Fourier-Transformation 
hergeleitet werden.
Anschliessend soll gezeigt werden, wie sich diese Eigenschaft auf 
periodische Funktionen und auf beliebige Funktionen auf $\mathbb R$
ausdehenen lässt.

\subsubsection{Diskrete Fourier-Transformation}
Die Berechnung der Fourier-Koeffizienten ist eine lineare Operation
mit der $N\times N$-Matrix:
\[
A
=
\begin{pmatrix}
c_0^t\\
c_1^t\\
s_1^t\\
\vdots\\
c_{n-1}^t\\
s_{n-1}^t\\
c_n^t
\end{pmatrix}
=
\begin{pmatrix}
1           &1           &\dots &1            \\
\cos t_1    &\cos t_2    &\dots &\cos t_N     \\
\sin t_1    &\sin t_2    &\dots &\sin t_N     \\
\cos 2t_1   &\cos 2t_2   &\dots &\cos 2t_N    \\
\sin 2t_1   &\sin 2t_2   &\dots &\sin 2t_N    \\
\vdots      &\vdots      &\ddots&\vdots       \\
\sin(n-1)t_1&\sin(n-1)t_2&\dots &\sin(n-1)t_N \\
\cos nt_1   &\cos nt_2   &\dots &\cos nt_N    
\end{pmatrix}
\]
Die Orthogonalitätsrelationen von
Satz~\ref{skript:fourier:orthogonalitaet}
können jetzt neu geschrieben werden:
\begin{align*}
AA^t
&=
\begin{pmatrix}
c_0\cdot c_0&
	c_0\cdot c_1&
		c_0\cdot s_1&
			\dots&
				c_0\cdot c_{n-1}&
					c_0\cdot s_{n-1}&
						c_0\cdot c_n\\
c_1\cdot c_0&
	c_1\cdot c_1&
		c_1\cdot s_1&
			\dots&
				c_1\cdot c_{n-1}&
					c_1\cdot s_{n-1}&
						c_1\cdot c_n\\
s_1\cdot c_0&
	s_1\cdot c_1&
		s_1\cdot s_1&
			\dots&
				s_1\cdot c_{n-1}&
					s_1\cdot s_{n-1}&
						s_1\cdot c_n\\
\vdots	&\vdots	&\vdots	&\ddots	&\vdots	&\vdots	&\vdots	\\
c_{n-1}\cdot c_0&
	c_{n-1}\cdot c_1&
		c_{n-1}\cdot s_1&
			\dots&
				c_{n-1}\cdot c_{n-1}&
					c_{n-1}\cdot s_{n-1}&
						c_{n-1}\cdot c_n\\
s_{n-1}\cdot c_0&
	s_{n-1}\cdot c_1&
		s_{n-1}\cdot s_1&
			\dots&
				s_{n-1}\cdot c_{n-1}&
					s_{n-1}\cdot s_{n-1}&
						s_{n-1}\cdot c_n\\
c_n\cdot c_0&
	c_n\cdot c_1&
		c_n\cdot s_1&
			\dots&
				c_n\cdot c_{n-1}&
					c_n\cdot s_{n-1}&
						c_n\cdot c_n\\
\end{pmatrix}
\\
&=
\begin{pmatrix}
N     &0        &0        &\dots    &0        &0        &0        \\
0     &\frac{N}2&0        &\dots    &0        &0        &0        \\
0     &0        &\frac{N}2&\dots    &0        &0        &0        \\
\vdots&\vdots   &\vdots   &\ddots   &\vdots   &\vdots   &\vdots   \\
0     &0        &0        &\dots    &\frac{N}2&0        &0        \\
0     &0        &0        &\dots    &0        &\frac{N}2&0        \\
0     &0        &0        &\dots    &0        &0        &\frac{N}2
\end{pmatrix}.
\end{align*}
Bis auf die Faktoren $N$ und $\frac{N}2$ auf der Diagonalen ist
${\cal F}{\cal F}^t$ 
eine Diagonalmatrix.
Wir können die Matrix zu einer Einheitsmatrix machen, indem wir 
sie mit der Diagonalmatrix
\begin{equation}
D
=
\begin{pmatrix}
\sqrt{\frac1N}&0&\dots&0\\
0&\sqrt{\frac{2}{N}}&\dots&0\\
\vdots&\vdots&\ddots&\vdots\\
0&0&\dots&\sqrt{\frac{2}{N}}
\end{pmatrix}
\end{equation}
multiplizieren.
Wir schreiben
\begin{align*}
{\cal F}
&=
D\, A
\end{align*}
Wir nennen $\cal F$ die {\em Fourier-Matrix}.
\index{Fourier-Matrix}%
Die Fourier-Matrix $\cal F$ ist orthogonal, es gilt
\[
{\cal F}{\cal F}^t
=
DAA^tD^t
=
DD^tAA^t
=
E,
\]
wobei wir im letzten Schritt $D^t$ mit $AA^t$ vertauschen durften,
weil beide Diagonalmatrizen sind und damit vertauschen.
Insbesondere erhält $\cal F$ das Skalarprodukt, womit wir natürlich
nur die Parseval-Identität anders formuliert haben.

\subsubsection{Fast Fourier Transform}


\subsubsection{Fourier-Transformation von $2\pi$-periodischen Funktionen}

\subsubsection{Fourier-Transformation von Funktionen auf $\mathbb R$}






