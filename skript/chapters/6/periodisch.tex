%
% periodisch.tex -- periodische Funktionen
%
% (c) 2018 Prof Dr Andreas Müller, Hochschule Rapperswil
%
\section{Periodische Funktionen}
Das ursprüngliche Budyko-Modell modelliert die Jahresmittel-Temperatur,
ignoriert also die Temperatur-Entwicklung der Temperatur im Laufe des
Jahres.
Damit ist es aber zum Beispiel nicht möglich, das Einsetzen von
Eiszeiten zu modellieren.
Wie in Kapitel~\ref{chapter:neigung} gezeigt wird, ist die jährlich
auf die Erde eingetrahlte Leistung unabhängig von der Neigung der
Erdachse.
Ein Modell, welches den Zusammenhang zwischen Eiszeiten und der
Achsneigung modellieren soll muss also die periodischen Schwankungen
der Einstrahlung der Sonne auf jede Erd-Halbkugel berücksichtigen.

Ein geeignetes Modell wird also nicht mehr autonom sein, sondern auf der
rechten Seite der Differentialgleichung
\[
\frac{dx}{dt} = f(x,t)
\]
muss man berücksichtigen, dass $f(x,t)$ nun auch von der Zeit $t$ abhängt
und periodisch in $t$ mit Periode $T$ ist.
Dies bedeutet $f(x,t+T)=f(x,t)$ für beliebige Vektoren $x$ und 
Zeiten $t$.

Man kann natürlich nicht mehr erwarten, dass es eine zeitunabhängige
Gleichgewichtslösung gibt.
Vielmehr erwarten wir statt konstanter Gleichgewichtslösungen 
periodische Lösungen mit der gleichen Periode $T$, dass also
$x(t+T)=x(t)$.

\subsection{Fourierreihen}
Die Funktionen
\begin{equation}
1,
\cos \frac{2\pi kt}{T},
\sin \frac{2\pi kt}{T},\qquad k\in \mathbb N
\label{skript:periodisch:fourierbasis}
\end{equation}
sind alle periodisch mit Periode $T$, aber auch eine beliebige
Linearkombination
\begin{equation}
f(t)
=
a_0  +\sum_{k=1}^\infty\biggl(
a_k \cos \frac{2\pi kt}{T} + b_k\sin\frac{2\pi kt}{T}
\biggr)
\label{skript:periodisch:fourierreihe}
\end{equation}
hat diese Eigenschaft.

Es war eine bedeutende Erkenntnis von Joseph Fourier, dass bis auf ein
paar technische Bedingungen, welche die Konvergenz der Funktionenreihe
sicherstellen sollen, jede periodische Funktion $f(t)$ 
als Reihe von Vielfachen der Funktionen
\eqref{skript:periodisch:fourierbasis}
in der Form
\eqref{skript:periodisch:fourierreihe}
dargestellt werden kann.
Die Berechnung der Koeffizienten $a_0,a_1,a_2,\dots$ und $b_1,b_2,b_3,\dots$
erfolgt mit Hilfe von Fourier-Integralen, deren Theorie wir hier weiter
nicht entwickeln wollen.

Für unsere Zwecke ist die vollständige Theorie der Fourier-Reihen nicht
notwendig, denn die Daten, an die wir unsere Klimamodelle anpassen
müssen, stellen bestenfalls diskrete Approximationen von stetigen
Funktionen dar.
Wir müssen die allgemeine Fourier-Theorie daher spezialisieren auf diese
diskrete Situation.

\subsection{Diskrete Fourierreihen: trigonometrische Polynome}
Wir wollen im Folgenden periodische diskrete Funktionen möglichst gut
approximieren.
Da für diskrete Funktionen die Skala des Argumentes nicht so wichtig ist,
verwenden wir als Basis die Funktionen 
\begin{equation}
1,\cos kt\quad\text{und}\quad \sin kt,\quad k\in\mathbb N.
\end{equation}
Die Funktion $f(t)$ soll also geschrieben werden als sogenanntes 
trigonometrisches Polynom
\begin{equation}
f(t)
=
a_0 + \sum_{k=1}^n \bigl(a_k \cos kt + b_k\sin kt).
\end{equation}
Die Funktion soll die Werte $y_j$ in den äquidistanten $t$-Werten 
$t_j=2\pi j/N$ mit $0\le j<N$ möglichst gut wiedergeben.

Zur Bestimmung der Koeffizienten $a_0$, $a_k$ und $b_k$ stehen also nur
die $N$ Bedingungen
\begin{equation}
f(t_j)=y_j
\label{skript:periodisch:fourierbedingungen}
\end{equation}
zur Verfügung.
Die Gleichungen \eqref{skript:periodisch:fourierbedingungen} sind
lineare Gleichungen für die Unbekannten $a_k$ und $b_k$ mit
$\cos kt_j$ und $\sin kt_j$ als Koeffizienten.
Sie können daher höchstens endlich viele Koeffizienten $a_k$ und $b_k$
bestimmen.
Dieses Problem wird im nächsten Abschnitt gelöst.




