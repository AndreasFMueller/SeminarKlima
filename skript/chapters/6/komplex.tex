%
% komplex.tex
%
% (c) 2018 Prof Dr Andreas Müller, Hochschule Rapperswil
%
\subsubsection{Komplexe Fouriertransformation}
Bisher wurden alle Rechnungen nur mit reellen Zahlen durchgeführt.
Es stellt sich aber heraus, dass komplexe Zahlen für die Beschreibung
der Fourier-Transformation sehr viel praktischer sind.
Der Grund dafür ist die Eulersche Beziehung
\[
e^{it} = \cos t + i \sin t.
\]
und die Rechenregel
\[
e^{a+b}=e^a\cdot e^b
\qquad\Rightarrow\qquad
e^{ikt}=\cos kt+i\sin kt
\]
für die Exponentialfunktion.
Für die Fourier-Koeffizienten werden die Summen
\[
a_0
=
\frac{1}{N}\sum_{j=1}^N y_j,\qquad
a_l
=
\frac{2}{N}\sum_{j=1}^N y_j cos lt_j,
\qquad\text{und}\qquad
b_l
=
\frac{2}{N}\sum_{j=1}^N y_j sin lt_j
\]
benötigt.
Fassen wir $a_l$ und $b_l$ als Real- und Imaginärteil einer komplexen
Zahl auf, dann können wir 
\begin{align*}
c_l
=
a_l+ib_l
&=
\frac2{N} \sum_{j=1}^N y_j (\cos lt_j + i \sin lt_j)
=
\frac2{N} \sum_{j=1}^N y_j e^{lt_j}
\end{align*}
berechnen.

Auch die Rekonstruktion~\eqref{skript:fourier:rekonstruktion} ist
mit komplexen Zahlen darstellbar.
Dazu verwendet man 
\[
\cos kt = \operatorname{Re} e^{ikt}
\qquad\text{und}\qquad
\sin kt = \opeatorname{Im} e^{ikt}.
\]
In dieser Form
\[
f(t)
=
a_0
+\sum_{k=1}^n (a_k \cos kt + b_k \sin kt)
\]




