%
% komplex.tex
%
% (c) 2018 Prof Dr Andreas Müller, Hochschule Rapperswil
%
\subsubsection{Komplexe Fouriertransformation}
Bisher wurden alle Rechnungen nur mit reellen Zahlen durchgeführt.
Es stellt sich aber heraus, dass komplexe Zahlen für die Beschreibung
der Fourier-Transformation sehr viel praktischer sind.
Der Grund dafür ist die Eulersche Beziehung
\[
e^{it} = \cos t + i \sin t.
\]
und die Rechenregel
\[
e^{a+b}=e^a\cdot e^b
\qquad\Rightarrow\qquad
e^{ikt}=\cos kt+i\sin kt = (\cos t + i \sin t)^k
\]
für die Exponentialfunktion.
Für die Fourier-Koeffizienten werden die Summen
\[
a_0
=
\frac{1}{N}\sum_{j=1}^N y_j,\qquad
a_l
=
\frac{2}{N}\sum_{j=1}^N y_j \cos lt_j,
\qquad\text{und}\qquad
b_l
=
\frac{2}{N}\sum_{j=1}^N y_j \sin lt_j
\]
benötigt.

Fassen wir $a_l$ und $b_l$ als Real- und Imaginärteil einer komplexen
Zahl auf, dann können wir 
\begin{align*}
c_l
=
\frac12(a_l-ib_l)
&=
\frac{1}{N} \sum_{j=1}^N y_j (\cos lt_j - i \sin lt_j)
=
\frac1{N} \sum_{j=1}^N y_j e^{-lt_j}
=
\frac{1}{N} \sum_{j=1}^N y_j e^{-2\pi ilj/N}
\label{skript:complex:cl}
\end{align*}
berechnen.
Für $l=0$ und $l=n$ gibt es keine $b$-Koeffizienten, man setzt daher
$c_0 = a_0$ und $c_n=a_n/2$.

Für reelle Werte $y_j$ haben die Koeffizienten $c_l$ zusätzliche
Symmetrieeigenschaften.
Die komplex konjugierte $\bar c_l$ ist
\[
\bar c_l
=
\overline{\sum_{j=0}^{N-1} y_j e^{2\pi ijl/N}}
=
\sum_{j=0}^{N-1} y_j e^{-2\pi i jl/N}
=
c_{-l}.
\]
Daraus liest man zum Beispiel ab, dass $c_0\in\mathbb R$ ist, nämlich
\[
c_0 = \frac{1}{N} \sum_{j=0}^{N-1} y_j = a_0.
\]
Für die anderen Koeffizienten gilt zunächst
\begin{align*}
c_l
&=
\frac{1}{N}\sum_{j=0}^{N-1} y_j e^{-2\pi ilj/N}
=
\frac{1}{N}\sum_{j=0}^{N-1} y_j \biggl(\cos \frac{2\pi lj}{N} + i \sin\frac{2\pi lj}{N}\biggr)
\\
&=
\frac{1}{N}\sum_{j=0}^{N-1} y_j
\cos lt_j
+
i
\frac{1}{N}\sum_{j=0}^{N-1} y_j
\sin lt_j
\\
&=
\frac12(a_l + ib_l)
\end{align*}
Zusammen mit der Beziehung $\bar c_l=c_{-l}$, können wir die reellen
Fourier-Koeffizienten auch verwenden, um die reellen Koeffizienten
\begin{align*}
a_l
&=
c_l + c_{-l}
&&\text{und}&
b_l
&=
\frac1{i}
(c_l - c_{-l})
\end{align*}
durch die komplexen Koeffizienten auszudrücken.

Auch die Rekonstruktion~\eqref{skript:fourier:rekonstruktion} ist
mit komplexen Zahlen darstellbar.
Dazu verwendet man 
\[
\cos x = \operatorname{Re} e^{ix} = \frac{e^{ix}+e^{-ix}}{2}
\qquad\text{und}\qquad
\sin x = \operatorname{Im} e^{ix} = \frac{e^{ix}-e^{-ix}}{2i}.
\]
Damit wird das trigonometrische Polynom
\begin{align*}
f(t)
&=
a_0
+\sum_{k=1}^{n-1} (a_k \cos kt + b_k \sin kt)
+a_n \cos nt
\\
&=
a_0
+\sum_{k=1}^{n-1} (a_k \operatorname{Re} e^{ikt} + b_k\operatorname{Im}e^{ikt})
+ a_n \cos nt
\\
&=
\operatorname{Re} \sum_{k=0}^{n} (a_k-ib_k) e^{ikt}
\intertext{Zusammen mit den Koeffizienten $c_{-l}$ folgt}
=
\sum_{l=0}^{N-1} c_k e^{ikt}.
\end{align*}
Damit sieht die Rekonstruktionsformel bis auf das Vorzeichen
im Exponenten gleich aus wie die
Transformationsformel~\eqref{skript:complex:cl}.
Diese Eigenschaft kann dazu verwendet werden, die Rücktransformation
im Wesentlichen mit demselben Code zu berechnen wie die Transformation.


