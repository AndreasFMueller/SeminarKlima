%
% fftaufwand.tex -- Aufwand für die Berechnung der Fourier-Transformation
%
% (c) 2018 Prof Dr Andreas Müller, Hochschule Rapperswil
%
\documentclass[tikz]{standalone}
\usepackage{times}
\usepackage{amsmath}
\usepackage{txfonts}
\usepackage[utf8]{inputenc}
\usepackage{graphics}
\usepackage{color}
\usetikzlibrary{arrows,intersections}
\usepackage{pgfplots}
\begin{document}
\definecolor{darkgreen}{rgb}{0,0.6,0}
\input{auf.tex}
\begin{tikzpicture}[>=latex,thick,scale=0.25]

\draw[->] (1.9,0)--(42.1,0) coordinate[label=$\log N$];
\draw[->] (2,0)--(2,29.1) coordinate[label={right:$\log g(N)$}];

\draw[color=blue] plot[domain=2:42,samples=2] ({\x},{2*\x/3});
\node[color=blue] at (19,14) [above] {$O(N^2)$};

\nlogn
\node[color=red] at (26,9) [below] {$O(N\log_2N)$};

\end{tikzpicture}
\end{document}
