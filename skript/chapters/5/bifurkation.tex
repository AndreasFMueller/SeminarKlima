%
% bifurkation.tex
%
% (c) 2018 Prof Dr Andreas Müller, Hochschule Rapperswil
%
\documentclass[tikz]{standalone}
\usepackage{amsmath}
\usepackage{times}
\usepackage{txfonts}
\usepackage[utf8]{inputenc}
\usepackage{graphics}
\usepackage{ifthen}
\usepackage{color}
\usetikzlibrary{arrows,intersections,math}
\begin{document}

\def\Q{342}

\tikzmath{
	real \epseins, \Teinsoben, \Teinsunten, \Teinsmitte;
	\epseins = 0.491;
	\Teinsoben = 304.5;
	\Teinsunten = 252;
	\Teinsmitte = (\Teinsoben + \Teinsunten) / 2;
	real \epszwei, \Tzweioben, \Tzweiunten, \Tzweimitte;
	\epszwei = 0.689;
	\Tzweioben = 274.4;
	\Tzweiunten = 226.5;
	\Tzweimitte = (\Tzweioben + \Tzweiunten) / 2;
	real \Teins, \Tzwei, \Tdrei;
	\Teins = 234.5;
	\Tzwei = 264.2;
	\Tdrei = 289.3;
}

\begin{tikzpicture}[>=latex,thick,scale=10]

\fill[color=red!20]
	(0,0)--
	plot[domain=220:320,samples=100]
	({2*((\Q / ((0.015431 * \x)^4) * (0.5 + 0.2*tanh((\x-265)/10))-0.3)},{\x/100-2.2})
	--(0,1)--cycle;

\fill[color=blue!20]
	(1,0)--(1,1)--
	plot[domain=320:220,samples=100]
	({2*((\Q / ((0.015431 * \x)^4) * (0.5 + 0.2*tanh((\x-265)/10))-0.3)},{\x/100-2.2})
	--cycle;

\draw[color=black,line width=1.4pt] 
	plot[domain=220:320,samples=100]
	({2*((\Q / ((0.015431 * \x)^4) * (0.5 + 0.2*tanh((\x-265)/10))-0.3)},{\x/100-2.2});

\draw ({2*(0.6-0.3)},0)--({2*(0.6-0.3)},1);

\foreach \e in {0.3,0.4,0.5,0.6,0.7,0.8}{
	\node at ({2*(\e-0.3)},0) [below] {$\e$};
	\draw ({2*(\e-0.3)},-0.01)--({2*(\e-0.3)},0.01);
}

\foreach \T in {220,230,...,320}{
	\draw (-0.01,{\T/100-2.2})--(0.01,{\T/100-2.2});
	\node at (-0.01,{\T/100-2.2}) [left] {$\T$};
}

\draw[color=red,line width=1.4pt,->] 
	({2*(\epseins-0.3)},{\Teinsunten/100-2.2})--
	({2*(\epseins-0.3)},{\Teinsmitte/100-2.2});
\draw[color=red,line width=1.4pt] 
	({2*(\epseins-0.3)},{\Teinsmitte/100-2.2-0.03})--
	({2*(\epseins-0.3)},{\Teinsoben/100-2.2});
\fill[color=red]
	({2*(\epseins-0.3)},{\Teinsunten/100-2.2}) circle[radius=0.008];
\fill[color=red]
	({2*(\epseins-0.3)},{\Teinsoben/100-2.2}) circle[radius=0.008];

\node[color=black] at ({2*(\epseins-0.3)},{\Teinsunten/100-2.2})
	[left] {$\varepsilon_1$};

\draw[color=blue,line width=1.4pt,->] 
	({2*(\epszwei-0.3)},{\Tzweioben/100-2.2})--
	({2*(\epszwei-0.3)},{\Tzweimitte/100-2.2});
\draw[color=blue,line width=1.4pt] 
	({2*(\epszwei-0.3)},{\Tzweimitte/100-2.2+0.03})--
	({2*(\epszwei-0.3)},{\Tzweiunten/100-2.2});
\fill[color=blue]
	({2*(\epszwei-0.3)},{\Tzweiunten/100-2.2}) circle[radius=0.008];
\fill[color=blue]
	({2*(\epszwei-0.3)},{\Tzweioben/100-2.2}) circle[radius=0.008];
\node[color=black] at ({2*(\epszwei-0.3)},{\Tzweioben/100-2.2})
	[right] {$\varepsilon_2$};

\node at 
	({2*(\epszwei-0.3)},{\Tzweiunten/100-2.2}) [above right] {Snowball Earth};

\fill ({2*(0.6-0.3)},{\Tdrei/100-2.2}) circle[radius=0.008];
\node at ({2*(0.6-0.3)},{\Tdrei/100-2.2}) [above right] {$T_3^*$};
\node at ({2*(0.6-0.3)},{\Tdrei/100-2.2}) [below left] {heute};

\fill ({2*(0.6-0.3)},{\Tzwei/100-2.2}) circle[radius=0.008];
\node at ({2*(0.6-0.3)},{\Tzwei/100-2.2}) [above left] {$T_2^*$};

\fill ({2*(0.6-0.3)},{\Teins/100-2.2}) circle[radius=0.008];
\node at ({2*(0.6-0.3)},{\Teins/100-2.2}) [above right] {$T_1^*$};

\draw[->] (-0.01,0)--(1.04,0) coordinate[label={$\varepsilon$}];
\draw[->] (0,-0.01)--(0,1.03) coordinate[label={right:$T\,[\text{K}]$}];

\end{tikzpicture}

\end{document}
