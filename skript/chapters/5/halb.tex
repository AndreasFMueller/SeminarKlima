%
% halb.tex
%
% (c) 2018 Prof Dr Andreas Müller, Hochschule Rapperswil
%
\documentclass[tikz]{standalone}
\usepackage{amsmath}
\usepackage{times}
\usepackage{txfonts}
\usepackage[utf8]{inputenc}
\usepackage{graphics}
\usepackage{ifthen}
\usepackage{color}
\usetikzlibrary{arrows,intersections}
\begin{document}

\def\R{4}
\def\winkel{30}
\def\neigung{23.4}

\begin{tikzpicture}[scale=1,>=latex,thick]

\begin{scope}[rotate=-150]
\fill[color=gray!10]
	plot[domain=0:180,samples=100]
		({\R*cos(\x)},{\R*sin(\x)})
	--
	plot[domain=0:180,samples=100]
		({\R*cos(180-\x)},{\R*sin(\neigung)*sin(180-\x)});
	
\end{scope}


\draw[color=gray]
	({\R*cos(\winkel-180)},{\R*sin(\winkel-180)})
		--({\R*cos(\winkel)},{\R*sin(\winkel)});
\draw[color=gray]
	(0,0)--({\R*sin(\neigung)*cos(\winkel-90)},{\R*sin(\neigung)*sin(\winkel-90)});

\node at ({\R*cos(\winkel)/2},{\R*sin(\winkel)/2}) [above left] {$R$};
\node at ({\R*sin(\neigung)*cos(\winkel-90)*0.6},{\R*sin(\neigung)*sin(\winkel-90)*0.6}) [below left] {$R\sin\gamma$};

\begin{scope}[rotate=30]
\draw[color=gray,line width=1pt]
	plot[domain=0:180,samples=100]
		({\R*cos(\x)},{\R*sin(\neigung)*sin(\x)});
\end{scope}

\begin{scope}[rotate=-150]
\draw[line width=1pt]
	plot[domain=0:180,samples=100]
		({\R*cos(\x)},{\R*sin(\neigung)*sin(\x)});
\end{scope}

\draw[line width=1pt] (0,0) circle[radius={\R}];

\fill[color=gray] (0,0) circle[radius=0.06];
\fill ({\R*cos(\neigung)*cos(\winkel+90)},{\R*cos(\neigung)*sin(\winkel+90)})
	circle[radius=0.06];
\node at ({\R*cos(\neigung)*cos(\winkel+90)},{\R*cos(\neigung)*sin(\winkel+90)})
	[below left] {$N$};

\end{tikzpicture}

\end{document}
