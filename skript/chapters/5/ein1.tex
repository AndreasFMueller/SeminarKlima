%
% ein1.tex
%
% (c) 2018 Prof Dr Andreas Müller, Hochschule Rapperswil
%
\documentclass[tikz]{standalone}
\usepackage{amsmath}
\usepackage{times}
\usepackage{txfonts}
\usepackage[utf8]{inputenc}
\usepackage{graphics}
\usepackage{ifthen}
\usepackage{color}
\usetikzlibrary{arrows,intersections}
\begin{document}

\def\pivalue{3.1415926}
\def\n{30}
\def\yscale{0.25}

\def\curve#1{
\draw[color=red!50,line width=1.0pt]
	plot[domain=0:{#1},samples=100]
	({\pivalue*\x/180},{\yscale * 2 * \pivalue * cos(\x)*sin(\x)*sin(#1)})
	--
	plot[domain={#1+1}:89,samples=100]
	({\pivalue*\x/180},{\yscale*sin(\x)*(
		+2 * cos(\x) * cos(#1) * sqrt(tan(\x)^2-tan(#1)^2)
		+4 * cos(\x) * sin(#1) *
			atan(sqrt((tan(\x)+tan(#1))/(tan(\x)-tan(#1))))
				* (\pivalue/180))
	})
	--
	plot[domain=91:{179-#1},samples=100]
	({\pivalue*\x/180},{\yscale*sin(\x)*(
		-2 * cos(\x) * cos(#1) * sqrt(tan(\x)^2-tan(#1)^2)
		+4 * cos(\x) * sin(#1) *
			atan(sqrt((tan(\x)+tan(#1))/(tan(\x)-tan(#1))))
				* (\pivalue/180))
	})
	--
	({\pivalue*(180-#1)/180},0)--({\pivalue},0);
}

\begin{tikzpicture}[scale=4,>=latex]

\foreach \neigung in {10,20,...,80}{
	\curve{\neigung}
}

\draw[color=red!50,line width=1pt]
	plot[domain=0:180,samples=100]
		({\pivalue*\x/180},{\yscale*2*sin(\x)^2});

\draw[color=red!50,line width=1pt]
	plot[domain=0:90,samples=100]
		({\pivalue*\x/180},{\yscale*2 *\pivalue*cos(\x)*sin(\x)})
		--(\pivalue,0);

\draw[color=red,line width=1.4pt]
	plot[domain=0:{\n},samples=10]
	({\pivalue*\x/180},{\yscale * 2 * \pivalue * cos(\x)*sin(\x) *sin(\n)})
	--
	plot[domain={\n+1}:89,samples=100]
	({\pivalue*\x/180},{\yscale*sin(\x)*(
		+2 * cos(\x) * cos(\n) * sqrt(tan(\x)^2-tan(\n)^2)
		+4 * cos(\x) * sin(\n) *
			atan(sqrt((tan(\x)+tan(\n))/(tan(\x)-tan(\n))))
				* (\pivalue/180))
	})
	--
	plot[domain=91:{179-\n},samples=100]
	({\pivalue*\x/180},{\yscale*sin(\x)*(
		-2 * cos(\x) * cos(\n) * sqrt(tan(\x)^2-tan(\n)^2)
		+4 * cos(\x) * sin(\n) *
			atan(sqrt((tan(\x)+tan(\n))/(tan(\x)-tan(\n))))
				* (\pivalue/180))
	})
	--
	({\pivalue*(180-\n)/180},0)--({\pivalue},0);

\draw[->] (-0.1,0)--({\pivalue+0.3},0) coordinate[label={$\vartheta$}];
\draw[->] (0,-0.1)--(0,0.9) coordinate[label={right:$E_\text{in}(\vartheta)$}];

\draw ({\pivalue/2},0.02)--({\pivalue/2},-0.02);
\node at ({\pivalue/2},0) [below] {$\frac{\pi}2$};

\draw ({\pivalue},0.02)--({\pivalue},-0.02);
\node at ({\pivalue},0) [below] {$\pi$};

\node at (0,0) [below right] {$0$};

\node at (0.8,0.2) {$\varepsilon=0$};
\node at (1.3,0.51) [above] {$\varepsilon=\frac{\pi}6$};
\node at ({\pivalue/4},{\pivalue/4-0.01}) [above] {$\varepsilon=\frac{\pi}2$};

%\draw (-0.02,{\pivalue/2})--(0.02,{\pivalue/2});
%\node at (0,{\pivalue/2}) [left] {$2\pi$};

\draw (-0.02,{\pivalue/4})--(0.02,{\pivalue/4});
\node at (0,{\pivalue/4}) [left] {$\pi$};

\end{tikzpicture}

\end{document}
