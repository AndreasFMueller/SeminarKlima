%
% total.tex
%
% (c) 2018 Prof Dr Andreas Müller, Hochschule Rapperswil
%
\documentclass[tikz]{standalone}
\usepackage{amsmath}
\usepackage{times}
\usepackage{txfonts}
\usepackage[utf8]{inputenc}
\usepackage{graphics}
\usepackage{ifthen}
\usepackage{color}
\usetikzlibrary{arrows,intersections}
\begin{document}

\def\pivalue{3.1415926}

\begin{tikzpicture}[scale=4,>=latex]

\draw[color=red, line width=1.4pt] (0.000, 0.521)
--(0.031, 0.521)--(0.062, 0.523)--(0.093, 0.525)--(0.124, 0.528)--(0.156, 0.533)--(0.187, 0.538)--(0.218, 0.544)--(0.249, 0.552)--(0.280, 0.561)--(0.311, 0.571)--(0.342, 0.583)--(0.373, 0.596)--(0.404, 0.612)--(0.435, 0.632)--(0.467, 0.655)--(0.498, 0.678)--(0.529, 0.702)--(0.560, 0.727)--(0.591, 0.752)--(0.622, 0.777)--(0.653, 0.802)--(0.684, 0.826)--(0.715, 0.851)--(0.747, 0.875)--(0.778, 0.898)--(0.809, 0.921)--(0.840, 0.943)--(0.871, 0.965)--(0.902, 0.986)--(0.933, 1.006)--(0.964, 1.026)--(0.995, 1.045)--(1.026, 1.063)--(1.058, 1.080)--(1.089, 1.096)--(1.120, 1.111)--(1.151, 1.126)--(1.182, 1.139)--(1.213, 1.152)--(1.244, 1.163)--(1.275, 1.174)--(1.306, 1.183)--(1.338, 1.192)--(1.369, 1.199)--(1.400, 1.205)--(1.431, 1.211)--(1.462, 1.215)--(1.493, 1.218)--(1.524, 1.220)--(1.555, 1.221)--(1.586, 1.221)--(1.617, 1.220)--(1.649, 1.218)--(1.680, 1.215)--(1.711, 1.211)--(1.742, 1.205)--(1.773, 1.199)--(1.804, 1.192)--(1.835, 1.183)--(1.866, 1.174)--(1.897, 1.163)--(1.929, 1.152)--(1.960, 1.139)--(1.991, 1.126)--(2.022, 1.111)--(2.053, 1.096)--(2.084, 1.080)--(2.115, 1.063)--(2.146, 1.045)--(2.177, 1.026)--(2.208, 1.006)--(2.240, 0.986)--(2.271, 0.965)--(2.302, 0.943)--(2.333, 0.921)--(2.364, 0.898)--(2.395, 0.875)--(2.426, 0.851)--(2.457, 0.826)--(2.488, 0.802)--(2.519, 0.777)--(2.551, 0.752)--(2.582, 0.727)--(2.613, 0.702)--(2.644, 0.678)--(2.675, 0.655)--(2.706, 0.632)--(2.737, 0.612)--(2.768, 0.596)--(2.799, 0.583)--(2.831, 0.571)--(2.862, 0.561)--(2.893, 0.552)--(2.924, 0.544)--(2.955, 0.538)--(2.986, 0.533)--(3.017, 0.528)--(3.048, 0.525)--(3.079, 0.523)--(3.110, 0.521)--(3.142, 0.521);


\draw[->] (-0.1,0)--({\pivalue+0.3},0) coordinate[label={$\vartheta$}];
\draw[->] (0,-0.1)--(0,1.6) coordinate[label={right:$s_\gamma(\vartheta)$}];

\node at (0,0) [below right] {$0$};

\draw ({\pivalue/4},-0.02)--({\pivalue/4},0.02);
\draw ({\pivalue/2},-0.02)--({\pivalue/2},0.02);
\draw ({3*\pivalue/4},-0.02)--({3*\pivalue/4},0.02);
\draw ({\pivalue},-0.02)--({\pivalue},0.02);

\node at (0,1) [left] {$1.0$};
\node at (0,0.5) [left] {$0.5$};
\node at (0,1.5) [left] {$1.5$};

\node at ({\pivalue/4},0) [below] {$\frac{\pi}4$};
\node at ({\pivalue/2},0) [below] {$\frac{\pi}2$};
\node at ({3*\pivalue/4},0) [below] {$\frac{3\pi}4$};
\node at ({\pivalue},0) [below] {$\pi$};

\foreach \ytic in {0.1,0.2,...,1.5}{
	\draw (-0.02,\ytic)--(0.02,\ytic);
}

\end{tikzpicture}

\end{document}
