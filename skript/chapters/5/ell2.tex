%
% ell2.tex
%
% (c) 2018 Prof Dr Andreas Müller, Hochschule Rapperswil
%
\documentclass[tikz]{standalone}
\usepackage{amsmath}
\usepackage{times}
\usepackage{txfonts}
\usepackage[utf8]{inputenc}
\usepackage{graphics}
\usepackage{ifthen}
\usepackage{color}
\usetikzlibrary{arrows,intersections}
\begin{document}

\def\pivalue{3.1415926}

\begin{tikzpicture}[scale=4,>=latex]

\input{elltheta.tex}

\draw[->] (-0.1,0)--({\pivalue+0.1},0) coordinate[label={$\vartheta$}];
\draw[->] (0,-0.1)--(0,1.6) coordinate[label={right:$s_\gamma(\vartheta)$}];

\node at (0,0) [below right] {$0$};

\draw ({\pivalue/4},-0.02)--({\pivalue/4},0.02);
\draw ({\pivalue/2},-0.02)--({\pivalue/2},0.02);
\draw ({3*\pivalue/4},-0.02)--({3*\pivalue/4},0.02);
\draw ({\pivalue},-0.02)--({\pivalue},0.02);

\node at (0,1) [left] {$1.0$};
\node at (0,0.5) [left] {$0.5$};
\node at (0,1.5) [left] {$1.5$};

\node at ({\pivalue/4},0) [below] {$\frac{\pi}4$};
\node at ({\pivalue/2},0) [below] {$\frac{\pi}2$};
\node at ({3*\pivalue/4},0) [below] {$\frac{3\pi}4$};
\node at ({\pivalue},0) [below] {$\pi$};

\foreach \ytic in {0.1,0.2,...,1.5}{
	\draw (-0.02,\ytic)--(0.02,\ytic);
}

\end{tikzpicture}

\end{document}
