%
% kugelkoordinaten.tex -- Kugelkoordinaten
%
% (c) 2018 Prof Dr Andreas Müller, Hochschule Rapperswil
%

\subsection{Kugelkoordinaten}
Spektrale Methoden verwenden auf entscheidende Art und Weise die
Besonderheiten des natürlichen Koordinatensystems auf der Kugeloberfläche.
In diesem Abschnitt sollen daher Kugelkoordinaten und die zugehörigen 
Differentialoperatoren genauer untersucht werden.

\subsubsection{Koordinatenumrechnung}
\index{Kugelkoordinaten}%
Wir verwenden in diesem Abschnitt Kugelkoordinaten mit der üblichen
Konvention, dass die geographische Breite als Winkel $\vartheta$
ausgehend vom Nordpol oder der $z$-Achse gemessen wird,
dass also $\vartheta\in[0,\pi]$.
Die geographische Breite wird ausgehen von der $x$-Achse als
Winkel $\varphi$ gemessen.
Schliesslich bezeichnet $r$ die Entfernung eines Punktes vom Nullpunkt.

Ein Breitenkreis zur geographischen Breite $\vartheta$ hat Radius
$r\sin\vartheta$.
Damit ergeben sich die Formeln
\begin{align*}
x
&=
r\sin\vartheta\cos\varphi
\\
y
&=
r\sin\vartheta\sin\varphi
\\
z
&=
r\cos\vartheta
\end{align*}
für die Umrechnung von Kugelkoordinaten in kartesische Koordinaten.
\index{Umrechnungsformel!Kugel-Koordinaten in kartesische Koordinaten}%

\subsubsection{Differentialoperatoren}
Das Ziel ist, den Laplace-Operator in Kugelkoordinaten auszudrücken.
Zu diesem Zweck müssen die partiellen Ableitungsoperatoren nach den
Koordinaten $x$, $y$ und $z$ durch die Operatoren
\[
\frac{\partial}{\partial r},
\;
\frac{\partial}{\partial\vartheta}
\quad\text{und}\quad
\frac{\partial}{\partial\varphi}
\]
ausgedrückt werden.

Für die Ableitungsoperatoren gilt die Kettenregel in der Form
\begin{equation}
\begin{aligned}
\frac{\partial}{\partial x}
&=
\frac{\partial r}{\partial x}\frac{\partial}{\partial r}
+
\frac{\partial\vartheta}{\partial x}\frac{\partial}{\partial\vartheta}
+
\frac{\partial\varphi}{\partial x}\frac{\partial}{\partial\varphi},
\\
\frac{\partial}{\partial y}
&=
\frac{\partial r}{\partial y}\frac{\partial}{\partial r}
+
\frac{\partial\vartheta}{\partial y}\frac{\partial}{\partial\vartheta}
+
\frac{\partial\varphi}{\partial y}\frac{\partial}{\partial\varphi},
\\
\frac{\partial}{\partial z}
&=
\frac{\partial r}{\partial z}\frac{\partial}{\partial r}
+
\frac{\partial\vartheta}{\partial z}\frac{\partial}{\partial\vartheta}
+
\frac{\partial\varphi}{\partial z}\frac{\partial}{\partial\varphi}.
\end{aligned}
\label{skript:kugel:kettenregel}
\end{equation}
Die gesuchte Darstellung der Ableitungsoperatoren läuft also darauf
hinaus, die Ableitungen von Kugelkoordinaten nach nach kartesischen
Koordinaten zu bestimmen.

Die Ableitungen von $r$ werden einfacher zur berechnen durch die
Beziehung
\[
\frac{\partial r^2}{\partial x}
=
2r\frac{\partial r}{\partial x}
\qquad\Rightarrow\qquad
\frac{\partial r}{\partial x}
=\frac1{2r}\frac{\partial r^2}{\partial x}.
\]
Da $r^2=x^2+y^2+z^2$ folgt
\begin{equation}
\begin{aligned}
\frac{\partial r}{\partial x}
&=
\frac1{2r}\frac{\partial}{\partial x}(x^2+y^2+z^2)
=
\frac1{2r}2x
=
\frac{x}{r}
=
\sin\vartheta\cos\varphi,
\\
\frac{\partial r}{\partial y}
&=
\frac1{2r}\frac{\partial}{\partial x}(x^2+y^2+z^2)
=
\frac1{2r}2y
=
\frac{y}{r}
=
\sin\vartheta\sin\varphi
\\
\text{und}
\qquad
\frac{\partial r}{\partial z}
&=
\frac1{2r}\frac{\partial}{\partial z}(x^2+y^2+z^2)
=
\frac1{2r}2z
=
\frac{z}{r}
=
\cos\vartheta.
\end{aligned}
\label{skript:kugel:rableitungen}
\end{equation}

Auf ähnliche Weise lassen sich die Ableitungen von $\vartheta$ bestimmen.
Dazu geht man aus von der Identität $z=r\cos\vartheta$ und leitet sie
nach den kartesischen Koordinaten ab:
\begin{align*}
0
=
\frac{\partial z}{\partial x}
&=
\frac{\partial r}{\partial x}\cos\vartheta
-
r\sin\vartheta\frac{\partial \vartheta}{\partial x}
&&\Rightarrow&
\frac{\partial\vartheta}{\partial x}
&=
\frac{1}{r\sin\vartheta}
\cos\vartheta
\frac{\partial r}{\partial x},
\\
0
=
\frac{\partial z}{\partial y}
&=
\frac{\partial r}{\partial y}\cos\vartheta
-
r\sin\vartheta\frac{\partial \vartheta}{\partial y}
&&\Rightarrow&
\frac{\partial \vartheta}{\partial y}
&=
\frac{1}{r\sin\vartheta}
\cos\vartheta
\frac{\partial r}{\partial y},
\\
1
=
\frac{\partial z}{\partial z}
&=
\frac{\partial r}{\partial y}\cos\vartheta
-
r\sin\vartheta\frac{\partial \vartheta}{\partial y}
&&\Rightarrow&
\frac{\partial\vartheta}{\partial y}
&=
-\frac1{r\sin\vartheta}
\biggl(1-
\cos\vartheta
\frac{\partial r}{\partial z}
\biggr).
\end{align*}
Die Ableitungen von $r$ wurden in \eqref{skript:kugel:rableitungen}
bereits berechnet, so dass wir nach den
Ableitungen von $\vartheta$ auflösen können.
Wir erhalten
\begin{equation}
\begin{aligned}
\frac{\partial\vartheta}{\partial x}
&=
\frac1{r\sin\vartheta} \cos\vartheta \sin\vartheta\cos\varphi
=
\frac{\cos\vartheta\cos\varphi}{r},
\\
\frac{\partial\vartheta}{\partial y}
&=
\frac{1}{r\sin\vartheta} \cos\vartheta \sin\vartheta\sin\varphi
=
\frac{\cos\vartheta\sin\varphi}{r},
\\
\frac{\partial\vartheta}{\partial z}
&=
-\frac{1}{r\sin\vartheta}\underbrace{(1-\cos^2\vartheta)}_{\displaystyle\sin^2\vartheta}
=
-
\frac{\sin\vartheta}r.
\end{aligned}
\label{skript:kugel:thetaableitungen}
\end{equation}
Diese Umformungen waren möglich, weil $z$ nur von $r$ und $\vartheta$ abhing.
Die Kettenregel hat dann eine Beziehung zwischen den beiden Ableitungen
dieser Variablen geliefert.

Die verbleibenden Anleitungen können auf ähnliche Weise aus dem Ausdruck
$x=r\sin\vartheta\cos\varphi$
oder
$y=r\sin\vartheta\sin\varphi$
gewonnen werden.
Wenn man ihn partiell nach kartesischen Koordinaten ableitet,
steht auf der linken Seite eine 1 oder 0,
auf der rechten Seite ein Ausdruck mit allen drei Ableitungen von
Kugelkoordinaten nach $x$, wovon die Ableitungen von $r$ und $\vartheta$
nach den kartesischen Koordinaten bereits bekannt sind.
Man kann also nach den Ableitungen von $\varphi$ auflösen.
Die etwas mühsame Berechnung liefert
\begin{equation}
\begin{aligned}
\frac{\partial \varphi}{\partial x}
&=
-\frac{\sin\varphi}{r\sin\vartheta},
\\
\frac{\partial \varphi}{\partial y}
&=
\frac{\cos\varphi}{r\sin\vartheta},
\\
\frac{\partial \varphi}{\partial z}
&=
0.
\end{aligned}
\label{skript:kugel:phiableitungen}
\end{equation}

In den Gleichungen
\eqref{skript:kugel:rableitungen},
\eqref{skript:kugel:thetaableitungen}
und
\eqref{skript:kugel:phiableitungen}
haben wir alle Koeffizienten für die Kettenregel
\eqref{skript:kugel:kettenregel}
gefunden.
Zusammen erlauben sie, Ableitungen nach kartesischen Koordinaten
durch Ableitungen nach Kugelkoordinaten auszudrücken.

\subsubsection{Laplace-Operator}
In kartesischen Koordinaten ist der Laplace-Operator gegeben durch die
Definition
\[
\Delta
=
\frac{\partial^2}{\partial x^2}
+
\frac{\partial^2}{\partial y^2}
+
\frac{\partial^2}{\partial z^2}.
\]
Durch Einsetzen der Ableitungsoperatoren nach
\eqref{skript:kugel:kettenregel} unter Verwendung der Gleichungen
\eqref{skript:kugel:rableitungen},
\eqref{skript:kugel:thetaableitungen}
und
\eqref{skript:kugel:phiableitungen}
kann man nach ziemlich langwieriger
Rechnung~\cite[Anhang B.3]{skript:mathsem-qm}
den Ausdruck
\begin{equation}
\Delta
=
\frac1{r^2}\frac{\partial}{\partial r}
\biggl(r^2\frac{\partial}{\partial r}\biggr)
+
\frac{1}{r^2\sin\vartheta}\frac{\partial}{\partial\vartheta}
\biggl(\sin\vartheta\frac{\partial}{\partial\vartheta}\biggr)
+
\frac{1}{r^2}
\frac{1}{\sin^2\vartheta}
\frac{\partial^2}{\partial\varphi^2}.
\end{equation}
für den Laplace-Operator in Kugelkoordinaten finden.
\index{Laplace-Operator in Kugelkoordinaten}%

