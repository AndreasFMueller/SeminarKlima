%
% spektral.tex -- Einführung in spektrale Methoden
%
% (c) 2018 Prof Dr Andreas Müller, Hochschule Rapperswil
%
\section{Spektrale Methoden\label{section:spektrale methoden}}
\rhead{Spektrale Methoden}
In den Ausführungen zum Lorenzmodell in Abschnitt~\ref{section:lorenz-modell}
haben wir gesehen, wie man mit Hilfe einer geeigneten Wahl von Basisfunktionen
die komplexen fluiddynamischen partiellen Differentialgleichungen zu einem
System von gewöhnlichen Differentialgleichungen vereinfachen kann.
Die Basis wurde so gewählt, dass einerseits möglichst viel geometrische
Information, im speziellen Fall die rechteckige Form des Definitionsgebietes,
bereits darin einfliesst.
Andererseits sollen sich die wesentlichsten Eigenschaften der Lösung bereits
aus wenigen Basisfunktionen rekonstruieren lassen.
Wie Kapitel~\ref{chapter:lorenz2} zeigt, lässt sich die Idee von
Abschnitt~\ref{section:lorenz-modell} sogar maschinell in eine
immer genaueres Modell erweitern, wenn sich nur eine geeignete Menge
von Basisfunktionen gefunden werden kann.
Ziel diese Abschnittes ist zu illustrieren, wie so eine Basis von
Funktionen aussehen könnte, mit der man globale Modelle vereinfachen
könnte.

%
% kugelkoordinaten.tex -- Kugelkoordinaten
%
% (c) 2018 Prof Dr Andreas Müller, Hochschule Rapperswil
%

\subsection{Kugelkoordinaten}
Spektrale Methoden verwenden auf entscheidende Art und Weise die
Besonderheiten des natürlichen Koordinatensystems auf der Kugeloberfläche.
In diesem Abschnitt sollen daher Kugelkoordinaten und die zugehörigen 
Differentialoperatoren genauer untersucht werden.

\subsubsection{Koordinatenumrechnung}
\index{Kugelkoordinaten}%
Wir verwenden in diesem Abschnitt Kugelkoordinaten mit der üblichen
Konvention, dass die geographische Breite als Winkel $\vartheta$
ausgehend vom Nordpol oder der $z$-Achse gemessen wird,
dass also $\vartheta\in[0,\pi]$.
Die geographische Breite wird ausgehen von der $x$-Achse als
Winkel $\varphi$ gemessen.
Schliesslich bezeichnet $r$ die Entfernung eines Punktes vom Nullpunkt.

Ein Breitenkreis zur geographischen Breite $\vartheta$ hat Radius
$r\sin\vartheta$.
Damit ergeben sich die Formeln
\begin{align*}
x
&=
r\sin\vartheta\cos\varphi
\\
y
&=
r\sin\vartheta\sin\varphi
\\
z
&=
r\cos\vartheta
\end{align*}
für die Umrechnung von Kugelkoordinaten in kartesische Koordinaten.
\index{Umrechnungsformel!Kugel-Koordinaten in kartesische Koordinaten}%

\subsubsection{Differentialoperatoren}
Das Ziel ist, den Laplace-Operator in Kugelkoordinaten auszudrücken.
Zu diesem Zweck müssen die partiellen Ableitungsoperatoren nach den
Koordinaten $x$, $y$ und $z$ durch die Operatoren
\[
\frac{\partial}{\partial r},
\;
\frac{\partial}{\partial\vartheta}
\quad\text{und}\quad
\frac{\partial}{\partial\varphi}
\]
ausgedrückt werden.

Für die Ableitungsoperatoren gilt die Kettenregel in der Form
\begin{equation}
\begin{aligned}
\frac{\partial}{\partial x}
&=
\frac{\partial r}{\partial x}\frac{\partial}{\partial r}
+
\frac{\partial\vartheta}{\partial x}\frac{\partial}{\partial\vartheta}
+
\frac{\partial\varphi}{\partial x}\frac{\partial}{\partial\varphi},
\\
\frac{\partial}{\partial y}
&=
\frac{\partial r}{\partial y}\frac{\partial}{\partial r}
+
\frac{\partial\vartheta}{\partial y}\frac{\partial}{\partial\vartheta}
+
\frac{\partial\varphi}{\partial y}\frac{\partial}{\partial\varphi},
\\
\frac{\partial}{\partial z}
&=
\frac{\partial r}{\partial z}\frac{\partial}{\partial r}
+
\frac{\partial\vartheta}{\partial z}\frac{\partial}{\partial\vartheta}
+
\frac{\partial\varphi}{\partial z}\frac{\partial}{\partial\varphi}.
\end{aligned}
\label{skript:kugel:kettenregel}
\end{equation}
Die gesuchte Darstellung der Ableitungsoperatoren läuft also darauf
hinaus, die Ableitungen von Kugelkoordinaten nach nach kartesischen
Koordinaten zu bestimmen.

Die Ableitungen von $r$ werden einfacher zur berechnen durch die
Beziehung
\[
\frac{\partial r^2}{\partial x}
=
2r\frac{\partial r}{\partial x}
\qquad\Rightarrow\qquad
\frac{\partial r}{\partial x}
=\frac1{2r}\frac{\partial r^2}{\partial x}.
\]
Da $r^2=x^2+y^2+z^2$ folgt
\begin{equation}
\begin{aligned}
\frac{\partial r}{\partial x}
&=
\frac1{2r}\frac{\partial}{\partial x}(x^2+y^2+z^2)
=
\frac1{2r}2x
=
\frac{x}{r}
=
\sin\vartheta\cos\varphi,
\\
\frac{\partial r}{\partial y}
&=
\frac1{2r}\frac{\partial}{\partial x}(x^2+y^2+z^2)
=
\frac1{2r}2y
=
\frac{y}{r}
=
\sin\vartheta\sin\varphi
\\
\text{und}
\qquad
\frac{\partial r}{\partial z}
&=
\frac1{2r}\frac{\partial}{\partial z}(x^2+y^2+z^2)
=
\frac1{2r}2z
=
\frac{z}{r}
=
\cos\vartheta.
\end{aligned}
\label{skript:kugel:rableitungen}
\end{equation}

Auf ähnliche Weise lassen sich die Ableitungen von $\vartheta$ bestimmen.
Dazu geht man aus von der Identität $z=r\cos\vartheta$ und leitet sie
nach den kartesischen Koordinaten ab:
\begin{align*}
0
=
\frac{\partial z}{\partial x}
&=
\frac{\partial r}{\partial x}\cos\vartheta
-
r\sin\vartheta\frac{\partial \vartheta}{\partial x}
&&\Rightarrow&
\frac{\partial\vartheta}{\partial x}
&=
\frac{1}{r\sin\vartheta}
\cos\vartheta
\frac{\partial r}{\partial x},
\\
0
=
\frac{\partial z}{\partial y}
&=
\frac{\partial r}{\partial y}\cos\vartheta
-
r\sin\vartheta\frac{\partial \vartheta}{\partial y}
&&\Rightarrow&
\frac{\partial \vartheta}{\partial y}
&=
\frac{1}{r\sin\vartheta}
\cos\vartheta
\frac{\partial r}{\partial y},
\\
1
=
\frac{\partial z}{\partial z}
&=
\frac{\partial r}{\partial y}\cos\vartheta
-
r\sin\vartheta\frac{\partial \vartheta}{\partial y}
&&\Rightarrow&
\frac{\partial\vartheta}{\partial y}
&=
-\frac1{r\sin\vartheta}
\biggl(1-
\cos\vartheta
\frac{\partial r}{\partial z}
\biggr).
\end{align*}
Die Ableitungen von $r$ wurden in \eqref{skript:kugel:rableitungen}
bereits berechnet, so dass wir nach den
Ableitungen von $\vartheta$ auflösen können.
Wir erhalten
\begin{equation}
\begin{aligned}
\frac{\partial\vartheta}{\partial x}
&=
\frac1{r\sin\vartheta} \cos\vartheta \sin\vartheta\cos\varphi
=
\frac{\cos\vartheta\cos\varphi}{r},
\\
\frac{\partial\vartheta}{\partial y}
&=
\frac{1}{r\sin\vartheta} \cos\vartheta \sin\vartheta\sin\varphi
=
\frac{\cos\vartheta\sin\varphi}{r},
\\
\frac{\partial\vartheta}{\partial z}
&=
-\frac{1}{r\sin\vartheta}\underbrace{(1-\cos^2\vartheta)}_{\displaystyle\sin^2\vartheta}
=
-
\frac{\sin\vartheta}r.
\end{aligned}
\label{skript:kugel:thetaableitungen}
\end{equation}
Diese Umformungen waren möglich, weil $z$ nur von $r$ und $\vartheta$ abhing.
Die Kettenregel hat dann eine Beziehung zwischen den beiden Ableitungen
dieser Variablen geliefert.

Die verbleibenden Anleitungen können auf ähnliche Weise aus dem Ausdruck
$x=r\sin\vartheta\cos\varphi$
oder
$y=r\sin\vartheta\sin\varphi$
gewonnen werden.
Wenn man ihn partiell nach kartesischen Koordinaten ableitet,
steht auf der linken Seite eine 1 oder 0,
auf der rechten Seite ein Ausdruck mit allen drei Ableitungen von
Kugelkoordinaten nach $x$, wovon die Ableitungen von $r$ und $\vartheta$
nach den kartesischen Koordinaten bereits bekannt sind.
Man kann also nach den Ableitungen von $\varphi$ auflösen.
Die etwas mühsame Berechnung liefert
\begin{equation}
\begin{aligned}
\frac{\partial \varphi}{\partial x}
&=
-\frac{\sin\varphi}{r\sin\vartheta},
\\
\frac{\partial \varphi}{\partial y}
&=
\frac{\cos\varphi}{r\sin\vartheta},
\\
\frac{\partial \varphi}{\partial z}
&=
0.
\end{aligned}
\label{skript:kugel:phiableitungen}
\end{equation}

In den Gleichungen
\eqref{skript:kugel:rableitungen},
\eqref{skript:kugel:thetaableitungen}
und
\eqref{skript:kugel:phiableitungen}
haben wir alle Koeffizienten für die Kettenregel
\eqref{skript:kugel:kettenregel}
gefunden.
Zusammen erlauben sie, Ableitungen nach kartesischen Koordinaten
durch Ableitungen nach Kugelkoordinaten auszudrücken.

\subsubsection{Laplace-Operator}
In kartesischen Koordinaten ist der Laplace-Operator gegeben durch die
Definition
\[
\Delta
=
\frac{\partial^2}{\partial x^2}
+
\frac{\partial^2}{\partial y^2}
+
\frac{\partial^2}{\partial z^2}.
\]
Durch Einsetzen der Ableitungsoperatoren nach
\eqref{skript:kugel:kettenregel} unter Verwendung der Gleichungen
\eqref{skript:kugel:rableitungen},
\eqref{skript:kugel:thetaableitungen}
und
\eqref{skript:kugel:phiableitungen}
kann man nach ziemlich langwieriger
Rechnung~\cite[Anhang B.3]{skript:mathsem-qm}
den Ausdruck
\begin{equation}
\Delta
=
\frac1{r^2}\frac{\partial}{\partial r}
\biggl(r^2\frac{\partial}{\partial r}\biggr)
+
\frac{1}{r^2\sin\vartheta}\frac{\partial}{\partial\vartheta}
\biggl(\sin\vartheta\frac{\partial}{\partial\vartheta}\biggr)
+
\frac{1}{r^2}
\frac{1}{\sin^2\vartheta}
\frac{\partial^2}{\partial\varphi^2}.
\end{equation}
für den Laplace-Operator in Kugelkoordinaten finden.
\index{Laplace-Operator in Kugelkoordinaten}%


%
% kugelfunktionen.tex -- Kugelfunktionen
%
% (c) 2018 Prof Dr Andreas Müller, Hochschule Rapperswil
%
\subsection{Kugelfunktionen}
Die Basisfunktionen im Lorenz-Modell waren aus zwei Gründen besonders
erfolgreich.
\begin{enumerate}
\item
Die Basisfunktionen waren Produkte von Funktionen, die jeweils nur von
einer Koordinate abhängen.
In einem Produkt 
$f(x,y)=X(x)\cdot Y(y)$ 
sind die Ableitungen nach den Koordinaten besonders einfach auszurechnen,
da gilt
\[
\begin{aligned}
\frac{\partial f}{\partial x}(x,y) &= X'(x)\cdot Y(y)
&&\text{und}&
\frac{\partial f}{\partial y}(x,y) &= X(x)\cdot Y'(y).
\end{aligned}
\]
\item
Die Basisfunktionen waren Eigenfunktionen des Laplace-Operators, also
\[
\Delta f = \lambda f.
\]
Da in den Gleichungen der Strömungsdynamik der Laplace-Operator
prominent vorkommt, bedeutet diese Eigenschaft, dass die Wirkung des
Laplace-Operators auf die Basisfunktionen durch Multiplikation mit
dem Eigenwert ersetzt werden kann.
Dadurch werden die Gleichungen sehr vereinfacht und die Ordnung
der Differentialgleichung reduziert sich.
\end{enumerate}
Wenn der Erfolg der speziellen Basiswahl im Lorenz-System für ein
Wetter- oder Klimamodell auf der Kugeloberfläche repliziert werden
soll, dann ist nahe liegend, dass dazu Funktionen mit den gleichen
Eigenschaften in Kugelkoordinaten gefunden werden müssen.

\subsubsection{Separationsansatz}
Die Produkteigenschaft bedeutet, dass die Basisfunktionen in der Form
\[
f(r,\varphi,\vartheta)
=
R(r)\cdot \Phi(\varphi)\cdot \Theta(\vartheta)
\]
gefunden werden müssen.
Das in Abschnitt~\ref{section:pdeloesungen} dargestellte
Separationsverfahren für partielle
Differentialgleichungen~\cite[Chapter 4]{skript:pde}
basiert genau auf dieser Art von Ansatz.

\subsubsection{Eigenwertgleichung}
Die Eigenwerteigenschaft bedeutet, dass die Funktionen Eigenfunktionen
des Laplace-Operators sein müssen, also Lösungen der partiellen
Differentialgleichungen
\[
\Delta f = \lambda f.
\]
Wenden wir den Laplace-Operator in Kugelkoordinaten auf $f$ an, finden
wir
\begin{align*}
\Delta f
&=
\biggl(
\frac{1}{r^2} \frac{\partial}{\partial r}
\biggl(r^2\frac{\partial f}{\partial r}\biggr)
+
\frac1{r^2\sin\vartheta}\frac{\partial}{\partial\vartheta}
\biggl(\sin\vartheta\frac{\partial f}{\partial\vartheta}\biggr)
+
\frac1{r^2\sin^2\vartheta}\frac{\partial^2 f}{\partial\varphi^2}
\biggr)
R(r)\cdot \Phi(\varphi)\cdot \Theta(\vartheta)
\\
&=
\frac1{r^2}\frac{\partial}{\partial r}\bigl(r^2R'(r)\bigr)
\cdot \Theta(\vartheta)\cdot \Phi(\varphi)
+
\frac1{r^2\sin\vartheta}\frac{d}{d\vartheta}
\bigl(\sin\vartheta \Theta'(\vartheta)\bigr)
\cdot R(r)\cdot \Phi(\varphi)
\\
&\qquad
+
\frac1{r^2\sin^2\vartheta}\Phi''(\varphi)
\cdot R(r)\cdot \Theta(\vartheta)
\\
&=
\frac1{r^2}\bigl(2rR'(r)+r^2R''(r)\bigr)
\Theta(\vartheta)\Phi(\varphi)
+
\frac1{r^2\sin\vartheta}
\bigl(\sin\vartheta\Theta'(\vartheta)\bigr)'
\cdot R(r)\cdot\Phi(\varphi)
\\
&\qquad
+
\frac1{r^2\sin^2\vartheta}
\Phi''(\varphi)
\cdot R(r)\cdot \Theta(\vartheta)
\\
&=
\lambda
R(r)\cdot \Theta(\vartheta)\cdot \Phi(\varphi).
\end{align*}

\subsubsection{Separation von $r$}
Um die einzelnen Funktionen zu isolieren, teilen wir durch $f$.
Zwar kann $f$ Nullstellen haben, aber für die meisten Werte der Koordinaten
ist $f$ von Null verschieden, für diese Punkte ist die Division unproblematisch
und ausreichend, um die Faktoren zu bestimmen.
Wir erhalten
\begin{align*}
\frac{2rR'(r)+r^2R''(r)}{r^2R(r)}
+
\frac{
\bigl(\sin\vartheta\Theta'(\vartheta)\bigr)'
}{r^2\sin\vartheta\Theta(\vartheta)}
+
\frac{1}{r^2\sin^2\vartheta}\frac{\Phi''(\varphi)}{\Phi(\varphi)}
&=\lambda
\end{align*}
Um die Variable $r$ allein auf die linke Seite zu bringen, multiplizieren wir
mit $r^2$, subtrahieren $\lambda r^2$ und bringen den zweiten und dritten 
Term auf der linken Seite auf die rechte Seite.
So erhalten wir
\begin{align*}
\frac{r^2R''(r)+2rR'(r)-\lambda r^2R(r)}{R(r)}
&=
-
\frac{
\bigl(\sin\vartheta\,\Theta'(\vartheta)\bigr)'
}{\sin\vartheta\,\Theta(\vartheta)}
-\frac{1}{\sin^2\vartheta}\frac{\Phi''(\varphi)}{\Phi(\varphi)}
\end{align*}
Die linke Seite hängt nur von $r$ ab, die rechte Seite nur von $\vartheta$
und $\varphi$.
Dies ist nur möglich, wenn beide Seiten konstant sind.
Es gibt also ein Zahl $\mu$ derart, dass
\begin{align}
\frac{r^2R''(r)+2rR'(r)-\lambda r^2R(r)}{R(r)}&=\mu,
\\
\frac{\bigl(\sin\vartheta\,\Theta'(\vartheta)\bigr)'}{\sin\vartheta\,\Theta(\vartheta)}
+
\frac{1}{\sin^2\vartheta}\frac{\Phi''(\varphi)}{\Phi(\varphi)}
&=
-\mu.
\label{skript:kugel:thetaphi}
\end{align}
Die erste Gleichung kann man vereinfachen zu
\begin{equation}
r^2R''(r)+2rR'(r)-(\lambda r^2-\mu)R(r) = 0,
\end{equation}
eine gewöhnliche lineare Differentialgleichung zweiter Ordnung.
$\mu$ kann nicht beliebig gewählt werden, der Wert muss so sein,
dass \eqref{skript:kugel:thetaphi} gelöst werden kann.

Doch auch $\lambda$ ist nicht beliebig, sein Wert muss so gewählt
werden, dass eventuelle Randbedingungen für die zugehörigen Funktion
$R(r)$ erfüllt sind.
Da die $r$-Abhängigkeit für die folgende Diskussion nicht wichtig ist,
verfolgen wir diese Frage hier nicht weiter.

\subsubsection{Separation von $\vartheta$ und $\varphi$}
Die Gleichung~\eqref{skript:kugel:thetaphi} enthält nur noch die
Variablen $\vartheta$ und $\varphi$. 
Wir versuchen den gleichen Trick erneut: indem wir mit $\sin^2\vartheta$
multiplizieren, den Term mit $\mu$ auf die linke Seite bringen und
den zweiten Term auf die rechte Seite bringen, erhalten wir
\[
\sin\vartheta
\frac{1}{\Theta(\vartheta)}
\frac{d}{d\vartheta}\bigl(\sin\vartheta\,\Theta'(\vartheta)\bigr)
+
\mu\sin^2\vartheta
=
-\frac{\Phi''(\varphi)}{\Phi(\varphi)}.
\]
Erneut haben wir eine Gleichung, deren linke Seite nur von $\vartheta$
und deren rechte Seite nur von $\varphi$ abhängt.
Also sind wieder beide Seiten konstant, es gibt also eine Konstante
$\nu$ derart, dass $\Theta(\vartheta)$ und $\Phi(\varphi)$ die 
Gleichungen
\begin{align}
\sin\vartheta\frac{d}{d\vartheta}
\biggl(
\sin\vartheta\frac{d}{d\vartheta}\Theta(\vartheta)
\biggr)
&=
(-\mu\sin^2\vartheta+\nu)\Theta(\vartheta)
\label{skript:kugel:thetagl}
\\
-\frac{\Phi''(\varphi)}{\Phi(\varphi)}&=\nu
\label{skript:kugel:phigl}
\end{align}
erfüllen.

\subsubsection{Lösungsfunktionen $\Phi(\varphi)$}
Die zweite Gleichung~\eqref{skript:kugel:phigl} ist gleichbedeutend mit
\begin{equation}
\Phi''(\varphi)=-\nu\Phi(\varphi),
\label{skript:kugel:philsg}
\end{equation}
wobei $\Phi(\varphi)$ eine $2\pi$-periodische Funktion ist.
Die Lösungen der Gleichung~\eqref{skript:kugel:philsg}
sind $\cos\sqrt{\nu}\varphi$ und $\sin\sqrt{\nu}\varphi$,
aber die Periodizität verlangt, dass $\sqrt{\nu}$ eine ganze Zahl ist.
Es muss also gelten $\nu=k^2$ mit $k\in \mathbb N$.

\subsubsection{Lösungsfunktionen $\Theta(\vartheta)$}
Im vorangegangen Absatz wurde gezeigt, dass $\nu=k^2$ ist, was die
Gleichung~\eqref{skript:kugel:thetagl} zu
\begin{equation}
\sin\vartheta\frac{d}{d\vartheta}\sin\vartheta\frac{d}{d\vartheta}\Theta(\vartheta)
=
\biggl(\sin\vartheta\frac{d}{d\vartheta}\biggr)^2 \Theta(\vartheta)
=
(-\mu\sin^2\vartheta+k^2)\Theta(\vartheta)
\label{skript:kugel:legendredgl0}
\end{equation}
In dieser Form ist die Differentialgleichung nicht so leicht zu
erkennen.
Schreibt man aber $z=\sin\vartheta$, dann wird die Ableitung einer
Funktion $P(z)=\Theta(\vartheta)$
\begin{align*}
\frac{d}{d\vartheta}\Theta(\vartheta)
=
\frac{d}{d\vartheta}P(\cos\vartheta)
=
-P'(\cos\vartheta) \sin\vartheta
&=
-\sqrt{1-\cos^2\vartheta}P'(\cos\vartheta)
\\
&=
-\sqrt{1-z^2}P'(z)
=
-\sqrt{1-z^2}\frac{d}{dz}P(z).
\end{align*}
Ableitungen nach $\vartheta$ sind also zu ersetzen durch Ableitungen
nach $z$ gefolgt von Multiplikation mit $-\sqrt{1-z^2}$.
In der Differentialgleichung~\eqref{skript:kugel:legendredgl0}
wird die Ableitung nach $\vartheta$ jeweils auch noch mit $\sin\vartheta=\sqrt{1-z^2}$
multipliziert.
Der Operator
\begin{equation}
\sin\vartheta\frac{d}{d\vartheta}
\qquad
\text{bekommt daher die Form}
\qquad
-(1-z^2)\frac{d}{dz}.
\label{skript:kugel:zoperator}
\end{equation}
Das Vorzeichen ist nicht wichtig, da der Operator in der Differentialgleichung
\eqref{skript:kugel:legendredgl0} nur im Quadrat vorkommt.

Wir setzen jetzt die Form \eqref{skript:kugel:zoperator}
des Differentialoperators in die Differentialgleichung
\eqref{skript:kugel:legendredgl0} ein
und erhalten 
\begin{align}
(1-z^2)\frac{d}{dz}\bigl((1-z^2)P'(z)\bigr)
&=
(-\mu(1-z^2)+m^2)P(z)
\notag
\\
\Rightarrow\qquad
(1-z^2)P''(z) -2z P'(z)
+
\mu P(z)
-\frac{m^2}{1-z^2}P(z)
&=
0.
\label{skript:kugel:legendredgl}
\end{align}
Für $m=0$ ist
\eqref{skript:kugel:legendredgl}
die sogenannte Legendresche Differentialgleichung.
\index{Differentialgleichung!Legendresche}
\index{Legendresche Differentialgleichung}
Sie hat Lösungen für $\mu=l(l+1)$ mit $l\in\mathbb N$.
Für $m>0$ ist
\eqref{skript:kugel:legendredgl}
die assozierte Legendre-Differentialgleichung.
Für beide Gleichung lassen sich Lösungen angeben, es sind die
sogenannten Legendre-Polynome $P_l(z)$ im Fall $m=0$ und die zugeordneten
Legendre-Polynome $P_l^m(z)$ für beliebiges $m$.

\subsubsection{Kugelflächenfunktionen}
Mit den gefundenen Lösungen für $\Phi(\varphi)$ und $\Theta(\vartheta)$
finden wir jetzt die allgemeinen Lösungen 
der Differentialgleichung \eqref{skript:kugel:thetaphi}.
Es sind die Funktionen
\[
Y_l^m(\vartheta,\varphi)
=
N_{lm}
P_l^m(\cos\vartheta) e^{im\varphi}
\]
mit einem geeigneten Normierungsfaktor $N_{lm}$.
Diese Funktionen haben genau die Eigenschaften, die wir in der Einleitung
dieses Abschnitts als Voraussetzungen für eine geeignete Basis
gefordert haben.




%
% spektralegleichungen.tex
%
% (c) 2018 Prof Dr Andreas Müller, Hochschule Rapperswil
%
\subsection{Spektrale Gleichungen\label{subsection:spektrale gleichungen}}
Mit den Kugelfunktionen steht jetzt eine Basis für Funktionen auf der
Kugeloberfläche oder auch für eine Kugelschicht wie die Atmosphäre zur
Verfügung.
Um die zeitliche Entwicklung zu verstehen, wie sie zum Beispiel vom
Modell~\eqref{skript:psidgl} beschrieben wird, müssen diese Gleichungen
umformuliert werden als Gleichungen für die Koeffizienten einer Darstellung
der Lösungsfunktion als Linearkombination 

Um das Prinzip zu veranschaulichen, wird dies für die Wärmeleitungsgleichung
\[
\frac{\partial T}{\partial t}
=
\kappa \Delta T
\]
auf der Kugeloberfläche gezeigt.
Die Temperatur kann also als Linearkombination
\begin{equation}
T(t, \vartheta, \varphi)
=
\sum_{l=0}^\infty \sum_{m=-l}^{l} a_{lm}(t) Y^m_l(\vartheta,\varphi)
\label{skript:spektral:ansatz}
\end{equation}
geschrieben werden.
Die Funtionen $Y^m_l(\vartheta,\varphi)$ sind nach Konstruktion
Eigenfunktionen des Laplace-Operators, der zugehörige Eigenwert soll
mit $\lambda^m_l$ abgekürzt werden.

Setzt man \eqref{skript:spektral:ansatz} in die Differentialgleichung
ein, wird sie zu
\begin{align*}
\frac{\partial T}{\partial t}
&=
\sum_{l=0}^\infty \sum_{m=-l}^{l} \frac{\partial a_{lm}(t)}{\partial t} Y^m_l(\vartheta,\varphi)
=
\sum_{l=0}^\infty \sum_{m=-l}^{l} \dot a_{lm}(t) Y^m_l(\vartheta,\varphi)
\\
\kappa\Delta T
&=
\kappa
\sum_{l=0}^\infty \sum_{m=-l}^{l} a_{lm}(t) \Delta Y^m_l(\vartheta,\varphi)
=
\kappa
\sum_{l=0}^\infty \sum_{m=-l}^{l} a_{lm}(t) \lambda^m_l Y^m_l(\vartheta,\varphi)
\end{align*}
Mittels Koeffizientenvergleich folgen die gewöhnlichen
Differentialgleichungen
\[
\dot a_{lm}(t)
=
\kappa\lambda^m_l
a_{lm},
\]
deren Lösungen man sofort angeben kann, sie sind
\[
a_{lm}(t)
=
a_{lm}(0) e^{\kappa\lambda^m_l t}.
\]
Damit kann man die Lösung der Wärmeleitungsgleichung sofort hinschreiben,
sie ist
\begin{equation}
T(t,\vartheta,\varphi)
=
\sum_{l=0}^\infty \sum_{m=-l}^{l}
a_{lm}(0) e^{\kappa\lambda^m_l t}
Y^m_l(\vartheta,\varphi).
\end{equation}
Die Verwendung der Basis der Kugelfunktionen hat also zu einer besonders 
einfachen Lösung geführt.

Die Gleichung~\eqref{skript:psidgl} ist nicht linear, die Lösung
der Gleichung wird nicht mehr so einfach sein wie im Fall der
Wärmeleitungsgleichung.
Da aber die Kugelfunktionen eine Basis bilden, müssen sich auch
Produkte von Kugelfunktionen oder Ableitungen von Kugelfunktionen
als Linearkombinationen von Kugelfunktionen ausdrücken lassen,
wie das im Beispiel auf Seite~\pageref{subsubsection:komplexeres}
für die nichtlineare Gleichung von Burgers und die Basis der
Exponentialfunktionen vorgeführt wurde.
Auch auf der Kugeloberfläche lässt sich daher ein System von gewöhnlichen
Differentialgleichungen für die Koeffizienten $a_{lm}(t)$ aufstellen,
mit dem Unterschied, dass sie nicht mehr linear sein werden.










