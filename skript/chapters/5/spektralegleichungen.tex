%
% spektralegleichungen.tex
%
% (c) 2018 Prof Dr Andreas Müller, Hochschule Rapperswil
%
\subsection{Spektrale Gleichungen\label{subsection:spektrale gleichungen}}
Mit den Kugelfunktionen steht jetzt eine Basis für Funktionen auf der
Kugeloberfläche oder auch für eine Kugelschicht wie die Atmosphäre zur
Verfügung.
Um die zeitliche Entwicklung zu verstehen, wie sie zum Beispiel vom
Modell~\eqref{skript:psidgl} beschrieben wird, müssen diese Gleichungen
umformuliert werden als Gleichungen für die Koeffizienten einer Darstellung
der Lösungsfunktion als Linearkombination 

Um das Prinzip zu veranschaulichen, wird dies für die Wärmeleitungsgleichung
\[
\frac{\partial T}{\partial t}
=
\kappa \Delta T
\]
auf der Kugeloberfläche gezeigt.
Die Temperatur kann also als als Linearkombination
\begin{equation}
T(t, \vartheta, \varphi)
=
\sum_{l=0}^\infty \sum_{m=-l}^{l} a_{lm}(t) Y^m_l(\vartheta,\varphi)
\label{skript:spektral:ansatz}
\end{equation}
geschrieben werden.
Die Funtionen $Y^m_l(\vartheta,\varphi)$ sind nach Konstruktion
Eigenfunktionen des Laplace-Operators, der zugehörige Eigenwert soll
mit $\lambda^m_l$ abgekürzt werden.

Setzt man \eqref{skript:spektral:ansatz} in die Differentialgleichung
ein, wird sie zu
\begin{align*}
\frac{\partial T}{\partial t}
&=
\sum_{l=0}^\infty \sum_{m=-l}^{l} \frac{\partial a_{lm}(t)}{\partial t} Y^m_l(\vartheta,\varphi)
=
\sum_{l=0}^\infty \sum_{m=-l}^{l} \dot a_{lm}(t) Y^m_l(\vartheta,\varphi)
\\
\kappa\Delta T
&=
\kappa
\sum_{l=0}^\infty \sum_{m=-l}^{l} a_{lm}(t) \Delta Y^m_l(\vartheta,\varphi)
=
\kappa
\sum_{l=0}^\infty \sum_{m=-l}^{l} a_{lm}(t) \lambda^m_l Y^m_l(\vartheta,\varphi)
\end{align*}
Mittels Koeffizientenvergleich folgen die gewöhnlichen
Differentialgleichungen
\[
\dot a_{lm}(t)
=
\kappa\lambda^m_l
a_{lm},
\]
deren Lösungen man sofort angeben kann, sie sind
\[
a_{lm}(t)
=
a_{lm}(0) e^{\kappa\lambda^m_l t}.
\]
Damit kann man die Lösung der Wärmeleitungsgleichung sofort hinschreiben,
sie ist
\begin{equation}
T(t,\vartheta,\varphi)
=
\sum_{l=0}^\infty \sum_{m=-l}^{l}
a_{lm}(0) e^{\kappa\lambda^m_l t}
Y^m_l(\vartheta,\varphi).
\end{equation}
Die Verwendung der Basis der Kugelfunktionen hat also zu einer besonders 
einfachen Lösung geführt.

Die Gleichung~\eqref{skript:psidgl} ist nicht linear, die Lösung
der Gleichung wird nicht mehr so einfach sein wie im Fall der
Wärmeleitungsgleichung.
Da aber die Kugelfunktionen eine Basis bilden, müssen sich auch
Produkte von Kugelfunktionen oder Ableitungen von Kugelfunktionen
als Linearkombinationen von Kugelfunktionen ausdrücken lassen,
wie das im Beispiel auf Seite~\pageref{subsubsection:komplexeres}
für die nichtlineare Gleichung von Burgers und die Basis der
Exponentialfunktionen vorgeführt wurde.
Auch auf der Kugeloberfläche lässt sich daher ein System von gewöhnlichen
Differentialgleichungen für die Koeffizienten $a_{lm}(t)$ aufstellen,
mit dem Unterschied, dass sie nicht mehr linear sein werden.







