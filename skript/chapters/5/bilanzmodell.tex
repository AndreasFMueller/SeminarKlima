%
% bilanzmodell.tex
%
% (c) 2018 Prof Dr Andreas Müller, Hochschule Rapperswil
%
\documentclass[tikz]{standalone}
\usepackage{amsmath}
\usepackage{times}
\usepackage{txfonts}
\usepackage[utf8]{inputenc}
\usepackage{graphics}
\usepackage{ifthen}
\usepackage{color}
\usetikzlibrary{arrows,intersections}
\begin{document}

\def\Q{342}
\def\Teins{234.5} \def\Peins{103}
\def\Tzwei{264.2} \def\Pzwei{165.5}
\def\Tdrei{289.3} \def\Pdrei{238.3}

\begin{tikzpicture}[>=latex,thick,scale=1.35]

\foreach \T in {220,230,...,310}{
	\draw[color=gray!50] ({\T/10-22},0)--({\T/10-22},6);
	\draw ({\T/10-22},-0.05)--({\T/10-22},0.05);
	\node at ({\T/10 - 22},-0.1) [below] {$\T$};
}

\foreach \P in {50,100,...,350}{
	\draw[color=gray!50] (0,{\P/50-1})--(9,{\P/50-1});
	\node at (-0.1,{\P/50-1}) [left] {$\P$};
	\draw (-0.05,{\P/50-1})--(0.05,{\P/50-1});
}

\draw[->] (-0.05,0)--(9.3,0) coordinate[label={$T\, [\text{K}]$}];
\draw[->] (0,-0.05)--(0,6.3) coordinate[label={right:$P\,[\text{Wm}^{-2}]$}];

\draw[color=red,line width=1.4pt] plot[samples=100,domain=220:310]
	({\x/10-22},{(0.6 * (0.015431 * \x)^4) / 50 - 1});

\draw[color=blue,line width=1.4pt] plot[samples=100,domain=220:310]
	({\x/10-22},{(0.5 + 0.2 * tanh((\x - 265)/10)) * \Q / 50 -1});

\draw ({\Teins/10-22},{\Peins/50 - 1}) circle[radius=0.05];
\node at ({\Teins/10-22},{\Peins/50 -1 }) [above] {$T^*_1$};

\draw ({\Tzwei/10-22},{\Pzwei/50 - 1}) circle[radius=0.05];
\node at ({\Tzwei/10-22},{\Pzwei/50 -1 }) [above left] {$T^*_2$};

\draw ({\Tdrei/10-22},{\Pdrei/50 - 1}) circle[radius=0.05];
\node at ({\Tdrei/10-22},{\Pdrei/50 -1 }) [above] {$T^*_3$};

\end{tikzpicture}

\end{document}
