%
% zonen.tex -- Zonenmodelle
%
% (c) 2018 Prof Dr Andreas Müller, Hochschule Rapperswil
%
\section{Zonenmodelle}
\rhead{Zonenmodelle}
Das einfachste Bilanzmodell kommt mit einer einzigen modellierten
Grösse, der globalen Mitteltemperatur $T(t)$ aus.
Es reicht aus, den Zusammenhang zwischen Treibhausgaskonzentration
und globaler Erwärmung zu erklären.
Es ist aber zu grob, den zum Beispiel für die thermohaline Zirkulation
wesentliche Abhängigkeit der Temperatur von der Breite wiederzugeben.

Im Kapitel~\ref{chapter:fluiddynamik} wurde gezeigt, dass der
Coriolis-Effekt dazu führt, dass die Atmosphäre mindestens im Mittel
in Zonen organisiert ist, die umso ausgeprägter sind, je schneller
die Umdrehungsgeschwindigkeit ist.
Die in Abbildung~\ref{skript:globalezirkulation} sichtbaren Zellen
zeigen, dass der Austausch von Energie zwischen den Zonen erschwert
ist, weil die Zellengrenzen für die Strömung undurchlässig sind.
Die Strömung innerhalb der Zellen und die vergleichsweise schnelle
Rotation der Erde führt also dazu, dass sich die Atmosphäre in
Zonen aufteilen lässt, die einzeln mit einer Zonenmitteltemperatur
modellieren lassen.

Ein erstes solches Modell könnte die Nord- und die Südhalbkugel voneinander
trennen.
Der Energieaustausch zwischen den beiden Halbkugeln ist dadurch 
eingeschränkt, dass die globale Zirkulation den Äquator nicht überquert.
Die mit den Jahreszeiten schwankende Einstrahlung auf eine Halbkugel
wird also nicht durch die andere Halbkugel kompensiert, so dass die
Mitteltemperaturen der Halbkugeln ausgeprägte Jahresrythmen zeigen.
In Kapitel~\ref{chapter:neigung} wird gezeigt, wie sich daraus ableiten
lässt, wie die schwankende Neigung der Erdachse zu Eiszeiten führen kann.

Ein einfaches diskretes Modell könnte wie folgt aufgebaut werden.
Man unterteilt die Atmosphäre in Zonen, numeriert mit $i=1,\dots,n$.
Die Einstrahlung in eine Zone kann durch Integration der Einstrahlung
\eqref{skript:einstrahlung:mittlereinsolation}
über das $\vartheta$-Interval ermittelt werden.
Wir bezeichnen die Einstrahlung in Zone $i$ mit $Q_i$.
Wie bei einem Bilanzmodell könnten wir jetzt die zeitliche
Entwicklung mit einer Gleichung der Form
\[
C_i\frac{dT_i}{dt}
=
(1-\alpha_i(T_i)) Q_i - \varepsilon_i \sigma T_i^4
\]
modellieren.
Dabei würden wir aber den Wärmeaustausch mit den Nachbarzonen
vernachlässigen.
Die Gleichung muss daher mit zusätzlichen Termen für die Nachbarzonen
korrigiert werden, also
\begin{equation}
C_i\frac{dT_i}{dt}
=
(1-\alpha_i(T_i)) Q_i - \varepsilon_i \sigma T_i^4
+
\kappa_{i,i+1} (T_{i+1}-T_i)
-
\kappa_{i-1,i} (T_i-T_{i-1}).
\end{equation}
Die Konstanten $C_i$ geben die Wärmekapazität einer Zone wieder.
Je grösser der Anteil ist, zu dem eine Zone mit Wasser bedeckt ist,
desto grösser ist ihre Wärmekapazität.
Die Albedo einer Zone hängt ebenfalls von deren Bodenbedeckung ab, 
daher braucht es eine eigene Albedo-Funktionen $\alpha_i$ für jede
einzelne Zone.
Die Ausstrahlung modellieren wir wieder mit dem Stefan-Boltzmannschen
Gesetz.
Die Konstanten $\varepsilon_i$ geben einerseits Abweichungen von
der Strahlung eines schwarzen Körpers wieder oder die Tatsache, dass
auch der Treibhauseffekt nicht in allen Zonen gleich ist.

Die Konstanten $\kappa_{ij}$ geben die Wärmeleitung zwischen den
Zonen wieder.
Indem wir abkürzen
\[
k_{ij}
=
\begin{cases}
\kappa_{i,i+1}&\qquad j=i+1
\\
-\kappa_{i,i+1}-\kappa_{i-1,i}&\qquad i=j
\\
\kappa_{i-1,i}&\qquad j=i-1
\\
0&\qquad\text{sonst}
\end{cases}
\]
Die Modellgleichung wird dann
\begin{equation}
C_i \frac{dT_i}{dt}
=
(1-\alpha_i(T_i)) Q_i - \varepsilon_i \sigma T_i^4
+
\sum_{j=1}^n k_{ij}T_j.
\label{skript:zonen:modellgl}
\end{equation}
Die Gleichungen
\eqref{skript:zonen:modellgl}
enthalten sehr viele experimentell zu bestimmenden Konstanten,
wir können daher ohne zusätzliche Informationen keine Aussagen
über die zeitliche Entwicklung der Lösungen machen.

Die Wärmeleitungsgleichung auf der Kugeloberfläche bietet eine
Grundlage zur Bestimmung der unbekannten Koeffizienten.
Ein direkterer Weg ist jedoch, nicht eine Aufteilung in Zonen
zu verwenden, sondern auch die Basisfunktionen für die Beschreibung
der Temperaturverteilung der Geometrie der Kugel anzupassen.
Dies ist, was die spektralen Methoden versuchen, die im nächsten
Abschnitt erläutert werden.

