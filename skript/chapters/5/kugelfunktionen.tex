%
% kugelfunktionen.tex -- Kugelfunktionen
%
% (c) 2018 Prof Dr Andreas Müller, Hochschule Rapperswil
%
\subsection{Kugelfunktionen}
Die Basisfunktionen im Lorenzmodell waren aus zwei Gründen besonders
erfolgreich.
\begin{enumerate}
\item
Die Basisfunktionen waren Produkte von Funktionen, die jeweils nur von
einer Koordinate abhängen.
In einem Produkt 
$f(x,y)=X(x)\cdot Y(y)$ 
sind die Ableitungen nach den Koordinaten besonders einfach auszurechnen,
da gilt
\[
\begin{aligned}
\frac{\partial f}{\partial x}(x,y) &= X'(x)\cdot Y(y)
&&\text{und}&
\frac{\partial f}{\partial y}(x,y) &= X(x)\cdot Y'(y).
\end{aligned}
\]
\item
Die Basisfunktionen waren Eigenfunktionen des Laplace-Operators, also
\[
\Delta f = \lambda f.
\]
Da in den Gleichungen der Strömungsdynamik der Laplace-Operator
prominent vorkommt, bedeutet diese Eigenschaft, dass die Wirkung des
Laplace-Operators auf die Basisfunktionen durch Multiplikation mit
dem Eigenwert ersetze werden kann.
Dadurch werden die Gleichungen sehr vereinfacht und die Ordnung
der Differentialgleichung reduziert sich.
\end{enumerate}
Wenn der Erfolg der speziellen Basiswahl im Lorenz-System für ein
Wetter- oder Klimamodell auf der Kugeloberfläche repliziert werden
soll, dann ist naheliegend, dass dazu Funktionen mit den gleichen
Eigenschaften in Kugelkoordinaten gefunden werden müssen.

\subsubsection{Separationsansatz}
Die Produkteigenschaft bedeutet, dass die Basisfunktionen in der Form
\[
f(r,\varphi,\vartheta)
=
R(r)\cdot \Phi(\varphi)\cdot \Theta(\vartheta)
\]
gefunden werden müssen.
Das in Abschnitt~\ref{section:pdeloesungen} dargestellte
Separationsverfahren für partielle
Differentialgleichungen~\cite[Chapter 4]{skript:pde}
basiert genau auf dieser Art von Ansatz.

\subsubsection{Eigenwertgleichung}
Die Eigenwerteigenschaft bedeutet, dass die Funktionen Eigenfunktionen
des Laplace-Operators sein müssen, also Lösungen der partiellen
Differentialgleichungen
\[
\Delta f = \lambda f.
\]
Wenden wir den Laplace-Operator in Kugelkoordinaten auf $f$ an, finden
wir
\begin{align*}
\Delta f
&=
\biggl(
\frac{1}{r^2} \frac{\partial}{\partial r}
\biggl(r^2\frac{\partial f}{\partial r}\biggr)
+
\frac1{r^2\sin\vartheta}\frac{\partial}{\partial\vartheta}
\biggl(\sin\vartheta\frac{\partial f}{\partial\vartheta}\biggr)
+
\frac1{r^2\sin^2\vartheta}\frac{\partial^2 f}{\partial\varphi^2}
\biggr)
R(r)\cdot \Phi(\varphi)\cdot \Theta(\vartheta)
\\
&=
\frac1{r^2}\frac{\partial}{\partial r}\bigl(r^2R'(r)\bigr)
\cdot \Theta(\vartheta)\cdot \Phi(\varphi)
+
\frac1{r^2\sin\vartheta}\frac{d}{d\vartheta}
\bigl(\sin\vartheta \Theta'(\vartheta)\bigr)
\cdot R(r)\cdot \Phi(\varphi)
\\
&\qquad
+
\frac1{r^2\sin^2\vartheta}\Phi''(\varphi)
\cdot R(r)\cdot \Theta(\vartheta)
\\
&=
\frac1{r^2}\bigl(2rR'(r)+r^2R''(r)\bigr)
\Theta(\vartheta)\Phi(\varphi)
+
\frac1{r^2\sin\vartheta}
\bigl(\sin\vartheta\Theta'(\vartheta)\bigr)'
\cdot R(r)\cdot\Phi(\varphi)
\\
&\qquad
+
\frac1{r^2\sin^2\vartheta}
\Phi''(\varphi)
\cdot R(r)\cdot \Theta(\vartheta)
\\
&=
\lambda
R(r)\cdot \Theta(\vartheta)\cdot \Phi(\varphi).
\end{align*}

\subsubsection{Separation von $r$}
Um die einzelnen Funktionen zu isolieren, teilen wir durch $f$.
Zwar kann $f$ Nullstellen haben, aber für die meisten Werte der Koordinaten
ist $f$ von Null verschieden, für diese Punkte ist die Division unproblematisch
und ausreichend, um die Faktoren zu bestimmen.
Wir erhalten
\begin{align*}
\frac{2rR'(r)+r^2R''(r)}{r^2R(r)}
+
\frac{
\bigl(\sin\vartheta\Theta'(\vartheta)\bigr)'
}{r^2\sin\vartheta\Theta(\vartheta)}
+
\frac{1}{r^2\sin^2\vartheta}\frac{\Phi''(\varphi)}{\Phi(\varphi)}
&=\lambda
\end{align*}
Um die Variable $r$ allein auf die linke Seite zu bringen, multiplizieren wir
mit $r^2$, subtrahieren $\lambda r^2$ und bringen den zweiten und dritten 
Term auf der linken Seite auf die rechte Seite.
So erhalten wir
\begin{align*}
\frac{r^2R''(r)+2rR'(r)-\lambda r^2R(r)}{R(r)}
&=
-
\frac{
\bigl(\sin\vartheta\,\Theta'(\vartheta)\bigr)'
}{\sin\vartheta\,\Theta(\vartheta)}
-\frac{1}{\sin^2\vartheta}\frac{\Phi''(\varphi)}{\Phi(\varphi)}
\end{align*}
Die linke Seite hängt nur von $r$ ab, die rechte Seite nur von $\vartheta$
und $\varphi$.
Dies ist nur möglich, wenn beide Seiten konstant sind.
Es gibt also ein Zahl $\mu$ derart, dass
\begin{align}
\frac{r^2R''(r)+2rR'(r)-\lambda r^2R(r)}{R(r)}&=\mu
\\
\frac{\bigl(\sin\vartheta\,\Theta'(\vartheta)\bigr)'}{\sin\vartheta\,\Theta(\vartheta)}
+
\frac{1}{\sin^2\vartheta}\frac{\Phi''(\varphi)}{\Phi(\varphi)}
&=
-\mu
\label{skript:kugel:thetaphi}
\end{align}
Die erste Gleichung kann man vereinfachen zu
\begin{equation}
r^2R''(r)+2rR'(r)-(\lambda r^2-\mu)R(r) = 0,
\end{equation}
eine gewöhnliche lineare Differentialgleichung zweiter Ordnung.
$\mu$ kann nicht beliebig gewählt werden, der Wert muss so sein,
dass \eqref{skript:kugel:thetaphi} gelöst werden kann.

Doch auch $\lambda$ ist nicht beliebg, sein Wert muss so gewählt
werden, dass eventuelle Randbedingungen für die zugehörigen Funktion
$R(r)$ erfüllt sind.
Da die $r$-Abhängigkeit für die folgende Diskussion nicht wichtig ist,
verfolgen wir diese Frage hier nicht weiter.

\subsubsection{Separation von $\vartheta$ und $\varphi$}
Die Gleichung~\eqref{skript:kugel:thetaphi} enthält nur noch die
Variablen $\vartheta$ und $\varphi$. 
Wir versuchen den gleichen Trick erneut: indem wir mit $\sin^2\vartheta$
multiplizieren, den Term mit $\mu$ auf die linke Seite bringen und
den zweiten Term auf die rechte Seite bringen, erhalten wir
\[
\sin\vartheta
\frac{1}{\Theta(\vartheta)}
\frac{d}{d\vartheta}\bigl(\sin\vartheta\,\Theta'(\vartheta)\bigr)
+
\mu\sin^2\vartheta
=
-\frac{\Phi''(\varphi)}{\Phi(\varphi)}.
\]
Erneut haben wir eine Gleichung, deren linke Seite nur von $\vartheta$
und deren rechte Seite nur von $\varphi$ abhängt.
Also sind wieder beide Seiten konstant, es gibt also eine Konstante
$\nu$ derart, dass $\Theta(\vartheta)$ und $\Phi(\varphi)$ die 
Gleichungen
\begin{align}
\sin\vartheta\frac{d}{d\vartheta}
\biggl(
\sin\vartheta\frac{d}{d\vartheta}\Theta(\vartheta)
\biggr)
&=
(-\mu\sin^2\vartheta+\nu)\Theta(\vartheta)
\label{skript:kugel:thetagl}
\\
-\frac{\Phi''(\varphi)}{\Phi(\varphi)}&=\nu
\label{skript:kugel:phigl}
\end{align}
erfüllen.

\subsubsection{Lösungsfunktionen $\Phi(\varphi)$}
Die zweite Gleichung~\eqref{skript:kugel:phigl} ist gleichbedeutend mit
\begin{equation}
\Phi''(\varphi)=-\nu\Phi(\varphi),
\label{skript:kugel:philsg}
\end{equation}
wobei $\Phi(\varphi)$ eine $2\pi$-periodische Funktion ist.
Die Lösungen der Gleichung~\eqref{skript:kugel:philsg}
sind $\cos\sqrt{\nu}\varphi$ und $\sin\sqrt{\nu}\varphi$,
aber die Periodizität verlangt, dass $\sqrt{\nu}$ eine ganze Zahl ist.
Es muss also gelten $\nu=m^2$ mit $m\in \mathbb N$.

\subsubsection{Lösungsfunktionen $\Theta(\vartheta)$}
Im vorangegangen Absatz wurde gezeigt, dass $\nu=k^2$ ist, was die
Gleichung~\eqref{skript:kugel:thetagl} zu
\begin{equation}
\sin\vartheta\frac{d}{d\vartheta}\sin\vartheta\frac{d}{d\vartheta}\Theta(\vartheta)
=
\biggl(\sin\vartheta\frac{d}{d\vartheta}\biggr)^2 \Theta(\vartheta)
=
(-\mu\sin^2\vartheta+k^2)\Theta(\vartheta)
\label{skript:kugel:legendredgl0}
\end{equation}
In dieser Form ist die Differentialgleichung nicht so leicht zu
erkennen.
Schreibt man aber $z=\sin\vartheta$, dann wird die Ableitung einer
Funktion $P(z)=\Theta(\vartheta)$
\begin{align*}
\frac{d}{d\vartheta}\Theta(\vartheta)
=
\frac{d}{d\vartheta}P(\cos\vartheta)
=
-P'(\cos\vartheta) \sin\vartheta
&=
-\sqrt{1-\cos^2\vartheta}P'(\cos\vartheta)
\\
&=
-\sqrt{1-z^2}P'(z)
=
-\sqrt{1-z^2}\frac{d}{dz}P(z).
\end{align*}
Ableitungen nach $\vartheta$ sind also zu ersetzen durch Ableitungen
nach $z$ gefolgt von Multiplikation mit $-\sqrt{1-z^2}$.
In der Differentialgleichung~\eqref{skript:kugel:legendredgl0}
wird die Ableitung nach $\vartheta$ jeweils auch noch mit $\sin\vartheta=\sqrt{1-z^2}$
multipliziert.
Der Operator
\begin{equation}
\sin\vartheta\frac{d}{d\vartheta}
\qquad
\text{bekommt daher die Form}
\qquad
-(1-z^2)\frac{d}{dz}.
\label{skript:kugel:zoperator}
\end{equation}
Das Vorzeichen ist nicht wichtig, da der Operator in der Differentialgleichung
\eqref{skript:kugel:legendredgl0} nur im Quadrat vorkommt.

Wir setzen jetzt die Form \eqref{skript:kugel:zoperator}
des Differentialoperators in die Differentialgleichung
\eqref{skript:kugel:legendredgl0} ein
und erhalten 
\begin{align}
(1-z^2)\frac{d}{dz}\bigl((1-z^2)P'(z)\bigr)
&=
(-\mu(1-z^2)+m^2)P(z)
\notag
\\
\Rightarrow\qquad
(1-z^2)P''(z) -2z P'(z)
+
\mu P(z)
-\frac{m^2}{1-z^2}P(z)
&=
0.
\label{skript:kugel:legendredgl}
\end{align}
Für $m=0$ ist
\eqref{skript:kugel:legendredgl}
die sogenannte Legendresche Differentialgleichung.
\index{Differentialgleichung!Legendresche}
\index{Legendresche Differentialgleichung}
Sie hat Lösungen für $\mu=l(l+1)$ mit $l\in\mathbb N$.
Für $m>0$ ist
\eqref{skript:kugel:legendredgl}
die assozierte Legendre-Differentialgleichung.
Für beide Gleichung lassen sich Lösungen angeben, es sind die
sogenannten Legendre-Polynome $P_l(z)$ im Fall $m=0$ und die zugeordneten
Legendre-Polynome $P_l^m(z)$ für beliebiges $m$.

\subsubsection{Kugelflächenfunktionen}
Mit den gefundenen Lösungen für $\Phi(\varphi)$ und $\Theta(\vartheta)$
finden wir jetzt die allgemeinen Lösungen 
der Differentialgleichung \eqref{skript:kugel:thetaphi}.
Es sind die Funktionen
\[
Y_l^m(\vartheta,\varphi)
=
N_{lm}
P_l^m(\cos\vartheta) e^{im\varphi}
\]
mit einem geeigneten Normierungsfaktor $N_{lm}$.
Diese Funktionen haben genau die Eigenschaften, die wir in der Einleitung
dieses Abschnitts als Voraussetzungen für eine geeignete Basis
gefordert haben.



