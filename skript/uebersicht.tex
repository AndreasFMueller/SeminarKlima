%
% uebersicht.tex -- Uebersicht ueber die Seminar-Arbeiten
%
% (c) 2018 Prof Dr Andreas Mueller, Hochschule Rapperswil
%
\chapter*{"Ubersicht}
\lhead{"Ubersicht}
\rhead{}
\label{skript:uebersicht}
Im zweiten Teil kommen die Teilnehmer des Seminars selbst zu Wort.
Die im ersten Teil dargelegten mathematischen Methoden und
grundlegenden Modelle werden dabei verfeinert, verallgemeinert
und auch numerisch überprüft.
Die Beispiele zeigen auch, dass umfassende Klimamodell erwartungsgemäss
schnell sehr komplex werden können.
Sie zeigen aber auch, dass es möglich ist, wesentliche Aspekte
des Klimas aus einfachen Überlegungen und mit übersichtlichen
mathematischen Modellen zu erfassen und zu verstehen.

{\em Matthias Baumann} und {\em Oli Dias} zeigen zusätzliche
Informationen zum Lorenz-System, insbesondere zum Lorenz-Attraktor.
{\em Hansruedi Patzen} verallgemeinert das Vorgehen, mit dem das
Lorenz-System gewonnen wurde, im Sinne des Separationsverfahrens
für partielle Differentialgleichungen weiter, und gewinnt eine
Reihe von höherdimensionalen Lorenz-Systemen und untersucht weiter,
ob das beim dreidimensionalen System gefundene chaotische Verhalten 
auch in den höherdimensionalen Versionen zu finden ist.
Diese Untersuchungen zeigen, dass das Verhalten wahrscheinlich
nicht nur ein Artefakt der Reduktion auf drei Dimensionen ist, sondern
eine inhärente Eigenschaft des Lorenz-Systems.

Die exakte Lösung von Klimamodellen ist oft sehr schwierig, oft sind
Parameter oder Abhängigkeiten nicht bekannt.
Ein alternativer Ansatz ist daher, die zeitliche Entwicklung
im Sinne von machine learning zu lernen.
{\em Martin Stypinski} untersucht diesen Ansatz an zwei Beispielen,
der Wärmeleitungsgleichung und der nichtlinearen Gleichung von Burgers.
Die Beispiele zeigen vor allem auch, wie mathematisches Wissen über das
Modell hilft, die neuronalen Netzwerke zweckmässig zu entwerfen.

Die im Text entwickelten Modelle sind meist noch rudimentär, eine
reihe von Arbeiten haben sich daher mit Erweiterungen und Verfeinerungen
befasst.
{\em Jonas Gründler} hat das 2-Box Modell der thermohalinen Zirkulation
auf 3 Boxen verallgemeinert.
{\em Silvio Marti} studiert den Einfluss von Eis auf das Klima.
Die verzögerte Differentialgleichung des El Niño-Systems kann
numerische gelöst werden. {\em Raphael Unterer} zeigt, wie dies
funktioniert und findet insbesondere auch eine Stabilitätsbedingung
für sein Verfahren.
{\em Nicolas Tobler} hat versucht, die Modelle auf verschiedene
Planeten zu verallgemeinern und damit die Unterschiede der Atmosphären
der Planeten Venus, Erde und Mars zu erklären.
Die Einflüsse anderer Planeten zeigen sich zum Beispiel in der
Neigung der Erdachse gegen die Erdbahnebene.
Wie {\em Sebastian Lenhard} vorführt, können auch kleine Änderungen
der Neigung das Klima dramatisch verändern.
Schliesslich gibt es auch eine Kopplung zwischen Klima und
Vegetation, die {\em Matthias Dunkel} modelliert hat.

Die letzten zwei Kapitel befassen sich mit der Aufarbeitung der
Daten, auf grund derer wir zum Schluss kommen, dass 
der Klimawandel real ist.
Für Wetter- wie auch für Klimamodelle sind grosse Datenmengen
von Wetter- und Klimamessstationen in die Rechnung zu integrieren.
Der Kalman-Filter ist die Basis vieler moderner Lösungsansätze
für dieses Problem.
Am Beispiel des erweiterten Kalman-Filters für das Lorenz-System
zeigt {\em Michael Müller}, wie es möglich ist, den Systemzustand
selbst eines chaotischen nichtlinearen Systems zu 
Extreme Ereignisse sind in den letzten Jahren häufiger geworden
und werden oft als Indikatoren für die Klimaveränderung angeführt.
Wie man mit statistischen Daten über solche Ereignisse beweisen
kann, dass der Klimawandel tatsächlich stattfindet, zeigt
{\em Melina Staub}.





