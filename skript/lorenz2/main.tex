%
% main.tex -- Paper zum Thema <thema>
%
% (c) 2018 Hansruedi Patzen, Hochschule Rapperswil
%
\chapter{H"oherdimensionales Lorenzsystem\label{chapter:lorenz2}}
\lhead{H"oherdimensionales Lorenzsystem}
\begin{refsection}
\chapterauthor{Hansruedi Patzen}

Auf das dreidimensionale Lorenzsystem wurde bereits in 
\cref{section:lorenz-modell} sowie in \cref{chapter:lorenz} detailliert 
eingegangen. In diesem Kapitel wollen wir nun aber Lorenzsysteme h"oherer 
Ordnung untersuchen. Das kommt mit einiger Arbeit einher. So brauchen wir 
entweder komplett neue Basisfunktionen, oder aber wir erweitern unsere 
bestehenden, wie in \cref{section:lorenz2:basic_function} aufgezeigt. Mit den 
dann gefunden Basisfunktionen k"onnen wir dann unsere Gleichungen aufbauen und 
versuchen daf"ur eine L"osung zu finden, letzteres geschieht in 
\cref{section:lorenz2:new_equations}.

\section{Einf"uhrung}
\rhead{Einf"uhrung}

Bevor wir uns aber auf die Suche nach Basisfunktionen machen, hier nun ein paar 
Angaben zu den in diesem Kapitel verwendeten Notation und einigen grundlegenden 
Funktionen und Umformungen, welche sp"ater verwendet werden, ein kleiner 
\"Teaser\" zu dem was uns noch erwartet sozusagen.

\subsection{Notation}
Die verwendete Notation entspricht grunds"atzlich derjenigen, welche man 
bereits aus anderen mathematischen B"uchern kennt. Einzig die Verwendung der 
sogenannten Multi-Index-Notationen k"onnte f"ur einige Leser zu Unklarheiten 
f"uhren, daher hier eine kurze Einf"uhrung.

Unter der Multi-Index-Notation versteht versteht man einen Index, der einem 
$n$-Tupel
\begin{equation*}
	\alpha = (\alpha_1, \alpha_2, \dotsc, \alpha_n) \qquad \alpha_k \in 
	\mathbb{N}_{0}
\end{equation*}
entspricht. Zur klaren Unterscheidung zu normalen Indexen, wird meist ein 
griechischer Buchstabe verwendet. Multi-Indexe haben zudem die Eigenschaft, 
dass ihr absoluter Wert wie folgt definiert ist
\begin{equation*}
	|\alpha| = \alpha_1 + \alpha_2 + \dots + \alpha_n.
\end{equation*}
 
N"utzlich ist diese Notation insbesondere, wenn mit Summen gearbeitet wird. So 
kann mittels einer Forderung wie beispielsweise $|\alpha| = k$, die Summe
\begin{equation}
	\sum_{k = 0}^{\infty}\sum_{l = 0}^{k}a_{l, k - l}
	\label{equation:lorenz2:doublesum}
\end{equation}
vereinfacht werden zu
\begin{equation}
	\sum_{k = 0}^{\infty}a_{\alpha} \qquad |\alpha| = k.
	\label{equation:lorenz2:mmsum}
\end{equation}
M"oglich ist dies, da $|\alpha| = k$ f"ur alle Kombinationen der verschiedenen 
$|\alpha_m|$ gilt. Zum Beispiel ist $|\alpha| = k = 2$ f"ur $2$-Tupel 
Multi-Indexe
\begin{align*}
	\alpha &= (0, 2), \\
	\alpha &= (1, 1), \\
	\alpha &= (2, 0)
\end{align*}
erf"ullt, was genau dem in \cref{equation:lorenz2:doublesum} f"ur $k = 2$ 
geforderten Index-Tupel $(l, k - l)$ entspricht.

\subsection{Formel Sammelserum}
Einige Umformungen die wir in den n"achsten Abschnitten vornehmen werden, 
verwenden Teils nicht ganz gel"aufige Funkionen. Damit nicht immer gleich 
Wikipedia bem"uht werden muss, haben wir diese hier in einem kleinen 
Formel Sammelserum zusammengestellt.

Die Signum-Funktion (kurz $\sgn(x)$):
\begin{equation}
\sgn(x) =
\begin{cases*}
1 & f"ur x > 0 \\
0 & f"ur x = 0 \\
-1 & f"ur x < 0
\end{cases*}
\label{equation:lorenz2:signum}
\end{equation}

Betragsfunktion (kurz $|x|$)
\begin{equation}
|x| =
\begin{cases*}
x & f"ur x $\geq$ 0 \\
-x & f"ur x < 0
\end{cases*}
\label{equation:lorenz2:absfunction}
\end{equation}

Sinus und Kosinus eines absoluten Wertes (\cref{equation:lorenz2:absfunction}):
\begin{equation}
\begin{split}
\sin(|x|) &= \sgn(x)\sin(x) \qquad \Leftrightarrow \qquad \sin(x) = 
\sgn(x)\sin(|x|)
\\
\cos(|x|) &= \cos(x)
\end{split}
\label{equation:lorenz2:sincosabs}
\end{equation}

Produkte von Sinus und Kosinus Kombinationen:
\begin{align*}
\cos(x)\sin(y) &= \frac{1}{2} \left(\sin(x + y) - \sin(x - y)\right)
\\
\sin(x)\cos(y) &= \frac{1}{2} \left(\sin(x - y) + \sin(x + y)\right)
\\
\sin(x)\sin(y) &= \frac{1}{2} \left(\cos(x - y) - \cos(x + y)\right)
\\
\cos(x)\cos(y) &= \frac{1}{2} \left(\cos(x - y) + \cos(x + y)\right)
\end{align*}

\section{Erweiterung der Basisfunktionen\label{section:lorenz2:basic_function}}
\rhead{Erweiterung der Basisfunktionen}


\section{L"osen der neuen Gleichungen\label{section:lorenz2:new_equations}}
\rhead{L"osen der neuen Gleichungen}



\section{Schlussfolgerung}
\rhead{Schlussfolgerung}

\printbibliography[heading=subbibliography]
\end{refsection}
