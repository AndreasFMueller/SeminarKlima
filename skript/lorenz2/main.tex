%
% main.tex -- Paper zum Thema <thema>
%
% (c) 2018 Hansruedi Patzen, Hochschule Rapperswil
%
\chapter{H"oherdimensionales Lorenzsystem\label{chapter:lorenz2}}
\lhead{H"oherdimensionales Lorenzsystem}
\begin{refsection}
\chapterauthor{Hansruedi Patzen}

Auf das dreidimensionale Lorenzsystem wurde bereits in 
\cref{section:lorenz-modell} sowie in \cref{chapter:lorenz} detailliert 
eingegangen. In diesem Kapitel wollen wir nun aber Lorenzsysteme h"oherer 
Ordnung untersuchen. Das kommt mit einiger Arbeit einher. So brauchen wir 
entweder komplett neue Basisfunktionen, oder aber wir erweitern unsere 
bestehenden, wie in \cref{section:lorenz2:basic_function} aufgezeigt. Mit den 
dann gefunden Basisfunktionen k"onnen wir dann unsere Gleichungen aufbauen und 
versuchen daf"ur eine L"osung zu finden, letzteres geschieht in 
\cref{section:lorenz2:ho-model}.

\section{Einf"uhrung}
\rhead{Einf"uhrung}

Bevor wir uns aber auf die Suche nach Basisfunktionen machen, hier nun ein paar 
Angaben zu den in diesem Kapitel verwendeten Notation und einigen grundlegenden 
Funktionen und Umformungen, welche sp"ater verwendet werden, ein kleiner 
\"Teaser\" zu dem was uns noch erwartet sozusagen.

\subsection{Notation}
Die verwendete Notation entspricht grunds"atzlich derjenigen, welche man 
bereits aus anderen mathematischen B"uchern kennt. Einzig die Verwendung der 
sogenannten Multi-Index-Notationen k"onnte f"ur einige Leser zu Unklarheiten 
f"uhren, daher hier eine kurze Einf"uhrung.

Unter der Multi-Index-Notation versteht versteht man einen Index, der einem 
$n$-Tupel
\begin{equation*}
	\alpha = (\alpha_1, \alpha_2, \dotsc, \alpha_n) \qquad \alpha_k \in 
	\mathbb{N}_{0}
\end{equation*}
entspricht. Zur klaren Unterscheidung zu normalen Indexen, wird meist ein 
griechischer Buchstabe verwendet. Multi-Indexe haben zudem die Eigenschaft, 
dass ihr absoluter Wert wie folgt definiert ist
\begin{equation*}
	|\alpha| = \alpha_1 + \alpha_2 + \dots + \alpha_n.
\end{equation*}
 
N"utzlich ist diese Notation insbesondere, wenn mit Summen gearbeitet wird. So 
kann mittels einer Forderung wie beispielsweise $|\alpha| = k$, die Summe
\begin{equation}
	\sum_{k = 0}^{\infty}\sum_{l = 0}^{k}a_{l, k - l}
	\label{equation:lorenz2:doublesum}
\end{equation}
vereinfacht werden zu
\begin{equation}
	\sum_{k = 0}^{\infty}a_{\alpha} \qquad |\alpha| = k.
	\label{equation:lorenz2:mmsum}
\end{equation}
M"oglich ist dies, da $|\alpha| = k$ f"ur alle Kombinationen der verschiedenen 
$|\alpha_m|$ gilt. Zum Beispiel ist $|\alpha| = k = 2$ f"ur $2$-Tupel 
Multi-Indexe
\begin{align*}
	\alpha &= (0, 2), \\
	\alpha &= (1, 1), \\
	\alpha &= (2, 0)
\end{align*}
erf"ullt, was genau dem in \cref{equation:lorenz2:doublesum} f"ur $k = 2$ 
geforderten Index-Tupel $(l, k - l)$ entspricht.

\subsection{Formel Sammelserum}
Einige Umformungen die wir in den n"achsten Abschnitten vornehmen werden, 
verwenden Teils nicht ganz gel"aufige Funkionen. Damit nicht immer gleich 
Wikipedia bem"uht werden muss, haben wir diese hier in einem kleinen 
Formel Sammelserum zusammengestellt.

Die Signum-Funktion (kurz $\sgn(x)$):
\begin{equation}
\sgn(x) =
\begin{cases*}
1 & f"ur x > 0 \\
0 & f"ur x = 0 \\
-1 & f"ur x < 0
\end{cases*}
\label{equation:lorenz2:signum}
\end{equation}

Betragsfunktion (kurz $|x|$)
\begin{equation}
|x| =
\begin{cases*}
x & f"ur x $\geq$ 0 \\
-x & f"ur x < 0
\end{cases*}
\label{equation:lorenz2:absfunction}
\end{equation}

Sinus und Kosinus eines absoluten Wertes (\cref{equation:lorenz2:absfunction}):
\begin{equation}
\begin{split}
\sin(|x|) &= \sgn(x)\sin(x) \qquad \Leftrightarrow \qquad \sin(x) = 
\sgn(x)\sin(|x|)
\\
\cos(|x|) &= \cos(x)
\end{split}
\label{equation:lorenz2:sincosabs}
\end{equation}

Produkte von Sinus und Kosinus Kombinationen:
\begin{align*}
\cos(x)\sin(y) &= \frac{1}{2} \left(\sin(x + y) - \sin(x - y)\right)
\\
\sin(x)\cos(y) &= \frac{1}{2} \left(\sin(x - y) + \sin(x + y)\right)
\\
\sin(x)\sin(y) &= \frac{1}{2} \left(\cos(x - y) - \cos(x + y)\right)
\\
\cos(x)\cos(y) &= \frac{1}{2} \left(\cos(x - y) + \cos(x + y)\right)
\end{align*}

\section{Erweiterung der Basisfunktionen\label{section:lorenz2:basic_function}}
Wir beginnen unsere Suche neuer Basisfunktionen mit denjenigen, die wir bereits 
in \cref{skript:funktionsauswahl} gefunden haben. Bei genauerer Betrachtung 
stellt man fest, dass man diese auch ein wenig anders schreiben kann
\begin{align*}
\sin(ax)\sin(y) &= \sin(1ax)\sin(1y),\\
\cos(ax)\sin(y) &= \cos(1ax)\sin(1y),\\
\sin(2y) &= \cos(0ax)\sin(2y).
\end{align*}
Wenn wir nun noch ein paar zus"atzliche, auf den ersten Blick etwas nutzlose 
Gleichungen hinzuf"ugen:
\begin{align*}
0 &= \sin(0ax)\sin(2y) \\
\sin(ax)\sin(y) &= \sin(1ax)\sin(1y)\\
0 &= \sin(2ax)\sin(0y) \\
0 &= \cos(2ax)\sin(0y)\\
\cos(ax)\sin(y) &= \cos(1ax)\sin(1y)\\
\sin(2y) &= \cos(0ax)\sin(2y)
\end{align*}
Daraus l"asst sich nun das Pattern
\begin{equation}
\begin{split}
\sin(\alpha_1 ax)\sin(\alpha_2 y) \\
\cos(\alpha_1 ax)\sin(\alpha_2 y)
\end{split}
\label{equation:lorenz2:basic-functions}
\end{equation}
ableiten, wobei $\alpha_1$ und $\alpha_2$ Teil eines Multi-Index sind und die 
Bedingung $|\alpha| = k = 2$ erf"ullen.

F"ur unser dreidimensionales Modell haben wir also Basisfunktionen zweiten 
Grades ($k = 2$) gefunden. Mit diesen Funktionen k"onnen wir nicht nur die uns 
bekannten generieren, sondern diese auch noch erweitern indem wir den Grad $k$ 
varieren.

Leicht zu erkennen ist, dass es f"ur $k = 0$ nur die $0$-Funktion 
"ubrig bleibt, da einzig das $(0, 0)$-Tupel die Bedingung $|\alpha| = 0$ 
erf"ullt. $k = 1$ generiert die beiden Tupel $(0, 1)$ und $(1, 0)$, womit die 
Funktion $\sin(y)$ "ubrig bleibt. Somit k"onnen wir Grad $k \geq 1$ 
veraussetzen. So wird auch gleich das Problem einer m"oglichen Division durch 
$0$ elegant umgangen, wie wir sp"ater sehen werden.

\section{Hoherdimensionales Lorenzsystem\label{section:lorenz2:ho-model}}
\rhead{Lorenz-Modell h"oherer Ordnung}
Nehmen wir nun also die Grundgleichungen aus 
\cref{skript:lorenzausgangsgleichung} und unsere neuen Basisfunktionen aus 
\cref{equation:lorenz2:basic-functions} und bauen daraus unser f"ur ein 
h"oherdimensionales Lorenzsystem.

Als ersten Schritt erweitern wir unsere in \cref{skript:psiansatz} und 
\cref{skript:thetaansatz} gefundenen L"osungen, erweitern diese und erhalten
\begin{equation}
\psi(x,y,t) =
\sum_{k = 1}^{\infty}
a_{\alpha}(t)
\sin(\alpha_1 ax) \sin(\alpha_2 y)
\qquad |\alpha| = k,
\label{equation:lorenz2:extendedpsi}
\end{equation}
sowie
\begin{equation}
\vartheta(x,y,t) =
\sum_{k = 1}^{\infty}
b_{\alpha}(t)
\cos(\alpha_1 ax) \sin(\alpha_2 y)
\qquad |\alpha| = k.
\label{equation:lorenz2:extendedtheta}
\end{equation}

Als n"achstes betrachten wir die \cref{skript:lorenzausgangsgleichung}
\begin{align*}
\frac{\partial\Delta\psi}{\partial t}
&=
\nu\Delta^2\psi 
+c\frac{\partial\vartheta}{\partial x}
-\frac{\partial(\psi,\Delta\psi)}{\partial(x,y)},
\\
\frac{\partial\vartheta}{\partial t}
&=
\kappa\Delta\vartheta
+ \frac{T_0}{\pi}\frac{\partial\psi}{\partial x}
- \frac{\partial(\psi,\vartheta)}{\partial(x,y)}.
\end{align*}
Statt einfach Terme zu ersetzen, macht es Sinn systematisch vorzugehen. Es ist 
leicht zu erkennen, dass wir sowohl partielle Ableitungen nach $x$ als auch $y$ 
brauchen werden. Daher fangen wir an uns Bausteine zurechtzulegen indem wir die 
\cref{equation:lorenz2:extendedpsi,equation:lorenz2:extendedtheta} 
zwei Mal je nach $x$ und $y$ ableiten.

F"ur $\psi(x,y,t)$ erhalten wir
\begin{align*}
\frac{\partial\psi}{\partial x}
&=
a
\sum_{k = 1}^{\infty}
\alpha_1
a_{\alpha}(t)
\cos(\alpha_1 ax) \sin(\alpha_2 y)
&&|\alpha| = k
\\
\frac{\partial\psi}{\partial y}
&=
\sum_{k = 1}^{\infty}
\alpha_2
a_{\alpha}(t)
\sin(\alpha_1 ax) \cos(\alpha_2 y)
&&|\alpha| = k
\\
\frac{\partial^2\psi}{\partial x^2}
&=
-
a^2
\sum_{k = 1}^{\infty}
\alpha_1^2
a_{\alpha}(t)
\sin(\alpha_1 ax) \sin(\alpha_2 y)
&&|\alpha| = k
\\
\frac{\partial^2\psi}{\partial y^2}
&=
-
\sum_{k = 1}^{\infty}
\alpha_2^2
a_{\alpha}(t)
\sin(\alpha_1 ax) \sin(\alpha_2 y)
&&|\alpha| = k
\end{align*}
und f"ur $\vartheta(x,y,t)$
\begin{align*}
\frac{\partial\vartheta}{\partial x}
&=
-
a
\sum_{k = 1}^{\infty}
\alpha_1
b_{\alpha}(t)
\sin(\alpha_1 ax) \sin(\alpha_2 y)
&&|\alpha| = k
\\
\frac{\partial\vartheta}{\partial y}
&=
\sum_{k = 1}^{\infty}
\alpha_2
b_{\alpha}(t)
\cos(\alpha_1 ax) \cos(\alpha_2 y)
&&|\alpha| = k
\\
\frac{\partial^2\vartheta}{\partial x^2}
&=
-
a^2
\sum_{k = 1}^{\infty}
\alpha_1^2
b_{\alpha}(t)
\cos(\alpha_1 ax) \sin(\alpha_2 y)
&&|\alpha| = k
\\
\frac{\partial^2\vartheta}{\partial y^2}
&=
-
\sum_{k = 1}^{\infty}
\alpha_2^2
b_{\alpha}(t)
\cos(\alpha_1 ax) \sin(\alpha_2 y)
&&|\alpha| = k.
\end{align*}

Damit haben wir alles zusammen um die den einfachen Laplace Operator $\laplace$ 
f"ur $\psi(x,y,t)$
\begin{align*}
\laplace\psi
&= 
\frac{\partial^2\psi}{\partial x^2}
+
\frac{\partial^2\psi}{\partial y^2}
\\
&=
-
a^2
\sum_{i = 1}^{\infty}
\gamma_1^2
a_{\gamma}(t)
\sin(\gamma_1 ax) \sin(\gamma_2 y)
-
\sum_{j = 1}^{\infty}
\delta_2^2
a_{\delta}(t) 
\sin(\delta_1 ax) \sin(\delta_2 y)
&&|\gamma| = i, |\delta| = j
\\
&=
-
\sum_{k = 1}^{\infty}
\left(\alpha_1^2 a^2 + \alpha_2^2 \right)
a_{\alpha}(t)
\sin(\alpha_1 ax) \sin(\alpha_2 y)
&&|\alpha| = k
\end{align*}
und $\vartheta(x,y,t)$
\begin{align*}
\laplace\vartheta &= 
\frac{\partial^2\vartheta}{\partial x^2}
+
\frac{\partial^2\vartheta}{\partial y^2} \\
&=
-
a^2
\sum_{i = 1}^{\infty}
\alpha_1^2
b_{\gamma}(t)
\cos(\gamma_1 ax) \sin(\gamma_2 y)
-
\sum_{j = 1}^{\infty}
\delta_2^2
b_{\delta}(t)
\cos(\delta_1 ax) \sin(\delta_2 y)
&&|\gamma| = i, |\delta| = j
\\
&=
-
\sum_{k = 1}^{\infty}
\left(\alpha_1^2 a^2 + \alpha_2^2\right)
b_{\alpha}(t)
\cos(\alpha_1 ax) \sin(\alpha_2 y)
&&|\alpha| = k
\end{align*}
aufzul"osen. In \cref{equation:lorenz2:extendedpsi} brauchen wir zudem 
$\laplace^2\psi(x,y,t)$. Daf"ur ben"otigen wir die zweifachen Ableitungen von 
$\laplace\psi(x,y,t)$ nach $x$ und $y$
\begin{align*}
\frac{\partial\laplace\psi}{\partial x} &=
-
a
\sum_{k = 1}^{\infty}
\alpha_1
\left(\alpha_1^2 a^2 + \alpha_2^2 \right)
a_{\alpha}(t)
\cos(\alpha_1 ax) \sin(\alpha_2 y)
&&|\alpha| = k
\\
\frac{\partial^2\laplace\psi}{\partial x^2}
&=
a^2
\sum_{k = 1}^{\infty}
\alpha_1^2
\left(\alpha_1^2 a^2+\alpha_2^2\right)
a_{\alpha}(t)
\sin(\alpha_1 ax) \sin(\alpha_2 y)
&&|\alpha| = k
\\
\frac{\partial\laplace\psi}{\partial y}
&=
-
\sum_{k = 1}^{\infty}
\alpha_2
\left(\alpha_1^2 a^2 + \alpha_2^2 \right)
a_{\alpha}(t)
\sin(\alpha_1 ax) \cos(\alpha_2 y)
&&|\alpha| = k
\\
\frac{\partial^2\laplace\psi}{\partial y^2}
&=
\sum_{k = 1}^{\infty}
\alpha_2^2
\left(\alpha_1^2 a^2+\alpha_2^2\right)
a_{\alpha}(t)
\sin(\alpha_1 ax) \sin(\alpha_2 y)
&&|\alpha| = k
\end{align*}
und k"onnen dann
\begin{align*}
\laplace^2\psi
&= 
\frac{\partial^2\laplace\psi}{\partial x^2} + 
\frac{\partial^2\laplace\psi}{\partial y^2} \\
&=
a^2
\sum_{i = 1}^{\infty}
\gamma_1^2
\left(\gamma_1^2 a^2+\gamma_2^2\right)
a_{\gamma}(t)
\sin(\gamma_1 ax) \sin(\gamma_2 y)
\\
&\phantom{={}}
+
\sum_{j = 1}^{\infty}
\delta_2^2
\left(\delta_1^2 a^2+\delta_2^2\right)
a_{\delta}(t)
\sin(\delta_1 ax) \sin(\delta_2 y)
&&|\gamma| = i, |\delta| = j
\\
&=
\sum_{k = 1}^{\infty}
\left(\alpha_1^2 a^2+\alpha_2^2\right)^2
a_{\alpha}(t)
\sin(\alpha_1 ax) \sin(\alpha_2 y)
&&|\alpha| = k
\end{align*}
zusammenbauen.

Bei einigen werden sich bei den Gleichungen f"ur die Laplace Operatoren beim 
Addieren zweier unendlicher Summen wahrscheinlich die Nackenhaare hochgestellt 
haben. Diese Umformung sei uns aber erlaubt, da wir davon ausgehen m"ussen, 
dass die einzelnen Serien konvergent sind. Zudem h"alt uns nichts davon ab, die 
Indexe der beiden unabh"angigen Summen so umzubenennen, dass diese Gleich sind. 
Damit ist klar, dass die gleichen Basisfunktionen generiert werden und die 
Summen somit vereinfacht werden k"onnen.

Mit unseren bisherigen Bausteinen haben wir jetzt fast alles zusammen, um 
\cref{skript:lorenzausgangsgleichung} aufzul"osen. Einzig die 
Funktionsdeterminanten fehlen noch, was auch dem haarigen Teil der Gleichung 
entspricht, was uns erst die Kopplung der beiden Terme beschert.

Beginnen wir also damit die erste Funktionsdeterminante
\begin{align*}
\frac{\partial(\psi, \laplace\psi)}{\partial(x,y)}
&=
\frac{\partial\psi}{\partial x}
\frac{\partial\laplace\psi}{\partial y}
-
\frac{\partial\psi}{\partial y}
\frac{\partial\laplace\psi}{\partial x}
\\
&=
\left(
a
\sum_{i = 1}^{\infty}
\gamma_1
a_{\gamma}(t)
\cos(\gamma_1 ax) \sin(\gamma_2 y)
\right)
\left(
-
\sum_{j = 1}^{\infty}
\delta_2
\left(\delta_1^2 a^2 + \delta_2^2 \right)
a_{\delta}(t)
\sin(\delta_1 ax) \cos(\delta_2 y)
\right)
\\
&\phantom{={}}
-
\left(
\sum_{q = 1}^{\infty}
\xi_2
a_{\xi}(t)
\sin(\xi_1 ax) \cos(\xi_2 y)
\right)
\left(
-
a
\sum_{r = 1}^{\infty}
\eta_1
\left(\eta_1^2 a^2 + \eta_2^2 \right)
a_{\eta}(t)
\cos(\eta_1 ax) \sin(\eta_2 y)
\right)
\\
&|\gamma| = i, |\delta| = j, |\xi| = q, |\eta| = r
\\
&=
-
a
\sum_{i = 1}^{\infty}
\gamma_1
a_{\gamma}(t)
\sum_{j = 1}^{\infty}
\delta_2
\left(\delta_1^2 a^2 + \delta_2^2 \right)
a_{\delta}(t)
\cos(\gamma_1 ax) \sin(\gamma_2 y)
\sin(\delta_1 ax) \cos(\delta_2 y)
\\
&\phantom{={}}
+
a
\sum_{q = 1}^{\infty}
\xi_2
a_{\xi}(t)
\sum_{r = 1}^{\infty}
\eta_1
\left(\eta_1^2 a^2 + \eta_2^2 \right)
a_{\eta}(t)
\sin(\xi_1 ax) \cos(\xi_2 y)
\cos(\eta_1 ax) \sin(\eta_2 y)
\\
&=
-
a
\sum_{i = 1}^{\infty}
\gamma_1
a_{\gamma}(t)
\sum_{j = 1}^{\infty}
\delta_2
\left(\delta_1^2 a^2 + \delta_2^2 \right)
a_{\delta}(t) 
\cos(\gamma_1 ax) \sin(\gamma_2 y)
\sin(\delta_1 ax) \cos(\delta_2 y)
\\
&\phantom{={}}
+
a
\sum_{j = 1}^{\infty}
\delta_2
a_{\delta}(t)
\sum_{i = 1}^{\infty}
\gamma_1
\left(\gamma_1^2 a^2 + \gamma_2^2 \right)
a_{\gamma}(t)
\sin(\delta_1 ax) \cos(\delta_2 y)
\cos(\gamma_1 ax) \sin(\gamma_2 y)
\\
&=
a
\sum_{i = 1}^{\infty}
\gamma_1
a_{\gamma}(t)
\sum_{j = 1}^{\infty}
\delta_2
\left(
a^2 \left(\gamma_1^2 - \delta_1^2 \right)
+ \left(\gamma_2^2 - \delta_2^2 \right)
\right)
a_{\delta}(t)
\\[-2.5ex]
&\phantom{=abbbbb\sum\sum.....{}}
\cdot
\cos(\gamma_1 ax) \sin(\gamma_2 y)
\sin(\delta_1 ax) \cos(\delta_2 y)
\\
&=
\frac{a}{4}
\sum_{i = 1}^{\infty}
\gamma_1
a_{\gamma}(t)
\sum_{j = 1}^{\infty}
\delta_2
\left(
a^2 \left(\gamma_1 - \delta_1 \right)\left( \gamma_1 + \delta_1 \right)
+ \left(\gamma_2 - \delta_2 \right)\left( \gamma_2 + \delta_2 \right)
\right)
a_{\delta}(t)
\\[-2.5ex]
&\phantom{=abbbbb\sum\sum.....{}}
\cdot
\Big(
\sin((\gamma_1 + \delta_1) ax)\sin((\gamma_2 + \delta_2) y)
\\[-1.0ex]
&\phantom{=abbbbb\sum\sum.....{}}
+
\sin((\gamma_1 + \delta_1) ax)\sin((\gamma_2 - \delta_2) y)
\\[-1.0ex]
&\phantom{=abbbbb\sum\sum.....{}}
-
\sin((\gamma_1 - \delta_1) ax)\sin((\gamma_2 + \delta_2) y)
\\[-1.0ex]
&\phantom{=abbbbb\sum\sum.....{}}
-
\sin((\gamma_1 - \delta_1) ax)\sin((\gamma_2 - \delta_2) y)
\Big)
\\
&=
\frac{a}{4}
\sum_{i = 1}^{\infty}
\gamma_1
a_{\gamma}(t)
\sum_{j = 1}^{\infty}
\delta_2
\left(
a^2 \left(\gamma_1 - \delta_1 \right)\left( \gamma_1 + \delta_1 \right)
+ \left(\gamma_2 - \delta_2 \right)\left( \gamma_2 + \delta_2 \right)
\right)
a_{\delta}(t)
\\[-2.5ex]
&\phantom{=abbbbb\sum\sum.....{}}
\cdot
\Big(
\sgn(\gamma_1 + \delta_1)\sgn(\gamma_2 + \delta_2)
\sin(|\gamma_1 + \delta_1|ax)\sin(|\gamma_2 + \delta_2|y)
\\[-1.0ex]
&\phantom{=abbbbb\sum\sum.....{}}
+
\sgn(\gamma_1 + \delta_1)\sgn(\gamma_2 - \delta_2)
\sin(|\gamma_1 + \delta_1|ax)\sin(|\gamma_2 - \delta_2|y)
\\[-1.0ex]
&\phantom{=abbbbb\sum\sum.....{}}
-
\sgn(\gamma_1 - \delta_1)\sgn(\gamma_2 + \delta_2)
\sin(|\gamma_1 - \delta_1|ax)\sin(|\gamma_2 + \delta_2|y)
\\[-1.0ex]
&\phantom{=abbbbb\sum\sum.....{}}
-
\sgn(\gamma_1 - \delta_1)\sgn(\gamma_2 - \delta_2)
\sin(|\gamma_1 - \delta_1|ax)\sin(|\gamma_2 - \delta_2|y)
\Big)
\\
&|\gamma| = i, |\delta| = j
\end{align*}
und analog dazu kann auch die zweite
\begin{align*}
\frac{\partial(\psi, \vartheta)}{\partial(x,y)}
&=
\frac{\partial\psi}{\partial x}
\frac{\partial\vartheta}{\partial y}
-
\frac{\partial\psi}{\partial y}
\frac{\partial\vartheta}{\partial x}
\\
&=
\left(
a
\sum_{i = 1}^{\infty}
\gamma_1
a_{\gamma}(t)
\cos(\gamma_1 ax) \sin(\gamma_2 y)
\right)
\left(
\sum_{j = 1}^{\infty}
\delta_2
b_{\delta}(t)
\cos(\delta_1 ax) \cos(\delta_2 y)
\right)
\\
&\phantom{={}}
-
\left(
\sum_{q = 1}^{\infty}
\xi_2
a_{\xi}(t)
\sin(\xi_1 ax) \cos(\xi_2 y)
\right)
\cdot
\left(
-
a
\sum_{r = 1}^{\infty}
\eta_1
b_{\eta}(t)
\sin(\eta_1 ax) \sin(\eta_2 y)
\right)
\\
&|\gamma| = i, |\delta| = j, |\xi| = q, |\eta| = r
\\
&=\phantom{+}
a
\sum_{i = 1}^{\infty}
\gamma_1
a_{\gamma}(t)
\sum_{j = 1}^{\infty}
\delta_2
b_{\delta}(t)
\cos(\gamma_1 ax) \sin(\gamma_2 y)
\cos(\delta_1 ax) \cos(\delta_2 y)
\\
&\phantom{={}}
+
a
\sum_{q = 1}^{\infty}
\xi_2
a_{\xi}(t)
\sum_{r = 1}^{\infty}
\eta_1
b_{\eta}(t)
\sin(\xi_1 ax) \cos(\xi_2 y)
\sin(\eta_1 ax) \sin(\eta_2 y)
\\
&=\phantom{+}
a
\sum_{i = 1}^{\infty}
\gamma_1
a_{\gamma}(t)
\sum_{j = 1}^{\infty}
\delta_2
b_{\delta}(t)
\cos(\gamma_1 ax) \sin(\gamma_2 y)
\cos(\delta_1 ax) \cos(\delta_2 y)
\\
&\phantom{={}}
+
a
\sum_{i = 1}^{\infty}
\gamma_2
a_{\gamma}(t)
\sum_{j = 1}^{\infty}
\delta_1
b_{\delta}(t)
\sin(\gamma_1 ax) \cos(\gamma_2 y)
\sin(\delta_1 ax) \sin(\delta_2 y)
\\
&=
a
\sum_{i = 1}^{\infty}
a_{\gamma}(t)
\sum_{j = 1}^{\infty}
b_{\delta}(t)
\left(
\gamma_1 \delta_2
\cos(\gamma_1 ax) \sin(\gamma_2 y)
\cos(\delta_1 ax) \cos(\delta_2 y)
\right.
\\[-2.5ex]
&\phantom{=abbb\sum\sum.....{}}
\left.
+
\gamma_2 \delta_1
\sin(\gamma_1 ax) \cos(\gamma_2 y)
\sin(\delta_1 ax) \sin(\delta_2 y)
\right)	
\\
&=
\frac{a}{4}
\sum_{i = 1}^{\infty}
a_{\gamma}(t)
\sum_{j = 1}^{\infty}
b_{\delta}(t)
\left(
\gamma_1 \delta_2
\left(
\cos((\gamma_1 - \delta_1)ax)\sin((\gamma_2 + \delta_2)y)
\right.
\right.
\\[-2.5ex]
&\phantom{=abbb\sum\sum.....{}}
+\cos((\gamma_1 - \delta_1)ax)\sin((\gamma_2 - \delta_2)y)
\\[-1.0ex]
&\phantom{=abbb\sum\sum.....{}}
+\cos((\gamma_1 + \delta_1)ax)\sin((\gamma_2 + \delta_2)y)
\\[-1.0ex]
&\phantom{=abbb\sum\sum.....{}}
\left.
+\cos((\gamma_1 + \delta_1)ax)\sin((\gamma_2 - \delta_2)y)
\right)
\\[-1.0ex]
&\phantom{=abbb\sum\sum.....{}}
+
\gamma_2 \delta_1
\left(
\cos((\gamma_1 - \delta_1)ax)\sin((\gamma_2 + \delta_2)y)
\right.
\\[-1.0ex]
&\phantom{=abbb\sum\sum.....{}}
-\cos((\gamma_1 - \delta_1)ax)\sin((\gamma_2 - \delta_2)y)
\\[-1.0ex]
&\phantom{=abbb\sum\sum.....{}}
-\cos((\gamma_1 + \delta_1)ax)\sin((\gamma_2 + \delta_2)y)
\\[-1.0ex]
&\phantom{=abbb\sum\sum.....{}}
\left.
\left.
+\cos((\gamma_1 + \delta_1)ax)\sin((\gamma_2 - \delta_2)y)
\right)
\right)
\\
&=
\frac{a}{4}
\sum_{i = 1}^{\infty}
a_{\gamma}(t)
\sum_{j = 1}^{\infty}
b_{\delta}(t)
\bigg(
\left(\gamma_1 \delta_2 + \gamma_2 \delta_1 \right)
\Big(
\cos((\gamma_1 - \delta_1)ax)\sin((\gamma_2 + \delta_2)y)
\\[-2.5ex]
&\phantom{=abbb\sum\sum.....{}}
+
\cos((\gamma_1 + \delta_1)ax)\sin((\gamma_2 - \delta_2)y)
\Big)
\\[-1.0ex]
&\phantom{=abbb\sum\sum.....{}}
+
\left(\gamma_1 \delta_2 - \gamma_2 \delta_1 \right)
\Big(
\cos((\gamma_1 - \delta_1)ax)\sin((\gamma_2 - \delta_2)y)
\\[-1.0ex]
&\phantom{=abbb\sum\sum.....{}}
+
\cos((\gamma_1 + \delta_1)ax)\sin((\gamma_2 + \delta_2)y)
\Big)
\bigg)
\\
&=
\frac{a}{4}
\sum_{i = 1}^{\infty}
a_{\gamma}(t)
\sum_{j = 1}^{\infty}
b_{\delta}(t)
\bigg(
\left(\gamma_1 \delta_2 + \gamma_2 \delta_1 \right)
\Big(
\sgn(\gamma_2 + \delta_2)
\cos(|\gamma_1 - \delta_1|ax)\sin(|\gamma_2 + \delta_2|y)
\\[-2.5ex]
&\phantom{=abbb\sum\sum.....{}}
+
\sgn(\gamma_2 - \delta_2)
\cos(|\gamma_1 + \delta_1|ax)\sin(|\gamma_2 - \delta_2|y)
\Big)
\\[-1.0ex]
&\phantom{=abbb\sum\sum.....{}}
+
\left(\gamma_1 \delta_2 - \gamma_2 \delta_1 \right)
\Big(
\sgn(\gamma_2 - \delta_2)
\cos(|\gamma_1 - \delta_1|ax)\sin(|\gamma_2 - \delta_2|y)
\\[-1.0ex]
&\phantom{=abbb\sum\sum.....{}}
+
\sgn(\gamma_2 + \delta_2)
\cos(|\gamma_1 + \delta_1|ax)\sin(|\gamma_2 + \delta_2|y)
\Big)
\bigg)
\\
&|\gamma| = i, |\delta| = j
\end{align*}
aufzul"osen. Jetzt haben wir alle Bausteine zusammen und k"onnen diese in die 
einzelnen Gleichungen einsetzen. Damit erhalten wir einerseits
\begin{align*}
&\frac{\partial\Delta\psi}{\partial t}
=
\nu\Delta^2\psi 
+c\frac{\partial\vartheta}{\partial x}
-\frac{\partial(\psi,\Delta\psi)}{\partial(x,y)}
\\
&\qquad \qquad \qquad \Leftrightarrow
\\
&-
\sum_{k = 1}^{\infty}
\dot{a}_{\gamma}(t)
\left(\gamma_1^2 a^2 + \gamma_2^2 \right)
\sin(\gamma_1 ax) \sin(\gamma_2 y)
\\
&=
\nu
\sum_{q = 1}^{\infty}
a_{\xi}(t)
\left(\xi_1^2 a^2+\xi_2^2\right)^2
\sin(\xi_1 ax) \sin(\xi_2 y)
\\
&\phantom{={}}
-
ca
\sum_{s = 1}^{\infty}
b_{\eta}(t)
\eta_1
\sin(\eta_1 ax) \sin(\eta_2 y)
\\
&\phantom{={}}
-
\frac{a}{4}
\sum_{i = 1}^{\infty}
\zeta_1
a_{\zeta}(t)
\sum_{j = 1}^{\infty}
\delta_2
\left(
a^2 \left(\zeta_1 - \delta_1 \right)\left( \zeta_1 + \delta_1 \right)
+ \left(\zeta_2 - \delta_2 \right)\left( \zeta_2 + \delta_2 \right)
\right)
a_{\delta}(t)
\\[-2.5ex]
&\phantom{=-abbbbb\sum\sum.....{}}
\left(
\sgn(\zeta_1 + \delta_1)\sgn(\zeta_2 + \delta_2)
\sin(|\zeta_1 + \delta_1|ax)\sin(|\zeta_2 + \delta_2|y)
\right.
\\[-1.0ex]
&\phantom{=-abbbbb\sum\sum.....{}}
+
\sgn(\zeta_1 + \delta_1)\sgn(\zeta_2 - \delta_2)
\sin(|\zeta_1 + \delta_1|ax)\sin(|\zeta_2 - \delta_2|y)
\\[-1.0ex]
&\phantom{=-abbbbb\sum\sum.....{}}
-
\sgn(\zeta_1 - \delta_1)\sgn(\zeta_2 + \delta_2)
\sin(|\zeta_1 - \delta_1|ax)\sin(|\zeta_2 + \delta_2|y)
\\[-1.0ex]
&\phantom{=-abbbbb\sum\sum.....{}}
\left.
-
\sgn(\zeta_1 - \delta_1)\sgn(\zeta_2 - \delta_2)
\sin(|\zeta_1 - \delta_1|ax)\sin(|\zeta_2 - \delta_2|y)
\right)
\\
&|\gamma| = k, |\xi| = q, |\eta| = s, |\zeta| = i, |\delta| = j
\\
&\qquad \qquad \qquad \Leftrightarrow
\\
&
\sum_{k = 1}^{\infty}
\dot{a}_{\gamma}(t)
\left(\gamma_1^2 a^2 + \gamma_2^2\right)
\sin(\gamma_1 ax) \sin(\gamma_2 y)
\\
&=
\sum_{i = 1}^{\infty}
\Bigg(
\left(
-\nu
\left(\alpha_1^2 a^2+\alpha_2^2\right)^2
a_{\alpha}(t)
+
\alpha_1 c a
b_{\alpha}(t)
\right)
\sin(\alpha_1 ax) \sin(\alpha_2 y)
\\[-2ex]
&\phantom{=bbbb}
+
\frac{\alpha_1 a}{4}
a_{\alpha}(t)
\sum_{j = 1}^{\infty}
\beta_2
\left(
a^2 \left(\alpha_1 - \beta_1 \right)\left(\alpha_1 + \beta_1 \right)
+ \left(\alpha_2 - \beta_2 \right)\left(\alpha_2 + \beta_2 \right)
\right)
a_{\beta}(t)
\\[-2.5ex]
&\phantom{=bb-4paatabb\sum\sum.{}}
\left(
\sgn(\alpha_1 + \beta_1)\sgn(\alpha_2 + \beta_2)
\sin(|\alpha_1 + \beta_1|ax)\sin(|\alpha_2 + \beta_2|y)
\right.
\\[-1.0ex]
&\phantom{=bb-4paatabb\sum\sum.{}}
+
\sgn(\alpha_1 + \beta_1)\sgn(\alpha_2 - \beta_2)
\sin(|\alpha_1 + \beta_1|ax)\sin(|\alpha_2 - \beta_2|y)
\\[-1.0ex]
&\phantom{=bb-4paatabb\sum\sum.{}}
-
\sgn(\alpha_1 - \beta_1)\sgn(\alpha_2 + \beta_2)
\sin(|\alpha_1 - \beta_1|ax)\sin(|\alpha_2 + \beta_2|y)
\\[-1.0ex]
&\phantom{=bb-4paatabb\sum\sum.{}}
\left.
-
\sgn(\alpha_1 - \beta_1)\sgn(\alpha_2 - \beta_2)
\sin(|\alpha_1 - \beta_1|ax)\sin(|\alpha_2 - \beta_2|y)
\right)
\Bigg)
\\
&|\gamma| = k, |\alpha| = i, |\beta| = j
\end{align*}
und andererseits
\begin{align*}
&\frac{\partial\vartheta}{\partial t}
=
\kappa\Delta\vartheta
+ \frac{T_0}{\pi}\frac{\partial\psi}{\partial x}
- \frac{\partial(\psi,\vartheta)}{\partial(x,y)}
\\
&\qquad \qquad \qquad \Leftrightarrow
\\
&
\sum_{k = 1}^{\infty}
\dot{b}_{\gamma}(t)
\cos(\gamma_1 ax) \sin(\gamma_2 y)
\\
&=
-
\kappa
\sum_{q = 1}^{\infty}
b_{\xi}(t)
\left(\xi_1^2 a^2 + \xi_2^2\right)
\cos(\xi_1 ax) \sin(\xi_2 y)
\\
&\phantom{={}}
+
\frac{a T_0}{\pi}
\sum_{s = 1}^{\infty}
\eta_1
a_{\eta}(t)
\cos(\eta_1 ax) \sin(\eta_2 y)
\\
&\phantom{={}}
-
\frac{a}{4}
\sum_{i = 1}^{\infty}
a_{\zeta}(t)
\sum_{j = 1}^{\infty}
b_{\delta}(t)
\left(
\left(\zeta_1 \delta_2 + \zeta_2 \delta_1 \right)
\left(
\sgn(\zeta_2 + \delta_2)
\cos(|\zeta_1 - \delta_1|ax)\sin(|\zeta_2 + \delta_2|y)
\right.
\right.
\\[-2.5ex]
&\phantom{=-abbb\sum\sum.....{}}
\left.
+
\sgn(\zeta_2 - \delta_2)
\cos(|\zeta_1 + \delta_1|ax)\sin(|\zeta_2 - \zeta_2|y)
\right)
\\[-1.0ex]
&\phantom{=-abbb\sum\sum.....{}}
+
\left(\zeta_1 \delta_2 - \zeta_2 \delta_1 \right)
\left(
\sgn(\zeta_2 - \delta_2)
\cos(|\zeta_1 - \delta_1|ax)\sin(|\zeta_2 - \delta_2|y)
\right.
\\[-1.0ex]
&\phantom{=-abbb\sum\sum.....{}}
\left.
\left.
+
\sgn(\zeta_2 + \delta_2)
\cos(|\zeta_1 + \delta_1|ax)\sin(|\zeta_2 + \delta_2|y)
\right)
\right)
\\
&|\gamma| = k, |\xi| = q, |\eta| = s, |\zeta| = i, |\delta| = j
\\
&\qquad \qquad \qquad \Leftrightarrow
\\
&\sum_{k = 1}^{\infty}
\dot{b}_{\gamma}(t)
\cos(\gamma_1 ax) \sin(\gamma_2 y)
\\
&=
\sum_{i = 1}^{\infty}
\Bigg(
\left(
-
\kappa
\left(\alpha_1^2 a^2 + \alpha_2^2\right)
b_{\alpha}(t)
+
\frac{\alpha_1 a T_0}{\pi}
a_{\alpha}(t)
\right)
\cos(\alpha_1 ax) \sin(\alpha_2 y)
\\
&\phantom{={}}
-
\frac{a}{4}
a_{\alpha}(t)
\sum_{j = 1}^{\infty}
b_{\beta}(t)
\left(
\left(\alpha_1 \beta_2 + \alpha_2 \beta_1 \right)
\left(
\sgn(\alpha_2 + \beta_2)
\cos(|\alpha_1 - \beta_1|ax)\sin(|\alpha_2 + \beta_2|y)
\right.
\right.
\\[-2.5ex]
&\phantom{=-aat\sum\sum...{}}
\left.
+
\sgn(\alpha_2 - \beta_2)
\cos(|\alpha_1 + \beta_1|ax)\sin(|\alpha_2 - \beta_2|y)
\right)
\\[-1.0ex]
&\phantom{=-aat\sum\sum...{}}
+
\left(\alpha_1 \beta_2 - \alpha_2 \beta_1 \right)
\left(
\sgn(\alpha_2 - \beta_2)
\cos(|\alpha_1 - \beta_1|ax)\sin(|\alpha_2 - \beta_2|y)
\right.
\\[-1.0ex]
&\phantom{=-aat\sum\sum...{}}
\left.
\left.
+
\sgn(\alpha_2 + \beta_2)
\cos(|\alpha_1 + \beta_1|ax)\sin(|\alpha_2 + \beta_2|y)
\right)
\right)
\Bigg)
\\
&|\gamma| = k, |\alpha| = i, |\beta| = j
\end{align*}
als Gleichungen die es nun zu l"osen gilt.

Da wir die $x$ und $y$ Komponenten loswerden wollen, damit nur noch $t$ "ubrig 
bleibt, brauchen wir Gleichungen f"ur einzelne $\dot{a}_\gamma(t)$
\begin{align}
\left(\gamma_1^2 a^2 + \gamma_2^2\right)
\dot{a}_\gamma(t)
&=
\sum_{i = 1}^{\infty}
\Bigg(
\left(
-\nu
\left(\alpha_1^2 a^2+\alpha_2^2\right)^2
a_{\alpha}(t)
+
\alpha_1 c a
b_{\alpha}(t)
\right)
f_\gamma(\alpha_1, \alpha_2) \nonumber
\\[-2ex]
&\phantom{=aaaa{}}
+
\frac{\alpha_1 a}{4}
a_{\alpha}(t)
\sum_{j = 1}^{\infty}
\beta_2
\left(
a^2 \left(\alpha_1 - \beta_1 \right)\left(\alpha_1 + \beta_1 \right)
+ \left(\alpha_2 - \beta_2 \right)\left(\alpha_2 + \beta_2 \right)
\right)
a_{\beta}(t) \nonumber
\\[-2ex]
&\phantom{=-4aapaata..\sum\sum.....{}}
\left(
f_\gamma(\alpha_1 + \beta_1, \alpha_2 + \beta_2)
+
f_\gamma(\alpha_1 + \beta_1, \alpha_2 - \beta_2)
\right. \nonumber
\\[-1.5ex]
&\phantom{=-4aapaata..\sum\sum.....{}}
\left.
-
f_\gamma(\alpha_1 - \beta_1, \alpha_2 + \beta_2)
-
f_\gamma(\alpha_1 - \beta_1, \alpha_2 - \beta_2)
\right)
\Bigg) \nonumber
\\
\Leftrightarrow \qquad
\dot{a}_\gamma(t)
&=
\left(
-\nu
\left(\gamma_1^2 a^2+\gamma_2^2\right)
a_{\gamma}(t)
+
\frac{\gamma_1 ca}{\gamma_1^2 a^2 + \gamma_2^2}
b_{\gamma}(t)
\right)
\sgn(\gamma_1)\sgn(\gamma_2) \nonumber
\\
&\phantom{={}}
+
\frac{a}{4 \left(\gamma_1^2 a^2 + \gamma_2^2\right)}
\sum_{i = 1}^{\infty}
\alpha_1
a_{\alpha}(t) \nonumber
\\[-3ex]
&\phantom{=a.-4aapaata...a\sum..{}}
\sum_{j = 1}^{\infty}
\beta_2
\left(
a^2 \left(\alpha_1 - \beta_1 \right)\left(\alpha_1 + \beta_1 \right)
+ \left(\alpha_2 - \beta_2 \right)\left(\alpha_2 + \beta_2 \right)
\right)
a_{\beta}(t) \nonumber
\\[-2ex]
&\phantom{=a.-4aapaata...a\sum....\sum..{}}
\left(
f_\gamma(\alpha_1 + \beta_1, \alpha_2 + \beta_2)
+
f_\gamma(\alpha_1 + \beta_1, \alpha_2 - \beta_2)
\right. \nonumber
\\[-.5ex]
&\phantom{=a.-4aapaata...a\sum....\sum..{}}
\left.
-
f_\gamma(\alpha_1 - \beta_1, \alpha_2 + \beta_2)
-
f_\gamma(\alpha_1 - \beta_1, \alpha_2 - \beta_2)
\right)
\label{equation:lorenz2:dota}
\\
&|\gamma| > 0, |\alpha| = i, |\beta| = j \nonumber
\end{align}
und wieder analog f"ur
\begin{align}
\dot{b}_\gamma(t)
&=
\sum_{i = 1}^{\infty}
\Bigg(
\left(
-
\kappa
\left(\alpha_1^2 a^2 + \alpha_2^2\right)
b_{\alpha}(t)
+
\frac{\alpha_1 a T_0}{\pi}
a_{\alpha}(t)
\right)
g_\gamma(\alpha_1, \alpha_2) 
\nonumber
\\
&\phantom{={}}
-
\frac{a}{4}
a_{\alpha}(t)
\sum_{j = 1}^{\infty}
b_{\beta}(t)
\left(
\left(\alpha_1 \beta_2 + \alpha_2 \beta_1 \right)
\left(
g_\gamma(\gamma_1 - \beta_1, \alpha_2 + \beta_2)
+
g_\gamma(\gamma_1 + \beta_1, \alpha_2 - \beta_2)
\right)
\right. \nonumber
\\[-3ex]
&\phantom{=-aat\sum\sum...{}}
\left.
+
\left(\alpha_1 \beta_2 - \alpha_2 \beta_1 \right)
\left(
g_\gamma(\gamma_1 - \beta_1, \alpha_2 - \beta_2)
+
g_\gamma(\gamma_1 + \beta_1, \alpha_2 + \beta_2)
\right)
\right)
\Bigg) \nonumber
\\
\Leftrightarrow \qquad
\dot{b}_\gamma(t)
&=
\left(
-
\kappa
\left(\gamma_1^2 a^2 + \gamma_2^2\right)
b_{\gamma}(t)
+
\frac{\gamma_1 a T_0}{\pi}
a_{\gamma}(t)
\right)
\sgn(\gamma_2) \nonumber
\\
&\phantom{={}}
-
\frac{a}{4}
\sum_{i = 1}^{\infty}
a_{\alpha}(t)
\sum_{j = 1}^{\infty}
b_{\beta}(t)
\left(
\left(\alpha_1 \beta_2 + \alpha_2 \beta_1 \right)
\left(
g_\gamma(\alpha_1 - \beta_1, \alpha_2 + \beta_2)
+
g_\gamma(\alpha_1 + \beta_1, \alpha_2 - \beta_2)
\right)
\right.\nonumber
\\[-2ex]
&\phantom{=-aat\sum..a..\sum...{}}
\left.
+
\left(\alpha_1 \beta_2 - \alpha_2 \beta_1 \right)
\left(
g_\gamma(\alpha_1 - \beta_1, \alpha_2 - \beta_2)
+
g_\gamma(\alpha_1 + \beta_1, \alpha_2 + \beta_2)
\right)
\right)
\label{equation:lorenz2:dotb}
\\
&|\gamma| > 0, |\alpha| = i, |\beta| = j.\nonumber
\end{align}

Die Definition der beiden Hilfsfunktionen lautet
\begin{equation*}
f_\gamma(q, s)
=
\begin{cases}
\sgn(q)\sgn(s) & |q| = \gamma_1, |s| = \gamma_2 \\
0 & \text{sonst}
\end{cases}
\end{equation*}
und
\begin{equation*}
g_\gamma(q, s)
=
\begin{cases}
\sgn(s) & |q| = \gamma_1, |s| = \gamma_2 \\
0 & \text{sonst}.
\end{cases}
\end{equation*}

\section{Lorenzsystem vierten Grades}
\rhead{Lorenzsystem vierten Grades}
Anhand des Lorenzsystems vierten Grades wollen wir nun zeigen, welche 
Information verloren geht, wenn wir uns alleinig auf ein Systme zweiten Grades 
konzentrieren.

Zuerst brauchen wir allerdings alle Gleichungen von $k = 1$
\begin{align*}
|\gamma| = 1
\qquad &
\dot{a}_{10}(t) = 0
\\
&
\dot{a}_{01}(t) = 0
\\
\\
&
\dot{b}_{10}(t) = 0
\\
&
\dot{b}_{01}(t)
=
-
\kappa
b_{01}(t)
\\
&\phantom{aaaaaaaaaa}
\color{blue}
+
\frac{a}{4} a_{11}(t) b_{12}(t)
+
\frac{a}{4} a_{12}(t) b_{11}(t)
\\
&\phantom{aaaaaaaaaa}
\color{red}
+
\frac{a}{4} a_{12}(t) b_{13}(t)
+
\frac{a}{4} a_{13}(t) b_{12}(t)
+
\frac{a}{2} a_{21}(t) b_{22}(t)
+
\frac{a}{2} a_{22}(t) b_{21}(t)
\end{align*}
"uber $k = 2$
\begin{align*}
|\gamma| = 2
\qquad &
\dot{a}_{20}(t) = 0
\\
&
\dot{a}_{11}(t)
=
-
(a^2+1)
\nu
a_{11}(t)
+
\frac{a c}{a^2+1} b_{11}(t)
\\
&\phantom{aaaaaaaaaa}
\color{blue}
+
\frac{9 a (a^2 - 1)}{4 (a^2+1)} a_{12}(t) a_{21}(t)
\\
&\phantom{aaaaaaaaaa}
\color{red}
+
\frac{a (3 a^2 - 5)}{a^2+1} a_{13}(t) a_{22}(t)
+
\frac{a (5 a^2 - 3)}{a^2+1} a_{22}(t) a_{31}(t)
\\
&
\dot{a}_{02}(t) = 0
\\
\\
&
\dot{b}_{20}(t) = 0
\\
&
\dot{b}_{11}(t)
=
-
(a^2+1)
\kappa
b_{11}(t)
+
\frac{a T_{0}}{\pi} a_{11}(t)
\\
&\phantom{aaaaaaaaaa}
+
a
a_{11}(t) b_{02}(t)
\\
&\phantom{aaaaaaaaaa}
\color{blue}
-
\frac{a}{2} a_{12}(t) b_{01}(t)
+
\frac{3 a}{2} a_{12}(t) b_{03}(t)
+
\frac{3 a}{4} a_{12}(t) b_{21}(t)
+
\frac{3 a}{4} a_{21}(t) b_{12}(t)
\\
&\phantom{aaaaaaaaaa}
\color{red}
-
a
a_{13}(t) b_{02}(t)
+
a
a_{13}(t) b_{22}(t)
+
2 a
a_{13}(t) b_{04}(t)
\\
&\phantom{aaaaaaaaaa}
\color{red}
+
a
a_{22}(t) b_{13}(t)
+
a
a_{22}(t) b_{31}(t)
+
a
a_{31}(t) b_{22}(t)
\\
&
\dot{b}_{02}(t)
=
-
4
\kappa
b_{02}(t)
\\
&\phantom{aaaaaaaaaa}
-
\frac{a}{2} a_{11}(t) b_{11}(t)
\\
&\phantom{aaaaaaaaaa}
\color{blue}
-
a
a_{21}(t) b_{21}(t)
\\
&\phantom{aaaaaaaaaa}
\color{red}
+
\frac{a}{2} a_{11}(t) b_{13}(t)
+
\frac{a}{2} a_{13}(t) b_{11}(t)
-
\frac{3 a}{2} a_{31}(t) b_{31}(t)
\end{align*}
mit $k = 3$
\begin{align*}
|\gamma| = 3
\qquad &
\dot{a}_{30}(t) = 0
\\
&
\dot{a}_{21}(t)
=
-
(4 a^2+1)
\nu
a_{21}(t)
+
\frac{2 a c}{4 a^2+1} b_{21}(t)
\\
&\phantom{aaaaaaaaaa}
+
\frac{9 a}{4 (4 a^2+1)} a_{11}(t) a_{12}(t)
\\
&\phantom{aaaaaaaaaa}
\color{red}
+
\frac{5 a (8 a^2 - 3)}{4 (4 a^2+1)} a_{12}(t) a_{31}(t)
+
\frac{25 a}{4 (4 a^2+1)} a_{12}(t) a_{13}(t)
\\
&
\dot{a}_{12}(t)
=
-
(a^2+4)
\nu
a_{12}(t)
+
\frac{a c}{a^2+4} b_{12}(t)
\\
&\phantom{aaaaaaaaaa}
-
\frac{9 a^3}{4 (a^2+4)} a_{11}(t) a_{21}(t)
\\
&\phantom{aaaaaaaaaa}
\color{red}
+
\frac{5 a (3 a^2 - 8)}{4 (a^2+4)} a_{13}(t) a_{21}(t)
-
\frac{25 a^3}{4 (a^2+4)} a_{21}(t) a_{31}(t)
\\
&
\dot{a}_{03}(t) = 0
\\
\\
&
\dot{b}_{30}(t) = 0
\\
&
\dot{b}_{21}(t)
=
-
(4 a^2+1)
\kappa
b_{21}(t)
+
\frac{2 a T_{0}}{\pi} a_{21}(t)
\\
&\phantom{aaaaaaaaaa}
+
\frac{3 a}{4} a_{11}(t) b_{12}(t)
-
\frac{3 a}{4} a_{12}(t) b_{11}(t)
+
2 a
a_{21}(t) b_{02}(t)
\\
&\phantom{aaaaaaaaaa}
\color{red}
+
\frac{5 a}{4} a_{12}(t) b_{13}(t)
-
\frac{5 a}{4} a_{13}(t) b_{12}(t)
-
a
a_{22}(t) b_{01}(t)
\\
&\phantom{aaaaaaaaaa}
\color{red}
+
\frac{5 a}{4} a_{12}(t) b_{31}(t)
+
\frac{5 a}{4} a_{31}(t) b_{12}(t)
+
3 a
a_{22}(t) b_{03}(t)
\\
&
\dot{b}_{12}(t)
=
-
(a^2+4)
\kappa
b_{12}(t)
+
\frac{a T_{0}}{\pi} a_{12}(t)
\\
&\phantom{aaaaaaaaaa}
-
\frac{a}{2} a_{11}(t) b_{01}(t)
\\
&\phantom{aaaaaaaaaa}
+
\frac{3 a}{2} a_{11}(t) b_{03}(t)
-
\frac{3 a}{4} a_{11}(t) b_{21}(t)
-
\frac{3 a}{4} a_{21}(t) b_{11}(t)
\\
&\phantom{aaaaaaaaaa}
\color{red}
-
\frac{5 a}{4} a_{21}(t) b_{31}(t)
-
\frac{5 a}{4} a_{31}(t) b_{21}(t)
+
\frac{5 a}{4} a_{13}(t) b_{21}(t)
\\
&\phantom{aaaaaaaaaa}
\color{red}
+
\frac{5 a}{4} a_{21}(t) b_{13}(t)
-
\frac{a}{2} a_{13}(t) b_{01}(t)
+
2 a
a_{12}(t) b_{04}(t)
\\
&
\dot{b}_{03}(t)
=
-
9
\kappa
b_{03}(t)
\\
&\phantom{aaaaaaaaaa}
-
\frac{3 a}{4} a_{11}(t) b_{12}(t)
-
\frac{3 a}{4} a_{12}(t) b_{11}(t)
\\
&\phantom{aaaaaaaaaa}
\color{red}
-
\frac{3 a}{2} a_{21}(t) b_{22}(t)
-
\frac{3 a}{2} a_{22}(t) b_{21}(t)
\end{align*}
und $k = 4$
\begin{align*}
|\gamma| = 4
\qquad &
\dot{a}_{40}(t) = 0
\\
&
\dot{a}_{31}(t)
=
-
(9 a^2+1)
\nu
a_{31}(t)
+
\frac{3 a c}{9 a^2+1} b_{31}(t)
\\
&\phantom{aaaaaaaaaa}
+
\frac{15 a (1 - a^2)}{4 (9 a^2+1)} a_{12}(t) a_{21}(t)
\\
&\phantom{aaaaaaaaaa}
+
\frac{12 a (1 + a^2)}{4 (9 a^2+1)} a_{11}(t) a_{22}(t)
+
\frac{8 a (5 - 3 a^2)}{4 (9 a^2+1)} a_{13}(t) a_{22}(t)
\\
&
\dot{a}_{22}(t)
=
-
4
(a^2+1)
\nu
a_{22}(t)
+
\frac{a c}{2 (a^2+1)} b_{22}(t)
\\
&\phantom{aaaaaaaaaa}
+
\frac{2 a}{a^2+1} a_{11}(t) a_{13}(t)
-
\frac{2 a^3}{a^2+1} a_{11}(t) a_{31}(t)
\\
&\phantom{aaaaaaaaaa}
+
\frac{4 a^3}{a^2+1} a_{13}(t) a_{31}(t)
-
\frac{4 a}{a^2+1} a_{13}(t) a_{31}(t)
\\
&
\dot{a}_{13}(t)
=
-
(a^2+9)
\nu
a_{13}(t)
+
\frac{a c}{a^2+9} b_{13}(t)
\\
&\phantom{aaaaaaaaaa}
+
\frac{15 a (1 - a^2)}{4 (a^2+9)} a_{12}(t) a_{21}(t)
\\
&\phantom{aaaaaaaaaa}
-
\frac{12 a (1 + a^2)}{4 (a^2+9)} a_{11}(t) a_{22}(t)
+
\frac{8 a (3 - 5 a^2)}{4 (a^2+9)} a_{22}(t) a_{31}(t)
\\
&
\dot{a}_{04}(t) = 0
\\
\\
&
\dot{b}_{40}(t) = 0
\\
&
\dot{b}_{31}(t)
=
-
(9 a^2+1)
\kappa
b_{31}(t)
+
\frac{3 a T_{0}}{\pi} a_{31}(t)
\\
&\phantom{aaaaaaaaaa}
+
\frac{5 a}{4} a_{21}(t) b_{12}(t)
-
\frac{5 a}{4} a_{12}(t) b_{21}(t)
\\
&\phantom{aaaaaaaaaa}
+
3 a
a_{31}(t) b_{02}(t)
+
a
a_{11}(t) b_{22}(t)
-
a
a_{22}(t) b_{11}(t)
\\
&\phantom{aaaaaaaaaa}
-
2 a
a_{13}(t) b_{22}(t)
+
2 a
a_{22}(t) b_{13}(t)
\\
&
\dot{b}_{22}(t)
=
-
(4 a^2+4)
\kappa
b_{22}(t)
+
\frac{2 a T_{0}}{\pi} a_{22}(t)
\\
&\phantom{aaaaaaaaaa}
-
a
a_{21}(t) b_{01}(t)
+
3 a
a_{21}(t) b_{03}(t)
\\
&\phantom{aaaaaaaaaa}
+
a
a_{11}(t) b_{13}(t)
-
a
a_{13}(t) b_{11}(t)
-
a
a_{11}(t) b_{31}(t)
-
a
a_{31}(t) b_{11}(t)
\\
&\phantom{aaaaaaaaaa}
+
2 a
a_{13}(t) b_{31}(t)
+
2 a
a_{31}(t) b_{13}(t)
+
4 a
a_{22}(t) b_{04}(t)
\\
&
\dot{b}_{13}(t)
=
-
(a^2+9)
\kappa
b_{13}(t)
+
\frac{a T_{0}}{\pi} a_{13}(t)
\\
&\phantom{aaaaaaaaaa}
-
a
a_{11}(t) b_{02}(t)
\\
&\phantom{aaaaaaaaaa}
-
\frac{a}{2} a_{12}(t) b_{01}(t)
-
\frac{5 a}{4} a_{12}(t) b_{21}(t)
-
\frac{5 a}{4} a_{21}(t) b_{12}(t)
\\
&\phantom{aaaaaaaaaa}
-
a
a_{11}(t) b_{22}(t)
-
a
a_{22}(t) b_{11}(t)
-
2 a
a_{22}(t) b_{31}(t)
-
2 a
a_{31}(t) b_{22}(t)
\\
&\phantom{aaaaaaaaaa}
+
2 a
a_{11}(t) b_{04}(t)
\\
&
\dot{b}_{04}(t)
=
-
16
\kappa
b_{04}(t)
\\
&\phantom{aaaaaaaaaa}
-
a
a_{12}(t) b_{12}(t)
\\
&\phantom{aaaaaaaaaa}
-
a
a_{11}(t) b_{13}(t)
-
a
a_{13}(t) b_{11}(t)
-
2a
a_{22}(t) b_{22}(t).
\end{align*}
Terme die verloren gehen, wenn man beim jeweiligen $k$ stoppen w"urde, sind 
$\color{blue}{blau}$ (dritter Grad), beziehungsweise $\color{red}{rot}$ 
(vierten Grades) hervorgehoben.

Vergleicht man die Resultate f"ur $k = 2$ mit denjenigen aus 
\cref{skript:lorenz:dim} stellt man fest, dass diese "ubereinstimmen, womit 
auch wieder gezeigt ist, dass unsere neuen Basisfunktionen eine echte 
Erweiterung sind. Bereits jetzt ist aber ersichtlich, dass die Anzahl 
Gleichungen die es zu l"osen gilt stark mit dem gew"ahlten Grad $k$ w"achst 
(\cref{table:lorenz2:degree}). Beispielsweise muss f"ur $k = 10$ ein 
Gleichungssystem mit
\begin{equation*}
	2\left(\frac{(10 + 1)(10 + 2)}{2} - 1\right) = 11 \cdot 12 - 2 = 130
\end{equation*}
Gleichungen gel"ost werden, das zudem noch aus Gleichungen besteht, die "uber 
etliche Kopplungen miteinander verbunden sind.

\begin{table}
	\centering
	\begin{tabular}{c | l}
		Grad $k$ & Anzahl Gleichungen \\
		1 & $2$ \\
		2 & $2 + 3$ \\
		3 & $2 + 3 + 4$\\
		4 & $2 + 3 + 4 + 5$\\
		\dots & \dots \\
		$n$ & $\dfrac{(n + 1)((n + 1) + 1)}{2} - 1
		= \dfrac{(n + 1)(n + 2)}{2} - 1$
	\end{tabular}
	\caption{Wachstun der Anzahl Gleichung mit dem Grad $k$}
	\label{table:lorenz2:degree}
\end{table}

\section{Numerische L"osung\label{section:lorenz2:numeric-solution}}
\rhead{Numerische L"osung}
Mit den \cref{equation:lorenz2:dota,equation:lorenz2:dotb} aus 
\cref{section:lorenz2:ho-model} haben wir Gleichungen gefunden, die wir 
mit einem Computer-Algebra Programm, wie beispielsweise \texttt{maxima}, 
generiert und dann mit einem ODE-Solver, zum Beispiel \texttt{lsode} in 
\texttt{octave}, gel"ost werden k"onnen.

\section{Schlussfolgerungen}
\rhead{Schlussfolgerungen}

\printbibliography[heading=subbibliography]
\end{refsection}
