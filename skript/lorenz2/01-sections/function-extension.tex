\section{Erweiterung der Basisfunktionen\label{section:lorenz2:basic_function}}
\rhead{Erweiterung der Basisfunktionen}
Die Suche neuer Basisfunktionen beginnt mit denjenigen, die bereits 
in \cref{skript:funktionsauswahl} erw"ahnt sind. Bei genauerer Betrachtung 
stellt man fest, dass diese auch ein wenig anders geschrieben werden k"onnen:
\begin{align*}
\sin(ax)\sin(y) &= \sin(1ax)\sin(1y),\\
\cos(ax)\sin(y) &= \cos(1ax)\sin(1y),\\
\sin(2y) &= \cos(0ax)\sin(2y).
\end{align*}
Nun f"ugen wir ein paar zus"atzliche, auf den ersten Blick etwas nutzlose
Gleichungen hinzu:
\begin{align*}
0 &= \sin(0ax)\sin(2y) \\
\sin(ax)\sin(y) &= \sin(1ax)\sin(1y)\\
0 &= \sin(2ax)\sin(0y) \\
\sin(2y) &= \cos(0ax)\sin(2y)\\
\cos(ax)\sin(y) &= \cos(1ax)\sin(1y)\\
0 &= \cos(2ax)\sin(0y)\\
\end{align*}
Daraus l"asst sich nun das Muster
\begin{equation}
\begin{split}
\sin(\alpha_1 ax)\sin(\alpha_2 y) \\
\cos(\alpha_1 ax)\sin(\alpha_2 y)
\end{split}
\label{equation:lorenz2:basic-functions}
\end{equation}
erkennen, wobei $\alpha_1$ und $\alpha_2$ Teile eines Multi-Indices sind und 
die Bedingung $|\alpha| = k = 2$ erf"ullen.

F"ur unser dreidimensionales Modell hatten wir also Basisfunktionen zweiten 
Grades ($k = 2$) gefunden. Mit den neuen Funktionen k"onnen wir aber nicht nur 
die uns bekannten generieren, sondern diese auch noch erweitern indem wir den 
Grad $k$ variieren.

Leicht zu erkennen ist, dass f"ur $k = 0$ nur die $0$-Funktion 
"ubrig bleibt, da einzig das $(0, 0)$-Tupel die Bedingung $|\alpha| = 0$ 
erf"ullt. $k = 1$ generiert die beiden Tupel $(0, 1)$ und $(1, 0)$, womit die 
Funktion $\sin(y)$ "ubrig bleibt. Somit k"onnen wir Grad $k \geq 1$ 
voraussetzen. Damit wird auch gleich das Problem einer m"oglichen Division 
durch $0$ elegant umgangen, wie wir sp"ater sehen werden.
