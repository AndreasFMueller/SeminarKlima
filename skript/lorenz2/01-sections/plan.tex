Auf das dreidimensionale Lorenz-System wurde bereits in 
\cref{section:lorenz-modell} sowie in \cref{chapter:lorenz} detailliert 
eingegangen. Trotzdem werden folgend nochmals die f"ur das Lorenz-Modell 
\index{Lorenz-Modell}%
relevanten Grundgleichungen
\begin{equation}
	\begin{aligned}
	\frac{\partial\Delta\psi}{\partial t}
	&=
	\nu\Delta^2\psi 
	+c\frac{\partial\vartheta}{\partial x}
	-\frac{\partial(\psi,\Delta\psi)}{\partial(x,y)}
	\\
	\frac{\partial\vartheta}{\partial t}
	&=
	\kappa\Delta\vartheta
	+ \frac{T_0}{\pi}\frac{\partial\psi}{\partial x}
	- \frac{\partial(\psi,\vartheta)}{\partial(x,y)}
	\end{aligned}
	\label{equation:lorenz2:base}
\end{equation}
genauer analysiert. Dank einer geschickten Wahl von Basisfunktionen
\begin{equation*}
	\sin(ax)\sin(y),
	\qquad
	\cos(ax)\sin(y)
	\qquad\text{und}\qquad
	\sin(2y)
\end{equation*}
und einigen Vereinfachungen, hatten wir es geschafft, das System auf drei 
gew"ohnliche Differential\-gleichungen
\begin{equation*}
	\begin{aligned}
	\dot X(t)
	&=
	-\nu(a^2+1)X(t)
	+\frac{ac}{a^2+1}Y(t)
	\\
	\dot Y(t)
	&=
	\frac{aT_0}{\pi}X(t)
	-(a^2+1)\kappa Y(t)
	-aX(t)Z(t)
	\\
	\dot Z(t)
	&=
	-4\kappa Z(t)
	+\frac{a}{2}X(t)Y(t)
	\end{aligned}
\end{equation*}
zu reduzieren, welche unser dreidimensionales Lorenz-System beschreiben.

In diesem Kapitel wollen wir nun diese mittels Separationsverfahren gefunden 
L"osungen, welche unsere Basisfunktionen bilden, erweitern. Diese neuen 
Basisfunktionen sollen es uns erm"oglichen auch Terme, die wir bisher bei 
unserer L"osung vernachl"assigt haben, mit einzubeziehen. Dies f"uhrt uns dann 
schlussendlich zu einem neuen Lorenz-System, welches nicht mehr 
dreidimensional sondern h"oherdimensional ist. Bis wir die daraus entstehenden 
Resultate in \cref{section:lorenz2:numeric-solution} begutachten k"onnen, 
m"ussen wir erst ein paar Dinge zur Notation und allgemeinen Formeln in 
\cref{section:lorenz2:einstieg} kl"aren. Danach machen wir uns auf die Suche 
neuer Basisfunktionen in \cref{section:lorenz2:basic_function} und generieren 
nach Einsetzen und Vereinfachen in \cref{section:lorenz2:ho-model} die zum 
L"osen unserer neuen h"oherdimensionalen Lorenz-Systems ben"otigten 
gew"ohnlichen Differentialgleichungen. \Cref{section:lorenz2:4degreelorenz} 
zeigt uns zudem an einem konkreten Beispiel eines Lorenz-System vierten 
Grades, welche Informationen verloren gehen, wenn Terme vernachl"assigt werden.
