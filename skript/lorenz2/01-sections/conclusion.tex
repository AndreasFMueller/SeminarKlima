\section{Schlussfolgerungen}
\rhead{Schlussfolgerungen}
Wir haben somit gezeigt, dass es m"oglich ist, ein h"oherdimensionales 
Lorenzsystem mittels Erweiterung der bekannten Basisfunktionen zu bestimmen. 
Wir k"onnen weiter die ben"otigten Gleichungen mittels Computer generieren und 
auswerten, auch wenn dies mit gr"osser werdendem Grad $k$ mit einem starken 
Anstieg der Laufzeit einher geht. Eine m"ogliche Verbesserung w"are 
beispielsweise eine parallele Berechnung der einzelnen gew"ohnlicher 
Differentialgleichungen, was aber nicht untersucht wurde. In einem weiteren 
Schritt m"usste zudem der Raum der Anfangsbedingungen und Parameter weiter 
untersucht werden, um zu zeigen, dass das chaotische Verhalten wirklich ab $k = 
4$ verloren geht oder ob es schlicht an den von uns gew"ahlten Anfangs- und 
Parameterwerten liegt. Zu guter Letzt k"onnte auch noch untersucht werden, was 
passiert, wenn der Grad $k$ noch weiter gesteigert wird, denn es ist m"oglich, 
dass wir einfach noch mehr Terme ben"otigen um die ``Schmetterlingsfl"ugel'' 
und somit das chaotische Verhalten wieder zu erzeugen. Daf"ur m"usste, mit der 
momentanen Implementation, allerdings viel Geduld mitgebracht werden.
