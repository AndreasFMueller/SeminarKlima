\section{Schlussfolgerungen}
\rhead{Schlussfolgerungen}
Wir haben somit gezeigt, dass es m"oglich ist, ein Lorenzsystem h"oherer 
Ordnung 
mittels Erweiterung der bekannten Basisfunktionen zu bestimmen. Wir k"onnen 
weiter die ben"otigten Gleichungen mittels Computer generieren und auswerten, 
auch wenn dies mit gr"osser werdendem Grad $k$ "ausserst lange dauert. Eine 
m"ogliche Verbesserung w"are Beispielsweise eine parallele Berechnung der 
einzelnen ODE Gleichungen.

Erstaunlich ist auch, dass ab Grad $k = 4$ das chaotische Verhalten nicht mehr 
aufzutreten scheint. Das k"onnte einerseits darauf hindeuten, dass dies durch 
die Reduktion des originalen Systems entstanden ist. Andererseits k"onnte dies 
auch einfach daran liegen, dass wir den Raum unserer Anfangsbedingungen und 
Parameter nicht weiter untersucht haben. So kann es durchaus sein, dass es 
etwaige Werte gibt, die uns wieder ein chaotisches Lorenzsystem liefern.
