\section{Lorenzsystem vierten Grades\label{section:lorenz2:4degreelorenz}}
\rhead{Lorenzsystem vierten Grades}
Anhand des Lorenzsystems vierten Grades wollen wir nun zeigen, welche 
Information verloren geht, wenn wir uns alleinig auf ein Systme zweiten Grades 
konzentrieren.

Zuerst brauchen wir allerdings alle Gleichungen von $k = 1$
\begin{align*}
|\gamma| = 1
\qquad &
\dot{a}_{(1,0)}(t) = 0
\\
&
\dot{a}_{(0,1)}(t) = 0
\\
\\
&
\dot{b}_{(1,0)}(t) = 0
\\
&
\dot{b}_{(0,1)}(t)
=
-
\kappa
b_{(0,1)}(t)
\\
&\phantom{aaaaaaaaaa}
\color{blue}
+
\frac{a}{4} a_{(1,1)}(t) b_{(1,2)}(t)
+
\frac{a}{4} a_{(1,2)}(t) b_{(1,1)}(t)
\\
&\phantom{aaaaaaaaaa}
\color{red}
+
\frac{a}{4} a_{(1,2)}(t) b_{(1,3)}(t)
+
\frac{a}{4} a_{(1,3)}(t) b_{(1,2)}(t)
\\
&\phantom{aaaaaaaaaa}
\color{red}
+
\frac{a}{2} a_{(2,1)}(t) b_{(2,2)}(t)
+
\frac{a}{2} a_{(2,2)}(t) b_{(2,1)}(t)
\end{align*}
"uber $k = 2$
\begin{align*}
|\gamma| = 2
\qquad &
\dot{a}_{(2,0)}(t) = 0
\\
&
\dot{a}_{(1,1)}(t)
=
-
(a^2+1)
\nu
a_{(1,1)}(t)
+
\frac{a c}{a^2+1} b_{(1,1)}(t)
\\
&\phantom{aaaaaaaaaa}
\color{blue}
+
\frac{9 a (a^2 - 1)}{4 (a^2+1)} a_{(1,2)}(t) a_{(2,1)}(t)
\\
&\phantom{aaaaaaaaaa}
\color{red}
+
\frac{a (3 a^2 - 5)}{a^2+1} a_{(1,3)}(t) a_{(2,2)}(t)
+
\frac{a (5 a^2 - 3)}{a^2+1} a_{(2,2)}(t) a_{(3,1)}(t)
\\
&
\dot{a}_{(0,2)}(t) = 0
\\
\\
&
\dot{b}_{(2,0)}(t) = 0
\\
&
\dot{b}_{(1,1)}(t)
=
-
(a^2+1)
\kappa
b_{(1,1)}(t)
+
\frac{a T_{0}}{\pi} a_{(1,1)}(t)
\\
&\phantom{aaaaaaaaaa}
+
a
a_{(1,1)}(t) b_{(0,2)}(t)
\\
&\phantom{aaaaaaaaaa}
\color{blue}
-
\frac{a}{2} a_{(1,2)}(t) b_{(0,1)}(t)
+
\frac{3 a}{2} a_{(1,2)}(t) b_{(0,3)}(t)
\\
&\phantom{aaaaaaaaaa}
\color{blue}
+
\frac{3 a}{4} a_{(1,2)}(t) b_{(2,1)}(t)
+
\frac{3 a}{4} a_{(2,1)}(t) b_{(1,2)}(t)
\\
&\phantom{aaaaaaaaaa}
\color{red}
-
a
a_{(1,3)}(t) b_{(0,2)}(t)
+
a
a_{(1,3)}(t) b_{(2,2)}(t)
+
2 a
a_{(1,3)}(t) b_{(0,4)}(t)
\\
&\phantom{aaaaaaaaaa}
\color{red}
+
a
a_{(2,2)}(t) b_{(1,3)}(t)
+
a
a_{(2,2)}(t) b_{(3,1)}(t)
+
a
a_{(3,1)}(t) b_{(2,2)}(t)
\\
&
\dot{b}_{(0,2)}(t)
=
-
4
\kappa
b_{(0,2)}(t)
\\
&\phantom{aaaaaaaaaa}
-
\frac{a}{2} a_{(1,1)}(t) b_{(1,1)}(t)
\\
&\phantom{aaaaaaaaaa}
\color{blue}
-
a
a_{(2,1)}(t) b_{(2,1)}(t)
\\
&\phantom{aaaaaaaaaa}
\color{red}
+
\frac{a}{2} a_{(1,1)}(t) b_{(1,3)}(t)
+
\frac{a}{2} a_{(1,3)}(t) b_{(1,1)}(t)
-
\frac{3 a}{2} a_{(3,1)}(t) b_{(3,1)}(t)
\end{align*}
mit $k = 3$
\begin{align*}
|\gamma| = 3
\qquad &
\dot{a}_{(3,0)}(t) = 0
\\
&
\dot{a}_{(2,1)}(t)
=
-
(4 a^2+1)
\nu
a_{(2,1)}(t)
+
\frac{2 a c}{4 a^2+1} b_{(2,1)}(t)
\\
&\phantom{aaaaaaaaaa}
+
\frac{9 a}{4 (4 a^2+1)} a_{(1,1)}(t) a_{(1,2)}(t)
\\
&\phantom{aaaaaaaaaa}
\color{red}
+
\frac{5 a (8 a^2 - 3)}{4 (4 a^2+1)} a_{(1,2)}(t) a_{(3,1)}(t)
+
\frac{25 a}{4 (4 a^2+1)} a_{(1,2)}(t) a_{(1,3)}(t)
\\
&
\dot{a}_{(1,2)}(t)
=
-
(a^2+4)
\nu
a_{(1,2)}(t)
+
\frac{a c}{a^2+4} b_{(1,2)}(t)
\\
&\phantom{aaaaaaaaaa}
-
\frac{9 a^3}{4 (a^2+4)} a_{(1,1)}(t) a_{(2,1)}(t)
\\
&\phantom{aaaaaaaaaa}
\color{red}
+
\frac{5 a (3 a^2 - 8)}{4 (a^2+4)} a_{(1,3)}(t) a_{(2,1)}(t)
-
\frac{25 a^3}{4 (a^2+4)} a_{(2,1)}(t) a_{(3,1)}(t)
\\
&
\dot{a}_{(0,3)}(t) = 0
\\
\\
&
\dot{b}_{(3,0)}(t) = 0
\\
&
\dot{b}_{(2,1)}(t)
=
-
(4 a^2+1)
\kappa
b_{(2,1)}(t)
+
\frac{2 a T_{0}}{\pi} a_{(2,1)}(t)
\\
&\phantom{aaaaaaaaaa}
+
\frac{3 a}{4} a_{(1,1)}(t) b_{(1,2)}(t)
-
\frac{3 a}{4} a_{(1,2)}(t) b_{(1,1)}(t)
+
2 a
a_{(2,1)}(t) b_{(0,2)}(t)
\\
&\phantom{aaaaaaaaaa}
\color{red}
+
\frac{5 a}{4} a_{(1,2)}(t) b_{(1,3)}(t)
-
\frac{5 a}{4} a_{(1,3)}(t) b_{(1,2)}(t)
-
a
a_{(2,2)}(t) b_{(0,1)}(t)
\\
&\phantom{aaaaaaaaaa}
\color{red}
+
\frac{5 a}{4} a_{(1,2)}(t) b_{(3,1)}(t)
+
\frac{5 a}{4} a_{(3,1)}(t) b_{(1,2)}(t)
+
3 a
a_{(2,2)}(t) b_{(0,3)}(t)
\\
&
\dot{b}_{(1,2)}(t)
=
-
(a^2+4)
\kappa
b_{(1,2)}(t)
+
\frac{a T_{0}}{\pi} a_{(1,2)}(t)
\\
&\phantom{aaaaaaaaaa}
-
\frac{a}{2} a_{(1,1)}(t) b_{(0,1)}(t)
\\
&\phantom{aaaaaaaaaa}
+
\frac{3 a}{2} a_{(1,1)}(t) b_{(0,3)}(t)
-
\frac{3 a}{4} a_{(1,1)}(t) b_{(2,1)}(t)
-
\frac{3 a}{4} a_{(2,1)}(t) b_{(1,1)}(t)
\\
&\phantom{aaaaaaaaaa}
\color{red}
-
\frac{5 a}{4} a_{(2,1)}(t) b_{(3,1)}(t)
-
\frac{5 a}{4} a_{(3,1)}(t) b_{(2,1)}(t)
+
\frac{5 a}{4} a_{(1,3)}(t) b_{(2,1)}(t)
\\
&\phantom{aaaaaaaaaa}
\color{red}
+
\frac{5 a}{4} a_{(2,1)}(t) b_{(1,3)}(t)
-
\frac{a}{2} a_{(1,3)}(t) b_{(0,1)}(t)
+
2 a
a_{(1,2)}(t) b_{(0,4)}(t)
\\
&
\dot{b}_{(0,3)}(t)
=
-
9
\kappa
b_{(0,3)}(t)
\\
&\phantom{aaaaaaaaaa}
-
\frac{3 a}{4} a_{(1,1)}(t) b_{(1,2)}(t)
-
\frac{3 a}{4} a_{(1,2)}(t) b_{(1,1)}(t)
\\
&\phantom{aaaaaaaaaa}
\color{red}
-
\frac{3 a}{2} a_{(2,1)}(t) b_{(2,2)}(t)
-
\frac{3 a}{2} a_{(2,2)}(t) b_{(2,1)}(t)
\end{align*}
und $k = 4$
\begin{align*}
|\gamma| = 4
\qquad &
\dot{a}_{(4,0)}(t) = 0
\\
&
\dot{a}_{(3,1)}(t)
=
-
(9 a^2+1)
\nu
a_{(3,1)}(t)
+
\frac{3 a c}{9 a^2+1} b_{(3,1)}(t)
\\
&\phantom{aaaaaaaaaa}
+
\frac{15 a (1 - a^2)}{4 (9 a^2+1)} a_{(1,2)}(t) a_{(2,1)}(t)
\\
&\phantom{aaaaaaaaaa}
+
\frac{12 a (1 + a^2)}{4 (9 a^2+1)} a_{(1,1)}(t) a_{(2,2)}(t)
+
\frac{8 a (5 - 3 a^2)}{4 (9 a^2+1)} a_{(1,3)}(t) a_{(2,2)}(t)
\\
&
\dot{a}_{(2,2)}(t)
=
-
4
(a^2+1)
\nu
a_{(2,2)}(t)
+
\frac{a c}{2 (a^2+1)} b_{(2,2)}(t)
\\
&\phantom{aaaaaaaaaa}
+
\frac{2 a}{a^2+1} a_{(1,1)}(t) a_{(1,3)}(t)
-
\frac{2 a^3}{a^2+1} a_{(1,1)}(t) a_{(3,1)}(t)
\\
&\phantom{aaaaaaaaaa}
+
\frac{4 a^3}{a^2+1} a_{(1,3)}(t) a_{(3,1)}(t)
-
\frac{4 a}{a^2+1} a_{(1,3)}(t) a_{(3,1)}(t)
\\
&
\dot{a}_{(1,3)}(t)
=
-
(a^2+9)
\nu
a_{(1,3)}(t)
+
\frac{a c}{a^2+9} b_{(1,3)}(t)
\\
&\phantom{aaaaaaaaaa}
+
\frac{15 a (1 - a^2)}{4 (a^2+9)} a_{(1,2)}(t) a_{(2,1)}(t)
\\
&\phantom{aaaaaaaaaa}
-
\frac{12 a (1 + a^2)}{4 (a^2+9)} a_{(1,1)}(t) a_{(2,2)}(t)
+
\frac{8 a (3 - 5 a^2)}{4 (a^2+9)} a_{(2,2)}(t) a_{(3,1)}(t)
\\
&
\dot{a}_{(0,4)}(t) = 0
\\
\\
&
\dot{b}_{(4,0)}(t) = 0
\\
&
\dot{b}_{(3,1)}(t)
=
-
(9 a^2+1)
\kappa
b_{(3,1)}(t)
+
\frac{3 a T_{0}}{\pi} a_{(3,1)}(t)
\\
&\phantom{aaaaaaaaaa}
+
\frac{5 a}{4} a_{(2,1)}(t) b_{(1,2)}(t)
-
\frac{5 a}{4} a_{(1,2)}(t) b_{(2,1)}(t)
\\
&\phantom{aaaaaaaaaa}
+
3 a
a_{(3,1)}(t) b_{(0,2)}(t)
+
a
a_{(1,1)}(t) b_{(2,2)}(t)
-
a
a_{(2,2)}(t) b_{(1,1)}(t)
\\
&\phantom{aaaaaaaaaa}
-
2 a
a_{(1,3)}(t) b_{(2,2)}(t)
+
2 a
a_{(2,2)}(t) b_{(1,3)}(t)
\\
&
\dot{b}_{(2,2)}(t)
=
-
(4 a^2+4)
\kappa
b_{(2,2)}(t)
+
\frac{2 a T_{0}}{\pi} a_{(2,2)}(t)
\\
&\phantom{aaaaaaaaaa}
-
a
a_{(2,1)}(t) b_{(0,1)}(t)
+
3 a
a_{(2,1)}(t) b_{(0,3)}(t)
\\
&\phantom{aaaaaaaaaa}
+
a
a_{(1,1)}(t) b_{(1,3)}(t)
-
a
a_{(1,3)}(t) b_{(1,1)}(t)
-
a
a_{(1,1)}(t) b_{(3,1)}(t)
-
a
a_{(3,1)}(t) b_{(1,1)}(t)
\\
&\phantom{aaaaaaaaaa}
+
2 a
a_{(1,3)}(t) b_{(3,1)}(t)
+
2 a
a_{(3,1)}(t) b_{(1,3)}(t)
+
4 a
a_{(2,2)}(t) b_{(0,4)}(t)
\\
&
\dot{b}_{(1,3)}(t)
=
-
(a^2+9)
\kappa
b_{(1,3)}(t)
+
\frac{a T_{0}}{\pi} a_{(1,3)}(t)
\\
&\phantom{aaaaaaaaaa}
-
a
a_{(1,1)}(t) b_{(0,2)}(t)
\\
&\phantom{aaaaaaaaaa}
-
\frac{a}{2} a_{(1,2)}(t) b_{(0,1)}(t)
-
\frac{5 a}{4} a_{(1,2)}(t) b_{(2,1)}(t)
-
\frac{5 a}{4} a_{(2,1)}(t) b_{(1,2)}(t)
\\
&\phantom{aaaaaaaaaa}
-
a
a_{(1,1)}(t) b_{(2,2)}(t)
-
a
a_{(2,2)}(t) b_{(1,1)}(t)
-
2 a
a_{(2,2)}(t) b_{(3,1)}(t)
-
2 a
a_{(3,1)}(t) b_{(2,2)}(t)
\\
&\phantom{aaaaaaaaaa}
+
2 a
a_{(1,1)}(t) b_{(0,4)}(t)
\\
&
\dot{b}_{(0,4)}(t)
=
-
16
\kappa
b_{(0,4)}(t)
\\
&\phantom{aaaaaaaaaa}
-
a
a_{(1,2)}(t) b_{(1,2)}(t)
\\
&\phantom{aaaaaaaaaa}
-
a
a_{(1,1)}(t) b_{(1,3)}(t)
-
a
a_{(1,3)}(t) b_{(1,1)}(t)
-
2a
a_{(2,2)}(t) b_{(2,2)}(t).
\end{align*}
Terme die verloren gehen, wenn man beim jeweiligen $k$ stoppen w"urde, sind 
$\color{blue}{blau}$ (dritter Grad), beziehungsweise $\color{red}{rot}$ 
(vierter Grad) hervorgehoben.

Vergleicht man die Resultate f"ur $k = 2$ mit denjenigen aus 
\cref{skript:lorenz:dim} stellt man fest, dass diese "ubereinstimmen, womit 
auch wieder gezeigt ist, dass unsere neuen Basisfunktionen eine echte 
Erweiterung sind. Bereits jetzt ist aber ersichtlich, dass die Anzahl 
zu l"osenden Gleichungen, in $\text{O}(k^2)$ mit dem Grad $k$ w"achst 
(\cref{table:lorenz2:degree}). Beispielsweise muss f"ur $k = 10$ ein 
Gleichungssystem mit
\begin{equation*}
	2\left(\frac{(10 + 1)(10 + 2)}{2} - 1\right) = 11 \cdot 12 - 2 = 130
\end{equation*}
Gleichungen gel"ost werden, das zudem noch aus Gleichungen besteht, die "uber 
etliche Kopplungen miteinander verbunden sind.

\begin{table}
	\centering
	\begin{tabular}{c | l}
		Grad $k$ & Anzahl Gleichungen \\
		\hline
		1 & $2$ \\
		2 & $2 + 3$ \\
		3 & $2 + 3 + 4$\\
		4 & $2 + 3 + 4 + 5$\\
		\dots & \dots \\
		$n$ & $\dfrac{(n + 1)((n + 1) + 1)}{2} - 1
		= \dfrac{(n + 1)(n + 2)}{2} - 1$
	\end{tabular}
	\caption{Wachstun der Anzahl Gleichung mit dem Grad $k$}
	\label{table:lorenz2:degree}
\end{table}
