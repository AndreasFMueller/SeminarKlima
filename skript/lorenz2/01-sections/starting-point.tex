\section{Einstieg\label{section:lorenz2:einstieg}}
\rhead{Einstieg}

Bevor wir uns auf die Suche nach Basisfunktionen machen, folgen hier ein paar 
Angaben zu den in diesem Kapitel verwendeten Notation und einigen grundlegenden 
Funktionen und Umformungen, welche sp"ater verwendet werden, ein kleiner 
``Teaser'' zu dem was uns noch erwartet sozusagen.

\subsection{Notation}
Die verwendete Notation entspricht grunds"atzlich derjenigen, welche man 
bereits aus der Literatur kennt. Einzig die Verwendung der 
sogenannten Multi-Index-Notationen k"onnte f"ur einige Leser zu Unklarheiten 
f"uhren, daher hier eine kurze Einf"uhrung.

Unter der Multi-Index-Notation versteht man einen Index, der einem 
$n$-Tupel
\begin{equation*}
	\alpha = (\alpha_1, \alpha_2, \dotsc, \alpha_n) \qquad \alpha_k \in 
	\mathbb{N}_{0}
\end{equation*}
entspricht. Zur klaren Unterscheidung zu normalen Indexen, wird meist ein 
griechischer Buchstabe verwendet. Der absolute Wert eiens Multi-Indices ist wie 
folgt definiert
\begin{equation*}
	|\alpha| = \alpha_1 + \alpha_2 + \dots + \alpha_n.
\end{equation*}
 
N"utzlich ist diese Notation insbesondere, wenn mit Summen gearbeitet wird. So 
kann mittels der Forderung $|\alpha| = k$ und unter Ausnutzung der ``Symetrie'' 
von $k$ und $l$ die Summe
\begin{equation}
	\sum_{k = 0}^{\infty}\sum_{l = 0}^{k}a_{l, k - l}
	\label{equation:lorenz2:doublesum}
\end{equation}
vereinfacht werden zu
\begin{equation*}
	\sum_{k = 0}^{\infty}\sum_{|\alpha| = k}a_{\alpha}
	\qquad \alpha = (\alpha_1, \alpha_2).
\end{equation*}
Fortf"uhrend werden wir die innere Summe $\sum_{|\alpha| = k}$ als implizit 
gegeben annehmen und daher nicht mehr ausschreiben. Es gilt also
\begin{equation}
	\sum_{k = 0}^{\infty}c_{\alpha}
	\qquad
	\text{wobei gilt }\alpha = (\alpha_1, \alpha_2),
	c_{\alpha} = \sum_{|\alpha| = k}a_{\alpha}.
	\label{equation:lorenz2:mmsum}
\end{equation}
Was ausgeschrieben
\begin{equation*}
	a_{(0,0)} + a_{(0,1)} + a_{(1,0)} + a_{(0,2)} + a_{(1,1)} + a_{(2,0)} + 
	a_{(0,3)} + 
	\dotsb
\end{equation*}
liefert. Wie man sehen kann, entspricht dies genau dem in 
\cref{equation:lorenz2:doublesum} geforderten Index-$2$-Tupel $(l, k - l)$.

\subsection{Formelzusammenstellung}
Einige Umformungen, die wir in den n"achsten Abschnitten vornehmen werden, 
verwenden teilweise nicht ganz gel"aufige Funkionen. Damit nicht immer gleich 
Wikipedia bem"uht werden muss, werden diese hier zusammengestellt.

Die Signum-Funktion (kurz $\sgn(x)$):
\begin{equation}
\sgn(x) =
\begin{cases*}
1 & f"ur $x$ > 0 \\
0 & f"ur $x$ = 0 \\
-1 & f"ur $x$ < 0
\end{cases*}
\label{equation:lorenz2:signum}
\end{equation}

Die \sgn-Funktion k"onnen wir dazu verwenden Sinus und Kosinus eines absoluten 
Wertes zu berechnen zu vereinfachen:
\begin{equation}
\begin{split}
\sin(|x|) &= \sgn(x)\sin(x) \qquad \Leftrightarrow \qquad \sin(x) = 
\sgn(x)\sin(|x|)
\\
\cos(|x|) &= \cos(x)
\end{split}
\label{equation:lorenz2:sincosabs}
\end{equation}

Produkte von Sinus und Kosinus k"onnen wir mit folgenden S"atzen in einfache 
Summen umgewandeln. Unter anderem kann dies mit Hilfe der Komplexen 
Zahlentheorie gezeigt werden:
\begin{align*}
\cos(x)\sin(y) &= \frac{1}{2} \left(\sin(x + y) - \sin(x - y)\right)
\\
\sin(x)\cos(y) &= \frac{1}{2} \left(\sin(x - y) + \sin(x + y)\right)
\\
\sin(x)\sin(y) &= \frac{1}{2} \left(\cos(x - y) - \cos(x + y)\right)
\\
\cos(x)\cos(y) &= \frac{1}{2} \left(\cos(x - y) + \cos(x + y)\right)
\end{align*}
