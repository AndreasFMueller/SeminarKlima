\section{Einstieg\label{section:lorenz2:einstieg}}
\rhead{Einstieg}

Bevor wir uns auf die Suche nach Basisfunktionen machen, folgen hier ein paar 
Angaben zu den in diesem Kapitel verwendeten Notation und zu einigen 
grundlegenden Funktionen und Umformungen, welche sp"ater verwendet werden. Eine 
kleine Vorschau zu dem was uns noch erwartet sozusagen.

\subsection{Notation}
Die verwendete Notation entspricht grunds"atzlich derjenigen, welche 
bereits aus der Literatur bekannt sind. Einzig die Verwendung der 
sogenannten Multi-Index-Notationen k"onnte f"ur einige Leser zu Unklarheiten 
f"uhren, daher hier eine kurze Einf"uhrung.

Unter der Multi-Index-Notation versteht man einen Index, der einem 
$n$-Tupel
\begin{equation*}
	\alpha = (\alpha_1, \alpha_2, \dotsc, \alpha_n) \qquad \alpha_k \in 
	\mathbb{N}_{0}
\end{equation*}
entspricht. Zur klaren Unterscheidung zu normalen Indices, wird meist ein 
griechischer Buchstabe verwendet. Der absolute Wert eines Multi-Indices ist wie 
folgt definiert
\begin{equation*}
	|\alpha| = \alpha_1 + \alpha_2 + \dots + \alpha_n.
\end{equation*}
 
N"utzlich ist diese Notation insbesondere, wenn mit Summen gearbeitet wird. So 
kann mittels der Forderung $|\alpha| = k$ und unter Ausnutzung der 
``Symmetrie'' von $k$ und $l$ die Summe
\begin{equation}
	\sum_{k = 0}^{\infty}\sum_{l = 0}^{k}a_{l, k - l}
	\label{equation:lorenz2:doublesum}
\end{equation}
auch geschrieben werden kann als
\begin{equation*}
	\sum_{k = 0}^{\infty}\sum_{|\alpha| = k}a_{\alpha}
	\qquad \alpha = (\alpha_1, \alpha_2).
\end{equation*}
Ausgeschrieben liefert uns dies
\begin{equation*}
	a_{(0,0)} + a_{(0,1)} + a_{(1,0)} + a_{(0,2)} + a_{(1,1)} + a_{(2,0)} + 
	a_{(0,3)} + 
	\dotsb
\end{equation*}
Wie man sehen kann, entspricht dies genau dem in 
\cref{equation:lorenz2:doublesum} geforderten $2$-Tupel $(l, k - l)$.

\subsection{Formelzusammenstellung}
Einige Umformungen, die wir in den n"achsten Abschnitten vornehmen werden, 
verwenden teilweise nicht ganz gel"aufige Funkionen. Damit nicht immer gleich 
Wikipedia bem"uht werden muss, werden diese hier zusammengestellt.

Die Signum-Funktion (kurz $\sgn(x)$):
\begin{equation}
\sgn(x) =
\begin{cases*}
1 & f"ur $x$ > 0 \\
0 & f"ur $x$ = 0 \\
-1 & f"ur $x$ < 0
\end{cases*}
\label{equation:lorenz2:signum}
\end{equation}

Die \sgn-Funktion k"onnen wir dazu verwenden, die Berechnung von Sinus und 
Kosinus eines absoluten Wertes zu vereinfachen:
\begin{equation}
\begin{split}
\sin(|x|) &= \sgn(x)\sin(x) \qquad \Leftrightarrow \qquad \sin(x) = 
\sgn(x)\sin(|x|)
\\
\cos(|x|) &= \cos(x)
\end{split}
\label{equation:lorenz2:sincosabs}
\end{equation}

Produkte von Sinus und Kosinus k"onnen wir mit folgenden S"atzen in einfache 
Summen umwandeln. Unter anderem kann dies mit Hilfe der komplexen 
Zahlentheorie gezeigt werden:
\begin{align*}
\cos(x)\sin(y) &= \frac{1}{2} \left(\sin(x + y) - \sin(x - y)\right)
\\
\sin(x)\cos(y) &= \frac{1}{2} \left(\sin(x - y) + \sin(x + y)\right)
\\
\sin(x)\sin(y) &= \frac{1}{2} \left(\cos(x - y) - \cos(x + y)\right)
\\
\cos(x)\cos(y) &= \frac{1}{2} \left(\cos(x - y) + \cos(x + y)\right)
\end{align*}
