\documentclass{book}
\usepackage{etex}
\usepackage{geometry}
\geometry{papersize={170mm,240mm},total={140mm,200mm},top=21mm,bindingoffset=10mm}
\usepackage[english,ngerman]{babel}
\usepackage[utf8]{inputenc}
\usepackage[T1]{fontenc}
\usepackage{cancel}
\usepackage{times}
\usepackage{amsmath,amscd}
\usepackage{amssymb}
\usepackage{amsfonts}
\usepackage{amsthm}
\usepackage{graphicx}
\usepackage{fancyhdr}
\usepackage{textcomp}
\usepackage{txfonts}
%\usepackage{alltt} 
\newcommand\hmmax{0}
\newcommand\bmmax{0}
\usepackage{bm}
\usepackage{verbatim}
\usepackage{paralist}
\usepackage{makeidx}
\usepackage{array}
%\usepackage[colorlinks=true]{hyperref}
\usepackage{hyperref}
\usepackage{subfigure}
\usepackage{tikz}
\usepackage{pgfplots}
\usepackage{pgfplotstable}
\usepackage{pdftexcmds}
\usepackage{pgfmath}
%\usepackage{placeins}
%\usepackage{subfigure}
\usepackage[autostyle=false,english=american]{csquotes}
%\usepackage{float}
%\usepackage{enumitem}
\usepackage{wasysym}
\usepackage{environ}
%\usepackage{pifont}
%\usepackage{feynmp}
\usepackage{appendix}
\usetikzlibrary{calc,intersections,through,backgrounds,graphs,positioning,shapes,arrows,fit}
\usetikzlibrary{patterns,decorations.pathreplacing}
\usetikzlibrary{decorations.pathreplacing}
\usetikzlibrary{external}
\usetikzlibrary{datavisualization}
\usepackage[europeanvoltages,
europeancurrents,
europeanresistors,   % rectangular shape
americaninductors,   % "4-bumbs" shape
europeanports,       % rectangular logic ports
siunitx,             % #1<#2>
emptydiodes,
noarrowmos,
smartlabels]         % lables are rotated in a smart way
{circuitikz}          %
\usepackage{siunitx}
\usepackage{tabularx}
\usetikzlibrary{arrows}
\usepackage{algpseudocode}
\usepackage{algorithm}
\usepackage{gensymb}
\usepackage{mathtools}



\tikzstyle{inputNode}=[draw,circle,minimum size=10pt,inner sep=0pt]
\tikzstyle{stateTransition}=[-stealth, thick]


\begin{document}

\begin{figure}
	\centering
	\begin{tikzpicture}
	\node[inputNode, thick] (i1) at (6, 1) {};
	\node[inputNode, thick] (i2) at (6, 0) {};
	\node[inputNode, thick] (i3) at (6, -1) {};
	
	\node[inputNode, thick] (h_pre) at (8, 2) {};
	
	\node[inputNode, thick] (h1) at (10, 1) {};
	\node[inputNode, thick] (h2) at (10, 0) {};
	\node[inputNode, thick] (h3) at (10, -1) {};
	
	
	\node[inputNode, thick] (o1) at (12, 0.0) {};
	
	
	\draw[stateTransition] (5, 1) -- node[above] {$I_1$} (i1);
	\draw[stateTransition] (5, 0) -- node[above] {$I_2$} (i2);
	\draw[stateTransition] (5, -1) -- node[above] {$I_3$} (i3);
	
	
	
	%\draw[stateTransition] (i1) -- (h1);
	%\draw[stateTransition] (i1) -- (h2);
	%\draw[stateTransition] (i1) -- (h3);
	\draw[stateTransition] (i1) -- (h_pre);
	
	\draw[stateTransition] (i2) -- (h1);
	\draw[stateTransition] (i2) -- (h2);
	\draw[stateTransition] (i2) -- (h3);
	\draw[stateTransition] (i2) -- (h_pre);
	
	
	%\draw[stateTransition] (i3) -- (h1);
	%\draw[stateTransition] (i3) -- (h2);
	%\draw[stateTransition] (i3) -- (h3);
	\draw[stateTransition] (i3) -- (h_pre);
	
	\draw[stateTransition] (h_pre) -- (h1);
	\draw[stateTransition] (h_pre) -- (h2);
	\draw[stateTransition] (h_pre) -- (h3);
	
	
	\draw[stateTransition] (h1) -- (o1);
	\draw[stateTransition] (h2) -- (o1);
	\draw[stateTransition] (h3) -- (o1);
	
	\node[above=of i1, align=center] (l1) {Input \\ layer};
	\node[right=1.1em of l1, align=center] (l2) {Multiplication \\ layer};
	\node[right=0.7em of l2, align=center] (l3) {Derivative \\ layer};
	
	\node[right=1.1em of l3, align=center] (l4) {Output \\ layer};
	
	
	\draw[stateTransition] (o1) -- node[above] {$O_1$} (13, 0);
	\end{tikzpicture}
	\label{fig:mst_variable_hidden_layer}
	\caption{Die vorgeschlagene Topologie des KNN zur Lösung der Burgersgleichung}
\end{figure}

\end{document}