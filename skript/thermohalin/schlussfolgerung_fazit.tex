\section{Schlussfolgerung}
\rhead{Schlussfolgerung}

Es ist erstaunlich, wie sich die Realität mittels eines auf den ersten Blick ziemlich simplen Modelles nachstellen lässt. Trotzdem ist die Simulation mit Vorsicht zu geniessen, da die Resultate nur qualitativ zu interpretieren sind. Genaue Temperaturen und Salinitäten abzulesen ist in dieser Simulation nicht möglich. Die Resultate sind eher als Trend in eine gewisse Richtung zu verstehen. Trotzdem ist es erfreulich, wie gut diese Simulation funktioniert. 
Als weiterer Schritt könnten noch die virtuellen Salzflüsse aufgeteilt werden um Nord- und Südpol unabhängig zu machen. Zusätzlich könnte die Simulation mit zuverlässigeren Daten füttern werden damit genauere Resultate entstehen. Auch sind die Zeitkonstanten der Salinität und Temperatur willkürlich gewählt, was aber keinen Einfluss auf die Simulation hat. Bei der genaueren Einstellung dieser Werte liesse sich vielleicht sogar eine Aussage über den Zeitraum machen. Dazu müsste aber das restliche Modell fehlerfrei sein, da diese Fehler wiederum die Resultate verfälschen würden.
Um tatsächlich Temperaturen und Salinitäten abzulesen müsste man die Simulation massiv erweitern, und grosse Datenmengen verwenden. Dies hätte jedoch den Rahmen dieses Projektes gesprengt. 
Die Forscher um Liu Wei \cite{thermohalin:liuwei} verwendeten die \texttt{\em NCAR SSCM3}\cite{thermohalin:sim}. Simulation, welche dieser hier meilenweit überlegen ist. Diese Simulationssoftware ist in Module aufgeteilt, so lassen sich Wasser, Atmosphäre, Eis und Land separat oder in kombination simulieren. Das hat den grossen Vorteil, dass man bei der Strömungssimulation nicht nur den Ozean, sondern auch seine Wechselwirkung mit Atmosphäre und Kontinenten berücksichtigen kann. 
Schlussendlich war das Verstehen und Programmieren dieser Modelle sehr lehrreich. Der Einblick in das mathematische Gebiet der Simulation und deren Resultate waren sehr spannend. Schön ist auch, dass sich tatsächlich eine Simulation erstellen liess, welche, wenn auch nur rudimentär, tatsächlich die Realität wiedergibt.

Der gesamte Simulationscode welcher im Rahmen dieses Seminars erstellt wurde, ist im Unterkapitel \texttt{thermohalin} des Gitrepository des Seminarbuches abgelegt\cite{thermohalin:gitrepo-klimawandel}.
