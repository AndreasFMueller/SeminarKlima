%
% buchcover.tex -- Cover für das Buch Klimawandel
%
% (c) 2018 Prof Dr Andreas Müller, Hochschule Rapperswil
%
\documentclass[11pt]{standalone}
\usepackage{tikz}
\usepackage{times}
\usepackage{geometry}
\usepackage{german}
\usepackage[utf8]{inputenc}
\usepackage[T1]{fontenc}
\usepackage{times}
\usepackage{amsmath,amscd}
\usepackage{amssymb}
\usepackage{amsfonts}
\usepackage{txfonts}
\usepackage{ifthen}
\usetikzlibrary{math}
\geometry{papersize={402mm,278mm},total={405mm,278mm},top=72.27pt, bottom=0pt, left=72.27pt, right=0pt}
\newboolean{guidelines}
\setboolean{guidelines}{true}
\setboolean{guidelines}{false}

\begin{document}
\begin{tikzpicture}[>=latex, scale=1]
\tikzmath{
	real \ruecken, \einschlag, \gelenk, \breite, \hoehe;
	\ruecken = 2.2;
	\einschlag = 1.6;
	\gelenk = 0.7;
	\breite = 16.7;
	\hoehe = 24.6;
	real \bogengreite, \bogenhoehe;
	\bogenbreite = 2 * (\breite + \einschlag + \gelenk) + \ruecken;
	\bogenhoehe = 2 * \einschlag + \hoehe;
}

%\clip (0,0) circle (6);

\draw[fill=blue](0,0) rectangle({\bogenbreite},{\bogenhoehe});
\hsize=13.6cm

\begin{scope}
\clip (0,0) rectangle({\bogenbreite},{\bogenhoehe});
\node at (17.2,9.3) [scale=0.2]{\includegraphics{bild.jpg}};
\end{scope}

\node at ({\einschlag+2*\gelenk+\ruecken +1.5*\breite},12.5)
	[color=white,scale=2.5]
	{$\displaystyle C\frac{dT}{dt} = \bigl(1-\alpha(T)\bigr) Q -\varepsilon \sigma T^4$};

\node at ({\einschlag+2*\gelenk+\ruecken+1.5*\breite},24.3)
	[color=white,scale=1]
	{\hbox to\hsize{\hfill%
	\sf \fontsize{24}{24}\selectfont Mathematisches Seminar}};

\node at ({\einschlag+2*\gelenk+\ruecken+1.5*\breite},21.9)
	[color=white,scale=1]
	{\hbox to\hsize{\hfill%
	\sf \fontsize{50}{50}\selectfont Klimawandel}};

\node at ({\einschlag+2*\gelenk+\ruecken+1.5*\breite},19.7)
	[color=white,scale=1]
	{\hbox to\hsize{\hfill%
	\sf \fontsize{13}{5}\selectfont Andreas Müller}};

\node at ({\einschlag+2*\gelenk+\ruecken+1.5*\breite},18.4)
	[color=white,scale=1]
	{\hbox to\hsize{\hfill%
	\sf \fontsize{13}{5}\selectfont
	Matthias Baumann,
	Oliver Dias-Lalcaca,
	Jonas Gründler%,
	%Matthias Dunkel
	}};

\node at ({\einschlag+2*\gelenk+\ruecken+1.5*\breite},17.75)
	[color=white,scale=1]
	{\hbox to\hsize{\hfill%
	\sf \fontsize{13}{5}\selectfont
	%Jonas Gründler,
	Sebastian Lenhard,
	Silvio Marti,
	Michael Müller,
	Hansruedi Patzen%
	}};

\node at ({\einschlag+2*\gelenk+\ruecken+1.5*\breite},17.1)
	[color=white,scale=1]
	{\hbox to\hsize{\hfill%
	\sf \fontsize{13}{5}\selectfont
	%Hansruedi Patzen,
	Melina Staub,
	Martin Stypinski,
	Nicolas Tobler,
	Raphael Unterer%
	}};
 
\node at ({\einschlag+2*\gelenk+\ruecken+1.5*\breite},16.45)
	[color=white,scale=1]
	{\hbox to\hsize{\hfill%
	\sf \fontsize{13}{5}\selectfont
%	Raphael Unterer
	}};
 
%\node at (0,3) [color=white] {\sf \LARGE Mathematisches Seminar 2017};

% Rücken
\node at ({\bogenbreite/2 + 0.05},20.5) [color=white,rotate=-90]
	{\sf\fontsize{35}{0}\selectfont Klimawandel};

% Buchrückseite
\node at ({\einschlag+0.5*\breite},18.6) [color=white] {\sf
\fontsize{13}{16}\selectfont
\vbox{%
\parindent=0pt
%\raggedright
Im Rahmen des Mathematischen Seminars der Hochschule für Technik Rapperswil
wurden im Frühjahrssemester 2018 verschiedene mathematische Methoden
und Modelle behandelt, die das Phänomen des Klimawandels zu verstehen
helfen.
Dieses Buch bringt das Skript des Vorlesungsteils mit den von den
Seminarteilnehmern beigetragenen Seminararbeiten zusammen.
}};


\ifthenelse{\boolean{guidelines}}{
\draw[white] (0,{\einschlag})--({\bogenbreite},{\einschlag});
\draw[white] (0,{\bogenhoehe-\einschlag})--({\bogenbreite},{\bogenhoehe-\einschlag});

\draw[white] ({\einschlag},0)--({\einschlag},{\bogenhoehe});
\draw[white] ({\einschlag+\breite},0)--({\einschlag+\breite},{\bogenhoehe});
\draw[white] ({\einschlag+\breite+\gelenk},0)--({\einschlag+\breite+\gelenk},{\bogenhoehe});
\draw[white] ({\bogenbreite-\einschlag-\breite-\gelenk},0)--({\bogenbreite-\einschlag-\breite-\gelenk},{\bogenhoehe});
\draw[white] ({\bogenbreite-\einschlag-\breite},0)--({\bogenbreite-\einschlag-\breite},{\bogenhoehe});
\draw[white] ({\bogenbreite-\einschlag},0)--({\bogenbreite-\einschlag},{\bogenhoehe});
}{}

\end{tikzpicture}
\end{document}
