%
% slides.tex -- slides
%
% (c) 2017 Prof Dr Andreas Müller, Hochschule Rapperswil
%
\theoremstyle{definition}
\newtheorem{trigopol}{Trigonometrische Polynome}
\newtheorem{fourier}{Fourier-Reihe}
\newtheorem{leastsquares}{Kleinste Quadrate}
\newtheorem{loesung}{Lösung}
\newtheorem{geometrie}{Schwerpunkt}
\newtheorem{produkte}{Produkte}

\begin{document}

\setlength{\abovedisplayskip}{3pt}
\setlength{\belowdisplayskip}{3pt}

\ifthenelse{\boolean{presentation}}{
\begin{frame}
\titlepage
\end{frame}
}{}

%
% Idee von Fourier
%
\begin{frame}
\frametitle{Die Idee von Fourier}
\begin{trigopol}
Ein trigonometrisches Polynom ist eine Funktion der Form
\begin{align*}
f(t)
&=
\frac{a_0}2 + a_1\cos t+b_1\sin t + a_2\cos 2t + b_2\sin 2t + \dots + a_n\cos nt+b_n\sin nt
\\
&=
\frac{a_0}2 + \sum_{k=1}^n \bigl( a_k \cos kt + b_k\sin kt\bigr)
\end{align*}
\end{trigopol}
\begin{fourier}
Eine periodische Funktion $f(t)$ kann durch eine Fourier-Reihe beliebig
genau approximiert werden:
\[
f(t)
=
\frac{a_0}2 + \sum_{k=1}^{\infty} \bigl( a_k \cos kt + b_k\sin kt\bigr)
\]
\end{fourier}
\end{frame}

%
% Kleinste Quadrate
%
\begin{frame}
\frametitle{Kleinste Quadrate}
\begin{columns}
\begin{column}{0.47\hsize}
\begin{leastsquares}
Zu vorgegebenen Funktionswerten $y_j=f(t_j)$ in den Punkten
\[
t_j = 2\pi \frac{j}{N},\qquad j=0,\dots,N-1
\]
finde Koeffizienten $\hat a_k$ und $\hat b_k$ derart, dass
\[
L=\sum_{j=0}^{N-1} \bigl(y_j - f(t_j)\bigr)^2
\]
minimal wird, wobei $f(t)$ ein trigonometrisches Polynom ist.
\end{leastsquares}
\end{column}
\begin{column}{0.5\hsize}
\begin{loesung}
$a_k$ und $b_k$ so bestimmen, dass
\begin{align*}
\frac{\partial L}{\partial a_k}&=0&
\frac{\partial L}{\partial b_k}&=0
\end{align*}
gilt.
Berechnung der Ableitungen:
\begin{align*}
\frac{\partial L}{\partial a_k}
&=
\sum_{j=0}^{N-1}2\bigl(y_j-f(t_j)\bigr)
\frac{\partial f(t_j)}{\partial a_k}
\\
\frac{\partial f(t_j)}{\partial a_k}
&=
\begin{cases}
\frac12&\qquad k=0
\\
\cos kt_j&\qquad\text{sonst}
\end{cases}
\end{align*}
und analog für $\sin kt_j$.
\end{loesung}
\end{column}
\end{columns}
\end{frame}

%
% Trigonometrische Summen
%
\begin{frame}
\frametitle{Trigonometrische Summen}
\begin{columns}
\begin{column}{0.48\hsize}
\begin{geometrie}
Die Punkte $(\cos kt_j,\sin kt_j)$ sind gleichmässig auf einem Kreis verteilt:
\\[3pt]
Schwerpunkt der roten Punkte ist $0$,
ausser für $k=0$
\end{geometrie}
\end{column}
\begin{column}{0.48\hsize}
\def\r{1}
\begin{tikzpicture}[>=latex,thick]
\draw (0,0) circle[radius={\r}];
\draw[->] (-1.3,0)--(1.3,0);
\draw[->] (0,-1.3)--(0,1.3);
\foreach\i in {0,1,...,16}{
\fill[color=red] ({\r*cos(360*\i/17)},{\r*sin(360*\i/17)}) circle[radius=1.5pt];
}
\node at (-0.7,-1.2) {$N=17$};
\node at (1.5,0) [right] {\begin{minipage}{3cm}\raggedright
\begin{align*}
\sum_{j=0}^{N-1} \cos kt_j&=N\delta_{k0}\\
\sum_{j=0}^{N-1} \sin kt_j&=0
\end{align*}
\end{minipage}};
\end{tikzpicture}
\end{column}
\end{columns}
\begin{produkte}
\vspace{-33pt}
\[
\left.
\begin{aligned}
\cos\alpha\cos\beta&={\textstyle\frac12}(\cos(\alpha-\beta)+\cos(\alpha+\beta))
\\
\sin\alpha\sin\beta&={\textstyle\frac12}(\cos(\alpha-\beta)-\cos(\alpha+\beta))
\\
\sin\alpha\cos\alpha&={\textstyle\frac12}(\sin(\alpha-\beta)+\sin(\alpha+\beta))
\end{aligned}
\right\}
\Rightarrow
\left\{
\begin{aligned}
\sum_{j=0}^{N-1}\cos kt_j\cos lt_j&=\frac{N}2\delta_{kl}
\\
\sum_{j=0}^{N-1}\sin kt_j\sin lt_j&=\frac{N}2\delta_{kl}
\\
\sum_{j=0}^{N-1}\sin kt_j\cos lt_j&=0
\end{aligned}
\right.
\]
\end{produkte}
\end{frame}

%
% Lösung a_0
%
\begin{frame}
\frametitle{Lösung $a_0$}
\begin{align*}
\frac{\partial L}{\partial a_0}
&=
\sum_{j=0}^{N-1} 2\biggl(y_j-\frac{a_0}2-\sum_{k=1}^\infty
\bigl(a_k\cos kt_j + b_k\sin kt_j\bigr)
\biggr)\frac12=0
\\
0
&=
\sum_{j=0}^{N-1} y_j - N\frac{a_0}{2} 
+\sum_{k=1}^\infty\biggl(
a_k\underbrace{\sum_{j=0}^{N-1}\cos kt_j}_{\displaystyle=0}
+
b_k\underbrace{\sum_{j=0}^{N-1}\sin kt_j}_{\displaystyle=0}
\biggr)
\\
\Rightarrow\;
\hat a_0&=\frac{2}{N}\sum_{j=0}^{N-1} y_j
\end{align*}
\end{frame}

%
% Lösung a_k
%
\begin{frame}
\frametitle{Lösung $a_k$}
\begin{align*}
\frac{\partial L}{\partial a_k}
&=
\sum_{j=0}^{N-1} 2\biggl(y_j-\frac{a_0}2-\sum_{l=1}^\infty
\bigl(a_l\cos lt_j + b_l\sin lt_j\bigr)
\biggr)\cos kt_j=0
\\
\sum_{j=0}^{N-1}y_j\cos kt_j
&=
\frac{a_0}{2}\underbrace{\sum_{k=1}^\infty\cos kt_j}_{\displaystyle=0}
+
\sum_{l=1}^\infty
\biggl(
a_l
\underbrace{
\sum_{j=1}^{N-1}
\cos lt_j
\cos kt_j
}_{\displaystyle=\frac{N}{2}\delta_{kl}}
+
b_l
\underbrace{
\sum_{j=1}^{N-1}
\sin lt_j
\cos kt_j
}_{\displaystyle=0}
\biggr)
\\
\Rightarrow\;
\hat a_k&=\frac{2}{N}\sum_{j=0}^{N-1}y_j\cos kt_j
\end{align*}
\end{frame}

%
% Lösung b_k
%
\begin{frame}
\frametitle{Lösung $b_k$}
\begin{align*}
\frac{\partial L}{\partial b_k}
&=
\sum_{j=0}^{N-1} 2\biggl(y_j-\frac{a_0}2-\sum_{l=1}^\infty
\bigl(a_l\cos lt_j + b_l\sin lt_j\bigr)
\biggr)\sin kt_j=0
\\
\sum_{j=0}^{N-1}y_j\cos kt_j
&=
\frac{a_0}{2}\underbrace{\sum_{k=1}^\infty\sin kt_j}_{\displaystyle=0}
+
\sum_{l=1}^\infty
\biggl(
a_l
\underbrace{
\sum_{j=1}^{N-1}
\cos lt_j
\sin kt_j
}_{\displaystyle=0}
+
b_l
\underbrace{
\sum_{j=1}^{N-1}
\sin lt_j
\sin kt_j
}_{\displaystyle=\frac{N}{2}\delta_{kl}}
\biggr)
\\
\Rightarrow\;
\hat b_k&=\frac{2}{N}\sum_{j=0}^{N-1}y_j\sin kt_j
\end{align*}
\end{frame}

\end{document}
