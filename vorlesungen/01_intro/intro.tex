%
% intro.tex -- XXX
%
% (c) 2017 Prof Dr Andreas Müller, Hochschule Rapperswil
%
\theoremstyle{definition}
\newtheorem{wetter}{Wetter}
\newtheorem{klima}{Klima}
\newtheorem{mitteltemperatur}{Global Mitteltemperatur}

\begin{document}

\begin{frame}
\frametitle{Wetter und Klima}
\begin{wetter}
Als {\em Wetter} bezeichnet man den
spürbaren, kurzfristigen Zustand der Atmosphäre (auch: messbarer
Zustand der Troposphäre) an einem bestimmten Ort der Erdoberfläche,
der unter anderem als Sonnenschein, Bewölkung, Regen, Wind, Hitze
oder Kälte in Erscheinung tritt.
\end{wetter}
\begin{klima}
Das {\em Klima} steht als Begriff für die Gesamtheit aller meteorologischen
Vorgänge, die für die über Zeiträume von mindestens 30 Jahren
regelmässig wiederkehrenden durchschnittlichen Zustände der Erdatmosphäre
an einem Ort verantwortlich sind.
\end{klima}
\end{frame}

\begin{frame}
\frametitle{Beispiel}
\begin{mitteltemperatur}
Globale Mitteltemperatur $T$ als Proxy für den Energieinhalt der Atmosphäre
\end{mitteltemperatur}

\bigskip

\pause
$\Rightarrow$ Die Änderung von $T$ hängt vom aktuellen Energieinhalt der
Atmosphäre ab:
\[
\frac{dT}{dt}
=
f(T)
\]

\bigskip
\pause
Weitere Parameter, z.~B.~$\text{CO}_2$-Gehalt $\lambda$
\[
\frac{dT}{dt}
=
f(T,\lambda)
\]
Wie ändert sich der Gleichgewichtszustand in Abhängigkeit von $\lambda$?

\end{frame}

\end{document}
